\adhyAya
\stitle{पीलातस्य समक्षं।}
\vakya ततः परं तेषां कृत्स्नो निकर उत्थाय तं पीलातस्य समीपमनैषीत्
\vakya तस्य विरुद्धञ्च तेऽभियोगं कुर्वन्तो भाषितुमारेभिरे, लक्षितोऽयमस्माभि र्यद् राजा ख्रीष्टोऽहमिति व्याहरन्नयं जातिमिमामुन्मार्गगामित्वे प्रवर्तयति कैसराय करदानं निवारयति च।
\vakya पीलातस्तदा तं पप्रच्छ, त्वं किं यिहूदीयानां राजा? स तं प्रतिबभाषे, भवांस्तथ्यं व्याहरति।
\vakya ततः पीलातो मुख्ययाजकान् जननिवहांश्च जगाद, नरोऽस्मिन् मया कोऽपि दोषो न लक्ष्यते।
\vakya ते तु गाढतरमवदन्, स प्रजा उत्तेजयति, गालीलत आरभ्यैतत् स्थानं यावद् यिहूदियायाः सर्वत्रोपदिशति च।
\stitle{हेरोदस्य समक्षं।}
\vakya इत्थं गालीलस्य नाम श्रुत्वा पीलातः पप्रच्छ, स नरः किं गालीलीयः?
\vakya ततः स हेरोदस्य कर्तृत्वाधीन इत्यवगम्य स तं हेरोदस्य समीपं प्राहिणोत्, यतस्तेषु दिनेषु सोऽपि यिरूशालेमेऽवर्तत।
\vakya यीशुं दृष्ट्वा हेरोदोऽतीव जहर्ष, यत आदीर्घकालात् स तमदिदृक्षत, यतः स तमधि बह्वीः कथा अशृणोत्, तेन प्रतर्शितं किञ्चनाभिज्ञानं द्रक्ष्यामीत्याशापि तस्य सञ्चाता।
\vakya तस्मात् स तं बहुप्रश्नानपृच्छत्। स तु तं किमपि न प्रत्यभाषत।
\vakya अपि च तत्र स्थिता मुख्ययाजकाः शास्त्राध्यापकाश्चैकाग्रं तस्याभियोगमकुर्वन्।
\vakya हेरोदस्तु स्वसैन्यानां साहाय्येन तं परुषीकृत्योपजहास शुभ्रं प्रावारं परिधाप्य च तं पुनः पीलातस्यान्तिकं प्रेषयामास।
\vakya तस्मिन् दिने पीलातहेरोदौ परस्परं प्रणयिनौ बभूवतुः, यतस्तत्पूर्वं तौ परस्परं वैरमास्थितौ
\stitle{पीलातस्य विचारो दण्डाज्ञा च।}
\vakya अथ पीलातो मुख्ययाजकान् नायकान् प्रजाजनञ्च समाहूयाब्रवीत्,
\vakya मनुष्योऽयं प्रजाजनस्य कुप्रवर्तक इव युष्माभि र्मदन्तिकमानीतः।
\vakya पश्यत तु यूयमस्मिन् यान् दोषानारोपयथ मया युष्मत्समक्षं विचारं कुर्वतास्मिन् नरे तेषां कोऽपि दोषो नु लक्षितः, हेरोदेनापि न लक्षितः, यतोऽहं युष्मान् तस्यान्तिकं प्रेषितवान्। पश्यत च नापराद्धमनेन प्राषदण्डहेतुकं किमपि।
\vakya अतोऽहमिमं शासित्वा मोचयिष्यामि।
\vakya प्रतिवत्सरन्तु पर्वाणि तेषां तोषणार्थमेकस्य मोचनं तेनावश्यं कर्तव्यमासीत्।
\vakya अतस्तेषां कृत्स्नजनतोत्क्रोशन्ती जगाद, संहरेमं बाराब्बाञ्चास्मदर्थं मोचय।
\vakya नरः स नगरे, सम्भूतस्योपप्लवस्य नरहत्यायाश्च कारणात् काराबद्ध आसीत्।
\vakya पीलातस्तदा यीशुं मोचयितुमिच्छंस्तान् पुनरुच्चैः सम्बोधयामास।
\vakya ते तूच्चै र्जगदुः, क्रुशमारोपयेमं, क्रुषमारोपय।
\vakya पीलातस्तृतीयवारं तान् बभाषे, किं न्वपराद्धमनेन? न लक्षितोऽस्मिन् मया प्राणदण्डहेतुकः कोऽपि दोषः। तदिमं शासित्वा मोचयिष्यामि।
\vakya ते तु महाशब्दैरुग्रतामास्थाय तस्य क्रुशारोपणं ययाचिरे, तेषाञ्च मुख्ययाजकानाञ्च शब्दाः प्रबभूवुः,
\vakya पीलातश्च तेषां याञ्चायाः साधनं निश्चिकाय।
\vakya इत्थमुपप्लवस्य नरहत्यायाश्च कारणात् काराबद्धो यो नरस्तै र्याचितः स तेषां तोषणार्थं तेन मोचितः, यीशुस्तु तेषामभिलाषे समर्पितः।
\stitle{यीशोः क्रुशारोपणं मृत्युश्च।}
\vakya अथ जना यदा तमपानयंस्तदा ग्रामादागच्छन्तं शिमोनाभिधमेकं कुरीणीयं नरं धृत्वा यीशोः पश्चाद् वहनार्थं तत्कन्धे क्रुशमर्पयामासुः।
\vakya महान् जननिवहश्च तमन्वगच्छत् तन्मध्ये योषितोऽप्यासन्, तास्तदर्थं स्ववक्षांस्यताडयन् व्यलपंश्च।
\vakya यीशुस्तु मुखं परिवर्त्य ता अब्रवीत्, भो यिरूशालेमस्य दुहितरः, मदर्थं रुदितापि तु स्वार्थं स्वसन्तानवृन्दार्थञ्च,
\vakya यतः पश्यतायान्ति तानि दिनानि यदा जनै र्वक्ष्यते, धन्या वन्ध्याः, धन्या जठरा अप्रसूताः, धन्याश्च स्तना अदोहिताः।
\vakya तदा मनुष्या गिरीन् सम्बोध्य वदिष्यन्ति, अस्मासु निपततेति, शैलांश्च वक्ष्यन्ति, अस्मान् प्रच्छादयतेति।
\vakya यतः सरसं तरुं प्रति यद्येतत् क्रियते, का तर्हि शुष्कस्य गति र्भविष्यति?
\vakya अपि च तेन सार्धमपरौ द्वौ दुष्कर्मिणौ व्यापादनायापानीयेताम्।
\vakya ततः परं ते कपालमित्यभिधं स्थानं प्राप्य तत्रैव तञ्च तौ दुष्कर्मिणौ च क्रुशान्यारोपयामासुरेकतरं तस्य दक्षिणे, वामे चान्यतरं।
\vakya यीशुस्तदोवाच, पितः, क्षमस्व तेषां, यतः किं कुर्वन्ति तन्न जानन्ति।
\vakya अनन्तरं ते तस्य वासांसि विभजमाना गुटिकापातमकार्षुः प्रजाजनाश्च निरीक्षमाणा अतिष्ठन्। तैः सार्धं नायका अपि तं न्यक्कुर्वन्तोऽवदन्, असावन्यांस्तारयामास, यद्यसावीश्वरस्य वरितः ख्रीष्टस्तर्ह्यात्मानं तारयतु।
\vakya सैनिका अपि तमुपहसन्त उपागत्याम्लरसमुपाहरन्नवदंश्च,
\vakya त्वं चेद् यिहूदीयानां राजा तर्ह्यात्मानं तारय।
\vakya तस्योर्ध्वे च यूनानीयै रोमीयैरिब्रीयैश्चाक्षरै र्लिखितमिदं लेख्यमासीत्, यिहूदीयानां राजायमिति।
\vakya तेन सार्धमुद्वद्धयो र्दुष्कर्मिणोरेकतरोऽपि तं निन्दन्नाह, त्वं यदि ख्रीष्टस्तर्ह्यात्मानमावाञ्च तारय।
\vakya अन्यतरस्तु तं भर्त्सयन्नब्रवीत्, ईश्वरादपि किं त्वं न बिभेषि तेनैव दण्डेनाक्रान्तः?
\vakya आवां हि तदर्हौ, यत आवाभ्यां यद्यदपराद्धं तदुपयुक्तानि फलानि भुज्यन्ते, अयन्त्वयुक्तं किमपि न कृतवान्।
\vakya ततः परं स यीशुं जगाद, प्रभो, भवान् यदा स्वराज्यत्वेनायास्यति, तदा मां स्मरसु।
\vakya यीशुश्च तमवादीत्, त्वामहं सत्यं ब्रवीमि, अद्य त्वं परमदेशे मम सङ्गी भविष्यसि।
\vakya तदा प्रायेणोपस्थितायां षष्ठघटिकायां कृत्स्ने भूमण्डलेऽन्धकारः सञ्जातः, स नवमघटिकां यावदवतस्थे।
\vakya सूर्यस्तदान्धकारमयो बभूव, मन्दिरस्य तिरस्करिणी च छिन्ना द्विधा बभूव।
\vakya यीशुश्चोच्चरवमुदीरयन् व्याजहार, पितस्तव हस्तयोर्ममात्मानं समर्पयामि। इदमुक्त्वा स प्राणांस्तत्याज।
\vakya तदितिवृत्तं दृष्ट्वा शतपतिरीश्वरं स्तुवन्नुवाच, सत्यं, नरोऽसौ धार्मिक आसीत्।
\vakya तत्कौतुकदर्शनार्थमेकत्रीभूता जननिवहाश्च सर्वे तदितिवृत्तं निरीक्ष्य स्ववक्षांसि ताडयन्तः प्रतिजग्मुः।
\vakya आसंश्च दूरे तिष्ठन्तस्तत्परिचितजनाः सर्वे याश्च गालीलतस्तेन सार्धमनुगतास्तासां काश्चिद् योषितस्तत् सर्वं निरीक्षमाणाः।
\stitle{यीशोः समाधिः।}
\vakya पश्य च यिहूदीयानाम् अरिमाथियानगरान्नर एको योषेफनामोपतस्थे।
\vakya स मन्त्री किन्तु भद्रो धार्मिकश्च नरोऽमीषां मन्त्रणाचाराभ्यामसम्पृक्तश्चासीत् स्वयमपीश्वरस्य राज्यं प्रत्यैक्षत च।
\vakya स एव पीलातस्यान्तिकं गत्वा तं यीशो र्देहं ययाचे।
\vakya तमवरोह्य च सूक्ष्मवाससा वेष्टयामास, यत्र च प्राक् कोऽपि कदापि न शयितस्तादृशो तक्षिते शवागारे तं निदधाय।
\vakya तद्दिनं सज्जनदिनं, विश्रामवारस्यारम्भश्चासन्न आसीत्।
\vakya गालीलतस्तेन सार्धमागता नार्योऽप्यनुगच्छन्त्यस्तच्छवागारमवलोकयामासु र्देहश्च कथं निधीयते तद् ददृशुः।
\vakya ततः परं प्रत्यावृत्य सुगन्धीनि द्रव्याणि तैलानि च सज्जीचक्रुः। तथापि विधानानुसारेण ता विश्रामवारे विश्राममसेवन्त\eoc