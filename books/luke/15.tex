\adhyAya
\stitle{हारितमेषहारितरूप्ययो र्दृष्टान्तः।}
\vakya अथ शुल्कादायिषु पापिषु च सर्वेषु तस्य वाक्यानि श्रोतुं तदन्तिकमागच्छत्सु फरीशिनः शास्त्राध्यापकाश्चासन्तुष्यावदन्,
\vakya पापिनोऽनुगृह्णात्यसौ तैः सार्धमश्नाति च।
\vakya स तु तेभ्यो दृष्टान्तकथामिमां कथयामास, युष्माकं को नरः शतमेषाणां स्वामी?
\vakya तेषामेकस्मिन् हारिते स किं नैकोनशतं प्रान्तरे परित्यज्य तमेकं हारितं यावन्न प्राप्नोति तावदनुसरति?
\vakya प्राप्य च स हृष्टस्तं स्कन्धदेशमारोपयति गृहमागत्य च बन्धून् समीपवासिनश्चाहूय वदति, मया सार्धमानन्दत,
\vakya यतो मम हारितो मेषो मया पुनर्लब्धः।
\vakya युष्मानहं ब्रवीमि, तथैव पापिन्येकस्मिन् मनः परावर्तयति स्वर्ग आनन्दो भविष्यति। न भविष्यति तादृश एकोनशतधार्मिकेष्वपि येषां मनः परावर्तनमप्रयोज्यम्।
\vakya यस्या दशमुद्रापादाः सन्ति तादृशी वा का योषित् एकस्मिन् मुद्रापादे हारिते न दीपिकां प्रज्वालयति यावच्च तं न प्राप्नोति तावद् गृहं परिमार्जन्ती समत्नं गवेषयति?
\vakya तस्मिन्नासादिते च सा सखीः प्रतिवेशिनीश्च समाहूय वदति, मया सार्धमानन्दत, यतः स हारितो मुद्रापादो मया पुनर्लब्धः।
\vakya युष्मानहं ब्रवीमि, तथैवैकस्मिन् पापिनि मनः परावर्तयतीश्वरस्य दूतानां समक्षमानन्दः सम्भवति।
\stitle{अपव्ययिपुत्रस्य दृष्टान्तः।}
\vakya स पुनः कथयामास, नरस्य कस्यचिद् द्वौ पुत्रावास्तां।
\vakya तयोः कनिष्ठः पितरमाह, पितः देहि मह्यं रिक्थस्य तमंशं यो मया प्राप्तव्यः। ततः स तयोः कृते रिक्थं विबभाज।
\vakya अनधिकदिवसेभ्यः परं स कनिष्ठः सर्वं सङ्गृह्य दूरदेशं प्रतस्थे तत्र च लाम्पट्येन जीवनं यापयन् स्ववित्तं प्राकारीत्।
\vakya व्यतिते तु तेन सर्वस्वे तस्मिन् देशे घोरतरं दुर्भिक्षं सञ्जातं, स च कष्टभागी भवितुमारेभे।
\vakya तदा स गत्वा तत्प्रदेशस्थे कस्मिंश्चित् पौरे ससज्जे, स च तं शूकरचारणार्थं स्वग्रामान्तीयक्षेत्राणि प्रेषयामास।
\vakya तत्र स शूकराणां भोज्यै र्वन्यमाषकोषैरुदरं परिपूरयितुमवाञ्छत्, कोऽपि तु तस्मै नाददात्।
\vakya ततः परं स चेतनां लब्ध्वा बभाषे, मत्पितुः कति वेतनजीविनः पूपैरतितृप्ता भवन्ति, अहन्त्वत्र क्षुधया नश्यामि।
\vakya अहमुत्थाय मत्पितरमभियास्यामि, तञ्च वदिष्यामि, पितः, स्वर्गस्य विरुद्धं भवतः समक्षञ्च पापं कृतं मया, भवतः पुत्र इति नाम मयि पुन र्न युज्यते, भवान् मां स्ववेतनजीविनामेकेन समानं करोतु।
\vakya अनन्तरं स उत्थाय स्वपितरमभिजगाम। तन्तु दूरे स्थितमेव पिता लक्षयामासानुकम्प्य च धावित्वा तस्य कण्ठं धृत्वा चुचुम्ब।
\vakya पुत्रस्तदा तमवादीत्, पितः स्वर्गस्य विरुद्धं भवतः समक्षञ्च पापं कृतं मया, भवतः पुत्र इति नाम मयि पुन र्न युज्यते।
\vakya तस्य पिता तु स्वदासानाह, सत्वरं सर्वोत्तमं प्रावारं बहिरानीयेमं परिधापयत, अस्य हस्ते चाङ्गुरीयकं चरणयोश्चोपानहावर्पयत,
\vakya अस्मत्पोषितं गोवत्समानीय मारयत च, अस्माभि र्भुज्यतां हर्षः सेव्यताञ्च,
\vakya यतो ममायं पुत्रो मृत आसीत् पुनर्जीवितश्चाभूत्, हारित आसीत् पुनर्लब्धश्चाभूत्। ततस्ते हर्षं सेवितुमारेभिरे।
\vakya तस्य ज्येष्ठः पुत्रस्तु तदा क्षेत्र आसीत्। आगमनकाले स यदा गृहान्तिकमुपातिष्ठत्, तदा वाद्यनृत्यानां शब्दमश्रौषीत्।
\vakya ततः स दासानामेकं समीपमाहूय पप्रच्छ, किमर्थमेतत्?
\vakya स तमवादीत्, भवतो भ्रातागतः, भवतः पिता च तं स्वस्थं लब्धवांस्तद्धेतोः पोषितं गोवत्सं मारयामास।
\vakya ततः स कुपित्वा प्रवेष्टुमनङ्गीचक्रे। तस्य पिता तदा बहिरागत्य तं प्रासादयत्।
\vakya स तु प्रतिभाषमाणः पितरमाह, पश्यैतावतो वत्सरानहं त्वां परिचरामि, कदापि तवाज्ञां न लङ्घामि स्म, मह्यं हि त्वं कदापि मित्रैः सार्धं हर्षसेवनार्थं छागशावकं नाददाः।
\vakya यस्तु वेश्याभिः सार्धं तव जीविकां ग्रस्तवांस्तव तस्मिन् पुत्रेऽमुष्मिन्नागतवत्येव तदर्थं पोषितं गोवत्सममारयः।
\vakya स तु तमब्रवीत्, वत्स, त्वं सततं मया सार्धं वर्तसे, मम च यद्यदस्ति तत् सर्वं तव।
\vakya प्रत्युत हर्षसेवनमाह्लादनञ्चास्मासु युज्येते स्म, यतस्तव भ्रातासौ मृत आसीत् पुनर्जीवितश्चाभूत्, हारित आसीत् पुनर्लब्धश्चाभूत्\eoc