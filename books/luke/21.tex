\adhyAya
\stitle{विधवाया दानकथनं।}
\vakya अथ स ऊर्ध्वदृष्टिं कृत्वा धनागार उपहारान् निक्षिपतो धनिनो ददर्श
\vakya विधवाञ्चैकं दरिद्रां ददर्श या क्षोधिष्ठमुद्राद्वयं तत्र न्यक्षेप्सीत्।
\vakya ततः स उवाच, युष्मानहं सत्यं ब्रवीमि, दरिद्रेयं विधवा सर्वेभ्योऽधिकं निक्षिप्तवती, यतः सर्वे प्रयोजनातिरिक्तं किञ्चिदीश्वरोद्देश्योपहारेषु निक्षिप्तवन्तः,
\vakya इयन्तु प्रयोजनीयस्याभावेऽपि सर्वस्वं निक्षिप्तवती, जीविकां कृत्स्नामेव।
\stitle{यिरूशालेमस्य विनाशः यीशोः पुनरुत्थानमधि शिक्षा।}
\vakya अथ धर्मधाम रुचिरैः प्रस्तरैरुत्सृष्टद्रव्यैश्च शोभितमिति कैश्चित् कथिते स व्याजहार,
\vakya अत्र युष्माभि र्यद्यद्दृश्यते तन्मध्ये यदा प्रस्तरोपरि प्रस्तर एकोऽप्यनिपातयितव्यो न विहायिष्यते, तादृशानि दिनान्यायास्यन्ति।
\vakya ते तदा तं पप्रच्छुः, गुरो, बाढं, कदा तत् सम्भविष्यति? किं वाभिज्ञानं तत्सम्भवकालस्य?
\vakya स उवाच, सावधानास्तिष्ठत, मा भ्राम्यत, यतो बहवो मन्नामध्वजिन आगत्य वदिष्यन्त्यहं सः, समयश्च निकटमागत इति। तद्यूयं तान् मानुगच्छत।
\vakya यदा च युद्धानामुपप्सवानाञ्च किंवदन्तीं श्रोष्यथ तदा मा त्रस्यत, यत एतानि प्रथममवश्यम्भावीनि, न त्वापाततः परिणामो भविष्यति।
\vakya स तदा तान् जगाद, जाति र्जाते र्विरुद्धं राज्यञ्च राज्यस्य विरुद्धमुत्थास्यति, स्थाने स्थाने च महाभूकम्पा दुर्भिक्षाणि मार्यश्च भविष्यन्ति, गगनाच्च महान्ति भीषणान्यभिज्ञानानि च सम्भविष्यन्ति।
\vakya सर्वस्मादेतस्मात् प्राक् तु मनुष्या युष्मासु हस्तानर्पयिष्यन्ति समाजगृहेषु कारासु च निधास्यन्त उपद्रोष्यन्ति च युष्मान् राज्ञां शास्तॄणाञ्च समक्षं नेतव्यान् मन्नामकारणात्।
\vakya तत्तु युष्मत्पक्षीयसाक्ष्येण परिणतं भविष्यति।
\vakya ततो यूयं मनःस्विदं निश्चिनुत यत् स्वपक्षवादार्थं युष्माभिः प्राक् न चिन्तयितव्यं।
\vakya यतो युष्माकं सर्वेऽपि विपक्षा यस्य प्रतिकूलवादं प्रतिरोधं वा कर्तुं न शक्ष्यन्ति, तादृशं वक्त्रं ज्ञानञ्च मयैव युष्मभ्यं दायिष्येते।
\vakya यूयन्तु पितृभ्यामपि भ्रातृभि र्ज्ञातिभि र्बन्धुभिश्च समर्पयिष्यध्वे, युष्माकं केचिच्च घानिष्यन्ते।
\vakya यूयञ्च सर्वेषां द्वेषपात्राणि भविष्यथ मन्नामकारणात्।
\vakya न तु विनंक्ष्यति युष्माकमेकोऽपि शिरोरुहः।
\vakya स्वधैर्येण यूयं स्वप्राणान् अर्जयिष्यथ।
\vakya यदा तु यिरूशालेमपुरीं पृतनाभिर्वेष्टितां द्रक्ष्यथ, तदा तस्य उच्छित्ति र्यन्निकटमागता तज्जानीत।
\vakya तदा यिहूदियानिवासिनः पलाय्य पर्वतानाश्रयन्तु मनुष्याश्च पुर्या मध्ये वर्तमाना निर्गच्छन्तु ग्रामेषु वर्तमानाश्च तां मा प्रविशन्तु।
\vakya यतो यद्यल्लिखितमास्ते तत्सर्वस्य सिद्धये तान्यधर्मप्रतीकारस्य दिनानि भविष्यन्ति।
\vakya भविष्यन्ति च दिनेषु तेषु गर्भिण्यः स्तन्यदायिन्यश्च सन्तापभाजनानि। यतो मेदिन्यां महादुर्गति र्भविष्यति कोपश्च जातिमिमां प्रति।
\vakya जना असिग्रस्ताः पतिष्यन्ति बन्दीकृताः सर्वजातिषु (विक्रायिष्यन्ते) च। परजातीनां कालश्च यावदपूर्णः स्थास्यति, तावद् यिरूशालेमं नित्यं परजातीनां चरणै र्मर्दयिष्यते।
\vakya सूर्ये चन्द्रे नक्षत्रेषु चाभिज्ञानानि भविष्यन्ति, भूतलेऽपि जातीनां सङ्कोचो व्याकुलानां समुद्रस्य नादचलनयो र्मनुष्येषु च
\vakya प्राणत्यागिषु भयाद् भूमण्डलमाक्रमिष्यन्तीनां (विपदां) प्रतीक्षणाच्च, यतो गगनस्य बलानि विचलिष्यन्ति।
\vakya तदा च महासामर्थ्यप्रतापाभ्यां सार्धं मेघरथेनागच्छन् मनुष्यपुत्रस्ते र्लक्षयिष्यते।
\vakya एतत्सम्भवस्यारम्भे तु यूयमूर्ध्वीभूय शिरांस्युच्चीकुरुत, यतो युष्माकं मुक्तिरासन्ना।
\vakya पुनः स तेभ्यो दृष्टान्तं कथयामास, उडुम्बराः सर्वे वृक्षाश्च युष्माभिरालोच्यन्तां,
\vakya ते स्पष्टं पल्लवानुत्पादयन्तीति दृष्ट्वा स्वयं जानाथ यूयमासन्नः ग्राष्मकाल इति।
\vakya तथैव यूयमपि यदैतत्सर्वस्य सम्भवं द्रक्ष्यथ ज्ञास्यथ तदा यदीश्वरस्य राज्यं समीपस्थं।
\vakya युष्मानहं सत्यं ब्रवीमि, यावत् सर्वमेतन्न सम्भवति तावन्न व्यत्येष्यन्ति मानवा एतत्कालिकाः।
\vakya द्यावापृथिव्यावत्येष्यतः, मम वाक्यानि तु नैवात्येष्यन्ति।
\vakya यूयन्तु स्वार्थं सावधाना भवत, नोचेत् युष्मच्चित्तानि मद्यभारेण मत्ततया जीविकाजन्यचिन्ताभि र्वा जडीभविष्यन्ति, तथा सति स दिवसः सहसा युष्मानुपस्थास्यति।
\vakya यतः कूटपाशवत् स भूतलस्य सर्वत्रासीनानां सर्वेषामुपरि वर्तिष्यते।
\vakya अतो यूयं जाग्रतः स्थित्वा सर्वकाले तथा प्रार्थयध्वं यथा विचार इदं निश्चीयेत् यत् सर्वाण्येतानि भाविदुःखानि तरितुं मनुष्यपुत्रस्य समक्षं स्थातुञ्च यूयमर्हथ।
\vakya तदा स प्रत्यहं धर्मधामन्युपदिशन् दिवसमयापयत्, जैतुनाख्यगिरिं निर्गत्य च तत्र रात्रिमयापयत्।
\vakya प्रत्यूषे च सर्वजनो धर्मधाम्नि तस्य वाक्यान्याकर्णयितुं तस्य समीपमागच्छत्\eoc