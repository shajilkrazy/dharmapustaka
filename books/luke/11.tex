\adhyAya
\stitle{प्रार्थनामधि यीशोरुपदेशः।}
\vakya अथ कस्मिंश्चित् स्थानेऽवस्थितिकाले स प्रार्थयत, यदा च विरराम तदा तच्छिष्याणामेकस्तमाह, प्रभो, शिक्षयत्वस्मान् प्रार्थयितुं योहनोऽपि यथा स्वशिष्यान् शिक्षितवान्।
\vakya ततः स तानवादीत्, यदा प्रार्थयध्वे तदा वदत, अस्माकं स्वर्गस्थ पितः, तव नाम पवित्रं पूज्यतां। तव राज्यमायातु। यथा स्वर्गे तथा मेदिन्यामपि तवेच्छा सिध्यतु।
\vakya अस्माकं श्वस्तनं भक्ष्यं प्रत्यहमस्मभ्यं देहि।
\vakya अस्माकं पापानि क्षमस्व च, यतो वयमप्यस्मदपराधिनामेकैकस्य क्षमामहे। अस्मांश्च परीक्षां मा नय, अपि तु दुरात्मत उद्धर।
\vakya पुनश्च स तानवादीत्, युष्मासु यस्य बन्धुरास्ते स यद्यर्धरात्रे तदन्तिकं गत्वा तं वदति, मित्र, पूपत्रयं मह्यमृणं देहि,
\vakya यतो मम बन्धुः पथिक एको मद्गृहमागतस्तं परिवेषयितुं मम किमपि नास्ति,
\vakya गृहाभ्यन्तरस्थः स प्रतिभाषमाणस्तर्हि किं वदिष्यति, मां मा क्लिशान, रुद्धं हि सम्प्रति द्वारं, मम बालकाश्च मया सार्धं शयने विद्यन्ते, तुभ्यं दानार्थमुत्थातुं न शक्नोमीति।
\vakya युष्मानहं ब्रवीमि, स मित्रमितिहेतोरुत्थाय तस्मै दातुमसम्मतः सन्नपि स तदीयाग्रहकारणादेवोत्थाय तस्मै यति पूपाः प्रयोज्यास्तति दास्यति।
\vakya अहमपि युष्मान् ब्रवीमि, याचध्वं तेन युष्मभ्यं दायिष्यते, अन्विष्यत तेनासादयिष्यथ।
\vakya यतो यः कश्चिद् याचते स लभते, यश्चान्विष्यति स आसादयति, यश्च द्वारमाहन्ति तदर्थमुद्घाट्यते।
\vakya युष्मासु कः पिता स्वपुत्रेण पूपं याचितस्तस्मै प्रस्तरं दास्यति? मीनं वा याचितः स किं मीनस्य परिवर्तेन तस्मै सर्पं दास्यति?
\vakya अण्डं वा यातितः किं तस्मै वृश्चिकं दास्यति?
\vakya तद् दुर्जना अपि यूयं चेत् स्वसन्तानेभ्यो हितदानानि वितरितुं जानीथ, तर्हि किमधिकं स स्वर्गस्थः पिता स्वयाचकेभ्यः पवित्रामात्मानं वितरिष्यति।
\stitle{भूतानामधि शिक्षा।}
\vakya अथैकदा स भूतं निरसारयत्, स भूतश्च मूकः। निःसृते च भूते स मूकोऽभाषत, ततो जननिवहा आश्चर्यं मेनिरे।
\vakya तेषां केचित् त्ववदन्, निःसारयत्यसौ भूतान् भूतराज्यस्य बेल्सबूबस्य साहाय्येन।
\vakya अपरे च परीक्षमाणास्तं गगनादभिज्ञानस्य प्रदर्शनं ययाचिरे।
\vakya स तु तेषां चिन्ता विज्ञाय तान् जगाद, उत्सीदति सकलं तद् राज्यं यद् भिन्नं स्वविरुद्धं, कुलस्योपरि कुलं पतति च।
\vakya शैतानोऽपि चेत् स्वविरुद्धं भिन्नो जातस्तस्य राज्यं तर्हि कथमवस्थास्यते? यूयं हि वदथ यदहं बेल्सबूबस्य नाम्ना भूतान् निःसारयामि।
\vakya तदहं यदि बेल्सबूबबलेन भूतान् निःसारयामि, युष्माकं पुत्रास्तर्हि केन निःसारयन्ति? अतस्ते युष्माकं विचारयितारो भविष्यन्ति।
\vakya यदि त्वीश्वरस्याङ्गुल्याहं भूतान् निःसारयामि तर्हीश्वरस्य राज्यं नूनं युष्मत्समीपमुपस्थितं।
\vakya स बलिष्ठो यावद् रणसज्जान्वितः स्वप्रासादं रक्षति तावत् तस्य वसूनि निरुपद्रवाणि तिष्ठन्ति।
\vakya तस्माद् बलवत्तरो नरस्तूपागत्य यदा तं पराजेष्यति, तदा तस्य विश्वासभूमिं सर्वाङ्गरक्षिकां रणसज्जामपहरिष्यति तदीयलोप्त्राणि बिभक्षति च।
\vakya यो न मम सहायः स मम विरोधी, यश्च मया सार्धं न सञ्चिनोति स विकिरति।
\vakya अशुचिरात्मा मनुष्यान्निर्याणात् परं निरुदकानि स्थानानि पर्यटन् विश्रामं मृगयते। तन्त्वप्राप्य स वदति, निर्गतोऽहं मामकाद् गेहाद् यस्मात् तत् पुनर्गच्छामि।
\vakya तत्रोपस्थाय तु स तन्मार्जितं शोभितञ्च पश्यति।
\vakya गत्वा च स तदापरान् स्वतो दुष्टतरान् सप्तात्मनः स्वसङ्गिनः करोति, सर्वे ते च तत्र प्रविश्य निवसन्ति। अनेन मनुष्यस्य तस्यान्तिमदशादिदशातो निकृष्टा भवति।
\vakya स यदैतान्यकथयत् तदा जनतामध्यात् कापि योषिदुच्चरवेण तमाह, धन्यः स जठरो यो भवन्तं धारितवान् धन्यञ्च तत् स्तनद्वयं यद् भवता दोहितं।
\vakya स तूवाच, भवतु धन्यास्तु ते यैरीश्वरस्य वाक्यं श्रूयते रक्ष्यते च।
\stitle{ऋजुतामधि शिक्षा।}
\vakya समागच्छत्सु तु जननिवहेषु स वक्तुमारेभे, वंशोऽयं दुष्टः, सोऽभिज्ञानमनुसन्धत्ते, तस्मै तु भाववादिनो योनाहस्याभिज्ञानादन्यदभिज्ञानं न दायिष्यते।
\vakya यतो योनाहो यथा नीनवीयानां निमित्तमभिज्ञानमभूत्, तथा मनुष्यपुत्रोऽप्यस्य वंशस्य निमित्तमभिज्ञानं भविष्यति।
\vakya दक्षिणदिशो राज्ञी विचारेऽस्य वंशस्य नरेः सार्धमुत्थापयिष्यते तान् दोषीकरिष्यति च, यतः सा शलोमनो विज्ञानोक्तीः श्रोतुं पृथिव्याः प्रान्तेभ्य आगतवती, पश्य त्वत्र शलोमनो महत्तरेण केनाप्युपस्थितं।
\vakya नीनवीयनरा विचारे वंशेनैतेन सार्धमुत्थास्यन्ति तं दोषीकरिष्यन्ति च, यतो योनाहस्य घोषणे तै र्मनांसि परावर्तितानि, पश्य त्वत्र योनाहान्महत्तरेण केनाप्युपस्थितं।
\vakya मनुष्यो दीपिकां प्रज्वाल्य न गृहाधःस्थायां गुप्त्यां द्रोणस्यधस्ताद् वा अपि तु दीपाधारस्योपरि स्थापयति, प्रवेशिनो यथा दीप्तिं पश्येयुः।
\vakya देहस्य दीपिका चक्षुः। अतस्तव चक्षुषि सरले सति तव कृत्स्नो देहो दीप्तिमयोऽस्ति, तस्मिंस्तु दुष्टे जाते तव देहोऽपि तिमिरमयो भवति।
\vakya ततो हेतोस्तथावलोकय तवान्तर्ज्योति र्यथा न तिमिरमयं भवेत्।
\vakya तद् यदि तव कृत्स्नो देहो दीप्तिमयो भवेत्, तस्य कोऽप्यंशस्तिमिरमयो न भवेत्, तर्हि तत्सर्वाङ्गं दीप्तिमयं भविष्यति, स्वप्रभया त्वां द्योतयन्ती दीपिकेव।
\stitle{आन्तरिकशुचितावश्यकत्वविषयकशिक्षा।}
\vakya स यदाभाषत तदा फरीशी कश्चित् तं निमन्त्रयामास यथा स पूर्वाह्निकाहारार्थं तद्गृहमागच्छेत्। ततः स प्रविश्य भोजनार्थमुपविवेश।
@V तद् दृष्ट्वा स फरीशी विस्मयं जगाम यतो भोजनात् प्राक् स नास्नाप्यत।
\vakya प्रभुस्तु तमवादीत्, इदानीं फरीशिनो यूयं पानपात्रस्य स्थलस्य च बहिर्देशं शुचिकुरुथ, युष्मदन्तर्देशस्तु परस्वापहारेण दुष्टतया च परिपूर्णोऽस्ति।
\vakya भो निर्बोधाः, बहिर्देशो येन रचितस्तेन किं नान्तर्देशोऽपि रचितः?
\vakya अपि तु यद्यदन्तःस्थं तद् भिक्षां दत्त, पश्यत च युष्मदर्थं सर्वमेव शुचीभूतं।
\vakya फरीशिनो यूयं सन्तापभाजनानि, यतो यूयं पोदिनायाः पीगनाख्यतृणस्य सर्वशाकानाञ्च दशमांशानुपहरथ, परिहरथ तु विचारमीश्वरस्य प्रेम च। द्वयमेतदासीद् युष्माभिरनुष्ठातव्यम् अमी च न त्यक्तव्याः।
\vakya परीशिनो यूयं सन्तापभाजनानि, यतः समाजगृहेषु श्रेष्ठासनानि हट्टेषु चाभिवन्दनान्याकाङ्क्षथ।
\vakya अरे कपटिनः शास्त्राध्यापकाः फरीशिनश्च, यूयं सन्तापभाजनानि, यतो यूयं प्रच्छन्नैः शवागारैः सदृशा येषामुपरि मनुष्या अज्ञात्वा विहरन्ति।
\vakya तदा व्यवस्थावेत्तॄणां कश्चित् प्रतिभाषमाणस्तमवादीत्, गुरो, वचांसीमानि व्याहरंस्त्वमस्मानपि न्यक्करोषि।
\vakya स तूवाच, रे व्यवस्थावेत्तारः यूयमपि सन्तापभाजनानि, यतो यूयं दुर्वहैर्भारै र्मनुष्यान् भारिणः कुरुथ, स्वयं त्वेकाङ्गुल्यापि तान् भारान् न स्पृशथ।
\vakya यूयं सन्तापभाजनानि, यतो यूयं भाववादिनां शवागाराणि निर्मिमीध्वे, युष्माकं पूर्वपुरुषास्तु तान् हतवन्तः।
\vakya सुतरां युष्मत्पूर्वपुरुषाणां कर्माणि सप्रमाणानि कुरुथ तेष्वनुमोदध्वे च। ते हि तान् हतवन्तो यूयञ्च तेषां शवागाराणि निर्मिमीध्वे।
\vakya अतो हेतोरीश्वरस्य प्रज्ञापि व्याजहार, तेषामन्तिकमहं भाववादिनः प्रेरितांश्च प्रहेष्यामि, तेषां मध्यात् केचित् तै र्घानिष्यन्ते प्रद्रावयिष्यन्ते च,
\vakya इत्थमाजगत्संस्थापनाद् विस्रावितं सर्वेषां भाववादिनां यच्छोणितं,
\vakya हेबलस्य शोणितमारभ्य निकेतनवेद्योरन्तराले नष्टस्य सखरियस्य शोणितं यावत् तत्सर्वस्य शोधो जनेभ्योऽधुनातनेभ्य आहारयिष्यते। युष्मानहं निश्चितं ब्रवीमि, जनेभ्योऽधुनातनेभ्यस्तच्छोध आहारयिष्यते।
\vakya रे व्यवस्थावेत्तारः यूयं सन्तापभाजनानि, यतो यूयं ज्ञानस्य कुञ्चिकामपहृतवन्तः। यूयमपि न प्रविष्टाः प्रविशन्तोऽपि युष्माभि र्निवारिताः।
\vakya तेनैतान्युच्यमानाः शास्त्राध्यापकाः फरीशिनश्च गाढं व्यग्रीभूय तद्विरुद्धं मन्त्रयमाणा अभियोगसूत्रलिप्सया तन्मुखनिःसृतं किमपि धर्तुं यतमानाश्च तं बहुप्रश्नानामाकस्मिकं प्रत्युत्तरं याचितुमारेभिरे\eoc