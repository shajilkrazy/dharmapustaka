\adhyAya
\vspace{25pt}
\vakya द्वितीय आद्यविश्रामवारे तु शस्यक्षेत्रेण तस्मिन् व्रजति तस्य शिष्या मञ्चरी र्भंक्त्वा पाणिभि र्मृद्नन्तोऽभुञ्चत।
\vakya फरीशिनः केचित् तदा तानवदन्, विश्रामदिने यन्न विधेयं तत् कर्म कथं कुरुथ?
\vakya ततो यीशुः प्रतिभाषमाणस्तान् जगाद, न किं पठितं भो दायूदस्तस्य सङ्गिनश्च यदाक्षुध्यंस्तदा स किमकार्षीत्?
\vakya स हीश्वरस्य गेहं प्रविश्य तानेव दर्शनीयपूपान् आदायाभुङ्क्त स्वसङ्गिभ्यश्च प्रादात्, यान् भोक्तुं याजकैः केवलैः र्विधेयम्।
\vakya पुनः स तान् जगाद, प्रभु र्हि मनुष्यपुत्रो विश्रामवारस्यापि।
\vakya अन्यस्मिन्नपि विश्रामवारे स तेषां समाजं प्रविश्य यदोपादिशत् तदासीत् तत्र नर एको यस्य दक्षिणो हस्तः शुष्कः।
\vakya स तु विश्रामवारे रोगप्रतिकारं करोति न वेति जिज्ञासवः शास्त्राध्यापकाः फरीशिनश्च तदभियोगस्य सूत्रं गवेषयन्तस्तमवैक्षन्त।
\vakya स तु तेषां तर्कान् ज्ञात्वा तं शुष्कहस्तं मनुष्यं जगाद, उत्थाय मध्यस्थाने तिष्ठ। ततः स उत्थायातिष्ठत्।
\vakya यीशुस्तदा तान् जगाद, युष्मानहं पृच्छामि, विश्रामवारे किं विधेयं, हिताचरणमथवा हिंसाचरणम्। प्राणरक्षाथवा प्राणनाशः?
\vakya अनन्तरं स परितस्तान् सर्वान् निरीक्ष्य तं नरं जगाद, तव हस्तं प्रसारय। ततस्तेन तथा कृते स करोऽन्यतर इव स्वस्थो जातः।
\vakya ते तु तमःपूर्णा बभूवु र्यीशुं प्रति किं कर्तव्यं तदधि समभाषन्त च।
\stitle{प्रेरितानां नियोगः।}
\vakya तेषु दिनेषु यीशुरेकदा प्रार्थयितुं पर्वतं गत्वेश्वरमुद्दिश्य प्रार्थयमानः कृत्स्नां रात्रिमयापयत्।
\vakya दिवसे तूपस्थिते स स्वशिष्यानाहूय तेषां मध्ये पश्चान्निर्दिष्टान् द्वादश नरान् वरयामास तेषाञ्च प्रेरितवर्ग इत्यभिधानं चकार।
\vakya ते शिमोनो यस्तेन पित्र इत्यभिहितस्तस्य भ्रातान्द्रियशच याकोबो योहनश्च फिलिपो बर्थलमयश्च
\vakya मथिस्थोमाश्च, आल्फेयसुतो याकोब उद्योगीत्यभिधः शिमोनश्च,
\vakya याकोबस्य (भ्राता) यिहूदा स चेष्करियोतीयो यिहूदा यः पश्चात् समर्पयिता बभूव।
\stitle{यीशोरुपदेशः।}
\vakya ततः परं स तैः सार्धमवरुह्य कस्मिंश्चित् समस्थले तस्थौ। उपतस्थिरे च तत्र तच्छिष्यवृन्दं कृत्स्नयिहूदियातो यिरूशालेमात् सोरसीदोनयोः समुद्रतीरस्थप्रदेशाच्चागतो महान् जननिवहश्च।
\vakya इमे तस्य वाक्यं श्रोतुं रोगमुक्तिं प्राप्तुञ्च तस्यान्तिकमागतवन्तः, भूताविष्टाश्चाशुच्यात्मभ्यो मुक्तिम् अलभन्त,
\vakya कृत्स्नो जननिवहश्च तं स्प्रष्टुमयतत, यतः प्रभावस्तस्मान्निर्गच्छन् सर्वेषामारोग्यमसाधयत्।
\vakya अनन्तरं स स्वशिष्यानुद्दिश्योर्ध्वदृष्टि र्भूत्वा व्याजहार, धन्या दीना यूयं, यत ईश्वरराज्यं युष्माकमेव।
\vakya धन्या अधुना क्षुधार्ता यूयं यतः परितर्प्स्यथ। धन्या अधुना रुदन्तो यूयं, यतो हसिष्यथ।
\vakya धन्या यूयं यदा जना मनुष्यपुत्रस्य हेतो र्युष्मभ्यम् द्विषन्ति, यदा युष्मान् पृथक् कुर्वन्ति न्यक्कुर्वन्ति अधममिव युष्मन्नाम निराकुर्वन्ति च।
\vakya तस्मिन् दिन आनन्दत नृत्यत च। यतः पश्यत सञ्चितं स्वर्गे युष्माकं प्रचुरं पारितोषिकं। वास्तवं तेषां पूर्वपुरुषा भाववादिनः प्रति तथैवाचरितवन्तः।
\vakya प्रत्युत धनिनो यूयं सन्तापपात्राणि, यतो यूयं स्वसान्त्वनां लब्धवन्तः।
\vakya परितृप्ता यूयं सन्तापपात्राणि यतः क्षोत्स्यथ।
\vakya अधुना हसन्तो यूयं सन्तापपात्राणि, यतः शोचिष्यथ रोदिष्यथ च। यूयं सन्तापपात्राणि यदा मनुष्याः सर्वे युष्माकं सुख्यातिं कुर्वन्ति। वास्तवं तेषां पूर्वपुरुषाः कूटभाववादिनः प्रति तथैवाचरितवन्तः।
\vakya परन्तु भो श्रोतारः, युष्मानहं ब्रवीमि, स्वशत्रून् प्रति प्रेम कुरुत, ये युष्मान् द्विषन्ति तेषां हितं कुरुथ,
\vakya ये युष्मान् शपन्ति तेभ्य आशिषं दत्त, ये युष्मान् अपवदन्ति तेषां हितं प्रार्थयध्वं।
\vakya यस्तवैकतरस्मिन् कपोले कराघातं करोति तं प्रत्यन्यतरं कपोलमपि व्याघोटय। यश्च तव प्रावारं हरति, तवाङ्गाच्छादकमपि हर्तुं तं मा वारय।
\vakya यश्च तव द्रव्याणि हरति तं तत्प्रतिदानं मा याचस्व।
\vakya अपि युष्मान् प्रति मनुष्याणां यादृशमाचारं यूयमभिवाञ्छथ, यूयमपि तान् प्रति तादृशमाचरत।
\vakya यदि तु युष्मत्‌प्रेमकारिणः प्रति प्रेम कुरुथ, कस्तर्हि युष्माकं साधुवादः? यतः पापिनोऽपि स्वप्रेमकारिणः प्रति प्रेम कुर्वन्ति।
\vakya यदि च स्वहितकारिणः प्रति हिताचारं कुरुथ, कस्तर्हि युष्माकं साधुवादः? यतः पापिनोऽपि तदेव कुर्वन्ति।
\vakya येभ्यः प्रत्यादानस्याशा युष्माकमस्ति तेषु चेदृणं समर्पयथ, कस्तर्हि युष्माकं साधुवादः? यतः समानप्रत्यादानार्थं पापिनोऽपि पापिष्वृणं समर्पयन्ति।
\vakya प्रत्युत यूयं स्वशत्रून् प्रति प्रेम कुरुत प्रत्यादानस्य प्रत्याशाभावेऽपि हिताचारमृणदानञ्च कुरुत। तथा कृते युष्माकं प्रचुरं पारितोषिकं भविष्यति यूयञ्च परात्परस्य पुत्रा भविष्यथ।
\vakya स ह्यकृतज्ञान् दुर्जनांश्च प्रति सुशीलः। अतो युष्माकं पिता यथा कृपावान्, यूयमपि तथैव कृपावन्तो भवत।
\vakya विचारं वा मा कुरुत, तेन युष्माकं विचारो न कारिष्यते। परं मा दोषीकुरुत, तेन यूयं न दोषीकारिष्यध्वे। क्षमध्वं तेन क्षमां लप्स्यध्वे।
\vakya दत्त, तेन युष्मभ्यं दायिष्यते, उत्तमं निपीडितं सञ्चालितम् अतिपूर्णं परिमाणपात्रं युष्मत्क्रोडे दायिष्यते। येन परिमाणेन यूयं मिमीध्वे तेनैव युष्मदर्थं पुन र्मायिष्यते।
\vakya अश स तेभ्य उपमां कथयामास, अन्धः किमन्धं नेतुं शक्नोति? किं न तावुभौ गर्त्ते पतिष्यतः?
\vakya नास्ति शिष्यो गुरुतः श्रेष्ठः।
\vakya परिपक्वस्तु यः कश्चित् स स्वगुरुणा तुल्यः।
\vakya त्वञ्च कुतः स्वभ्रातुश्चक्षुःस्थं शूककणं निरीक्षसे, तव चक्षुःस्थन्तु गेहकाष्ठं नावधारयसि? तव चक्षुःस्थं गेहकाष्ठमदृष्ट्वा त्वं कथं वा स्वभ्रातरं गदितुं शक्नोषि, भ्रातस्तव चक्षुःस्थशूककणस्य मयोद्धारणमनुमन्यस्वेति। कपटिन्, प्रथमं स्वचक्षुषस्तद् गेहकाष्ठमुद्धर, ततः स्वभ्रातुश्चक्षुषः शूककणस्योद्धरणार्थं स्पष्टं द्रक्ष्यसि।
\vakya नास्ति हि सुवृक्षः कुफलोत्पादकः नास्ति वा कुवृक्षः सुफलोत्पादकः, यत एकैको वृक्षः स्वफलेनाभिज्ञायते।
\vakya नैव कण्टकेभ्य उडुम्बरफलानि सञ्चीयन्ते नैव स्तम्बाद् वा द्राक्षाफलानि सङ्गृह्यन्ते।
\vakya सुजनो सुधनपूर्णाद्धृदयात् सुद्रव्यं निःसारयति, कुजनश्च कुधनपूर्णद्धृदयात् कुद्रव्यं निःसारयति, यतो हृदयस्यातिपूरणमेव वक्त्रं व्याहति।
\vakya किमर्थं वा यूयं मां प्रभो प्रभो इत्यभिभाषध्वे न तु मद्वाक्यानुरूपमाचरणं कुरुथ?
\vakya यः कश्चिन्मदन्तिकमागत्य मम वाक्यानि शृणोति समाचरति च स केन सदृशस्तदहं युष्मान् ज्ञापयिष्मामि।
\vakya स गृहनिर्माणे नियुक्तेन नरेण सदृशो यो गभीरं खनित्वा पाषाणे गृहमूलं स्थापयामास। ततः परं वन्यायामुपस्थितायां स्रोतस्तद् गृहमाजघान विचालयितुन्तु नाशक्नोत्, यतस्तत् पाषाणोपरि संस्थापितं।
\vakya यस्तु श्रुत्वा न समाचरति स तेन नरेण सदृशो येन भित्तिमूलं विना गृहं मृत्तिकोपरि निर्मितं। स्रोतसाहन्यमानं तत् सहसा पपात, तस्य गृहस्य भङ्गो घोरतरो बभूव\eoc