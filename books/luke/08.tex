\adhyAya
\vspace{25pt}
\vakya तदनन्तरं स देशं परिभ्राम्यन् नगरेषु ग्रामेषु चेश्वरस्य राज्यमघोषयत् तत्सुसंवादञ्च जनानज्ञापयत्।
\vakya आसंस्तु तेन सार्धं पूर्वोक्ता द्वादश नरा दुष्टात्मभ्यो व्याधिभ्यश्च तेन मुक्ताः काश्चिद् योषितोऽर्थतो यस्याः सप्त भूता निःसृताः सा मग्दलीन्यभिधा मरियम्
\vakya हेरोदस्य विषयाध्यक्षस्य कूशाः पत्नी योहाना शोशन्ना बह्वश्चान्याः, इमाः स्ववसुभिस्तं पर्यचरन्।
\stitle{बीजवापकस्य दृष्टान्तः।}
\vakya अथ महति जननिवहे समागच्छति मनुष्येषु च नगरेभ्यस्तस्यान्तिकं व्रजत्सु स दृष्टान्तकथयेदं व्याजहार,
\vakya वप्ता स्वबीजानि वप्तुं निर्जगाम, वपनकाले तु पथपार्श्वे कतिपयानि पतितानि, ततस्तानि पादतलै र्दलितानि खेचरपक्षिभिः खादितानि च।
\vakya पतितानि चापराणि पाषाणे, तानि प्ररुह्य रसाभावाच्छोषितानि।
\vakya पतितान्यपराणि तु कण्टकानां मध्ये तान्येकत्र वर्धमानैः कण्टकैः सम्पीडितानि।
\vakya अपराणि तुत्तममृत्तिकायां पतित्वा प्ररुह्य शतगुणं फलं फेलुः। इदमुक्त्वा स उच्चैः र्व्याजहार, शृणोतु यस्य श्रोतुं श्रोत्रे स्तः।
\vakya ततः परं तस्य शिष्यास्तं पप्रच्छुः, अस्या दृष्टान्तकथायास्तात्पर्यं किं स्यात्?
\vakya स तान् जगाद, युष्मभ्यम् ईश्वरराज्यस्य निगूढानां विषयाणां ज्ञानमदायि, अपरेभ्यस्तु दृष्टान्तच्छलेन (शिक्षा दीयते), ते यथा दृष्ट्वापि न पश्येयुः श्रुत्वापि च न बुध्येरन्।
\vakya सा दृष्टान्तकथा त्वेवम्भूता। बीजमीश्वरस्य वाक्यं।
\vakya पथपार्श्वस्थास्ते ये शृण्वन्ति ततः परं यथा ते विश्वस्य परित्राणं न प्रप्नुयुस्तदर्थं दियावल आगत्य तेषां हृदयतो वाक्यमपहरति।
\vakya पाषाणोपरिस्थाश्च ते ये श्रवणकाले सहसा सानन्दं वाक्यं गृह्णन्ति, ते तु मूलहीनाः स्वल्पकालं विश्वसन्ति। परीक्षाकाले चापक्राम्यन्ति।
\vakya कण्टकेषु पतितानि बीजानि च ते ये श्रुत्वा चिन्ताभि र्वसुभि र्जीवनसम्बन्धीयसुखभोगेन च भाराक्रान्ताश्चरन्तो निपीड्यन्ते पक्वफलहीनास्तिष्ठन्ति च।
\vakya उत्तममृत्तिकायां पतितानि बीजानि च ते ये भद्रेण साधुना च हृदयेन वाक्यं श्रुत्वा रक्षन्ति स्थैर्येण फलं फलन्ति च।
\vakya दीपिकां प्रज्वाल्य मनुष्यस्तां न पात्रेणाच्छादयति नापि खट्वाया अधो निदधाति, अपि तु दीपाधारस्योपरि स्थापयति यथा प्रविशन्तः सर्वे दीप्तिं पश्येयुः।
\vakya यतो नास्ति किमपि तादृशं तिरोहितं यन्नाविर्हितं भविष्यति, नापि तादृशं निगूढं यन्न ज्ञायिष्यते प्रकाशताञ्च यास्यति।
\vakya अतो युष्माभिः कथं श्रूयते तदालोच्यतां। यतो यस्यास्ति तस्मै दायिष्यते, यस्य तु नास्ति स यत् स्वं मन्यते तदपि तस्मादपहारिष्यते।
\vakya अथ तस्य माता भ्रातरश्च तत्समीपमागमन्, जनताहेतोस्तु तेन सह मिलितुं नाशक्नुवन्।
\vakya ततः केचित् तस्मै निवेदयामासुः, भवतो माता भ्रातरश्च भवन्तं द्रष्टुमिच्छतो बहिस्तिष्ठन्ति।
\vakya स तु तान् प्रतिजगाद, मम माता भ्रातरश्च ते य ईश्वरस्य वाक्यं शृण्वन्ति समाचरन्ति च।
\stitle{यीशोर्झञ्झानिवारणम्।}
\vakya कस्मिंश्चिद् दिने स स्वशिष्यैः सह नावमारुह्य तान् जगाद, ह्रदस्य पारं गम्यतां। ततस्ते नावं मुक्त्वा प्रस्थिताः।
\vakya गमनकाले तु सन्यद्रात्। तदा झञ्झा ह्रदमाचक्राम नावि तोयेन पूर्यमाणायां तेषां प्राणसंशयो बभूव च।
\vakya ततस्ते समीपं गत्वा तं जागरयित्वा जगदुः नाथ नश्यामो वयं। तदा स उत्थाय वायुं तोयभङ्गञ्च तर्जयामास तौ च निवृत्तौ शान्तिश्च सञ्जाता।
\vakya तांस्तु सोऽब्रवीत्, युष्माकं विश्वासः क्व? अनेन ते भीता विस्तिताश्च परस्परमाहुः, को न्वसौ यत् स वायून् अपश्चाप्याज्ञापयति तैश्च तस्यादेशो गृह्यते?
\stitle{यीशुना भूतग्रस्तस्यारोग्यकरणम्।}
\vakya ततः परं नावा वहमानास्ते गालीलस्य सम्मुखस्थं गादारीयणां देशं प्रापुः।
\vakya तत्र स्थलमवतीर्णे तस्मिन् नगरस्य कश्चिन्मनुष्यस्तस्य सम्मुखमुपतस्थे, य आबहुवत्सरेभ्यो भूतैराविष्टः। स न वस्त्रं पर्यधत्त, नापि गृहेऽवसत्, अपि तु शवागारेषु।
\vakya यीशुं दृष्ट्वैव स प्रणिपत्य क्रुष्ट्वा प्रोच्चै र्व्याजहार, भो परात्परस्येश्वरस्य पुत्र यीशो भवता सह मम कः सम्बन्धः? नार्हति भवान् मां यातयितुं।
\vakya यतः स तमशुचिमात्मानं तस्मान्मनुष्यान्निर्गन्तुमादिशत्। यत आदीर्घकालात् स तेनाक्रान्तः, यदा च शृङ्खलैर्निगडैश्च बद्धोऽरक्ष्यत तदा बन्धनानि विदारयन् भूतेन निर्जनानि स्थानान्यपानीयत।
\vakya यीशुस्तं पप्रच्छ, तव किं नाम? स व्याजहार वाहिनीति, यतः स बहुभूतैराविष्टः।
\vakya ते सानुनयं तमूचुः, अस्मान् अगाधस्थानं गन्तुं मैवाज्ञापयतु। तत्र तु बहुसङ्ख्यकशुकरव्रजो गिरावचरत्। अतस्ते यीशुं ययाचिरे, अमून् शूकरान् प्रवेष्टुमस्माननुजानातु।
\vakya ततः स तान् अनुजज्ञौ।
\vakya ते भूतास्तदा तस्मान्मनुष्यान्निःसृत्य तान् शूकरान् प्रविविशुस्ततः कृत्स्नो व्रजो धावन् पतित्वा शैलाग्रतो ह्रदे पञ्चत्वं जगाम।
\vakya ये च तमचारयंस्ते तदितिवृत्तं दृष्ट्वा पलाय्य नगरे ग्रामेषु च तज्ज्ञापयामासुः।
\vakya यद् वृत्तं जनास्तदा तद् द्रष्टुं बहिराजग्मुः, यीशोः समीपमुपस्थाय च ददृशुः, भूतास्ते यस्मान्निर्गताः स मनुष्यो वस्त्रान्वितः सुबुद्धिश्च यीशोश्चरणयोरुपविष्टोऽस्तीति, ततस्ते बिभ्युः।
\vakya भूताविष्टः स नरश्च कथं तारितस्तत् साक्षिभिस्तेभ्यो निवेदयाञ्चक्रे।
\vakya ततो गादारीयप्रदेशानिवासिनां कृत्स्नो जनौघो महात्रासापन्नत्वात् तमाह, भवानस्मत्सन्निधितोऽपगन्तुमर्हति।
\vakya ततः स नावमारुह्य प्रत्याववृते। ते भूतास्तु यस्मान्निःसृता स मनुष्यस्तं प्रार्थयाञ्चक्रे, भवत्समीपमवस्थातुं मामनुमन्यतां। यीशुस्तु तं विसृज्याब्रवीत्, त्वं स्वगृहं प्रत्यावर्तस्व, त्वदर्थमीश्वरो यद्यत् कृतवांस्तत्कथां कथय च।
\vakya स तदा प्रस्थाय तदर्थं यीशु र्यद्यत् कृतवान्, नगरस्य सर्वत्र तत्कथामघोषयत्।
\stitle{यीशुना प्रदररोगिण्या आरोग्यकरणम् मृतकन्यायै जीवनदानञ्च।}
\vakya प्रत्यावृत्ते तु यीशौ जननिवहस्तमनुजग्राह, यतः सर्वे तं प्रत्यैक्षन्त।
\vakya पश्य च यायीर इत्यभिध एको नर उपतस्थे, स समजाध्याक्ष आसीत्। स यीशोश्चरणयोः प्रणिपत्य स्वगृहप्रवेशार्थं तं प्रसादयामास,
\vakya यतः प्रायेण द्वादशवर्षीया तस्येकजाता या दुहिता सा मरणोद्यतासीत्।
\vakya गमनकाले तु जनता तमपीडयत्। अनन्तरं द्वादशवर्षान् यावद् रक्तस्रावातुरा काचिन्नारी या चिकित्सकैः सेवनार्थं सर्वस्वं व्ययितवती, तथापि कस्मादप्यारोग्यं लब्धुं नाशक्नोत्, सा पश्चाद्दिश्युपागत्य तदीयवस्त्रस्य प्रलम्बकं पस्पर्श,
\vakya तत्क्षणाञ्च तस्या रक्तस्रावो विरराम।
\vakya यीशुस्तदा पप्रच्छ, को मां स्पृष्टवान्? अनङ्गीकुर्वत्सु तु सर्वेषु पित्रस्तत्सङ्गिश्चाहुः, नाथ, भवान् जनतया परिवार्यते पीड्यते च, पुनः पृच्छति, को मां स्पृष्टवानिति।
\vakya यीशुरुवाच, कश्चिन्मां पस्पर्श, यतः प्रभावो मत्तो निःसृत इत्यज्ञायि मया।
\vakya ततः सा योषित् प्रच्छन्ना स्थातुं न शक्नोतीति बुद्ध्वा वेपमानोपाजगाम, तस्यान्तिकं प्रणिपत्य च तत्स्पर्शनस्य कारणं सद्य आरोग्यलाभस्य कथाञ्च सर्वजनानां समक्षं निवेदयामास।
\vakya स तदा तामब्रवीत्, आश्वसिहि, वत्से, तव विश्वासस्त्वां तारयामास, क्षेमेण याहि।
\vakya इदं भाषमाणे तस्मिन् समाजाध्यक्षस्य गृहात् कश्चिदागत्य तमाह, भवतो दुहितो दुहिता ममार, गुरुं मा क्लिश्नातु।
\vakya यीशुस्तच्छ्रुत्वा प्रतिभाषमाणस्तं जगाद, मा भैषीः, केवलं विश्वसिहि, तेन सा तरिष्यते।
\vakya ततः परं स तद् भवनमागत्य नान्यं कमपि प्रवेष्टुमनुजज्ञौ, केवलं पित्रं याकोबं योहनञ्च बालाया मातापितरौ च।
\vakya जनाः सर्वे त्वरुदंस्तस्याः कृते वक्षांस्यताडयंश्च। स तानाह, मा रुदित, न मृता सापि तु निद्राणा।
\vakya सा तु मृतवतीति ज्ञात्वा ते तमुपाहसन्।
\vakya स तदा तान् सर्वान् बहिष्कृत्य तस्या हस्तं धृत्वा प्रोच्चै र्व्याजहार, बालिके उत्तिष्ठेति।
\vakya तस्याः प्राणास्तदा प्रत्याजग्मुः, सा च तत्क्षणमुत्तस्थौ।
\vakya ततः परं यीशुस्तस्यै भक्ष्यदानमादिदेश, तस्या जनकौ च विस्मयापन्नौ। स तु तावाज्ञापयामास, इदं यद् वृत्तं युवां तत् कमपि ज्ञापयितुं नार्हथः\eoc