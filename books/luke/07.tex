\adhyAya
\stitle{यीशुना पीडितस्य आरोग्यकरणं मृताय जीवनदानञ्च।}
\vakya जनानां कर्णगोचरे सर्वाणि तानि वाक्यानि समाप्य स कफरनाहूमं प्रविवेश।
\vakya कस्यचिच्छतपते र्व्याधिग्रस्तो दासस्तदा मृतकल्प आसीत्, स स्वामिनो बहुमतः।
\vakya अतः स यीशोः संवादं श्रुत्वा यिहूदीयानां प्राचीनान् तस्यान्तिकं प्रहित्य तं याचयामास स यथागत्य तस्य दसं तारयेत्।
\vakya ते च यीशोः समीपमुस्थाय सयत्नं प्रसादयन्तस्तमूचुः, भवान् यदित्थं तमुपकुर्यात् स तस्य योग्यपात्रं,
\vakya यतोऽस्मज्जातिं प्रति स प्रेम करोति, स एव चास्माकं समाजगृहं निर्मापितवान्।
\vakya अतो यीशुस्तैः सार्धमव्रजत्। तद्गृहादनतिदूरन्तूपस्थिते तस्मिन् स शतपतिः कतिपयान् बन्धून् तदन्तिकं प्रहित्य तमब्रवीत्, प्रभो, मा श्राम्यतु नाहमर्हामि यद् भवान् मम वेश्म प्रविशेत्।
\vakya ततोहेतोरेवाहं स्वयं भवत्समीपं गन्तुमात्मानमयोग्यममन्ये। प्रत्युत भवान् वाक्यं व्याहरतु तेनैव मम किङ्करो निरामयो भविष्यति।
\vakya यतोऽहमपि क्षमताया वशीभूतो मनुष्यः, ममाधीनाश्च सैनीकाः सन्ति। तेषामेको गच्छेति मया कथिते गच्छति, अपरश्चायाहीति कथिते समायाति, मम दासश्चेदं कुर्विति कथिते तत् करोति।
\vakya इदं श्रुत्वा यीशुस्तस्मिन्नाश्चर्यं मन्यमानो मुखं प्रत्यावर्त्यानुगामिनं जननिवहं जगाद, युष्मानहं ब्रवीमि इस्रायेलेऽपि मयेदृग्‌दृढो विश्वासो न लक्षितः।
\vakya तदनन्तरं ते प्रहिता जनास्तद् गृहं परावृत्य तं रोगिणं दासं स्वस्थं ददृशुः।
\vakya परदिवसे स नायिनाख्यमेकं नगरमगच्छत् तेन सार्धञ्च तस्य बहवः शिष्या महान् जननिवहश्चाव्रजन्।
\vakya पश्य च नगरद्वारसमीपमुपस्थिते तस्मिन् कश्चिन्मृतः समाध्यर्थं बहिरुह्यते, स निजमातुरेकजातः सापि विधवा। नगरस्य महान् जननिवहश्च तया सार्धमासीत्।
\vakya तां दृष्ट्वा प्रभुरनुकम्प्य जगाद, मा रोदीः।
\vakya ततः परं स समीपमागत्य शवयानं पस्पर्श। वाहकेषु तदा तिष्ठत्सु स जगाद, भो युवन्नुत्तिष्ठेति त्वामादिशामि।
\vakya अनेन स मृत ऊर्ध्वो भूत्वोपविवेश भाषितुमारेभे च।
\vakya ततः परं स तं तन्मातरि समर्पयामास। सर्वे च भयाक्रान्ता ईश्वरं तुष्टुवुरूचुश्च, उत्पन्नोऽस्मन्मध्ये महान् भाववादीतीश्वरः स्वप्रजा अवेक्षितवानिति च।
\vakya तमधि सा किंवदन्ती च कृत्स्नां यिहूदियां परितःस्थं कृत्स्नं जनपदञ्च व्यानशे।
\stitle{योहनस्य प्रश्नः यीशोरुत्तरञ्च।}
\vakya योहनस्य शिष्यास्तस्मै तदैताः सर्वकथा निवेदयामासुः।
\vakya ततो योहनः स्वशिष्याणां द्वौ समीपमाहूय यीशुं प्रति प्रहित्य पप्रच्छ, येनागन्तव्यं स किं भवान्, वान्यः कश्चिदस्माभिः प्रतीक्षितव्यः?
\vakya अतस्तौ नरौ तस्यान्तिकमागत्य जगदतुः, स्नापको योहन आवां भवत्समीपं प्रहित्य पृच्छति, येनागन्तव्यं स किं भवान्, वान्यः कश्चिदस्माभिः प्रतीक्षितव्य इति।
\vakya तस्मिन् दण्डे तु स बहून् व्याधितो यातनातो दुष्टात्मभ्यश्च मोचितवान् बहुभ्योऽन्धेभ्यो दृक्‌शक्तिं दत्तवांश्च।
\vakya अतो यीशुः प्रतिभाषमाणस्तौ जगाद, युवाभ्यां यद्यद्दृष्टं श्रुतञ्च गत्वा तद् योहनाय निवेद्यतां। अन्धा दृष्टिं लभन्ते, खञ्जाः परिचरन्ति, कुष्ठिनः शुचीभूयन्ते, बधिराः शृण्वन्ति, मृता उत्थाप्यन्ते, दरिद्राश्च सुसंवादं ज्ञाप्यन्ते,
\vakya यश्च मयि न स्खलति स धन्यः।
\vakya योहनस्य दूतयोरपगतयो र्यीशु र्जननिवहेभ्यो योहनमधीदं कथयितुमारेभे, किं निरीक्षितुं यूयं मरुं निर्गतवन्तः? किं वायुना चाल्यमानं नलं?
\vakya किमथवा द्रष्टुं यूयं निर्गतवन्तः? किं सूक्ष्मवेषपरिहितं मनुष्यं? पश्यत ये शुभ्रं परिच्छदं सुखभोगञ्च सेवन्ते ते राजभवनेषु विद्यन्ते।
\vakya किमधवा द्रष्टुं यूयं निर्गतवन्तः? किं भाववादिनं? तथैव, युष्मांस्त्वहं ब्रवीमि, भाववादिनोऽपि श्रेष्ठतरं नरम्।
\vakya असौ स यमधीदं लिखितमास्ते,
\begin{poem}
\startwithline “पश्य त्वद्वदनस्याग्रे स्वदूतं प्रहिणोम्यहं।
\pline गन्तव्यं तव मार्गं स त्वदग्रे संस्करिष्यति॥”
\end{poem}
\vakya यतो युष्मानहं ब्रवीमि, नारीप्रसूतेषु स्नापकाद् योहनान्महान् भाववादी नास्ति। ईश्वरराज्ये तु यः क्षोदीयान् स तस्मादपि महत्तरः।
\vakya कृत्स्नं हि जनवृन्दं शुल्कादायिनश्च श्रुत्वा योहनस्य स्नानेन स्नापिता ईश्वरं धार्मिकं मत्वा प्रतिपेदिरे,
\vakya फरीशिनस्तु शास्त्राध्यापकाश्च तेन स्नानमस्वीकृत्येश्वरस्य मन्त्रणामात्मोद्देशे व्यर्थीचक्रिरे।
\vakya तदेतत्कालीया मनुष्या मया कैरुपमेयाः?
\vakya ते हट्ट उपविष्टैस्तैर्बालकैः सदृशा येषामेकेऽन्यानाह्वयन्तो वदन्ति, युष्मत्कृते वयं वंशीरवादयाम यूयन्तु नानृत्यत, युष्मत्कृते वयं व्यलपाम यूयन्तु नारुदित।
\vakya योहनो हि न पूपभोजनं न द्राक्षारसपानं वा सेवमान आगतः, युष्माभिस्तूच्यते, स भूताविष्टः।
\vakya मनुष्यपुत्रो भोजनं पानञ्च सेवमान आगतः, युष्माभिस्तूच्यते, पश्यासौ भोक्ता मद्यपश्च मनुष्यः, शुल्कादायिनां पापिनाञ्च बन्धुः।
\vakya प्रज्ञा तु स्वसर्वसन्तानानां स्वभावेन निर्दोषीकृता।
\stitle{अनुतापिनीं स्त्रियम् प्रति यीशोर्दया।}
\vakya अथ कश्चित् फरीशी स्वेन सह भोजनार्थं तं निमन्त्रयामास, ततः स तस्य फरीशिनो गृहं प्रविश्य भोजनार्थमुपविवेश।
\vakya पश्य च तन्नगरनिवासिनी काचित् पापिष्ठा योषित् तस्य फरीशिनो गृहे भोजनार्थं तदुपवेशनस्य संवादं श्रुत्वा सुगन्धितैलपूर्णं श्वेतोपलपात्रमादाय
\vakya पश्चात् तच्चरणयोः सन्निधौ तिष्ठन्ती रुदती च नेत्राम्बुना तस्य चरणौ निषेक्तुमारेभे, पुनः स्वशिरसः कचै र्मार्ष्ट्वा तौ चरणावचुम्बत् तैलेनामर्दयच्च।
\vakya तद्दृष्ट्वा तन्निमन्त्रयिता फरीशी स्वान्तरेऽवादीत्, असौ चेद् भाववाद्यभविष्यत्, तर्ह्यज्ञास्यत्, का कीदृशी च सा योषित् या तं स्पृशति, यतः सा पापिष्ठा।
\vakya ततो यीशुः प्रतिभाषमाणस्तमुवाच, शिमोन, तुभ्यं कथयितव्यं मम किञ्चिदास्ते। स उवाच, गुरो कथयतु।
\vakya एकस्योत्तमर्णस्य द्वावधमर्णावास्तां, तयोरेकः पञ्चशतान्यन्यतरश्च पञ्चाशत् मुद्रापादानधारयत्।
\vakya तयोस्तु प्रतिदानस्योपायाभावात् स उभयोश्चक्षमे। तद् वद, कतरस्तं प्रत्यधिकं प्रेम करिष्यति?
\vakya शिमोनः प्रत्युवाच, मन्ये यस्याधिकमृणं चक्षमे। स तमब्रवीत्, यथार्थो विचारोऽकारि त्वया।
\vakya ततः परं स तां योषितं प्रति मुखं प्रत्यावर्त्य जगाद, योषितममूं किं पश्यसि? मयि तव गृहं प्रविष्टे त्वं मत्पादप्रक्षालनार्थं तोयं नादाः, असौ तु स्वनेत्राम्बुना मम पादौ निषिक्तवती स्वशिरसः कचै र्मार्जितवती च।
\vakya त्वं मां नाचुम्बीरसौ त्वत्र मत्प्रवेशकालमारभ्य मम पादौ चुम्बितुं न निवृत्ता।
\vakya त्वं तैलेन मम शिरो नाममर्दः, असौ तु सुगन्धितैलेन मम पादौ मर्दितवती।
\vakya अतो हेतोस्त्वां ब्रवीमि, अमुष्याः पापानि यानि बहूनि तेषां क्षमाकारि, यतः (पश्य) सा बहुप्रेम चकार। यस्य तु स्तोकं क्षम्यते स स्तोकं प्रेम करोति।
\vakya ततः परं स तां जगाद, तव पापानां क्षमाकारि।
\vakya तदा सहभोजिनः स्वेषु वक्तुमारेभिरे, कोऽसौ यः पापक्षमामपि करोति?
\vakya स तु तां योषितं जगाद, तव विश्वासस्त्वां तारयामास, क्षेमेण याहि\eoc