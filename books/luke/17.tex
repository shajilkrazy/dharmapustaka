\adhyAya
\stitle{क्षमादयरधि उपदेशः।}
\vakya अथ स स्वशिष्यानब्रवीत्, स्खलनहेतवो यन्न भविष्यन्ति तदसम्भवं, ते तु येनायान्ति स सन्तापपात्रं।
\vakya तत्कण्ठे बृहत्पेषणीबन्धनं समुद्रे च तस्य निक्षेपणं श्रेयः, न चैतेषां क्षुद्राणामेकस्य निक्षेपणं श्रेयः, न चैतेषां क्षुद्राणामेकस्य स्खालनं।
\vakya युयमात्मार्थं सावधानास्तिष्ठत। तव भ्राता यदि त्वद्विरुद्धं पापं करोति, तर्हि तं भर्त्सय, यदि त्वनुतप्यते तर्हि तस्य क्षमस्व।
\vakya एकस्मिन् दिने सप्तकृत्वस्त्वद्विरुद्धं पापं कृत्वा यदि स एकस्मिन् दिने सप्तकृत्वस्त्वां प्रति परावृत्य वदति, अनुतप्य इति, तर्हि तस्य क्षमिष्यसे।
\vakya प्रेरितास्तदा प्रभुमूचुः, अस्माकं विश्वासं वर्धयतु।
\vakya प्रभुस्तु तान् जगाद, सति विश्वासे युष्माकं सर्षपबीजमिते यूयं चेदेनमुडुम्बरवृक्षं वदथ, त्वमुन्मूलितो भूत्वा समुद्रे रोपितो भवेति, तर्हि स युष्माकमाज्ञां ग्रहीष्यति।
\vakya युष्माकं कस्यचिद्धलधरः पशुचारको वा दासो यदा क्षेत्राद् गृहं प्रविशति, तदा स किं तं वदिष्यति, इदानीमेवोपागत्य भोजनार्थमुपविशेति।
\vakya स तं किं नेदं वदिष्यति, ममापराह्णिकाहारार्थं सर्वं सज्जीकुरु, यावच्चाहमश्नामि पिबामि च तावद् बद्धकटिः स्थितो मां परिचर, ततः परं त्वमप्यशिष्यसि पास्यसि चेति।
\vakya स दास आदिष्टानि कर्माणि कृतवानितिहेतोः स किं तं प्रति कृतज्ञतां स्वीकरिष्यति?
\vakya मन्येऽहं न तथा। तथैव यूयमपि यद्यदादिष्टास्तत्समापनात् परं ब्रूत, अनुपयोगिनो दासा वयं, अस्माभि र्यद्यद् ऋणशोध इव कर्तव्यमासीत् तदेवाकारि।
\stitle{यीशोर्दशकुष्ठिनां स्वास्थ्यकरणम्।}
\vakya अथ स यदा यिरूशालेममगच्छत् तदा शमरियागालीलयो र्मध्येनागच्छत्।
\vakya प्रविष्टे च तेन कस्मिंश्चिद् ग्रामे कुष्ठिनो दश नरास्तस्य सम्मुखीबभूवुः,
\vakya ते दूरे तिष्ठन्तः प्रोच्चैरवदन्, यीशो नाथ, अस्माननुकम्पतां।
\vakya तद् दृष्ट्वा स तानब्रवीत्, यूयं गत्वा याजकेभ्यः स्वान् दर्शयत।
\vakya ततस्तेषां यात्राकाले ते शुचीभूताः। तेषामेकस्तदात्मानं प्राप्तारोग्यं दृष्ट्वोच्चरवेणेश्वरं स्तुवन् प्रत्याववृते
\vakya तस्य चरणयोरधोमुखं प्रणिपत्य च तं प्रति कृतज्ञतामङ्गीचकार। स नरः शमरीयः।
\vakya यीशुस्तदा प्रतिभाषमाणो जगाद, किं न दश ते शुचीकृताः? ते नव तर्हि कुत्र?
\vakya ईश्वरं स्तोतुं प्रत्यावृत्ता अस्माद् विजातीयदन्ये केऽपि किं न लक्षिताः?
\vakya ततः परं स तमुवाच, उत्थाय याहि, तव विश्वासस्त्वां तारयामास।
\stitle{ख्रीष्टागमनस्य कथा।}
\vakya अथेश्वरस्य राज्यं कदायातीति फरीशिभिः पृष्टः स प्रतिभाषमाणस्तानवादीत्, नायातीश्वरस्य राज्यं प्रत्यक्षतया सहितं, जनाश्च न वदिष्यन्ति, पश्यात्रेति, पश्यामुत्रेति वा।
\vakya यतः पश्यत, ईश्वरस्य राज्यं युष्मदन्तःस्थम्।
\vakya अथ स स्वशिष्यानुवाच, आयाति स कालो यदा यूयं मनुष्यपुत्रस्य दिनानामेकं दिदृक्षिष्यथ न तु द्रक्ष्यथ।
\vakya जनाश्च युष्मान् वदिष्यन्ति, पश्यात्रेति, पश्यामुत्रेति वा। मैवापगच्छत, मैवानुधावत वा।
\vakya यतः प्रकाशमाना विद्युद् यथाकाशस्याधोदेशमारभ्याकाशस्याधोदेशं यावद् विराजते, तथैव मनुष्यपुत्रोऽपि स्वदिवसे प्रतिभास्यति।
\vakya अपि तु प्रथमं तेन बहु दुःखं भोक्तव्यम्, एतत्कालिकै र्मनुष्यैश्च तस्य निराकरणमवश्यम्भावि।
\vakya अपि च नोहस्य काले यथा सम्भूतं, तथैव मनुष्यपुत्रस्य कालेऽपि सम्भविष्यति।
\vakya पोते नोहस्य प्रवेशदिनं यावन्मनुष्या आश्नन्नपिबन्नुदवहन्नुदौह्यन्त च, तदा तु वन्योपस्थाय सर्वान् नाशयामास।
\vakya तथैव लोटस्य कालेऽपि सम्भूतं। मनुष्या आश्नन्नपिबन्नक्रीणन् व्यक्रीणन्नरोपयन् गृहनिर्माणमकुर्वंश्च।
\vakya यस्मिन् दिने तु लोटः सदोमान्निर्ययौ, तस्मिन्नाकाशाद् वर्षन्तौ वह्निगन्धकौ सर्वान् नाशयामासतुः।
\vakya मनुष्यपुत्रो यस्मिन् दिने प्रकाशिष्यते, तस्मिंस्तथैव भविष्यति।
\vakya तस्मिन् दिने यो गृहपृष्ठे विद्यते तस्य द्रव्येषु गृहमध्ये स्थितेषु मावरोहतु स तान्यादातुं। तथैव यः क्षेत्रे विद्यते स मा परावर्ततां।
\vakya लोटस्य भार्यां स्मरत।
\vakya यः कश्चित् स्वप्राणान् रक्षितुं यतिष्यते स तान् हारिष्यति, यश्च तान् हारयिष्यति स तान् जीवयिष्यति।
\vakya युष्मानहं ब्रवीमि, तस्यां रात्रावेकस्यां खट्वायां शयानयो र्द्वयो र्नरयोरेकतरो ग्राहीष्यतेऽन्यतरः परित्यक्ष्यते।
\vakya योषितो र्द्वयोरेकत्र पेषणीनियुक्तयोरेकतरा ग्राहीष्यतेऽन्यतरा परित्यक्ष्यते।
\vakya नरयो र्द्वयोः क्षेत्रे स्थितयोरेकतरो ग्राहीष्यतेऽन्यतरः परित्यक्ष्यते।
\vakya तदा ते प्रतिभाषमाणास्तं पप्रच्छुः, कुत्र, प्रभो? स तानवादीत्, यत्र कुणपं तत्रैव गृध्राः समागमिष्यन्ति\eoc