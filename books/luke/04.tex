\adhyAya
\stitle{यीशोः परीक्षा।}
\vakya अथ यीशुः पवित्रेणात्मना पूर्णो यर्दनतः प्रत्यावृत्त आत्मना च मरुं नीतश्चत्वारिंशद्दिनानि दियावलेन परीक्षितश्च।
\vakya तेषु दिनेषु स निरशनस्तस्थौ। तेषु त्वतीतेषु स चुक्षोद।
\vakya ततो दियावलस्तमाह, भवांश्चेदीश्वरस्य पुत्रस्तर्हि प्रस्तरोऽसौ यथा पूपो भवेत् तथाज्ञापयतु।
\vakya यीशुस्तं प्रतिजगाद, लिखितमास्ते, न केवलेन पूपेन मनुष्यो जीविष्यत्यपि त्वीश्वरस्य येन केनचिद् वचनेन।
\vakya ततः परं दियावलस्तमुच्चमेकं गिरिशिखरं नीत्वा निमिषैकमध्ये भूमण्डलस्य निखिलराज्यानि दर्शयन्नवादीत्,
\vakya कृत्स्नमिदं कर्तृत्वमेतेषां प्रतापञ्चाहं तुभ्यं दास्यामि, यतस्तत् सर्वं मयि समर्पितं, यस्मै च तद् दातुमिच्छामि तस्मै तद् ददामि,
\vakya अतस्त्वं चेन्मम समक्षं प्रणिपतेस्तर्हि तत् सर्वं तवैव भविष्यति।
\vakya यीशुस्तं प्रतिबभाषे मत्तोऽपसर शैतान, यतो लिखितमास्ते, निजेश्वरस्य प्रभो र्भजना त्वया कर्तव्या एकश्च स एव त्वयाराधित्वः।
\vakya पुनश्च स तं यिरूशालेमं नीत्वा धर्मधाम्नः शिखिरे स्थापयित्वा च जगाद, भवांश्चेदीश्वरस्य पुत्रस्तर्ह्यस्मात् स्थानाद् अधस्थात् प्रपततु, यतो लिखितमास्ते,
\begin{poem}
\startwithvakya त्वदर्थं निजदूतान् स तवादेक्ष्यति रक्षणं।
\vakya यन्नाहन्याः पदं शैले तत् त्वां वक्ष्यन्ति ते करैः॥
\end{poem}
\vakya यीशुस्तं प्रतिजगाद, उक्तमास्ते, त्वं स्वेश्वरस्य प्रभोः परीक्षां माकार्षीः।
\vakya इत्थं कृत्स्नां परीक्षां समाप्य दियावल उपयुक्तं समयं यावत् तस्मात् प्रतस्थे।
\stitle{नासरतनगरे यीशोरुपदेशः।}
\vakya यीशुश्चात्मनः प्रभावेन गालीलं प्रत्यावृत्तस्तस्य ख्यातिश्च तं कृत्स्नं प्रदेशं व्यानशे।
\vakya स च तेषां समाजगृहेषूपादिशत् सर्वैश्च प्राशस्यत।
\vakya इत्थं पुरा यत्र स पोषितस्तन्नासरतनगरमाजगाम, स्वरीत्यनुसारेण च विश्रामदिने समाजगृहं प्रविश्य पाठार्थमुत्तस्थौ।
\vakya तदा भाववादिनो यिशायाहस्य ग्रन्थे तस्मै दत्ते स ग्रन्थं विस्तार्य तत् स्थानमवाप यत्र लिखितमासीत्,
\begin{poem}
\startwithvakya आस्ते मयि प्रभोरात्मा यतो मां सोऽभिषिक्तवान्।
\pline ज्ञापयेयं सुसंवादं दरिद्रान् मानवान् यथा॥
\pline तेनाहं प्रहितः कर्तुं भग्नचित्तान् निरामयान्।
\pline वन्दीनां घोषितुं मुक्तिं दृक्‌शक्तिञ्च विचक्षुषाम्॥
\pline अप्यनुज्ञाय विस्रष्टुं मनुष्यान् परिपीडितान्।
\vakya प्रभो र्घोषयितुञ्चापि प्रसादावहवत्सरम्॥
\end{poem}
\vakya ततः परं स तं ग्रन्थं बद्ध्वा भृत्ये प्रत्यर्प्यासन उपविवेश। समाजस्थाः सर्वे चैकदृष्ट्या तं निरीक्ष्य तस्थुः।
\vakya तदा स तानिदं वक्तुमारेभे, शास्त्रीयवचनमिदमद्य युष्मत्कर्णगोचरे पुर्णतां गतम्।
\vakya सर्वे च तस्य पक्षे साक्ष्यमददुस्तन्मुखान्निःसरत्सु प्रीतिजनकेषु वचःस्वाश्चर्यममन्यन्त च। ते पुनरवदन्, किं न योषेफस्य पुत्रोऽसौ?
\vakya ततः स तान् जगाद, यूयमवश्यं मामिमं प्रवादं वक्ष्यथ, चिकित्सक, आत्मचिकित्सां कुरु, कफरनाहूमे सिद्धानां येषां कर्मणां किंवदन्त्यस्माभिरश्रावि, तादृशान्यत्र स्वदेशेऽपि कुर्विति।
\vakya स पुन र्जगाद, कोऽपि भाववादी स्वदेशे नानुगृह्यते।
\vakya युष्मानहं सत्यं ब्रवीमि, एलियस्य काले यदा त्रीन् वत्सरान् षण्मासांश्च यावदाकाशो रुद्धोऽतिष्ठत् कृत्स्ने देशे महादुर्भिक्षञ्चाभूत् इस्रायेले तदा बह्व्यो विधवा अविद्यन्त, एलियस्तु तासां कामपि प्रति न प्रहितः,
\vakya केवलं सीदोनदेशस्थं सारिफन्नगरं काञ्चिद्विधवां योषितं प्रति प्रहितः।
\vakya तथा भाववादिन इलीशायस्य काल इस्रायेले बहवः कुष्ठिनोऽविद्यन्त, तेषान्तु कोऽपि न शुचीकृतः, केवलः सुरीयो नामानः शुचीकृत इति।
\vakya इदं श्रुत्वा समाजस्थाः सर्वे रोषेण पुर्णा उत्थाय तं नगराद् बहिश्चक्रुः।
\vakya यत्र पर्वते च तेषां नगरं निर्मितमासीत् तदग्रभागं नीत्वा तं निपातयितुमैच्छन्।
\vakya स तु तेषां मध्येन व्रजन् प्रतस्थे।
\stitle{यीशुना बहुभ्यः पीडितभूतग्रस्तलोकेभ्यः आरोग्यदानम्।}
\vakya अनन्तरं स कफरनाहूमाख्यां गालीलस्थं नगरमवततार।
\vakya विश्रामवारे च यदा जनानशिक्षयत् तदा ते तस्य शिक्षायामाश्चर्यम् अमन्यन्त, यतस्तस्य वाक्यं सामर्थ्यविशिष्टमासीत्।
\vakya तत्र समाजे चाशुचिभूतस्यत्मनाधिष्ठित एको नरोऽविद्यत, स उच्चैःस्वरेणोत्क्रोशन् जगाद,
\vakya निवर्ततां, भो नासरतीय यीशो, अस्माकं भवतश्च किम्? भवान् किमस्मान् नाशयितुमागतः? भवान् कस्तदहं जानामि, ईश्वरस्य पवित्रो नरो भवानिति।
\vakya यीशुस्तु तं निर्भर्त्सयन् बभाषे, मौनी भव, अस्माच्च निःसर। ततः स भूतस्तं मध्यस्थाने निक्षिप्य किमपि न हिंसित्वा तस्मान्निःसृतः,
\vakya सर्वे च विस्मयेनापन्नाः परस्परं संलपन्तश्चावादिषुः, वाक्यमिदं कीदृशम्? यतोऽसौ सामर्थ्येन प्रभावेन चाशुचीनात्मनोऽप्याज्ञापयति ते च निःसरन्ति।
\vakya अनन्तरं तदीयकिंवदन्ती परितःस्थस्य जनपदस्य सर्वत्र व्यानशे।
\stitle{पित्रस्य भवने।}
\vakya अनन्तरं स उत्थाय समाजात् शिमोनस्य गृहं प्रविवेश। शिमोनस्य श्वश्रूस्तदा प्रबलेन ज्वरेणापीड्यत, ते च तस्या उपकारार्थं तं प्रसादयामासुः।
\vakya ततः स तस्या ऊर्ध्वे तिष्ठन् ज्वरं तर्जयामास, ज्वरञ्च तां तत्याज, सा च तत्क्षणमुत्थाय तान् पर्यचरत्।
\vakya ततः परमस्तं गच्छति सूर्ये विविधव्याधिग्रस्ता अस्वस्था जना येषामासंस्ते सर्वे तान् तस्यान्तिकमानिन्युः। स च तेषामेकैकस्मिन् हस्तावर्पयंस्तान् निरामयांश्चकार।
\vakya बहुभ्यश्च भूता निःसरन्त उच्चैःस्वरेणावदन्, भवानीश्वरस्य पुत्रः ख्रीष्टः। ततः स तान् भर्त्सयन् भाषितुं नान्वमन्यत, यतः स ख्रीष्ट इति तेऽजानन्।
\vakya जाते पुनः प्रातःकाले स निर्गत्य किञ्चन निर्जनं स्थानं ययौ, जननिवहास्तु तं मृगयाञ्चक्रिरे तस्यान्तिकमागत्य च तेषां सन्निधितस्तस्य प्रस्थानं निवारयितुं तमधरन्।
\vakya स तु तानब्रवीत्, अन्येष्वपि नगरेष्वीश्वरराज्यस्य सुसंवादो मया घोषयितव्यः, यतस्तदर्थमहं प्रहितः।
\vakya ततः परं स गालीलस्य समाजगृहेष्वघोषयत्\eoc