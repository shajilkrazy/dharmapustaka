\adhyAya
\stitle{यीशोः पुनरुत्थानम्।}
\vakya सप्ताहस्य प्रथमे दिने त्वतीव प्रत्यूषे ताः सज्जीकृतानि सुगन्धिद्रव्याणि वहन्त्यः शवागारान्तिकं जग्मुस्ताभिः सार्धमन्या अपि काश्चिन्नार्योऽव्रजन्।
\vakya तदा ताः शवागारद्वारात् प्रस्तरमपसारितं लक्षयामासुः,
\vakya प्रविश्य तु प्रभो र्यीशो र्देहं न ददृशुः।
\vakya पश्य चात्र चिन्ताकुलासु तासु दीप्रपरिच्छदवेष्टितौ द्वौ पुरुषौ ता उपतस्थतुः।
\vakya तासु च भीतासु भूमिं प्रति नतमुखीषु च तौ ता ऊचतुः, कथं मृतानां मध्ये तं जीवितं गवेषयथ?
\vakya सोऽत्र नास्ति, स उत्थापित एव।
\vakya पुरा गालीले विद्यमानेन तेन यूयं यदुक्तास्तत् स्मरत।
\vakya स हि कथितवान्, मनुष्यपुत्रस्येदमवश्यम्भावि, यत् स पापिमनुष्याणां करेषु समर्पयिष्यते, क्रुशमारोपयिष्यते च तृतीये दिने च पुनरुत्थास्यति। तदा तास्तस्योक्तीः सस्मरुः।
\vakya अनन्तरं ताः शवागारात् प्रत्यावृत्यैकादशभ्यः प्रेरितेभ्योऽन्येभ्यः सर्वेभ्यश्च संवादं ददुः।
\vakya मग्दलीनी मरियम् योहाना याकोबस्य माता मरियम् तासां सङ्गिन्योऽन्याश्चेमा एव प्रेरितेभ्यः सर्वमेतत् कथयामासुः।
\vakya तेषां दृष्टौ तु तासां कथा प्रलाप एव प्रत्यभात् ते च तासु न व्यश्वसन्।
\vakya पित्रस्तूत्थाय शवागारं प्रति दुद्राव, प्रह्वश्च केवलं वस्त्रखण्डान् भूमौ विस्तृतान् ददर्श, यच्च वृत्तं तत्र मनस्याश्चर्यं मन्यमानोऽपजगाम।
\vakya पश्य च तस्मिन् दिवसे तेषां द्वौ नरौ यिरूशालेमात् क्रोशचतुष्टयव्यवहितमिम्मायूनामकं ग्राममगच्छतां।
\vakya तावेतानि सर्वाणीतिवृत्तान्यधि मिथः समलपतां।
\vakya तयोरित्थं संलपत्यो र्विचारयत्योश्च स्वयं यीशुः समीपमागत्य ताभ्यां सार्धं व्रजितुं प्रववृते।
\vakya तयो र्नेत्राणि तु तथा रुद्धान्यासन् यथा तौ तमभिज्ञातुं नाशक्नुतां।
\vakya स तदा ताववादीत्, कास्ताः कथा व्रजनकाले या मिथो विचारयथो विषणौ?
\vakya तदा क्लियपानामा तयोरेकतरः प्रतिभाषमाणस्तमवादीत्, एको भवान् प्रवासीव यिरूशालेमे वर्तमानोऽपि किं नावगतवांस्तत् सर्वं यदेतेषु दिनेषु तत्र वृत्तं?
\vakya स तौ पप्रच्छ, तत् किं? तौ तमूचतुः, नासरतीयं यीशुमधि यद् वृत्तं तदेव। स भाववादी नर आसीत्, क्रियया वाचा च शक्तिमान्।
\vakya स यन्मुख्ययाजकैरस्माकं नायकैश्च प्राणदण्डार्थं समर्पितः क्रुशमारोपितश्चाभूत्।
\vakya अस्माकन्त्वियमाशासीत्, यत् स एवेस्रायेलस्य भावी मोचयिता। तद् यथास्तु तथास्तु, तस्मादितिवृत्तादारभ्याद्य तृतीयं दिनं स गतः।
\vakya अपि चास्मत्सङ्गिनीभिः काभिश्चिन्नारीभि र्वयं चमत्कारिताः।
\vakya ताः प्रत्यूषे शवागारं गतास्तदीयदेहमप्राप्य प्रत्यागताश्च, ता अवदन्, वयं स्वर्गदूतयो र्दर्शनं प्राप्तास्ताभ्यामुच्यते स जीवतीति।
\vakya ततोऽस्मत्सङ्गिनां केऽपि शवागारं जग्मुः, ते ताभि यथा कथितं तदनुरूपमेव सर्वं प्रापुस्तन्तु न ददृशुः।
\vakya स तदा तावुवाच, रे निर्बोधौ भाववादिभि र्यद्यदुक्तं तत्र विश्वसितुं मन्दधियौ च,
\vakya कृत्स्नं तद् दुःखं भुक्त्वा ख्रीष्टो यत् स्वप्रतापं प्रविशति, किं न तत् तस्यावश्यङ्कर्तव्यमासीत्?
\vakya ततः परं स मोशितः सर्वेभ्यो भाववादिभ्यश्चारभ्य कृत्स्ने शास्त्रे तमधि यद्यल्लिखितं तद् व्याख्यात्।
\vakya इत्थं ते यदा तयो र्गन्तव्यस्य स्थानस्य समीपमुपातिष्ठन्त स तदाग्रगमनेच्छाया लक्षणमदर्शयत्।
\vakya तौ तु साग्रहं प्रसादयन्तौ तमूचतुः, तिष्ठत्वावाभ्यां सार्धं, यतः सायंसमय आसन्नौ दिनावसानञ्च प्रतिभाति। ततः स ताभ्यां सार्धमवस्थानार्थं गृहं प्रविवेश।
\vakya ततः परं ताभ्यां सार्धं भोजनायोपविश्य स पूपं गृहीत्वा धन्यवादमकार्षीत्, तं भङ्क्वा च ताभ्यामददात्।
\vakya अनेन तयो र्नेत्राणि मुक्तानि तौ च तमभिजज्ञतुः। स तु तयोः समीपतोऽन्तर्दधे।
\vakya तौ तदा परस्परमवादिष्टां, अस्मद्धृदयं किं नाज्वलत् तदास्मदन्तरे यदा स पथ्यावाभ्यां समलपदस्मदर्थञ्च शास्त्राणि व्याख्यत्?
\vakya तस्मिन्नेव दण्डे तावुत्थाय यिरूशालेमं प्रतिजग्मतुः समवेतांश्चैकादशप्रेरितांस्तेषां सङ्गिनश्च प्रापतुः।
\vakya तेऽवदन् सत्यं प्रभुरुत्थापितो दत्तवांश्च शिमोनाय दर्शनमिति।
\vakya तावपि पथि यद् वृत्तं पूपभञ्जने च स यथा ताभ्यामभिज्ञातस्तन्निवेदयामासतुः। 
\vakya इत्थं तेषां कथनकाले स्वयं यीशुस्तेषां मध्ये तस्थौ तांश्च जगाद, युष्माकं शान्ति र्भूयात्।
\vakya ततस्ते क्षुब्ध्वा त्रासं गत्वा चान्वमिमत कञ्चिदात्मानं पश्याम इति।
\vakya स तु तानब्रवीत्, कथमुद्विग्नाः स्थ? किमर्थञ्च युष्माकं हृदयेषु वितर्का उद्भवन्ति?
\vakya मम हस्तौ चरणौ च निरीक्ष्य जानीत यत् स्वयमहं सोऽस्मि। स्पृशत मां समवलोकयत च। नास्त्यात्मा तादृङ्मांसास्थिविशिष्टो यादृशं मां पश्यथ।
\vakya एतद् भाषमाणः स तेभ्यः स्वहस्तौ स्वचरणौ च दर्शयामास।
\vakya तदापि त्वानन्दात् तेषु न विश्वसत्स्वाश्चर्यं मन्यमानेषु च स तान् पप्रच्छ, युष्माकमत्र कि़ञ्चित् खाद्यं किं विद्यते?
\vakya तदा ते तस्मै भृष्टमीनस्य खण्डं मधुकोषस्य खण्डञ्च ददुः,
\vakya स च तेषां समक्षं तद् गृहीत्वा बुभुजे।
\vakya तांश्चोवाच, मोशे र्व्यवस्थायां भाववादिनां ग्रन्थेषु गीतसंहितायाञ्च मामधि यद्यल्लिखितमास्ते सर्वेण तेन सिद्धेन भवितव्यमिति याः कथाः पुरा युष्माभिः सार्धं ममावस्थितिकाले यूयं मयोक्तास्तासां फलमेतत्।
\vakya तदा स तेषां बुद्धिं शास्त्रबोधायोद्घाटयामास।
\vakya ताश्च जगाद, इत्थं लिखितमास्ते इत्थञ्च ख्रीष्टस्यावश्यङ्कर्तव्यमासीत्, यत् तेन दुःखं भोक्तव्यं तृतीये दिने पुनरुत्थातव्यं,
\vakya यिरूशालेमादारभ्य च सर्वजातिषु तस्य नाम्ना मनःपरावर्तनस्य पापक्षमायाश्च घोषणेन व्याप्तव्यं।
\vakya यूयञ्च सर्वस्यैतस्य साक्षिणः।
\vakya पश्यत च मम पित्रा यत् प्रतिश्रुतं तदहं युष्माकमुपरि प्रेषयिष्यामि। यूयञ्च यावदूर्ध्वलोकतः प्रभावेन सज्जिता न भविष्यथ, तावद् यिरूशालेमपुर्यामवतिष्ठध्वम्।
\vakya अथ स तान् बैथनियां यावद् बहि र्निनाय, स्वकरावूर्ध्वीकृत्य च तेभ्य आशिषं ददौ।
\vakya आशीर्दानकाले च तेभ्यः पृथग्भूय स्वर्गमारोहयाञ्चक्रे।
\vakya ते च तस्य भजनं कृत्वा महानन्देन यिरूशालेमं प्रतिजग्मुः,
\vakya सततं धर्मधाम्नि वर्तमानाश्चेश्वरमस्तुवन् धन्यमवदंश्च\eoc