\adhyAya
\stitle{यीशोर्विरुद्धकुमन्त्रणा।}
\vakya निष्किण्वपूपानां पर्व तदासन्नमासीत्, तस्य नामान्तरं पास्खेति।
\vakya मुख्ययाजकाः शास्त्राध्यापकाश्च यीशो र्व्यापादनस्योपायमगवेषयन्, यतस्ते जनेभ्यो भीता आसन्।
\vakya अथ शैतानो द्वादशानां श्रेण्यां गणितमीष्करियोतीयाभिधं यिहूदाम् आविवेश।
\vakya स च मुख्ययाजकानां सेनापतीनाञ्च समीपं गत्वा तेषु तत्समर्पणस्योपायमधि तैः संललाप।
\vakya ततस्ते हर्षं गत्वा तस्मै रूप्यं दातुमङ्गीचक्रुः।
\vakya सोऽपि सम्मतो बभूव, जनताव्यतिरेकेण तेषु तत्समर्पणार्थमवकाशं प्राप्तुमयतत च।
\stitle{निस्तारपर्वपालनं प्रभोर्भोज्यस्थापनञ्च।}
\vakya ततः परं निष्किण्वपूपानां दिनमुपतस्थे यदा निस्तारोत्सवीयो मेषो घातयितव्यः।
\vakya यीशुस्तदा पित्रं योहनञ्चेदमादिशन् प्राहिणोत्, युवां गत्वास्मदर्थं निस्तारोत्सवीयं भोज्यं सज्जीकुरुतं तदस्माभि र्यथा भुज्येत।
\vakya तौ तं पप्रच्छतुः, भवदिच्छातः कुत्रावाभ्यां सज्जीकर्तव्यं?
\vakya स ताववादीत्, पश्यतां युवयो र्नगरं प्रविष्टयो र्जलपूर्णं कलसं वहन् कश्चिन्नरो युष्मत्सम्मुखमुपस्थास्यते, स यद् गृहं प्रवेक्ष्यति, तदभ्यन्तरं यावद् तमनुगच्छतं,
\vakya गृहस्वामिनञ्च वदतं, गुरु र्भवन्तं पृच्छति, कुत्र तदतिथिगृहं यत्राहं स्वशिष्यैः सार्धं निस्तारोत्सवीयं भोज्यं भोक्ष्ये?
\vakya स तदा वां भूमित उच्चतरामासन्नैः सज्जितां बृहतीं शालां दर्शयिष्यति। तत्रैवोपकल्पयतं।
\vakya ततस्तौै गत्वा तेन यत् कथितं तदनुरूपं दृष्ट्वा निस्तारोत्सवीयं भोज्यं सज्जीचक्रतुः।
\vakya समये तूपस्थिते स भोक्तुमुपविवेश द्वादश प्रेरिताश्च तेन सार्धमासन्।
\vakya स तदा तानुवाच, दृढाकाङ्क्षयेदं मया प्रत्याकाङ्क्षितं यन्ममम दुःखभोगात् प्राग् युष्माभिः सार्धमिदं निस्तारोत्सवीयं भोज्यं भुञ्जीय।
\vakya युष्मानहं ब्रवीमि, तद् यावदीश्वरस्य राज्ये सिद्धं न भविष्यति तावन्मया नैव भोक्ष्यते पुनः।
\vakya ततः परं स पानपात्रं गृहीत्वा धन्यवादं कृत्वा जगाद, गृह्णीतैतन्मिथो विभजध्वं च।
\vakya यतो युष्मान् ब्रवीमि, इतः परं यावदीश्वरस्य राज्यं नोपस्थास्यते तावद् गोस्तनीलतोत्पन्नो रसो मया न पायिष्यते।
\vakya ततः परं स पूपमादाय धन्यवादं कृत्वा च तं भङ्क्त्वा तेभ्यो ददौ बभाषे च, एतन्मम शरीरं यद् युष्मदर्थं दीयते। मम स्मरणार्थमिदं कुरुत।
\vakya भोजनात् परञ्च स तथैव पानपात्रमादाय बभाषे, पानपात्रमेतन्मम शोणितेन स्थापितो नूतनो नियमः, युष्मदर्थं तद् विस्राव्यते।
\vakya अपि च पश्यत, मत्समर्पयितु र्हस्तो मया सार्धं भोजनमञ्चे वर्तते।
\vakya यथा निरूपितं यथैव मनुष्यपुत्रः प्रयाति, नरेण येन तु स समर्प्यते स सन्तापपात्रं।
\vakya ततस्तेषां कस्मत् करिष्यत्येतदधि ते मिथो विचारयितुमारेभिरे।
\stitle{को महान्?}
\vakya पुनस्तेषां को महान् मान्यः, एतदधि तेषु वादानुवाद उत्पेदे।
\vakya स तु तान् जगाद, परजातीयानां राजानस्तेषामुपरि प्रभुत्वं कुर्वते, तेषां कर्तृत्वकारिणश्च हितकरा इत्यभिधीयन्ते।
\vakya यूयन्तीदृग्भावा मा भवत। प्रत्युत युष्मन्मध्ये यो महान् स कनिष्ठ इव भवतु, यश्च नायकः स परिचारक इव भवतु।
\vakya कतरो हि महान्, भोज्ये य आसीनोऽस्त्यथवा यः परिचरति? किं न स य आसीनोऽस्ति? युष्माकं मध्ये त्वहं परिचारक इव वर्ते।
\vakya मत्परीक्षासु तु यूयं स्थिरभावेन मया सार्धमवस्थितास्ततो
\vakya यथा मम पिता मह्यं दायवद् राज्यं दातुमङ्गीकृतवान्, तथाहं युष्मभ्यं दायवदिममधिकारं दातुमङ्गीकरोमि,
\vakya यन्मम राज्ये यूयं मया सार्धं मम भोजनमञ्चे भोक्ष्यध्वे पास्यथ च, सिंहासनेष्वासीना इस्रायेलस्य द्वादशानां वंशानां विचारं करिष्यथ च।
\stitle{पित्रस्यानङ्गीकारकथनम्।}
\vakya ततः परं प्रभु र्बभाषे, शिमोन, शिमोन, पश्य, शैतानस्तितडना शस्यमिव चालयिष्यन् युष्मान् लब्धुं ययाचे
\vakya अहन्तु त्वदर्थं तथा प्रार्थयितवान् यथा तव विश्वासो न विलुप्येत। त्वमपि स्वसमये प्रत्यावृत्तस्तव भ्रातॄन् सुस्थिरीकुरु।
\vakya स तु तमाह, प्रभो, भवता सार्धमहं कारामपि मृत्युमपि च गन्तुमुद्यतोऽस्मि।
\vakya स तूवाच, पित्र, त्वामहं ब्रवीमि, त्वं यन्मां जनासि तत् त्रिकृत्वो यावन्नानङ्गीकरिष्यसि तावत् कुक्कुटोऽद्य न रविष्यति।
\vakya ततः परं स तानुवाच, यदाहं युष्मान् मुद्राधारं चेलसम्पुटकमुपानहौ च विना प्राहिणवं, तदा युष्माकं किं कस्याप्यभावो जातः? ते प्रत्यूचुः, न जातः।
\vakya स पुनस्तानवादीत्, अधुना तु यस्य मुद्राधारोऽस्ति स तं गृह्णातु चेलसम्पुटकञ्च तथा।
\vakya यस्य तु नास्ति स स्ववस्त्रं विक्रीणात्वसिञ्च क्रीणातु। यतोऽहं युष्मान् ब्रवीमि, मामध्येतेनापि सिद्धेन भवितव्यं यल्लिखितमास्ते,
\begin{poem}
\startwithline “अधर्माचारिणां मध्ये सोऽप्येको गणितोऽभवत्॥”
\end{poem}
@V\eoc