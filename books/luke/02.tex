\adhyAya
\stitle{यीशुख्रीष्टस्य जन्म बाल्यकालश्च।}
\vakya तस्मिन् काले औगस्तकैसरस्य सकाशादिदं शासनं प्रचचार, यत् साम्राज्यस्य सर्वजनैः स्वनाम्नां लेखारोपणं कर्तव्यं।
\vakya नाम्नां तल्लेख्यारोपणं सुरियाप्रदेशाधिपतेः कुरीणीयस्याधिकारकालस्य प्रथममासीत्।
\vakya अतो नामानि लेख्यमारोपयितुं सर्वजनः प्रत्येकं स्वनगरं जगाम।
\vakya योषेफोऽपि गालीलस्थान्नासरताख्यनगराद् यिहूदियादेशस्थं बैतलेहमाभिधं दायूदस्य नगरं जगाम, यतः स दायूदस्य कुले गोष्ठ्याञ्च जातः।
\vakya स आत्मने वाग्दत्तया स्वजायया मरियमा सार्धं नाम्नो लेख्यारोपणार्थं जगाम। सा तु गर्भिण्यासीत्।
\vakya तयोश्च तत्र वर्तमानयोस्तस्याः प्रसूतिकाल उपतस्थे।
\vakya अतः सा निजं प्रथमजातं पुत्रं प्रासोष्ट, वस्त्रखण्डै र्वेष्टयित्वा च पशुशालास्थगवादन्यां तं शाययामास, यतः प्रवासगृहे तेषां कृते स्थानं नासीत्।
\vakya तस्मिन् प्रदेशे वर्तमानाः केचिन्मेषरक्षकाः प्रान्तरे रात्रिं यापयन्तः पर्यायक्रमेण यामिन्यां स्वव्रजम् अरक्षन्।
\vakya पश्य च प्रभोरेको दूतस्तेषामन्तिकम् उपतस्थे प्रभोः प्रतापश्च तेषां परितो विरेजे। अनेन ते महाभयेनाक्रान्ताः।
\vakya स दूतस्तु तान् जगाद, मा भैष्ट। पश्यताहं युष्मभ्यं महानन्दस्य सुसंवादं ददामि, स आनन्दः कृत्स्नेन प्रजाजनेन भोग्यः।
\vakya यतो युष्मदर्थमद्य दायूदस्य नगरे परित्राताजनि, स हि ख्रीष्टः प्रभुः।
\vakya इदञ्च युष्मदर्थं तदभिज्ञानं, वस्त्रखण्डै र्वेष्टितः शिशु र्गवादन्यां शयानो युष्माभिरासादयिष्यत इति।
\vakya अकस्माच्च बहुतरा स्वर्गीयसेना तस्य दूतस्य सङ्गिनी भूत्वेश्वरस्य स्तवं गायन्ती व्याजहार,
\begin{poem}
\startwithvakya “ईश्वरायोर्ध्वलोकेषु भूयान्माहात्म्यकीर्तनं।
\pline शान्ति र्भूयात् पृथिव्याञ्च प्रीतिपात्राणि मानवाः॥”
\end{poem}
\vakya तेषु दूतेषु तेभ्यः स्वर्गं प्रस्थितेषु मेषरक्षकास्ते मनुष्याः परस्परमूचुः, गम्यतामस्माभि र्बैतलेहमम्, अवलोक्यताञ्च तदितिवृत्तं यत् प्रभुरस्मान् ज्ञापयामास।
\vakya अतस्ते सत्वरं गत्वा मरियमं योषेफञ्च गवादन्यां शयानं शिशुञ्चासादयामासुः।
\vakya दृष्ट्वा च ज्ञापयामासुस्तं शिशुमधि तां कथां यां ते प्रोक्ताः,
\vakya श्रोतारश्च सर्वे मेषरक्षकैः कथितेषु वचःस्वाश्चर्यं मेनिरे।
\vakya मरियमपि चित्ते विचारयन्ती तानि सर्वाणि वचांस्यरक्षत्।
\vakya ते मेषरक्षकास्तु यथोक्तास्तथैव यद्यच्छ्रुतवन्तो दृष्टवन्तश्च, तन्निमित्तमीश्वरं महयन्तस्तुवन्तश्च प्रत्याजग्मुः।
\vakya अष्टमे दिने तूपस्थिते शिशोस्त्वक्छेदकाले तस्य नाम यीशुरित्यकारि। गर्भावस्थात् प्रागेव तस्य तन्नाम दूतेनोदाहृतम्।
\stitle{शिशुयीशुमधि हान्नाशिमियोनयोः कथा।}
\vakya मोशे र्व्यवस्थानुसारेण च तस्याः शुचीभवनकाल उपस्थिते तौ प्रभोः समक्षं शिशुमुपस्थापयितुं यिरूशालेमं निन्यतुः,
\vakya यतः प्रभो र्व्यवस्थायामित्थं लिखितमास्ते, गर्भाशयोद्घाटकः सर्वपुंसन्तानः प्रभोः कृते पवित्रोऽभिधायिष्यते।
\vakya पुनश्च प्रभो र्व्यवस्थायां यत् कथितमास्ते तदनुसारेण यज्ञीयोपहारार्थं कपोतयुग्मं पारावतशावकद्वयं वा तयो र्दातव्यमासीत्।
\vakya पश्य च यिरूशालेमे शिमियोननामको नर एकोऽविद्यत, स धार्मिकः श्रद्धाशील इस्रायेलस्य सान्त्वनां प्रतीक्षमाणः पवित्रेणात्मनाधिष्ठितश्चासीत्।
\vakya पवित्रेणात्मना तस्मै चेदं प्रत्यश्रावि यत् प्रभोरभिषिक्तं नररत्नं (ख्रीष्टं) न दृष्ट्वा स मृत्युं नास्वादिष्यति।
\vakya पवित्रस्यात्मन आवेशात् स धर्मधाम आगतवान्। यदा च शिशो र्यीशोः पितरौ व्यवस्थानिरूपितरीतिमाचरितुं तमभ्यन्तरमानयतां,
\vakya तदा सोऽपि तं क्रोडेऽग्रहीत्, ईश्वरस्य धन्यवादं कुर्वंश्चावादीत्,
\begin{poem}
\startwithvakya इदानीं भवता स्वीयः किङ्करोऽयं विसृज्यते।
\pline भो मत्स्वामिन् सकल्याणं भवद्वाक्यानुसारतः॥
\vakya मन्नेत्राभ्यां यतोऽदर्शि त्राणोपायस्त्वया कृतः।
\pline समक्षं सर्वजातीनां भवतैवोपकल्पितः॥
\vakya दीपः स परजातीनां तासां प्रद्योतनार्थकः।
\vakya त्वज्जनस्तु य इस्रायेल् तस्य भास्कर एव सः॥
\end{poem}
\vakya शिशुमधि यद्यद् अकथ्यत तत्र योषेफस्तस्य माता चाश्चर्यममन्येतां।
\vakya शिमियोनश्च ताभ्यामाशिषं ददौ तस्य मातरं मरियमं जगाद च, पश्येस्रायेलस्य मध्ये बहूनां पतनोत्थानयो र्हेतु र्विसंवादान्वितमभिज्ञानञ्च भवितुमयं नियुक्तः।
\vakya तव निजप्राणा अप्यसिना व्यत्स्यन्ते। इत्थं बहुहृदयेभ्य उद्गतैस्तर्कैः प्रकटीभवितव्यमिति।
\vakya तदानीमाशेरवंशोद्भवा पनूयेलसुता हान्नेतिनामिका भाववादिन्यासीत्, सातीव वयोवृद्धा, कौमार्यात् परं सप्तवत्सरान् भर्त्रा सार्धमुषित्वा प्रायेण चतुरशीतिवर्षीया विधवा जाता।
\vakya सा धर्मधामतो न निरगच्छत् नक्तं दिवा चोपवासप्रार्थनाभिरुपासनामकुरुत।
\vakya सापि तस्मिन् दण्ड उपस्थाय प्रभो र्माहात्म्यं स्वीकुर्वाणा यिरूशालेमे ये मोचनं प्रत्यैक्षन्त तेभ्यः सर्वेभ्यस्तस्य कथामकथयत्।
\vakya प्रभो र्व्यवस्थयादिष्टानि सर्वकर्माणि सम्पाद्य तौ स्वनगरं गालीलस्थं नासरतं प्रतिजग्मतुः।
\stitle{बालकयीशोर्यिरूशालेमयात्रा।}
\vakya स शिशुश्चावर्धतात्मना बलवान् ज्ञानपूर्णश्चाभवत्, ईश्वरस्यानुग्रहेणाधिष्ठितश्चासीत्।
\vakya अथ तस्य पितरौ प्रतिवर्षं निस्तारपर्वणि यिरूशालेममगच्छताम्।
\vakya अतस्तस्मिन् द्वादशवर्षवयस्के सति पार्वणरीत्यनुसारेण तौ यिरूशालेमं गत्वा पार्वणदिनानि यावयित्वा च यदा प्रत्यावर्तेतां तदा बालको यीशु र्यिरूशालेमेऽतिष्ठत्।
\vakya योषेफस्तु तस्य माता च तन्नाजानीताम्,
\vakya अपि तु स कश्चित् सहयात्रिकाणां मध्येऽस्तीत्यनुमाय तौ दिनैकगम्यं मार्गं जग्मतुः, ततः परं ज्ञातिजनेषु बन्धुषु च तम् अमृगयेताम्।
\vakya अनासाद्य तु तमन्विष्यन्तौ यिरूशालेमं प्रत्याजग्मतुः।
\vakya दिनत्रयात् परञ्च धर्मधाम्नि तमासादयामासतुः। स गुरूणां मध्य आसीनस्तेषां वचांस्यशृणोत् तांश्च प्रश्नानपृच्छत्।
\vakya यावन्तश्च तस्य वाक्यान्यशृण्वन् ते सर्वे तस्य बुद्ध्यां प्रत्युत्तरेषु चाश्चर्यममन्यन्त।
\vakya तं दृष्ट्वा तौ विस्मयापन्नौ। तस्य माता च तं जगाद, वत्स, आवां प्रति त्वयेदं किमकारि? पश्य तव पिताहञ्च शोचन्तौ त्वाम् अन्वैष्याव।
\vakya स तु ताववादीत् किमर्थं मामन्वैष्यतम्? किं नाजानीतं यन्मत्पितु र्विषयेष्ववस्थानं मयि युज्यते?
\vakya इदं यद् वचनं स ताववादीत्, तत् ताभ्यां नाबोधि।
\vakya ततः परं स ताभ्यां सार्धं यात्रां कृत्वा नासरतमाजगाम तयो र्वशवर्ती तस्थौ च। तस्य माता चैताः सर्वकथाश्चित्तेऽरक्षत्।
\vakya यीशुस्तु ज्ञाने वयसीश्वरस्य मनुष्याणाञ्चानुग्रहे च वर्धिष्णुरासीत्\eoc