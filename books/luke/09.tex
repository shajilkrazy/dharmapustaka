\adhyAya
\stitle{यीशुना घोषयितुं द्वादशशिष्याः प्रेषिताः।}
\vakya अथ स शिष्यान् द्वादश निजानेकत्राहूय तेभ्यः सर्वभूतानां (दमनार्थं) प्रभावं कर्तृत्वञ्च रोगप्रतीकारस्य सामर्थ्यञ्च ददौ,
\vakya तांश्चेश्वरस्य राज्यं घोषयितुमस्वस्थान् निरामयान् कर्तुञ्च प्रेषयमास।
\vakya तान् जगाद च, गमनार्थं यष्टिं चेलसम्पुटकं वा पूपं वा रूप्यं वा किमपि मादद्ध्वं, युष्माकञ्च कस्याप्यङ्गरक्षे द्वे मास्तां।
\vakya यद् गृहं प्रविवेश च तत्र तिष्ठत, तस्माच्च प्रतिष्ठध्वं।
\vakya यस्य नगरस्य जनास्तु युष्मान् न गृह्णन्ति, तस्मान्नगरान्निर्गच्छन्तः स्वचरणेभ्यो धूलिमप्यवधूनुत तेषां विरुद्धं साक्ष्यार्थम्।
\vakya ततस्ते प्रस्थाय ग्रामं ग्रामं पर्यटन्तः सर्वत्र सुसंवादमघोषयन् व्याधिप्रतीकारमकुर्वंश्च।
\vakya तेन तु यद्यदक्रियत तत्कथां चतुर्थांशपति र्हेरोदः शुश्रावास्थिरो बभूव च, यतः कैश्चिदूचे, योहनो मृतेभ्य उत्थापित इति,
\vakya अपरैः कैश्चित्, एलियो दर्शनं ददाविति, अपरैः पुनः, प्राचीनानां भाववादिनामेक उत्तस्थाविति।
\vakya हेरोदस्तदाब्रवीत्, योहनस्य शिरश्छेदनं कृतं मया, कः पुनरसौ यमधीदृशीः कथाः शृणोमि? स च तं द्रष्टुमयतत।
\stitle{यीशुना पञ्चसहस्रलोकेभ्यो भोजनदानम्।}
\vakya ततः परं प्रेरिताः प्रत्यावृत्य यद्यत् तैः कृतं तस्मै तन्निवेदयमासुः। स तदा तान् सङ्गिनः कृत्वा गुप्तं बैत्सैदाख्यनगरसमीपस्थं किञ्चन निर्जनं स्थानं जगाम।
\vakya जननिवहास्तु तदनुभूय तमनुजग्मुः, स च तान् गृहीत्वेश्वरस्य राज्यमधि तैः संललाप रोगप्रतीकारार्थिनश्च निरामयांश्चकार।
\vakya दिनावसाने त्वागच्छति ते द्वादश शिष्या उपागत्य तमूचुः, विसृज्यन्तां भवता जननिवहा यथामी परितःस्थाः पल्ली र्ग्रामांश्च गत्वा रात्रिं यापयेयुः खाद्यान्याप्नुयुश्च, वयं ह्यत्र निर्जने स्थाने विद्यामहे।
\vakya स तानब्रवीत्, यूयमेव तेभ्यो भक्ष्याणि दत्त। तेऽवादिषुः, पूपान् पञ्च मीनौ च द्वावपहायास्माकं किमपि नास्ति, तदेतेषां सर्वजनानां कृते खाद्यानि क्रेतुमस्माभि र्गन्तव्यं किं?
\vakya यतः प्रायेण पुरुषाः पञ्चसहस्रााण्यासन्। स तदा स्वशिष्यान् जगाद, यूयं पञ्चाशतं पञ्चाशतं, जनानेकत्र कृत्वा सर्वान् पङ्क्तिभिरुपवेशयत।
\vakya ते तदेव कृत्वा सर्वान् उपवेशयामासुः।
\vakya स तदा तान् पूपान् पञ्च मीनौ च द्वावादाय स्वर्गं प्रत्युच्चदृष्टिं कृत्वा तेभ्य आशिषं ददौ भङ्क्त्वा च जनानां कृते परिवेषणार्थं स्वशिष्येभ्यस्तानददात्।
\vakya भुक्त्वा च सर्वे ततृपुः। भग्नानामंशानां शेषेण पूर्णा द्वादश डल्लकाश्च जनैराददिरे।
\stitle{यीशोः स्वमरणपुनरुत्थानमधि च कथाकथनम्।}
\vakya एकदा स यदा निभृतं प्रार्थयत तदा तस्य शिष्यास्तेन सार्धमासन्। स च तान् पप्रच्छ, जना मां कं वदन्ति?
\vakya ते प्रत्यूचुः, स्नापकं योहनमिति। केचित्तु वदन्ति, भवान् एलिय इति। अन्ये पुनः, प्राचीनानां भाववादिनामेक उत्तस्थाविति। स तदा तान् पप्रच्छ, यूयं तु मां कं वदथ?
\vakya पित्रः प्रतिभाषमाणोऽब्रवीत्, ईश्वरस्य ख्रीष्टं।
\vakya ततः स तांस्तर्जयन् जगाद, यूयमिदं कमपि मा ज्ञापयत।
\vakya स पुनरब्रवीत्, मनुष्यपुत्रस्यावश्यम्भावि यत् स बहुदुःखं भोक्ष्यते प्राचीनै र्मुख्ययाजकैः शास्त्राध्यापकैश्च निराकारिष्यते घातयिष्यते च, तृतीये दिवसे तु पुनरुत्थापयिष्यते।
\vakya स पुनः सर्वानुवाच, कश्चिच्चेन्मामनुगन्तुमिच्छति स तर्ह्यात्मानं प्रत्याख्यातु प्रत्यहं स्वक्रुशमादाय मामनुव्रचतु च।
\vakya यतो यः कश्चित् स्वप्राणान् रिरक्षिषति स तान् हारयिष्यति, यस्तु मदर्थं स्वप्राणान् हारयति स एव तान् रक्षिष्यति,
\vakya यतो मनुष्यः कृत्स्नं जगल्लब्ध्वा यद्यात्मानं हारयति वात्महानिं कुरुते, तर्हि तस्य हितं वा किं भवति?
\vakya यतो यः कश्चिन्मत्तो मदीयवाक्येभ्यश्चापत्रपते, मनुष्यपुत्रो यदा स्वप्रतापेन स्वपितुः पवित्रदूतानाञ्च प्रतापेन चागमिष्यति तदा सोऽपि तस्मादपत्रपिष्यते।
\vakya युष्मांस्त्वहं सत्यं ब्रवीमि, अत्र ये तिष्ठन्ति तेषां केचिद् यावदीश्वरस्य राज्यं न द्रक्ष्यन्ति तावन्मृत्योरास्वादं न लप्स्यन्ते।
\stitle{यीशोरुज्ज्वलमूर्तिधारणम्।}
\vakya एताभ्यः कथाभ्यः परं व्यतीतेषु प्रायेणाष्टसु दिनेषु स पित्रं याकोबं योहनञ्च सङ्गिनः कृत्वा प्रार्थयितुं पर्वतमारुरोह।
\vakya प्रार्थयमाने च तस्मिंस्तदीयवदनस्याकृतिरन्यरूपा परिच्छदश्च सित उज्ज्वलश्च बभूव।
\vakya पश्य च द्वौ नरौ तेन समलपतां, तौ मोशिरेलियश्च।
\vakya इमौ सप्रतापं दर्शनं दत्त्वा यिरूशालेमे साधयितव्यां तस्य शेषगतिमकथयतां।
\vakya पित्रस्तत्सङ्गिनौ च निद्राभारेणाक्रान्ता आसन्। तथापि कथञ्चिज्जागृत्वा तस्य प्रतापं तेन सार्धं तिष्ठन्तौ तौ नरौ च ददृशुः।
\vakya तयोस्तु तस्मात् प्रस्थानकाले पित्रे यीशुं जगाद, नाथ भद्रमस्माकमत्रावस्थानम्, अतोऽस्माभिरुटजानि त्रीणि निर्मीयन्ताम्, एकं भवदर्थम्, एकं मोश्यर्थम्, एकमेलियार्थम्। स तु यदभाषत तन्नाबुध्यत।
\vakya तस्मिंस्त्वित्थं ब्रुवाणे मेघ एक आगत्य तेषामुपरि छायां ददौ।
\vakya तस्मिन् मेघे तयोः प्रवेशकाले ते बिभ्युः। मेघस्य मध्याच्च वाणीयमुदभूत्, अयं मम प्रियः पुत्रः, अस्य वचांसि युष्माभिः श्रूयन्तामिति।
\vakya सम्भूतायामेवास्यां वाण्यां यीशुरेकाक्यविद्यत। ते च मौनमवललम्बिरे यच्च दृष्टवन्तस्तदधि तेषु दिनेषु कमपि किमपि न जगदुः।
\stitle{यीशुना बालकस्यारोग्यकरणम्।}
\vakya परदिने तेषु गिरितोऽवरुढेषु महान् जननिवहस्तत्सम्मुखमुपतस्थे।
\vakya पश्य च जनताया नर एक उच्चैःस्वरेणोवाच, गुरो, मम पुत्रं प्रति भवान् दृक्‌पातं कर्तुमर्हति, यतः स ममैकजातः।
\vakya पश्यतु च कश्चिदात्मा तं धरति सहसा चोत्क्रोशयति फेनमुद्गमयंस्तदीयाङ्गानि कर्षति तञ्च चूर्णयन्निव कष्टेन तस्मादपसरति।
\vakya भवतः शिष्याश्च मया तस्य निःसारणं याचिताः, ते तु नाशक्नुवन्।
\vakya यीशुस्तदा प्रतिभाषमाणोऽब्रवीत्, रे अविश्वासिन्नुन्मार्गगामिंश्च वंश, कियन्तं कालं स्थातव्यं मया युष्माभिः सार्धं यूयञ्च सोढव्या मया?
\vakya तव पुत्रमत्रानय स तु यदागच्छत्, तदा स भूतस्तमाक्रम्याकर्षत्। यीशुस्तदा तमशुचिमात्मानं भर्त्सयामास बालकञ्च स्वस्थं कृत्वा तस्य पितरि समर्पयामास।
\vakya तदेश्वरस्य माहात्म्यतः सर्वे विस्मयापन्नाः।
\vakya यीशुना कृतेषु सर्वकर्मसु यदा सर्व आश्चर्यममन्यन्त, तदा स स्वशिष्यानब्रवीत्, सर्वा इमाः कथा युष्मत्कर्णै र्गृह्यन्तां, यतो मनुष्यपुत्रो मनुष्याणां हस्तेषु समर्पयिष्यत इति।
\vakya तद् वाक्यन्तु तै र्न बुबुधे यथा च तै र्नानुभूयेन तथा तेभ्यः प्रच्छन्नमासीत्। तद् वाक्यमधि तं कमपि प्रष्टुं तेऽबिभयुश्च।
\vakya अथ तर्क एषस्तान् प्रविवेश, तेषां कः श्रेष्ठ इति।
\vakya यीशुस्तु तेषां हृद्गतं तर्कं दृष्ट्वा बालकमेकं गृहीत्वा स्वपार्श्वे स्थापयित्वा च तान् उवाच, यः कश्चिन्मम नाम्नेमं बालकं गृह्णाति स मां गृह्णाति, माञ्च यो गृह्णाति स मत्प्रेषयितारं गृह्णाति।
\vakya यतो युष्मासु यः सर्वेषां क्षोदिष्ठः स महान् भविष्यति।
\vakya ततो योहनः प्रतिभाषमाणोऽब्रवीत्, नाथ, भवतो नाम्ना यो भूतान् निःसारयति तादृशः कश्चिदस्माभि र्लक्षितः, स त्वस्माभिः सार्धं नानुगच्छति तन्निमित्तं सोऽस्माभि र्निवारितः।
\vakya यीशुस्तमुवाच, तं मा निवारयत, यस्माद् यो न युष्मद्विपक्षः स युष्मत्स्वपक्षः।
\stitle{यीशो र्यिरूशालेमे शेषगमनम्।}
\vakya अथ स यदा स्वर्गमारोहयितव्यस्तस्मिन् काले पूर्यमाणे स स्थिरचेता यिरूशालेमं गन्तुमुन्मुखो बभूव।
\vakya ततः स स्वस्याग्रतो दूतान् प्राहिणोत्। ते च प्रस्थाय तदर्थं प्रयोज्यद्रव्याणि सज्जीकर्तुमिच्छन्तः शमरीयाणां कञ्चिद् ग्रामं प्रविविशुः।
\vakya स तु यिरूशालेमं गन्तुमुन्मुखस्ततो हेतो र्जनास्तं न जगृहुः।
\vakya तद् दृष्ट्वा याकोबो योहनश्चेतिनामानौ तस्य शिष्यौ पप्रच्छतुः, प्रभो, एलियेनापि यथाकारि तथैवावां किं वह्निमादेक्ष्यावो यत् स्वर्गादवरुह्य तान् संहरेत्? इदं किं भवतोऽभिरुचितम्?
\vakya स तु मुखं प्रत्यावर्त्य तौ भर्त्सयन् जगाद, युवां कीदृशस्यात्मनस्तन्न जानीथः। मनुष्यपुत्रो हि न मनुष्याणां प्राणान् नाशयितुमागतोऽपि तु तारयितुम्। ततस्तेऽपरं ग्रामं जग्मुः।
\vakya अथ तेषां गमनकाले कश्चित् पथि तं जगाद, प्रभो यत्र कुत्रचिद् भवता गन्तव्यं तत्राहमपि भवन्तमनुगमिष्यामि।
\vakya यीशुस्तं जगाद, सन्ति गर्तानि शृगालानां नीडाश्च विहायसो विहङ्गमानां, न स्थानं मनुष्यपुत्रस्य तु शिरः शाययितुं।
\vakya अपरमेकं नरं सोऽब्रवीत् मामनुगच्छ। स तु प्रतिजगाद, प्रभो, प्रथमं गत्वा स्वपितुः सत्कारं कर्तुमनुमन्यताम्।
\vakya यीशुस्तु तं जगाद, मृतानेव मृतानां स्वकीयानां सत्कारायानुजानीहि, त्वन्तु गत्वेश्वरराज्यस्य संवादं प्रचारय।
\vakya अपरः कश्चिदुवाच, प्रभो, अहं भवन्तमनुयास्यामि, प्रथमन्तु मदीयगृहे ये विद्यन्ते तानाप्रष्टुं मामनुजानातु।
\vakya यीशुस्तु तमब्रवीत्, लाङ्गले हस्तमर्पयित्वा येन पश्चाद्दिशि दृक्‌पातः क्रियते तादृशः कोऽपीश्वरराज्यार्थमुपयोगी नास्ति\eoc