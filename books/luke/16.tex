\adhyAya
\stitle{अपव्ययिगृहकार्याधीशदृष्टान्तः।}
\vakya अथ स स्वशिष्येभ्योऽपि कथयामास, नरस्य कस्यचिद् धनिनो धनाध्यक्ष आसीत्, स तस्य वित्तानि प्रकिरतीति तत्समीपं पर्युदितः।
\vakya ततः स तमाहूय जगाद, किमेतद् यच्छ्रूयते मया त्वामधि? त्वं स्वधनाध्यक्षत्वस्य गणनां देहि, यत इतःप्रभृति त्वं धनाध्यक्षः स्थातुं न शक्ष्यसि।
\vakya स धनाध्यक्षस्तदा स्वान्तरेऽचिन्तयत्, किं करवाणि? यतः मम स्वामी धनाध्यक्षत्वपदं मत्तोऽपहरति। खनितुं मम बलं नास्ति, भिक्षितुं लज्जे।
\vakya किं कर्तव्यं तन्मया बुद्धं, जना यथाध्यक्षत्वाच्च्यावितं मां स्वगृहेषु गृह्णीयुः।
\vakya ततः स स्वप्रभोरेकैकमधमर्णमाहूय प्रथमं पप्रच्छ, त्वं मम स्वामिनः कति धारयसि?
\vakya स जगाद, तैलस्य शतं कुतूः। ततः स तमादिशत्, तव पत्रमादायोपविश्य च तूर्णं पञ्चाशतं लिख।
\vakya ततः परं सोऽन्यमेकं पप्रच्छ, त्वञ्च कति धारयसि? सोऽब्रवीत्, गोधूमानां शतं द्रोणान्। स तमादिशत्, तव पत्रमादायाशीतिं लिख।
\vakya अनन्तरं स्वामी तमयाथार्थिकं धानाध्यक्षं सुबुद्ध्याचरितं मत्वा प्रशशंस, यतो दीप्तेः सन्तानेभ्य एतद्युगस्य सन्तानाः स्वजातिमुद्दिश्य बुद्धिमत्तराः।
\vakya अहमपि युष्मान् ब्रवीमि, यूयमयथार्थेन धनेन मित्रलाभं साधयत, तथा कृते यदा हीनस्वा भविष्यथ, तदा मित्रैरनन्तेष्वावासेषु ग्राहीष्यध्वे।
\vakya यो लघिष्ठे विश्वस्तः स प्रभूतेऽपि विश्वस्तः। यस्तु लघिष्ठेऽयाथार्थिकः स प्रभूतेऽप्ययाथार्थिकः।
\vakya अतो यूयमयथार्थो धने चेदविश्वस्ता जाताः कस्तर्हि विश्वस्य युष्मासु परमार्थं निक्षेप्स्यति?
\vakya परकीये च चेन्न विश्वस्ता जाताः कस्तर्हि युष्मदीयं युष्मभ्यं दास्यति? द्वयो स्वामिनो र्दास्यं कर्तुं केनापि दासेन न शक्यं।
\vakya यतः स एकतरं द्विषन्नन्यतरस्मिन् प्रेष्यते, न चेदेकतरस्मिन्नासज्जमानोऽन्यतरमवमंस्यते। ईश्वरस्य धनस्य चोभयो र्दास्यं कर्तुं युष्माभि र्न शक्यते।
\vakya धनलोभिनः फरीशिनोऽपि सर्वाण्येतान्यशृण्वन् तमुपाहसंश्च, स तु तानवादीत्, यूयमेव मनुष्माणां समक्षं स्वान् धार्मिकीकुरुथ, ईश्वरस्तु युष्माकं चित्तानि जानाति।
\vakya यतो मनुष्येषु यदुच्चमीश्वरस्य समक्षं तज्जघन्यं।
\vakya व्यवस्था भाववादिनश्च योहनपर्यन्ताः। तं कालमारभ्येश्वरराज्यस्य सुसंवादो घोष्यते, नराश्च प्रत्येकं व्यग्रत्वेन तत् प्रविशन्ति।
\vakya व्यवस्थाया एकस्य बिन्दोरपायाद् द्यावापृथिव्योरपायः सुसाध्यः।
\vakya यः कश्चित् स्वभार्यां त्यक्त्वान्यामुद्वहति स व्यभिचारं करोति, यश्च कश्चित् स्वामित्यक्तां स्त्रियमुद्वहति स व्यभिचारं करोति।
\stitle{धनिदरिद्रयो र्दृष्टान्तः।}
\vakya धनाढ्यः कश्चिन्नर आसीत्, स कृष्णलोहितानि सूक्ष्माणि च वसनानि पर्यधत्त प्रत्यहञ्च सप्रतापं सुखभोगमसेवत।
\vakya अपि च लासाराभिध एको दरिद्र आसीत्, स तस्य गोपुरान्तिकं शयानो व्रणयुक्तश्च
\vakya तस्य धनिनो भोज्यमञ्चाद् भ्रंशन्तीभि र्भक्ष्यफेलीभिस्त्रप्तुमैच्छत्। प्रत्युत श्वानोऽप्यागत्य तस्य व्रणानि पर्यलिहन्।
\vakya ततः परं स दरिद्रो ममार स्वर्गदूतैश्चोह्यमानोऽब्राहामस्य क्रोडं निन्ये। स धनवानपि ममार शवागारे निदधे च।
\vakya पाताले तूर्ध्वं निरीक्ष्य यन्त्रणास्थाने विद्यमानो दूरादब्राहामं तत्क्रोडस्थं लासारञ्च ददर्श।
\vakya ततः स प्रोच्चैराह, भो पितरब्राहाम, मामनुकम्पतां लासारञ्च प्रहिणोतु स यथाङ्गुल्या अग्रभागं तोये मज्जयित्वा मम जिह्वां शीतलीकुर्यात्, यतोऽत्राग्निशिखायां यातनां सहे।
\vakya अब्राहामस्त्वब्रवीत्, वत्स, स्मर यत् त्वं स्वजीवने तव सुखमभजः, लासारश्च तथैव दुःखमभत्। इदानीन्तु तस्यात्र सान्त्वना तव च यातना जायते।
\vakya सर्वमेतदपहाय त्वस्माकं युष्माकञ्चान्तरालं बृहच्छून्यस्थलं तथा दृढीकृतं, यथास्मात् स्थानाद् युष्मदन्तिकं जिगमिषवस्तरितुं न शक्नुयुः, नामुष्मात् स्थानादागन्तुकामा वास्मदन्तिकं तरेयुः।
\vakya स तदा जगाद, भवन्तं प्रार्थये, पितः, भवांस्तं मत्पितृगृहं प्रहिणोतु,
\vakya यतो मम पञ्च भ्रातरः सन्ति, तेऽपि यथेदं यातनास्थानं नागच्छेयुस्तथा स तेभ्यो दृढं साक्ष्यं ददातु।
\vakya अब्राहामस्तमाह, मोशि र्भाववादिनश्च तेषां (शिक्षकाः) सन्ति, तेषां वचांसि तैः श्रूयन्तां।
\vakya स उवाच, न तथा, पितरब्राहाम। प्रत्युत मृतानां समीपतो यदि कश्चित् तेषां समीपं गच्छेत्, तर्हि ते मनांसि परावर्तयिष्यन्ति।
\vakya स तु तमवादीत्, ते यदि मोशे र्भाववादिनाञ्च वचांसि न शृण्वन्ति, तर्हि मृतानां मध्यादुत्थितेऽपि कस्मिंश्चिन्न प्रत्येष्यन्ति\eoc