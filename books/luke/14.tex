\adhyAya
\stitle{भोजनसमये दत्तोपदेशः।}
\vakya अथ कस्मिंश्चिद् विश्रामवारे तेन भोजनार्थं कस्यचिन्मुख्यफरीशिनो गृहं प्रविष्टे ते तमवेक्ष्यातिष्ठन्।
\vakya पश्य च नर एक उदकोदरि तस्य सम्मुखमविद्यत।
\vakya यीशुस्तदा प्रतिभाषमाणो व्यवस्थावेत्तॄन् फरीशिनश्चाह, विश्रामवारे रोगप्रतीकारः किं विधेयः?
\vakya अनेन ते मौनीभूताः। स तु तं नरं स्पृशन् निरामयं कृत्वा विससर्ज, ततः परं प्रतिभाषमाणस्तानवादीत्, युष्माकं मध्ये कः कूपे पतितं स्वगर्दभं स्वगां वा तत्क्षणं विश्रामवारे नोद्धरति?
\vakya अस्य वचनस्य प्रत्युत्तरं कर्तुं तै र्नाशक्यत।
\vakya ततः परं स निमन्त्रितेभ्यो जनेभ्यो दृष्टान्तकथामकथयत्, यतस्ते कथं श्रेष्ठस्थानानि वरयन्ति तत् तेनालक्ष्यत।
\vakya स तानुवाच, केनचिद् विवाहोत्सवार्थं तव निमन्त्रणे कृते मैव श्रेष्ठस्थान उपविश, किंस्वित् त्वत्तो बहुमतः कश्चित् तेन निमन्त्रितः,
\vakya तथा सति तव तस्य च निमन्त्रयितोपागत्य त्वां वक्ष्यति, अमुष्मै स्थानं देहीति, त्वञ्च तदा लज्जितोऽन्त्यस्थानमाश्रयितुं प्रवर्तिष्यसे।
\vakya प्रत्युत कृते तव निमन्त्रणे गत्वान्त्यस्थानं उपविश, यथा त्वन्निमन्त्रयितागमनकाले त्वां वदेत् बन्धो त्वमुच्चतरं स्थानं गन्तुमर्हसि। तदा हि सहभोजिनां समक्षं तव गौरवं जनिष्यते।
\vakya यतो यः कश्चिदात्मानमुच्चीकरोति स नीचीकारिष्यते, यश्चात्मानं नीचीकरोति स उच्चीकारिष्यते।
\vakya अथ यस्तं निमन्त्रितवान्, तस्मै स कथयामास, त्वं यदा पूर्वाह्णिकमपराह्णिकं वा भोज्यं करोषि, तदा स्वबन्धून् स्वभ्रातॄन् वा स्वकुटुम्बिनो वा स्वसमीपवासिनो धनिनो वा मा निमन्त्रय, किस्वित् तैरपि तव निमन्त्रणे कृते तव प्रतिदानं भविष्यति।
\vakya प्रत्युत यदा महाभोज्यं करोषि, तदा दरिद्रान्, हीनाङ्गान्, खञ्जान्, अन्धान् निमन्त्रय,
\vakya तर्हि धन्यो भविष्यसि, यतस्तुभ्यं प्रतिदानस्योपायस्तेषां नास्ति, धार्मिकाणां पुनरुत्थाने तु तुभ्यं प्रतिदानं विधायिष्यते।
\vakya कथा एताः श्रुत्वा सहभोजिनामेकस्तमब्रवीत्, धन्यः स य ईश्वरस्य राज्ये भक्ष्यभागी भविष्यति।
\vakya स तु तमुवाच, कश्चिन्मनुष्योऽपराह्णिकं महाभोज्यं चक्रे बहून् निमन्त्रयामास च।
\vakya भोज्यस्य दण्डे च स्वदासं प्रहित्य तानवादीत्, आयात, यतः सर्वमेवेदानीं सज्जीभूतं।
\vakya ते सर्वे त्वेकमत्यानुज्ञां याचितुं प्रववृतिरे। प्रथमस्तमुवाच, क्षेत्रं क्रीतं मया तद्दर्शनार्थमवश्यं बहिर्गन्तव्यं। त्वां प्रार्थये, मामनुज्ञातं मन्यस्व।
\vakya द्वितीय आह, गोयुग्मानि पञ्च क्रीतानि मया, तानि परीक्षितुं गच्छामि, त्वां प्रार्थये, मामनुज्ञातं मन्यस्व।
\vakya अपर एको जगाद, दारा उदूढा मया, ततो गन्तुं न शक्यते।
\vakya ततः परं स दास उपस्थाय स्वप्रभवे तत् सर्वं निवेदयामास। गृहस्वामी तदा क्रुद्ध्वा स्वदासमाह, सत्वरं नगरस्य चत्वराणि रथ्याश्च गत्वा दरिद्रान्, हीनाङ्गान्, खञ्जान्, अन्धांश्चात्र समानय।
\vakya ततः परं तेन दासेनोक्तं, प्रभो, भवता यथादिष्टं तथैवाकारि, स्थानन्त्विदानीमपि शिष्यते।
\vakya प्रभुस्तदा दासमाह, त्वं बहीराजमार्गान् तरुतलानि च गत्वा तथाग्रहं विधाय जनान् समानय यथा मम गृहं परिपूर्येत।
\vakya यतोऽहं युष्मान् ब्रवीमि, नराणां तेषां निमन्त्रितानां कोऽपि मम भोज्यं नास्वादिष्यति।
\vakya अथ महत्सु जननिवहेषु तेन सार्धं व्रजत्सु स प्रत्यावृत्य तान् जगाद,
\vakya यो मदन्तिकमायाति स यदि स्वपितरं स्वमातरञ्च प्रति भार्यां सन्तानान् भ्रातॄन् भगिनीश्च प्रत्यपि च स्वप्राणान् प्रति न विरज्यते तर्हि मम शिष्यो भवितुं न शक्नोति।
\vakya यः कश्चिच्च स्वक्रुशं वहन् मां नानुगच्छति स मम शिष्यो भवितुं न शक्नोति।
\vakya यत उच्चगृहं निर्मातुमिच्छु र्युष्माकं को न प्रथममुपविश्य व्ययं गणयन् कार्यसिद्धेरपायस्तस्यास्ति न वेति विचारयति?
\vakya नोचेत् स्थापिते गृहमूले यदि समाप्तिस्तस्याशक्या जायते, तर्हि ये तद् द्रक्षन्ति ते सर्वे तमुपहसितुं प्रवृत्ता वदिष्यन्ति,
\vakya नरोऽसौ गृहनिर्माणमारेभे समापयितुन्तु न शशाकेति।
\vakya अथवापरेण राज्ञा सार्धं सम्पातार्थं युद्धयात्रां चिकीर्षुः को राजा न प्रथममुविश्य मन्त्रयते, विंशतिसहस्रै र्वृतो यो मद्विरुद्धमायाति, तं किं दशसहस्रै र्वृतोऽहं प्रतिरोद्धुं शक्ष्यामीति।
\vakya यद्यशक्यं प्रतिभाति, तर्हि यावदन्यतरो दूरे स्थितस्तावत् स दौत्यं प्रहित्य सन्धेः सम्भवं प्रार्थयिष्यते।
\vakya तथैव युष्माकं यः कश्चित् सर्वस्वाय जलाञ्जलिं न ददाति स मम शिष्यो भवितुं न शक्नोति।
\vakya लवणं हि भद्रं, लवणन्तु यदि विस्वादं जायते, तर्हि तत् केनोपायेन स्वादयुक्तं कारिष्यते?
\vakya ततःप्रभृति तन्न मृत्तिकायां नालवालराशौ वा युज्यते, बहिरेव निक्षिप्यते। शृणोतु यस्य श्रोतुं श्रोत्रे स्तः\eoc