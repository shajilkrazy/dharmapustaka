\adhyAya
\stitle{सक्केयस्य मनःपरिवर्तनम्।}
\vakya ततः परं स यरीहुं प्रविश्य तन्मध्येनागच्छत्। पश्य च सक्केयनामा कश्चिन्नरस्तत्रासीत्।
\vakya स मुख्यः शुल्कादायी धनी चासीत्।
\vakya को यीशुस्तन्निश्चेतुमिच्छन् स तं द्रष्टुम् अयतत जनताकारणात्तु नाशक्नोत्, यतः स ह्रस्वकाय आसीत्।
\vakya ततः सोऽग्रे धावित्वा तद्दिदृक्षयोडुम्बरवृक्षमारुरोह, यतस्तत्रैव तेन गन्तव्यं।
\vakya ततः परं यीशुस्तत् स्थानं प्राप्योर्ध्वदृष्टिं कृत्वा तं निरीक्ष्य जगाद, सक्केय, शीघ्रमवतर, यतस्तव गृहे मयाद्य वस्तव्यं।
\vakya स तदा त्वरयावरुह्य सानन्दं तं जग्राह।
\vakya तद् दृष्ट्वा सर्वेऽन्तुष्य प्राहुः, स रात्रियापनार्थं पापिष्ठनरस्य गृहं प्रविष्टः।
\vakya सक्केयस्तु स्थित्वा प्रभुमब्रवीत्, पश्यतु प्रभो, मदीयसर्वस्वस्यार्धमहं दरिद्रेभ्यो वितरामि, यदि च मृषादोषमारोप्य कस्यचित् किमपि मयापहृतं, तर्हि तच्चतुर्गुणं प्रत्यर्पयामि।
\vakya यीशुस्तदा तमाह, अद्य गृहेऽस्मिन् परित्राणं ववृते, यतोऽयमप्यब्राहामस्य सन्तानः।
\vakya मनुष्यपुत्रो हि हारितमन्वेषितुं तारयितुञ्चागतः।
\stitle{दशस्वर्णमुद्राया दृष्टान्तः।}
\vakya तेष्वेतच्छृण्वत्सु स दृष्टान्तकथामपि कथयामास, यतः स यिरूशालेमस्य सन्निधावासीत् तैश्चान्वमीयत यदीश्वरस्य राज्यं सहसा प्रकाशिष्यते।
\vakya अतः स व्याजहार, नरः कश्चित् कुलीनो दूरदेशं प्रतस्थे, स स्वार्थं राजत्वं ग्रहीतुं प्रत्यायातुञ्चैच्छत्।
\vakya स स्वदासान् दश समाहूय दश स्वर्णमुद्रास्तेषु समर्प्य च तान् जगाद, ममागमनं यावद् व्यापारं कुरुतेति।
\vakya तदीयपुरवासिनस्तु तमद्विषुस्तत्पश्चाच्च दौत्यं प्रहित्य निवेदयामासुः,
\vakya नरोऽसौ यदस्माकं राजा भवेत् तन्नाभिमतमस्माभिः। स तु यदा राजत्वं लब्ध्वा पुनरागमत् तदा केन व्यापारं कृत्वा किमुपार्जितं तज्जिज्ञासुः स तेषां दासानामाह्वानमादिदेश येष्वर्थमर्पितवान्।
\vakya ततः प्रथम उपस्थाय जगाद, प्रभो, भवतो मुद्रयापरा दश मुद्रा उपार्जिताः।
\vakya स तमाह, साधु, भद्र दास, त्वं क्षोधिष्ठे विश्वस्त आसीस्ततो दशानां नगराणामाधिपत्यविशिष्टो भव।
\vakya अनन्तरं द्वितीय आगत्य जगाद, प्रभो, भवतो मुद्रया पञ्च मुद्रा उपार्जिताः।
\vakya स तमप्याह, त्वं पञ्चत्सु नगरेष्वधिकृतो भव।
\vakya ततः परमन्य एक आगत्याब्रवीत्, प्रभो, पश्यत्वियं भवतो मुद्रा। स्वेदहरवस्त्रे निधाय सा मया रक्षिता भवतो भयात्।
\vakya यतो भवान् कठोरो नरः, यन्न निहितवांस्तदादत्ते, बीजान्यनुप्त्वा च शस्यं कृन्तति।
\vakya स तु तमवादीत्, तव मुखाद् यन्निःसृतं, रे दुष्ट दास, तेनैव तव विचारं करिष्यामि। त्वयेदं भो अज्ञायि यदहं कठोरो नरः, न निधाय द्रव्यमाददे, बीजान्यनुप्त्वा च शस्यं कृन्तामि।
\vakya कथं तर्हि ममार्थस्त्वया न वणिजां हस्ते निक्षिप्तः? तथा कृतेऽहमागमनकाले कुसीदेन सार्धं तं प्रत्यपत्स्ये।
\vakya निकटे तिष्ठतो जनांश्च स उवाच, तां स्वर्णमुद्रामस्मादपहृत्यामुष्मै दत्त यस्य दश मुद्रा सन्ति।
\vakya ते प्रोचुः, प्रभो, सन्ति तस्य दश मुद्राः।
\vakya युष्मानहं सत्यं ब्रवीमि, यस्य कस्यचिदास्ते तस्मै दायिष्यते, यस्य तु नास्ते तस्य यदस्ति तदपि तस्मादपहारिष्यते।
\vakya किन्त्वानयतात्र मत्समक्षं घातयत च मम तान् शत्रून् ये स्वेषामुपरि मम राजत्वम् अस्वीकृतवन्तः।
\stitle{यीशोर्यिरूशालेमे प्रवेशः।}
\vakya सर्वमेतत् कथयित्वा स पुरस्ताद् व्रजन् यिरूशालेमं गन्तुं प्रववृते।
\vakya यदा च स बैतफग्या बैथनियायाश्च निकटे जैतुनाख्यगिरेः पार्श्व उपातिष्ठत, तदा स्वशिष्याणां द्वौ प्रेषयन्नब्रवीत्,
\vakya सम्मुखस्थममुं ग्रागं गच्छतं, तत्र प्रविश्यैव तादृशमेकं बद्धं गर्दभवत्समासादयिष्यथो यस्य पृष्ठे मनुष्यः कोऽपि कदापि नोपविष्टवान्, तं मुक्त्वानयतं।
\vakya यदि तु कश्चिद् वां पृच्छति, कुतो मुञ्चथ इति तर्हि ब्रूतं, अनेन प्रभोः प्रयोजनमस्तीति।
\vakya ततस्तौ प्रेरितौ गत्वा तेन यत् कथितं तदनुरूपं प्रापतुः।
\vakya तौ तु यदा तं गर्दभवत्सममुञ्चतां तदा तत्स्वामिन ऊचुः, कुतो गर्दभवत्सं मुञ्चथ?
\vakya तौ प्राहतुः, अनेन प्रभोः प्रयोजनमस्ति।
\vakya ततः परं तौ तं गर्दभवत्सं यीशोः समीपं निन्यतुस्तत्पृष्ठे च स्ववासांसि क्षिप्त्वा यीशुं तदुपर्युपवेशयामासतुः।
\vakya तस्य यात्राकाले च जनाः स्ववस्त्राणि मार्गे तस्तरुः।
\vakya ततः परं यदा स जैतुनपर्वतादवरोहणमार्गस्य सन्निधावुपातिष्ठत, तदा कृत्स्नः शिष्यनिवहः सानन्दं पूर्वदृष्टानां सर्वेषां प्रभावसिद्धकर्मणां कारणादीश्वरस्य स्तवं कीर्तयितुमारभ्य बभाषे,
\vakya धन्यो राजासौ यः प्रभो र्नाम्नायाति, स्वर्गे शान्तिरूर्ध्वलोके च माहात्म्यकीर्तनं भूयात्।
\vakya जनताया मध्यात् तदा केचित् फरीशिनस्तमवादिषुः, गुरो, स्वशिष्यान् तर्जयतु।
\vakya स तु प्रतिभाषमाणस्तान् जगाद, युष्मानहं ब्रवीमि, अमी यदि तूष्णीम्भवन्ति, प्रस्तरास्तर्हि प्रक्रोक्ष्यन्ति।
\vakya ततः परं स यदा निकटस्थोऽभूत् तदा पुरीं निरीक्ष्य तामुद्दिश्य क्रन्दन्नब्रवीत्,
\vakya यत् तव शान्तिजनकं कथं तन्न त्वयाप्यनुभूतं? अस्मिन् त्वदीयदिवस एव कथ नानुभूतं? तत् त्वधुना तव दृष्टितः प्रच्छन्नं।
\vakya यत उपस्थास्यन्ति त्वां तानि दिनानि येषु तव शत्रवस्त्वत्परितः प्राकारं रचयित्वा त्वां वेष्टयिष्यन्ति समन्ताद्रोत्स्यन्ति च,
\vakya त्वां त्वदन्तःस्थांस्तव शिशूंश्च भूमिसात् करिष्यन्ति, त्वन्मध्ये प्रस्तरस्योपरि प्रस्तरं नैकमपि शेषयिष्यन्ति च, यतो हेतोस्त्वदीयतत्त्वावलोकनस्य समयस्त्वया नानुभूतः।
\vakya अथ स धर्मधाम प्रविश्य तत्र क्रयविक्रयकारिणो बहिष्कर्तुं प्रववृते,
\vakya तांश्चाब्रवीत्, लिखितमास्ते,
\begin{poem}
\startwithline “गृहं यन्मामकीनं तद् भविता प्रार्थनागृहं।
\pline तत्तु युष्माभिरकारि दस्यूनां गह्वरः।”
\end{poem}
\vakya ततः प्रभृति स प्रत्यहं धर्मधामन्युपादिशत्।
\vakya मुख्ययाजकास्तु शास्त्राध्यापकाः प्रजानां प्रथमाश्च तं नाशयितुमयतन्त, तथापि किं कर्तव्यं तन्निर्णेतुं नाशक्नुवन्, यतः सर्वजनस्तदुपदेशश्रवण आसक्त आसीत्\eoc