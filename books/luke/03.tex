\adhyAya
\stitle{योहनस्नापकस्य कर्माणि। यीशोः स्नानम्।}
\vakya अथ तिबिरियकैसरस्य राजत्वस्य पञ्चदशे वत्सरे यदा पन्तीयः पीलातो यिहूदियायाः शास्ता, चतुर्थांशाधिपतीनां मध्ये च हेरोदो गालीलस्य, तदीयभ्राता फिलिपो यितूरिया-त्राखोनीत्याख्ययोः प्रदेशयो र्लुषाणियश्चाबिलीन्या राजान् आसन्,
\vakya यदा च हाननः कायाफाश्च महायाजकावास्तां, तदा मरुभूमावीश्वरस्य वाक्यं सखरियसुतं योहनं प्रति प्रादुरभूत्।
\vakya ततः स यर्दनसमीपस्थस्य जनपदस्य सर्वत्रोपस्थाय पापमोचनार्थं मनःप्रत्यावर्तनसूचकम् स्नापनमघोषयत्,
\vakya भाववादिनो यिशायाहस्योक्तीनां ग्रन्थे यथा लिखितमास्ते,
\begin{poem}
\startwithline “मरौ घोषयतः प्रोच्चैरस्त्ययं कस्यचिद्रवः।
\pline प्रभोः संस्कुरुताध्वानं विधद्ध्वं तत्सृती ऋजूः॥
\pline सर्वाण्यापूरयिष्यन्ते निम्नस्थानानि यत्नतः।
\vakya कारिष्यन्ते पुन र्निम्नाः सर्वे भूधरपर्वताः॥
\pline सरलानि भविष्यन्ति कुटिलानि स्थलान्यपि।
\pline यानि वर्त्मानि रुक्षाणि तान्याप्स्यन्ति समानतां।
\vakya द्रक्ष्यते सर्वमर्त्यैश्च त्राणमीश्वरसाधितं॥
\end{poem}
\vakya ततो ये जननिवहास्तेन स्नापनार्थं बहिरगच्छंस्तान् सोऽवदत्, अरे सर्पवंशाः, यूयं भाविक्रोधात् पलायितुं केनादिष्टाः?
\vakya अतो मनःप्रत्यावर्तनस्य योग्यानि फलानि फलत। अस्माकं पिताब्राहामो विद्यत इति वाक्यं वान्तरे प्रयोक्तुं मा प्रवर्तध्वं, यतोऽहं युष्मान् ब्रवीमि, एतेभ्यः प्रस्तरेभ्योऽब्राहामस्य कृते सन्तानानुत्पादयितुम् ईश्वरः शक्तिमान्।
\vakya परन्त्वधुनैव पादपानां मूलेषु कुठारो लगन्नास्ते, अतो यः कश्चित् पादपः सत्फलं न फलति स उच्छिद्यते वह्नौ निक्षिप्यते च।
\vakya जननिवहास्तदा तं पप्रच्छुः, तर्हि किं कर्तव्यमस्माभिः?
\vakya स प्रतिभाषमाणस्तान् जगाद, यस्याङ्गरक्षके द्वे वस्त्रे स्तः स वस्त्रहीनायैकतरं ददातु। खाद्यद्रव्याणि वा यस्य विद्यन्ते स तथैव करोतु।
\vakya स्नापनार्थमागताः केचिच्छुल्कादायिनोऽपि तं जगदुः, गुरो किं कर्तव्यमस्माभिः?
\vakya स तान् जगाद, युष्मदर्थं यदादिष्टं ततोऽधिकं माहारयत।
\vakya केचिद् योद्धारोऽपि तं पप्रच्छुरस्माभि र्वा किं कर्तव्यं? स तान् जगाद, कमपि मा पीडयत, मृषादोषं वा कस्मिन्नपि मारोपयत, स्वभृतौ सन्तुष्य तिष्ठत च।
\vakya प्रजाजनस्तु प्रतैक्षत सर्वे च योहनमधि मनःस्वित्थम् अतर्कयन्, असौ किं ख्रीष्टः स्यादिति?
\vakya अतो योहनः प्रतिभाषमाणः सर्वान् जगाद, अहं युष्मांस्तोये स्नापयामि, मत्पश्चात्तु मत्तो बलवत्तरो नर आगच्छति, तदीयोपानहो र्बन्धन्यौ मोक्तुमप्यहं न योग्यः। स युष्मान् पवित्र आत्मनि वह्नौ च स्नापयिष्यति।
\vakya तस्य हस्ते सूर्पो विद्यते, स च स्वीयखलं सम्यक् संशोधयिष्यति, निजगोधूमान् कुशूले सङ्ग्रहीष्यति च, तुषांस्त्वनिर्वाणवह्निना दाहयिष्यति।
\vakya एतदन्येन प्रचुरेणोपदेशेनापि स प्रजाजनं सुसंवादमज्ञापत्।
\vakya ततः परं हेरोदः फिलिपनामकस्य स्वभ्रातु र्भार्यां हेरोदियामधि स्वकृतान्यपराणि सर्वाणि दुष्कर्माण्यधि च योहनेन भर्त्सितः।
\vakya अतः स तं दोषराशिमनेनापि वर्धयामास यद् योहनं कारायां रुरोध।
\vakya यदा तु कृत्स्नः प्रजाजनोऽस्नाप्यत, तदा यीशावपि स्नापिते प्रार्थयमाने च स्वर्ग उद्घाटितः
\vakya पवित्र आत्मा च दैहिक आकारे कपोत इव तस्योपर्यावततार, स्वर्गाच्चेयं वाणी समभूत्, मम प्रियः पुत्रस्त्वं त्वय्यहं प्रीतः।
\stitle{यीशुख्रीष्टस्य वंशावलि र्पत्रम्।}
\vakya स्वकार्यमारभमाणो यीशुः प्रायेण त्रिंशद्वर्षवयस्क आसीत्।
\vakya लौकिकज्ञाने स योषेफस्य सुतः, स पुनरेलेः सुतः, स मत्ततस्य सुतः, स लेवेः सुतः, स मल्केः सुतः, स यान्नस्य सुतः, स योषेफस्य सुतः,
\vakya स मत्तथियस्य सुतः, स आमोषस्य सुतः, स नहूमस्य सुतः, स इष्लेः सुतः, स नगेः सुतः,
\vakya स माटस्य सुतः, स मत्तथियस्य सुतः, स शिमियेः सुतः, स योषेकस्य, स यूदाः सुतः,
\vakya स योहनाः सुतः, स रीषाः सुतः, स सरुब्बाबिलस्य सुतः, स शल्‌टीयेलस्य सुतः, स नेरेः सुतः,
\vakya स मल्केः सुतः, स आद्देः सुतः, स कोषमस्य सुतः, स इल्मोदमस्य सुतः, स एरेः सुतः,
\vakya स योषेः सुतः, स इलीयेषरस्य सुतः, स योरीमस्य सुतः, स मत्ततस्य सुतः, स लेवेः सुतः,
\vakya स शिमियोनस्य सुतः, स यूदाः सुतः, स योषेफस्य सुतः, स योननस्य सुतः, स इलियाकीमस्य सुतः,
\vakya स मिलेयाः सुतः, स मैननस्य सुतः, स मत्तत्तस्य सुतः, स नाथनस्य सुतः, स दायूदस्य सुतः,
\vakya स यिशयस्य सुतः, स ओबेदस्य सुतः, स बोयसस्य सुतः, स सल्मोनस्य सुतः, स नहशोनस्य सुतः,
\vakya स अम्मीनादबस्य सुतः, सोऽरामस्य सुतः, स हिष्रोणस्य सुतः, स पेरसस्य सुतः, स यिहूदाः सुतः,
\vakya स याकोबस्य सुतः, स इस्‌हाकस्य सुतः, सोऽब्राहामस्य सुतः, स तेरहस्य सुतः,
\vakya स नाहोरस्य सुतः, स सरूगस्य सुतः, स रियोः सुतः, स पेलगस्य सुतः, स एबरस्य सुतः, स शेलहस्य सुतः,
\vakya स कैननस्य सुतः, सोऽर्फक्षदस्य सुतः, स शेमस्य सुतः, स नोहस्य सुतः, स लेमकस्य सुतः,
\vakya स मथूशेलहस्य सुतः, स हनोकस्य सुतः, स येरदस्य सुतः, स महललेलस्य सुतः, स कैननस्य सुतः,
\vakya स इनोशस्य सुतः, स शेथस्य सुतः, स आदमस्य सुतः, स ईश्वरस्य सुतः\eoc