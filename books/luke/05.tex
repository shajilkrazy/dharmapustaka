\adhyAya
\stitle{जाले बहुमत्स्योत्थानम्।}
\vakya एकदेश्वरस्य वाक्यं श्रोतुं जननिवहस्तमाकुलयत् स च गिनेषरताख्यह्रदस्य तीरेऽतिष्ठत्।
\vakya तदा स ह्रदकूलेऽवस्थिते द्वे नावौ ददर्श, ताभ्यां निर्गता धीवरा जालानि प्राक्षालयन्।
\vakya अतस्तयो र्या शिमोनस्यासीत् स तां प्रविश्य शिमोनमब्रवीत् त्वं स्थलात् किञ्चिद् दूरं तां वाहयितुमर्हसीति। अनन्तरं स नाव्युपविश्य जननिवहान् अशिक्षयत्।
\vakya यदा च कथनाद्विरराम तदा शिमोनं जगाद, गभीरं स्थानं यावन्नावं वाहय, तत्रैव मत्स्यधरणार्थं युष्माभि र्जालानि निक्षिप्यन्तां।
\vakya शिमोनस्तं प्रत्यब्रवीत्, नाथ, कृत्स्नां यामिनीं परिश्रमं कृत्वास्माभिः किमपि न धृतं, भवत आज्ञायान्त्वहं जालं निक्षेप्स्यामि।
\vakya तदनुरूपं कृत्वा च तै र्महान् मत्स्यनिवहो निबद्धः।
\vakya ततो विशीर्यमाणे तेषां जाले त उपकारायागमनार्थम् अन्यतरस्यां नावि स्थितान् स्वसङ्गिन आजुहुवुस्तेषु चागतेषूभे नावौ मग्नकल्पे कुर्वन्तः पूरयामासुः।
\vakya तद् विलोक्य पित्रो यीशो र्जानुनोः प्रणिपत्य व्याजहार, मत्तोऽपसरतु प्रभो, यतः पापिष्ठो नरोऽहमिति।
\vakya यतस्तैः सन्धृतान्मत्स्ययूथात् स तत्सङ्गिनश्च सर्वे विस्मयाक्रान्ताः,
\vakya शिमोनस्य सहभागिनोः सिबदियसुतयो र्याकोब-योहनयोरपि सैव दशा सम्भूता। यीशुस्तु शिमोनमब्रवीत्, मा भैषीरद्यारभ्य त्वं नरधारी भविष्यसि।
\vakya अनन्तरं नावौ स्थलं नीत्वा ते सर्वं विहाय तमनुजग्मुः।
\stitle{यीशुना जनैककुष्ठिने पक्षाघातिने च आरोग्यदानम्।}
\vakya अथ स यदा कस्मिंश्चिन्नगर आसीत्, तदा कुष्ठपूर्णो नर एक उपतस्थे। यीशुं दृष्ट्वा सोऽधोमुखं पतित्वा तं प्रसादयन् जगाद, प्रभो, भवान् यदीच्छति तर्हि मां शुचीकर्तुं शक्नोति।
\vakya ततः स हस्तं प्रसार्य तं स्पृष्ट्वा जगाद, इच्छामि, शुचि र्भव।
\vakya तत्क्षणञ्च कुष्ठं तस्मादपससार। यीशुस्तु तमाज्ञापयामास, कमप्येतन्मा वद। अपि तु गत्वा याजकमात्मानं दर्शय, तव शुचित्वलाभनिमित्तञ्च मोशिना यथादिष्टं तथैव तेभ्यः साक्ष्यदानार्थमुपहारं देहि।
\vakya तथापि तदीयकिंवदन्त्यधिकं देशं व्याप्नोत्, महान्तो जननिवहाश्च तस्य वाक्यानि श्रोतुं तेन व्याधिमुक्तिं लब्धुञ्च समागच्छन्।
\vakya स तु गुप्तं गत्वा निर्जनेषु स्थानेष्ववर्तत प्रार्थयत च।
\vakya अथ कस्मिंश्चिद् दिवसे यदा सोऽशिक्षयत् तदा गालीलस्य सर्वग्रामेभ्यो यिहूदियादेशाद् यिरूशालेमाच्चागताः फरीशिनो व्यवस्थावेत्तारश्च समासीना आसन् तेन व्याधिप्रतीकाराय च प्रभोः प्रभावः सतेजा आसीत्।
\vakya पश्य च केचिन्नराः खट्वायां शयानमेकमवशाङ्गं मनुष्यमभ्यन्तरं नेतुं तस्य समक्षं स्थापयितुञ्चायतन्त।
\vakya जनताकारणात्तु तदानयनार्थं मार्गमप्राप्य ते गृहपृष्ठमारुह्य फलकाकाराः कतिपया इष्टका अपसार्य तेन छिद्रेण तं सखट्वं गृहमध्ये यीशोः सम्मुखमवरोहयामासुः।
\vakya स च तेषां विश्वासं दृष्ट्वा तं जगाद, भो मानव, तव पापानि मोचितानि।
\vakya तदा ते शास्त्राध्यापकाः फरीशिनश्च मिथ इत्थं तर्कयितुमारेभिरे, कोऽसौ य ईश्वरनिन्दककथा व्याहरति? एकस्मादीश्वरादन्यः कः पापानि क्षमितुं शक्नोति?
\vakya यीशुस्तु तेषां तर्कान् विज्ञाय प्रतिभाषमाणस्तानब्रवीत्, युष्माभि र्मनःसु किं तर्क्यते?
\vakya ब्रूत, तव पापानि मोचितानीति वा त्वमुत्थाय परिव्रजेत्येतयोः कथनयोः कतरम् अनायासम्?
\vakya पृथिव्यान्तु पापानि मोचयितुं मनुष्यपुत्रस्य सामर्थ्यमस्तीति यथा युष्माभि र्ज्ञायेत, तदर्थं - तमवशाङ्गमभिभाष्य स उवाच - त्वामहमाज्ञापयामि, उत्थाय स्वखट्वामादाय स्वगेहं याहि।
\vakya स तु तत्क्षणं सर्वेषां समक्षमुत्थाय स्वशय्यामादाय चेश्वरं स्तुवन् स्वगृहं जगाम।
\vakya ततः सर्वे विस्मयापन्ना ईश्वरमस्तुवन् भयपूर्णाश्चावदन्, अद्भुतानीतिवृत्तान्यद्यास्माभि र्दृष्टानि।
\stitle{लेव्याह्वानम्।}
\vakya ततः परं स बहि र्गत्वा शुल्कादायस्थान उपविष्टं लेविनामकं शुल्कादायिनं निरीक्ष्य जगाद, मामनुगच्छ।
\vakya स च सर्वं त्यक्त्वोत्थाय तमनुजगाम।
\vakya स लेविश्च तस्य कृते निजगृहे महाभोज्यं चकार, तत्र भोक्तुमुपविष्टा बहवः शुल्कादायिनस्तेषां सङ्गिनोऽन्यजनाश्चाविद्यन्त।
\vakya अतस्तेषां शास्त्राध्यापकाः फरीशिनश्च तस्य शिष्यान् सम्बोध्य सामर्षमवादिषुः, किमर्थं भुंग्ध्वे शुल्कादायिभिः पापिभिश्च सार्धम्?
\vakya यीशुस्तु प्रतिभाषमाणस्तान् जगाद, चिकित्सको न स्वस्थानाम् अपि त्वस्वस्थानाम् आवश्यकः।
\vakya न धार्मिकान्, अपि तु मनःपरावर्तनाय पापिन आह्वातुमहमागतः।
\vakya ते पुनस्तमूचुः, किमर्थं योहनसय शिष्या भूय उपवसन्ति प्रार्थयन्ते च फरीशिनां शिष्या अपि तथैव कुर्वन्ति, तव शिष्यास्त्वश्नन्ति पिबन्ति च?
\vakya तदा स तानब्रवीत्, वरो यावद् वरसखैः सार्धं वर्तते तावत् किं तान् उपवासयितुं शक्नुथ?
\vakya दिनानि तावदुपस्थास्यन्ति यदा वरस्तेभ्योऽपहारिष्यते, तेष्वेव दिनेषु त उपवत्स्यन्ति।
\vakya स तानिमाम् उपमाकथामपि जगाद, जीर्णवसने न कोऽपि नूतनवसनस्य खण्डं सीव्यति, यतस्तथा कृते तन्नूतनमपि तेन छिद्यते पुरातने च न युज्यते नूतनादुद्धृतः स खण्डः।
\vakya नापि नवो द्राक्षारसः केनापि जीर्णकुतूषु निधीयते; यतस्तथा कृते नवो द्राक्षारसः कुतू र्विदारयति तेन स च विस्रोष्यति कुत्वश्च नंक्ष्यन्ति।
\vakya प्रत्युत नवद्राक्षारसो नवकुतूषु निधातव्यस्तेनोभयो रक्षा सम्भवति।
\vakya पुरातनं पीत्वा तु कोऽपि सहसा नवं नाभिवाञ्छति, यतः स ब्रूते, पुरातनः सुस्वादुतरः\eoc