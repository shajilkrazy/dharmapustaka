\adhyAya
\stitle{आभाषः।}
\vakya आदितो ये साक्षिणो वाक्यस्य सेवकाश्चासंस्तैरस्मासु यथा समर्पितं
\vakya तथैवास्मन्मध्ये दृढप्रत्ययेन गृहीतानाम् इतिवृत्तानां कथाप्रबन्धं रचयितुम् अनेक उपचक्रमिरे।
\vakya अत एव, भो महामहिम थियफिल, सूक्ष्मानुसन्धानेनारम्भात् सर्वमनुगतोऽहमपि भवदर्थमानुपूर्व्येण लेखितुं मतिमकार्षं,
\vakya तेन भवान् याः कथाः शिक्षितस्तासाम् अमोघतामनुभविष्यति।
\stitle{योहनस्नापकस्य जन्माधि आगमसंवादः।}
\vakya यिहूदियाया राज्ञो हेरोदस्य कालेऽबियस्य श्रेण्यां नियुक्तः सखरियनामा याजक आसीत्। तस्य भार्या हारोणवंशोद्भवा, नाम तस्या इलीशाबेता।
\vakya उभौ तावीश्वरस्य समक्षं धार्मिकावास्ताम् अनिन्दनीयौ च प्रभोः सर्वा आज्ञा विधींश्च समाचरताम्।
\vakya सन्तानस्तु तयो र्नासीत्, यत इलीशाबेता वन्ध्या तावुभौ च वयोवृद्धौ जातौ।
\vakya एकदा तु यदा स याजकत्वस्य नियमात् स्वश्रेण्याः पर्यायहेतोरीश्वरस्य समक्षं याजकत्वम् अन्वतिष्ठत्,
\vakya तदा गुटिकापातकृतनिरूपणात् स धूपदाहं कर्तुं प्रभो र्मन्दिरं प्रविष्टवान्।
\vakya धूपदाहकाले च कृत्स्नो जननिवहो बहिः प्रर्थयत।
\vakya तदा धूपवेद्या दक्षिणे तिष्ठन् प्रभोरेको दूतस्तस्मै दर्शनं ददौ।
\vakya तं दृष्ट्वा सखरिय उद्विज्य भयक्रान्तोऽभूत्।
\vakya स दूतस्तु तं जगाद, मा भैषीः सखरिय, यतस्तव प्रार्थनाश्रावि, तव भार्येलीशाबेता च त्वदर्थं पुत्रं प्रसोष्यते, त्वञ्च तस्य नाम योहन इति करिष्यसि।
\vakya तव हर्ष उल्लासश्च जनिष्यते बहवश्च तस्य जन्मन्यानन्दिष्यन्ति।
\vakya यतः प्रभोः समक्षं महान् भविष्यति, द्राक्षारसं मद्यं वा न पास्यति,
\vakya आमातृगर्भादेव पवित्रेणात्मना पूरयिष्यत इस्रायेलस्य सन्तानानाम् अनेकांस्तेषामीश्वरं प्रभुं प्रति प्रत्यावर्तयिष्यति च।
\vakya तस्याग्रतश्च स एलियस्यात्मना प्रभावेन च व्रजन् सन्तानान् प्रति पितॄणां हृदयानि प्रत्यावर्तयिष्यत्याज्ञात्यागिनश्च धार्मिकाणां मतौ प्रवर्त्य प्रभोः कृते सज्जितं प्रजावृन्दमुपकल्पयिष्यति।
\vakya सखरियस्तदा तं दूतमब्रवीत्, कथमेतज्ज्ञास्यामि? अहं हि स्थविरो मम भार्यापि वयोवृद्धा।
\vakya दूतस्तं प्रतिबभाषे, ईश्वरस्य समक्षं स्थातुं नियुक्तो गाब्रीयेलोऽहं त्वामालपितुमिमं मङ्गलसंवादं तुभ्यं दातुञ्च प्रहितः।
\vakya पश्य च यस्मिन् दिन एतत् सेत्स्यति तद्दिनं यावत् त्वं मौनी भाषितुमसमर्थश्च स्थास्यसि। यतो मम वचःसु त्वया प्रत्ययो नाकारि, स्वसमये तु तानि सेत्स्यन्ति।
\vakya एतावत्कालं प्रजाजनः सखरियं प्रत्यैक्षत मन्दिरे तस्य विलम्बने व्यस्मयच्च।
\vakya बहिरागतस्तु स तैः संलपितुं नाशक्नोत्, मन्दिरे तेन दर्शनं लब्धमिति तैरबोधि च।
\vakya तदीयोपासनानुष्ठानस्य काले सम्पूर्णे स स्वगृहं प्रत्याजगाम।
\vakya तस्मात् कालात् परं तस्य भार्येलीशाबेता गर्भवती बभूव पञ्च मासान् प्रच्छन्ना तस्थौ च,
\vakya यतः सावदत्, प्रभु र्यस्मिन् काले जनेषु ममापयशो हर्तुं दृक्पातं कृतवान्, तस्मिन् काले मां प्रतीत्थमाचरितवान्।
\stitle{यीशुख्रीष्टस्य जन्माधि आगमसंवादः।}
\vakya षष्ठे मासे पुन र्गाब्रीयेलो दूतो गालीलस्थं नासरताख्यां नगरं
\vakya दायूदकुलोद्भवाय योषेफाभिधाय पुरुषाय वाग्दत्तामेकां कुमारीं प्रतीश्वरेण प्रहितः। तस्याः कुमार्या नाम मरियमिति।
\vakya दूतो गृहं प्रविश्य तां जगाद, मङ्गलं ते भूयात्, अनुगृहीते, प्रभुस्तव सहायः, नारीषु त्वं धन्या।
\vakya एवं दृष्ट्वा सा तस्य कथायामुद्विग्ना समालोचयत् कीदृशोऽयं मङ्गलवाद इति।
\vakya दूतस्तदा तामवादीत्, मा भैषी र्मरियम्, यतस्त्वमीश्वरस्यानुग्रहपात्री,
\vakya पश्य च त्वं गर्भिणी भूत्वा पुत्रं प्रसोष्यते तस्य नाम यीशुरिति करिष्यसि च।
\vakya स महान् भविष्यति परात्परस्य पुत्र इत्यभिधायिष्यते च, ईश्वरः प्रभुस्तस्मै तत्पितु र्दायूदस्य सिंहासनं दास्यति च,
\vakya स च सर्वयुगानि यावद् याकोबकुलस्य राजा स्थास्यति तद्राज्यस्यान्तः कदापि न भविष्यति च।
\vakya मरियम् तदा दूतमवादीत् कथमेतत् सम्भविष्यति? पुरुषं हि न जानेऽहं।
\vakya दूतस्तां प्रतिबभाषे, पवित्र आत्मा त्वामधिष्ठास्यति परात्परस्य प्रभावस्तवोपरि छायां दास्यति च। एतत्कारणादेव तव तत् पवित्रं गर्भफलमीश्वरस्य पुत्र इत्यभिधायिष्यते।
\vakya पश्य च तव ज्ञातिरिलीशाबेतापि वार्धक्ये गर्भिणी जाता, सर्वै र्वन्ध्येत्यभिहितायाश्च तस्याः षष्ठो मासोऽयं
\vakya यत ईश्वरस्यासाध्यं वाक्यं किमपि नास्ति।
\vakya मरियम् तदाब्रवीत् पश्याहं प्रभो र्दासी। यथोक्तं भवता तथैव मम गति र्भूयात्। अनन्तरं दूतस्तस्याः समीपात् प्रतस्थे।
\vakya तस्मिन् काले मरियम् उत्थाय सत्वरं पर्वतमयप्रदेशस्थं यूदानगरं जगाम
\vakya सखरियस्य गृहं प्रविश्य चेलीशाबेतां मङ्गलवादेनाभिबभाषे।
\vakya इलीशाबेता च यदा मरियमो मङ्गलवादमश्रौषीत् तदैव तस्या जठरे शिशु र्ननर्ते।
\vakya इलीशाबेता च पवित्रेणात्मना पूर्णोच्चरवेणेमां वाचमुदीरयामास, धन्या त्वं नारीषु धन्यश्च तव गर्भफलं।
\vakya कथन्तु जातं ममेदं सौभाग्यं यन्मत्प्रभो र्जननी मदन्तिकमागता?
\vakya पश्य त्वदीयमङ्गलवादस्य रवे मम कर्णौ प्रविष्टवत्येव ननर्त शिशुरुल्लासान्ममोदरे।
\vakya धन्यैव त्वं विश्वासकारिणि यत् प्रभोरादेशात् त्वं यद्यदुक्ता तत्सिद्धि र्भविष्यति।
\vakya तदा मरियम् बभाषे,
\begin{poem}
\startwithline “कीर्तयन्ति मम प्राणा महिमानं सदाप्रभोः।
\vakya ममात्मा च प्रहृष्टोऽभूद् ईश्वरे तारके मम॥
\vakya स्वदास्या दीनतायां स दृक्पातं कृतवान् यतः।
\pline पश्याद्यारभ्य वक्ष्यन्ति धन्यां मां सार्वकालिकाः॥
\vakya मदर्थं गुरुकर्माणि कृतवान् सर्वशक्तिमान्।
\pline नामधेयञ्च यत् तस्य तत् पवित्रं स्मृतं किल॥
\vakya तस्माद् भीता मनुष्या ये तेषामेव कृते स्थिरा।
\pline पुत्रपौत्रादिभि र्भोग्या करुणा तस्य वर्तते॥
\vakya कर्म विक्रमसिद्धं स कृतवान् स्वीयबाहुना।
\pline उद्धतांश्चित्तसङ्कल्पै र्मनुष्यान् स विकीर्णवान्॥
\vakya स कर्तृत्वविशिष्टांश्च कृतवान् आसनच्युतान्।
\pline ये च नीचा मनुष्यांस्तान् चकारोच्चपदान्वितान्॥
\vakya क्षुधितान् पूरयामास स द्रवैरुत्तमैः पुनः।
\pline धनिनस्तु मनुष्यान् स रिक्तहस्तान् विसृष्टवान्॥
\vakya इस्रायेलं स्वदासञ्चोपाकार्षीत् स सहायवत्।
\pline दयां स्मर्तुं यथा प्रोक्तास्तेनास्मत्पूर्वपुरुषाः।
\vakya अब्राहामे फलिष्यन्तीं तद्वंशे च युगक्रमात्॥”
\end{poem}
\vakya अनन्तरं मरियम् प्रायेण मासत्रय तस्याः सन्निधाववस्थाय स्वगृहं प्रतिजगाम।
\stitle{योहनस्य जन्मः।}
\vakya इलीशाबेतायास्तु प्रसवकाले सम्पूर्णे सति सा पुत्रं प्रासोष्ट।
\vakya तदा तां प्रति प्रभुना महती दयाकारीति श्रुत्वा तस्याः समीपवासिनो ज्ञातिजनाश्च तया सार्धमाननन्दुः।
\vakya अष्टमे दिने च ते शिशोस्त्वक्छेदनार्थमागत्य पितु र्नामेव तस्यापि सखरिय इति नाम कर्तुं प्रावर्तन्त।
\vakya तस्य जननी तु प्रतिजगाद, मैवम्, अस्य नाम योहन इत्यभिधायिष्यते।
\vakya अनेन ते तामूचुः, एतन्नाम्ना ख्यातस्तव ज्ञातिजनेषु कोऽपि नास्ति।
\vakya ततस्त इङ्गितैस्तस्य पितरं पप्रच्छुः, भवदिच्छातोऽस्य किं नाम कर्तव्यम्?
\vakya ततः स फलकमेकं याचित्वा लिलेख, अस्य नाम योहन इति। अनेन सर्वे विस्मयं मेनिरे।
\vakya तत्क्षणमेव तस्य वक्त्रं जिह्वा च मोचिते, स च भाषमाण ईश्वरस्य धन्यवादं कर्तुं प्रवृत्तः।
\vakya तयोः समीपवासिनश्च सर्वे भयाक्रान्ताः।
\vakya यिहूदियायाः पर्वतमयप्रदेशस्य च सर्वत्रैतासां सर्वकथानां प्रसङ्गो व्यानशे, योवन्तश्च ता अश्रौषंस्ते हृदयेषु ता निधायावदन्, किं नु भवितव्यं शिशुनामुनेति। प्रभो र्हस्तश्च तस्य सहाय आसीत्।
\stitle{सखरियस्य भविष्यद्वाक्यं।}
\vakya तस्य पिता सखरियश्च पवित्रेणात्मना पूर्णो भाववाक्यमिदं व्याजहार,
\begin{poem}
\startwithvakya “धन्याः स्यादीस्रयेलीयजनानामीश्वरः प्रुभुः।
\pline अवेक्ष्य स्वप्रजानां स मोचनं कृतवान् यतः॥
\vakya शृङ्गञ्चोत्पादयामासास्मदर्थं त्राणसाधकं।
\pline उद्भूतं स्वीयदासस्य दायूदस्यैव वंशतः।
\vakya यथा प्राहादिकालीनैः पवित्रै र्भाववादिभिः॥
\pline परित्राणार्थमस्माकं परिपन्थिसमूहतः।
\vakya उद्धाराय च सर्वेषाम् अस्मद्विद्वेषिणां करात्॥
\vakya करुणां कर्तुकामोऽस्ति सोऽस्मदीयपितॄन् प्रति।
\pline नियमं स्मर्तुकामश्च पवित्रं स्वकृतं पुरा।
\vakya अस्मत्पित्रेऽब्रहामाय शप्त्वा दत्तञ्च तं वरं॥
\vakya निर्भया येन शक्ष्यामः शत्रुहस्तेभ्य उद्धृताः।
\vakya साधुत्वधार्मिकत्वाभ्यां तत्साक्षात् तं समर्चितुं।
\pline सकलान् दिवसान् यावदस्माकं परमायुषः॥
\vakya शिशो परात्परस्य त्वं भाववाद्यभिधास्यसे।
\pline प्रभो र्मार्गान् हि संस्कर्तुं तदग्रे त्वं व्रजिष्यसि॥
\vakya प्रजाभ्यस्तस्य दातव्यं ज्ञानं त्राणस्य च त्वया।
\pline पापानां मार्जने तासां कारुण्येनास्मदीशितुः॥
\vakya तत्स्नेहादूर्ध्वलोकीयास्मान् ऊषा प्रत्यवैक्षत।
\vakya ध्वान्ते मृत्युतमिस्रे चासीनान् द्योतयितुं नरान्।
\pline अस्मत्पादान् यथा कुर्याच्छान्तिमार्गावलम्बिनः॥”
\end{poem}
\vakya अथ स शिशुरवर्धतात्मना च बलवान् अजायत। इस्रायेलाय तु तत्प्रदर्शनस्य दिनं यावदनुपस्थितं तावत् स निर्जनस्थानेष्ववर्तत\eoc