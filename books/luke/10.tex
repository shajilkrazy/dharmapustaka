\adhyAya
\stitle{यीशोः सप्ततिजनानां प्रेरणं विविधशिक्षादानञ्च।}
\vakya ततः परं प्रभुरन्यान् सप्ततिजनानपि निरूपयामास, स स्वयं यद्यन्नगरं स्थानं वा गमिष्यति तत्र स्वस्याग्रतस्तान् द्वौ द्वौ प्रेषयामास।
\vakya तान् जगाद च, प्रचुरं तावच्छस्यं कर्तनीयं कार्यकारिणस्त्वल्पे, तत् तदेव प्रार्थयध्वं शस्यक्षेत्रस्य स्वामिनं यत् स स्वक्षेत्रे कार्यकारिणः प्रेरयेत्।
\vakya यात, पश्यत वृकाणां मध्ये शिशुमेषानिव युष्मान् प्रहिणोमि।
\vakya मुद्राधारं चेलसम्पुटकं वोपानहौ वा मा वहत, पथि च कमपि मङ्गलवादं मा वदत।
\vakya यदा च किञ्चन गृहं प्रविशथ तथा प्रथमं वदत, गृहस्यास्य शान्ति र्भूयात्।
\vakya ततो यदि तत्र शान्तेः पात्रं विद्यते तर्हि युष्मदुक्ता शान्तिस्तं नरमाश्रयिष्यति, नो चेत् पुन र्युष्मासु वर्तिष्यते।
\vakya यूयं तस्मिन् गृहे चावतिष्ठध्वं भुंध्वं पिबत च यद्यत् तत्रत्येभ्यो लप्स्यध्वे। यतः कार्यकारी स्ववेतनमर्हति। गृहाद् गृहं मा गच्छत।
\vakya किञ्चन नगरं प्रविष्टा यूयं यदि जनै र्गृह्यध्वे, तर्हि युष्मदर्थं यद्यत् परिवेषयिष्यते तद् भुंध्वं,
\vakya तत्रत्यान् अस्वस्थान् निरामयान् कुरुत, जनांश्च वदत, ईश्वरस्य राज्यं युष्मत्समीपमुपस्थितं।
\vakya किञ्चन नगरं प्रविष्टास्तु यदि जनै र्न गृह्यध्वे, तर्हि बहिस्तदीयचत्वराणि गत्वा वदत,
\vakya युष्मदीयनगरस्य या धूलिरस्मासु लग्ना तामपि युष्मदर्थमवधूनुमः। तथापि युष्माभिरिदं ज्ञायतां यदीश्वरस्य राज्यं युष्मत्समीपमुपस्थितम्।
\vakya युष्मांस्त्वहं ब्रवीमि, दिनेऽमुष्मिन् सदोमस्य दशा तस्य नगरस्य दशातः सह्यतरा भविष्यति।
\vakya हा कोरासिन, हा बैत्सैदे, युवां सन्तापार्हे, यतो युवयो र्मध्ये कृतानि यानि प्रभावसिद्धानि कर्माणि, तानि चेत् सोरे सीदोने चाकारिष्यन्त, प्रागेव तन्निवासिनस्तर्हि शाणं परिधाय भस्मन्युपविश्य च मनांसि परावर्तयिष्यन्।
\vakya अपि तु विचारे युवयो र्दशातः सोरस्य सीदोनस्य च दशा सह्यतरा भविष्यति।
\vakya त्वञ्च हा स्वर्गं यावदुन्नमिते कफरनाहूमपुरि, पातालं यावदवरोहयिष्यसे।
\vakya यो युष्माकं वाक्यानि गृह्णाति स मम वाक्यानि गृह्णाति। यश्च युष्मान् निराकरोति स मां निराकरोति। यश्च मां निराकरोति स मत्प्रेरकं निराकरोति।
\vakya ततः परं सप्ततिस्ते सानन्दं प्रत्यागत्य जगदुः, प्रभो भवतो नाम्ना भूता अप्यस्माकं वशीभवन्ति।
\vakya स तु तानवादीत्, अहं विद्युतमिव शैतानं स्वर्गात् पतन्तमवलोकयं।
\vakya पश्यत युष्मभ्यमहमिदं सामर्थ्यं दत्तवान्, तद् यूयं सर्पान वृश्चिकांश्च रिपोः कृत्स्नं बलञ्च चरणै र्मर्दिष्यथ।
\vakya किमप्येव युष्मान् न हिंसिष्यति। अपि तु मैवैतस्मिन्नानन्दत यदात्मानो युष्माकं वशीभवन्ति, प्रत्युतैतस्मिन् यद्युष्मन्नामानि स्वर्गे लिखितानि विद्यन्ते।
\vakya तस्मिन् दण्डे यीशुरात्मन्युल्लासमनुभूय बभाषे, भो स्वर्गमर्त्ययोः स्वामिन् पितः, त्वामहं सुधं वदामि यतस्त्वया विज्ञेभ्यस्तीक्ष्णबुद्धिभ्यश्चेमानि निगुह्य शिशूनामाविष्कृतानि। अतः किं पितः, यदित्थं तव दृष्टौ यत् प्रीतिकरं तदेव सिद्धं।
\vakya ततः परं स शिष्यान् प्रति परावृत्योवाच, मम पित्रा मयि सर्वमेव समर्पितं, अपि च पुत्रः कस्तत् पितरं विनापरः कोऽपि न जानाति, पिता वा कस्तत् पुत्रं विनापरः कोऽपि न जानाति, यस्मै च तत् प्रकाशयितुं पुत्राय रोचते सोऽपि तज्जानाति।
\vakya ततः परं स स्वशिष्यान् प्रति प्रत्यावृत्य विजनेऽवादीत्, धन्यानि तानि नेत्राणि यानि पश्यन्ति यद्यद् यूयं पश्यथ।
\vakya वस्तुतोऽहं युष्मान् ब्रवीमि, बहवो भाववादिनो राजानश्च द्रष्टुं वाञ्छितवन्तो युष्माभि र्यद्यद्दृष्यते श्रोतुञ्च युष्माभि र्यद्यच्छ्रूयते, न तु तानि दृष्टवन्तो न वा श्रुतवन्तः।
@Vकिं सर्वप्रधानाज्ञा एतद्विषयकशिक्षा।
\vakya पश्य च, व्यवस्थावेत्ता कश्चिदुत्थाय तं परीक्षमाणोऽब्रवीत्, गुरो, किं कृत्वा मयानन्तजीवनस्याधिकारो लप्स्यते?
\vakya स तं जगाद, व्यवस्थायां किं लिखितमास्ते? कीदृक् पठसि?
\vakya स प्रतिबभाषे, त्वं कृत्स्नान्तःकरणेन कृत्स्नप्राणैः कृत्स्नशक्त्या कृत्स्नचित्तेन च स्वेश्वरं प्रभुं प्रति, स्ववच्च स्वनिकटस्थं प्रति प्रेम कुर्विति।
\vakya स तमाह, यथार्थं प्रत्युक्तं त्वया, तदेवाचर तर्हि जीविष्यसि।
\vakya स त्वात्मानं निर्दोषीकर्तुमिच्छन् यीशुमब्रवीत्, वाढं, मम निकटस्थः कः?
\vakya यीशुस्तदा प्रतिभाषमाणः कथयामास, कश्चिन्मनुष्यो यिरूशालेमाद् यिरीहुमवारोहन् दस्युभिरासादितः, ते तं विवस्त्रीकृत्यास्त्रैराजघ्नु र्मृतप्रायञ्च त्यक्त्वा प्रतस्थिरे।
\vakya सङ्गत्या याजकः कश्चित् तेन मार्गेणावारोहत्, स तं दृष्ट्वापरपार्श्वेनापजगाम।
\vakya तथैव लेवीयः कश्चित् तस्मिंस्थान उपस्थायोपागत्य दृष्ट्वा चापरपार्श्वेनापजगाम।
\vakya तदन्तिकमुपस्थितः शमरियः कश्चित् पथिकस्तु तं दृष्ट्वानुचकम्पे, समीपं गत्वा च तैलं द्राक्षारसञ्च निषिच्य तत्क्षतानि बबन्ध, निजवाहनमारोह्य च तं पान्थशालां निनायाशुश्रूषत च।
\vakya परदिने च निर्गत्य मुद्रापादद्वयं गृहीत्वा पान्थशालाध्यक्षाय ददौ तञ्चाह, शुश्रूषस्वामुं, यदि चाधिकं व्ययेस्तर्ह्यहं प्रत्यागमनकाले तुभ्यं तत् प्रतिदास्यामीति।
\vakya अत्र त्वं किं मन्यसे? त्रयाणां तेषां को दस्युभिरासादितस्य तस्य निकटस्थो जातः? सोऽब्रवीत् यस्तं प्रति दयां चकार स एव।
\vakya ततो यीशुस्तमुवाच, याहि त्वमपि तथैवाचर।
\vakya अथ तेषां गमनकाले स कञ्चिद् ग्रामं प्रविवेश। मार्था नाम्नी योषिच्च स्वगृहे तस्यातिथ्यं चकार।
\vakya तस्या मरियमभिधा भगिन्यासीत्, सा यीशोश्चरणयोरुपविश्य तस्य वाक्यमाकर्णयत्।
\vakya मार्था तु बहुविधपरिचर्यायां व्याकुलाभूत्, उपागत्य चावोचत्, प्रभो, भवानत्र किमुदासीनो यन्मद्भगिनी मां त्यक्त्वेकस्यां मयि परिचर्यां समर्पितवती? तदादिशतु तां यथआ सा मम साहाय्यं कुर्यात्।
\vakya यीशुस्तु प्रतिभाषमाणस्तामाह, मार्थे, मार्थे, बहून्यधि चिन्तयसि व्यग्रा भवसि च, एकमेव तु प्रयोज्यं।
\vakya मरियम् हि तमुत्तममंशं वरितवती यतस्तस्या नैवापहारिष्यते\eoc