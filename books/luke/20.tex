\adhyAya
\stitle{यीशोः क्षमताविषयिणी शिक्षा।}
\vakya तत्कालस्य कस्मिंश्चिद्दिवसे स यदा धर्मधाम्नि जनानशिक्षयत् सुसंवादमज्ञापयच्च, तदा मुख्ययाजकाः शास्त्राध्यापकाः प्राचीनाश्च तमुपस्थाय जगदुः,
\vakya अस्मान् वद, केनाधिकारेण करोषि त्वं कर्माण्येतानि? को वा तुभ्यं तमधिकारं दत्तवान्?
\vakya यीशुस्तु प्रतिभाषमाणस्तानवादीत्, युष्मानहमपि कथामेकां प्रक्ष्यामि, मह्यं तदुत्तरं दत्त।
\vakya योहनस्य स्नापनं किं स्वर्गोत्पन्नमासीदुत मनुष्योत्पन्नं?
\vakya ते तदा मिथो विचारयन्तोऽवदन्, तत् स्वर्गोत्पन्नमित्युक्ते सोऽस्मान् प्रक्ष्यति, यूयं तर्हि तस्मिन् किमर्थं न विश्वसितवन्तः?
\vakya तत्तु मनुष्योत्पन्नमित्युक्ते सर्वजनोऽस्मान् प्रस्तरैराहत्य मारयिष्यति, यतो योहनो यद् भाववादी तत् सर्वै र्दृढं प्रतीयते।
\vakya तत् ते प्रत्यवदन् तत् कुत उत्पन्नं तन्न जानीमः।
\vakya यीशुरनेन तानवादीत्, नाहमपि युष्मान् ज्ञापयामि, करोम्यहं सर्वमेतत् केनाधिकारेण।
\stitle{द्राक्षोद्यानः कृषकाश्च।}
\vakya ततः स जनेभ्यो दृष्टान्तकथामिमां कथयितुं प्रववृते, नरः कश्चिद् द्राक्षालता रोपयन् द्राक्षोद्यानं चकार, कृषकेषु करदायिषु तत् समर्प्य च दीर्घकालार्थं देशान्तरं जगाम।
\vakya फलकाले च कृषका यद् द्राक्षोद्यानोत्पन्नफलानामंशं तस्मै दद्युस्तदर्थं स तेषां समीपं दासमेकं प्राहिणोत्, कृषकास्तु तं प्रहृत्य रिक्तहस्तं विससृजुः।
\vakya पुनः सोऽपरमेकं दासं प्राहिणोत्, ते तु तमपि प्रहृत्य न्यक्कृत्य च रिक्तहस्तं विससृजुः।
\vakya पुनः स तृतीयमेकं प्रेषयामास, ते तु तमप्यस्त्रैराहत्य बहि र्निचिक्षिपुः।
\vakya उद्यानस्य स्वामी तदा बभाषे, किं करवाणि? मम प्रियं पुत्रं प्रहेष्यामि, किंस्वित् तं दृष्ट्वा त आदरिष्यन्ते।
\vakya कृषकास्तु तं दृष्ट्वा परस्परं तर्कयन्तोऽवन्, असावुत्तराधिकारी, समायात, हन्ततामसावस्माभिस्तेन रिक्थमस्माकं भविष्यति।
\vakya ततः परं स तै र्द्राक्षोद्यानाद् बहि र्निक्षिप्तो हतश्च। तदोद्यानस्य स्वामी तान् प्रति किं करिष्यति?
\vakya स आगत्य तान् कृषकान् संहरिष्यति, द्राक्षोद्यानमन्येषु समर्पयिष्यति च।
\vakya तच्छ्रुत्वा जना आवदन्, मैवं भवतु। स तु तान् अवलोक्य व्याजहार, किं तर्ह्येतद् यल्लिखितमास्ते,
\begin{poem}
\startwithline “गृहनिर्मातृभिर्लोकै र्यः पाषाणो निराकृतः।
\pline स एव गृहकोणस्थः प्रमुख्यः प्रस्तरोऽभवत्॥”
\end{poem}
\vakya प्रस्तरे तस्मिन् यः कश्चित् पतिष्यति स खण्डशो भग्नो भविष्यति, प्रस्तरश्च स यस्मिन् पतिष्यति तं सञ्चूर्णयिष्यति।
\vakya मुख्ययाजकाः शास्त्राध्यापकाश्च तस्मिन्नेव दण्डे तस्मिन् हस्तार्पणं कर्तुमयतन्त, जनेभ्यस्त्वबिभयुः, यतः स तानुद्दिश्यामुं दृष्टान्तं कथितवानिति तैरबोधि।
\stitle{कैसराय करदानं।}
\vakya अथ शास्तु र्हस्ते देशाध्यक्षस्याधिकारे च तं समर्पयितुमिच्छवस्ते समालोच्य वाक्पाशेन यत् तं धरेयुस्तदर्थं कपटधार्मिकवेशान् कतिपयान् प्रणिधीन् प्रेषयामासुः।
\vakya ते तं पप्रच्छुः, भो गुरो, वयं जानीमो यद् भवान् ऋजु भाषते उपदिशति च, मुखापेक्षां न कुरुतेऽपि तु सत्येनेश्वरस्य मार्गं शिक्षयति।
\vakya अस्माभिः कैसराय करदानं विधेयं न वेति?
\vakya स तु तेषां धूर्ततां बुद्ध्वा तान् जगाद, किमर्थं मां परीक्षध्वे?
\vakya मुद्रामेकां दीनाराख्यां मां दर्शयत। अत्र कस्य मूर्ति र्लेखनञ्च भातः? ते प्रत्यूचुः, कैसरस्य। 
\vakya तदा स तानब्रवीत्, दत्त तर्हि कैसराय यद्यत् कैसरस्य, दत्त चेश्वराय यद्यदीश्वरस्य।
\vakya ततस्ते जनानां समक्षं वाक्पाशेन तं धर्तुं नाशक्नुवन्, तस्य प्रत्युत्तरे विस्मिता मौनीबभूवुश्च।
\stitle{पुनरुत्थानविषयिणी शिक्षा।}
\vakya अथ पुनरुत्थानमनङ्गीकुर्वाणाः केचित् सद्दूकिनो निकटमागत्य तं पप्रच्छुः,
\vakya भो गुरो, मोशिरस्मदर्थं लिखितवान्, कस्यचित् कृतदारो भ्राता यदि निःसन्तानो म्रियते, तर्हि तस्य भ्राता तां जायामुदुह्य स्वभ्रात्रे वंशमुत्पादयिष्यतीति।
\vakya वाढं, सप्त भ्रातर आसन्। प्रथमश्च विवाहं कृत्वा निःसन्तानो ममार।
\vakya द्वितीयस्तदा तस्य जायां लेभे, सोऽपि निःसन्तानो ममार।
\vakya ततस्तृतीयस्तां लेभे। इत्थमेव ते सप्त नराः सन्तानानत्यक्त्वा मम्रुः।
\vakya सर्वेषां पश्चात् सा योषिदपि ममार।
\vakya अतः पुनरुत्थाने सा तेषां कस्य भार्या भवेत्? यतस्ते सप्त तामुदूढवन्तः।
\vakya यीशुस्तदा तान् प्रत्यवादीत्, एतद्युगस्य सन्ताना उद्वहन्त्युदुह्यन्ते च,
\vakya ये त्वमुष्य युगस्य मृतोत्थानस्य च भागित्वमर्हन्तीति प्रतिपन्नास्ते नोद्वहन्ति नोदुह्यन्ते वा, न शक्यं हि तैः पुन र्मर्तुमपि।
\vakya वास्तवं स्वर्गदूतैस्तुल्यास्त ईश्वरस्य सन्तानाश्च जाताः, यतस्ते पुनरुत्थानस्य सन्तानाः।
\vakya मृतास्तु यदुत्थाप्यन्ते तत् स्तम्बवृत्तान्ते मोशिनापि सूचयाञ्चक्रे, यतस्तेन प्रभुरब्राहामस्येश्वर इस्हाकस्य चेश्वरो याकोबस्य चेश्वर इत्यभिधीयते।
\vakya ईश्वरो न मृतानामपि तु जीवितामीश्वरोऽस्ति, यतस्तस्मै सर्वे जीवन्ति।
\vakya शास्त्राध्यापकानां केचित् तदा प्रत्यूचुः, भो गुरो, सम्यगुक्तं भवता।
\vakya ततः प्रभृति पुनस्तं किमपि प्रष्टुं ते नोत्सेहिरे।
\stitle{ख्रीष्टो दायूदस्य पुत्रः।}
\vakya स तु तानवादीत्, जनाः ख्रीष्टं कथं दायूदस्य पुत्रं वदन्ति?
\vakya गीतग्रन्थे हि स्वयं दायूद इत्थं व्याहरति,
\begin{poem}
\startwithline “मम प्रभुमिदं वाक्यं बभाषे परमेश्वरः।
\vakya त्वच्छत्रून् पादपीठं ते यावन्न हि करोम्यहं।
\pline अवतिष्ठस्व तावत् त्वमासीनो मम दक्षिणे॥”
\end{poem}
\vakya अतएव स दायूदेन प्रभुरित्यभिधीयते, कथं तर्हि स तस्य पुत्रो भवेत्?
\vakya ततः परं स कृत्स्नस्य प्रजावृन्दस्य कर्णगोचरे स्वशिष्यानाह,
\vakya यूयं शास्त्राध्यापकेभ्यः सावधाना भवत, रोचयन्ति ते महापरिच्छदै र्विहारम्, आकाङ्क्षन्ति च हट्टेष्वभिवन्दनानि समाजगृहेषु श्रेष्ठासनानि भोज्येषु च श्रेष्ठस्थानानि।
\vakya ते विधवानां गृहाणि ग्रसन्ति, छलाच्च सुदीर्घं प्रार्थयन्ते। विचारे ते घोरतरं दण्डं लप्स्यन्ते\eoc