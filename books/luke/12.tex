\adhyAya
\stitle{कापट्यलोभादयोरधि च यीशोरुपदेशः।}
\vakya इतोमध्ये जननिवहा अयुतशः समागत्य परस्परं पादै र्मर्दयितुं प्रावर्तन्त। तदा स स्वशिष्यान् वक्तुमारेभे, यूयं प्रथमतः स्वरक्षार्थं फरीशिनां किण्वतः सावधानास्तिष्ठत। तद्धि कापट्यं।
\vakya नास्ति तु किमपि तादृशं तिरोहितं यन्नाविर्हितं भविष्यति, नापि तादृशं निगूढं यन्न ज्ञायिष्यते।
\vakya अतो यूयं तिमिरे यद्यदुक्तवन्तस्तद् गृहपृष्ठेषु घोषयिष्यते।
\vakya मदीयबन्धून् युष्मांस्त्वहं ब्रवीमि, मा भैष्ट तेभ्यो ये देहं घ्नन्ति तदुत्तरन्त्वधिकं किमपि कर्तुं न शक्नुवन्ति।
\vakya कस्मात्तु भेतव्यं तदहं युष्मान् बोधयिष्यामि। बिभीत तस्माद् यो हत्वा तदुत्तरं नरके निक्षेप्तुं सामर्थ्यविशिष्टः। युष्मानहं निश्चितं वदामि, तस्मादेव बिभीत।
\vakya चटकाः पञ्च किं न ताम्रखण्डद्वयेन विक्रीयन्ते? ईश्वरस्य समक्षन्तु तेषां नैकोऽप्यस्मृतः।
\vakya सन्तु च गणिताः सर्वेऽपि युष्मच्छिरसां कचाः। अतो मा बिभीत, यूयं बहुभ्यश्चटकेभ्यो विशिष्यध्वे।
\vakya युष्मांस्त्वहं ब्रवीमि, यः कश्चिन्मनुष्याणां समक्षं मामङ्गीकुरुते, तं मनुष्यपुत्रोऽपीश्वरस्य दूतानां समक्षमङ्गीकरिष्यते।
\vakya यस्तु मनुष्याणां समक्षं मां प्रत्याख्याति, स ईश्वरस्य दूतानां समक्षं प्रत्याख्यायिष्यते।
\vakya यः कश्चिच्च मनुष्यपुत्रस्य प्रतिकूलं वाक्यं व्याहरति तस्य क्षमिष्यते, यस्तु पवित्रमात्मानं परिनिन्दति तस्य नैव क्षमिष्यते।
\vakya यदा तु जनैः समाजानां शास्तॄणां कर्तॄणाञ्च समक्षं नायिष्यध्वे, कथं किं वा तदा प्रतिवदिष्यथ कथयिष्यथ वा, तच्चिन्तया माकुलीभवत।
\vakya यतो यद् वक्तव्यं तत् तस्मिन्नेव दण्डे पवित्र आत्मा युष्मान् शिक्षयिष्यति।
\vakya अथ जनताया मध्यात् कश्चित् तं जगाद, गुरो, मम भ्रातरं तथादिशतु यथा स मया सार्धं रिक्थं विभजेत्।
\vakya स तु तमब्रवीत्, हे मनुष्य, युवयोः प्राड्विवाकं विभक्तारं वा मां कोऽधिकृतवान्?
\vakya ततः स जनानवादीत्, सतर्का भवत सावधानाश्च सर्वस्माद्धनलोभात्, यतो धनाढ्यस्य जीवनं न तद्वित्तनिष्ठं।
\vakya स तेभ्यो दृष्टान्तकथामपि कथयामास, यथा, मनुष्यस्य कस्यचिद् धनिनो भूमिः प्रचुरशस्यादीनि द्रव्याण्युदपीपदत्।
\vakya तदा स मनसा व्यतर्कयत् किं करवाणि? यतो मम भूम्युत्पन्नद्रव्याणि सञ्चेतुं मम स्थानं नास्ति।
\vakya स पुनरब्रवीत्, इदं करिष्यामि, मम कुशूलान् भङ्क्त्वा महत्तरान् निर्माय तत्रैव मम सर्वाणि भूम्युत्पन्नद्रव्याणि वित्तानि च सञ्चेष्यामि,
\vakya मत्प्राणांश्च वदिष्यामि, भो प्राणाः, सन्ति वः प्रचुरवित्तानि बहुवर्षार्थं सञ्चितानि। विश्रामः क्रियतां भुज्यतां पीयताम् अमादः सेव्यतां।
\vakya ईश्वरस्तु तमाह, रे निर्बोध, रात्रावस्यां त्वं प्राणानां प्रत्यर्पणमाज्ञापयिष्यसे। कस्य तर्हि भविष्यति सर्वं तद् यत् सज्जीकृतं त्वया?
\vakya ईदृश्येव गतिस्तस्य य आत्मार्थं धनसञ्चयकारी न त्वीश्वरार्थं धनवान्।
\vakya स्वशिष्यांस्तु स जगाद, ततो हेतोरहं युष्मान् ब्रवीमि, किं भक्षिष्याम इति विचिन्त्य स्वप्राणानधि, किं वसिष्यामह इति विचिन्त्य स्वदेहमधि वा माकुलीभवत।
\vakya भक्ष्यात् प्राणा हि श्रेष्ठा वसनाच्च देहः श्रेष्ठः।
\vakya वायसानालोचयत, तै र्नोप्यते नापि कृत्यते, भाण्डागारं कुशूलो वा तेषां नास्ति, तथापीश्वरस्तान् पुष्णाति। विहङ्गमेभ्यो यूयं कतिगुणमधिकं विशिष्यध्वे?
\vakya चिन्तयित्वा वा युष्माकं केन स्ववयो हस्तमेकं वर्धयितुं शक्यते?
\vakya तद् यदि लघिष्ठमपि युष्माकमशक्यं, कथं तर्ह्यन्यान्यधि चिन्ताकुला भवथ?
\vakya शूशनाख्यानि क्षेत्रपुष्पाणि पर्यालोचयत, कथं तानि वर्धन्ते? न तानि श्रमं कुर्वते नापि सूत्राणि तन्वन्ति। युष्मांस्त्वहं ब्रवीमि, शलोमापि स्वकृत्स्नप्रतापे तेषामेकमिव न पर्यधीयत।
\vakya यदि त्वद्य क्षेत्रे वर्तमानं श्वश्चुल्ल्यां निक्षेप्तव्यं तृणमीदृशमीश्वरः परिधापयति, तर्हि भो स्तोकविश्वासिनः स कतिगुणमधिकं युष्मान् परिधापयिष्यति।
\vakya अतो यूयं किं भक्षिष्यथ किं पास्यथ वा तन्मानुसन्धध्वं संशयारूढा वा मा भवत।
\vakya जगतिस्थाः परजातीयजना हि सर्वाण्येतान्यनुसन्दधते। युष्मत्पिता तु जानीते यदिमानि सर्वाणि युष्माकमावश्यकानि।
\vakya ईश्वरस्य राज्यमेवान्विष्यत, तथा कृते सर्वाणीमान्यपि युष्मभ्यं प्रदायिष्यन्ते।
\vakya मा भैषीः क्षुद्र मेषव्रज, यतो युष्मभ्यं राज्यं दातुं युष्मत्पिता रोचितवान्। युष्माकं यद्यदास्ते तद् विक्रीय भिक्षां दत्त।
\vakya सज्जीकुरुत स्वार्थमजरान् मुद्राधारान् अक्षयं धनं स्वर्गे सञ्चितं, यत्र न चौर उपसर्पति न कीटो वा क्षिणोति।
\vakya यतो युष्माकं यत्र वित्तं तत्र युष्माकं चित्तमपि स्थास्यति।
\vakya युष्माकं कटयो बद्धा दीपिकाश्चोज्ज्वलास्तिष्ठन्तु,
\vakya यूयञ्च भवत सदृशास्तै र्जनै र्ये विवाहोत्सवात् स्वप्रभोः प्रस्थानकालमाकाङ्क्षन्तस्तथा तं प्रतीक्ष्यावतिष्ठन्ते, यथा तस्मिन्नायाते द्वारं ताडयति च तत्क्षणं तदर्थं द्वारमुद्घाटयेयुः।
\vakya धन्यास्ते दासाः प्रभुरागत्य यान् जाग्रत आसादयिष्यति। युष्मानहं सत्यं ब्रवीमि स कटिं बद्ध्वा तान् भोजनायोपवेशयिष्यत्युपागत्य च परिचरिष्यति।
\vakya स द्वितीये यामे तृतीये यामे वागत्य यानेवम्भूतान् आसादयिष्यति, त एव दासा धन्याः।
\vakya जानीत परन्त्विदं यद् यदि गृहस्वाम्यज्ञास्यद् आयाति चौरः कस्मिन् दण्ड इति, स तर्ह्यजागरिष्यत् कुड्यभेदञ्च स्वगेहस्य नासहिष्यत।
\vakya तद् यूयमपि ससज्जास्तिष्ठत, यत आयाति मनुष्यपुत्रस्तस्मिन्नेव दण्डे दण्डो यो युष्माभि र्नानुभूयते।
\vakya पित्रस्तदा तं पप्रच्छ, प्रभो, दृष्टान्तोऽयं भवतास्मभ्यं कथ्यते किंवा सर्वेभ्य एव?
\vakya प्रभुरब्रवीत् को नु खलु स विश्वस्तो बुद्धिमांश्च धनाध्यक्षो यो यथासमये निरूपितभक्ष्यवितरणाय स्वामिना स्वभृत्येष्वधिक्रियते?
\vakya धन्यः स दासः स्वामी यमागमनकाले तथैवाचरन्तमासादयिष्यति।
\vakya अहं युष्मान् सत्यं ब्रवीमि, स तं कृत्स्ने सर्वस्वेऽधिकरिष्यति।
\vakya मम स्वामी त्वागन्तुं विलम्बत इति मनसि ध्यात्वा दासः स यदि दासान् दासीश्च ताडयितुं, भोक्तुं पातुं प्रमत्तो भवितुञ्च प्रवर्तते,
\vakya स तर्हि दिने यस्मिन् नापेक्षते दण्डञ्च यं स न जनाति, तदैव दासस्य स्वामी तस्योपस्थाय तं द्विधाकरिष्यति भाग्यञ्च तस्याविश्वासिभिः सार्धं निरूपयिष्यति।
\vakya यो दासस्तु स्वामिनोऽभीष्टं विज्ञाय न ससज्जोऽभूत् न वा तदभीष्टानुरूपमाचरितवान् स प्रभूतं प्रहारिष्यते।
\vakya यस्तु न विज्ञाय प्रहारयोग्यमाचारं कृतवान् स स्तोकं प्रहारिष्यते। अपि च यस्मै कस्मैचिद् बहु दत्तं बहु तत्सन्निधावनुसन्धायिष्यते, यस्मिंश्च बहु समर्पितं, जनैः सोऽधिकतरं याचिष्यते।
\vakya मेदिन्यामग्निं निक्षेप्तुमहमागतः, स चेदधुना प्रज्वलितस्तर्ह्यपरं किं वाञ्छामि?
\vakya स्नानेन केनचित्तु मया स्नातव्यं तच्च यावदसिद्धं तावत् कीदृक् सङ्कोच्ये!
\vakya युष्माभिः किमनुमीयते यदहं पृथिव्यामैक्यं निधातुमागतः? युष्मानहं ब्रवीमि, न तथापि तु विभेदमेव।
\vakya यतोऽद्यप्रभृत्येकस्मिन् गृहे पञ्च जना विभिन्ना भविष्यन्ति, त्रयो द्वयो र्द्वौ च त्रयाणां विरुद्धं।
\vakya विभिन्ना हि भविष्यन्ति पिता पुत्रस्य, पुत्रः पितु र्विरुद्धं, माता दुहितु र्दुहिता मातु र्विरुद्धं, श्वश्रूः पुत्रवध्वाः पुत्रवधूः श्वश्र्वाश्च विरुद्धम्।
\vakya अपि च स जननिवहान् जगाद, पश्चिमदिश उद्गच्छन्तं मेघं दृष्ट्वैव यूयं वदथ, वृष्टिरायातीति, तथैव जायते च।
\vakya दक्षिणवायु र्वातीति निरूप्य च वदथ, ग्रीष्मो भविष्यतीति, तच्च जायते।
\vakya रे कपटिनः, यूयं भूतलस्याकाशस्य च रूपं निर्णेतुं जानीथ, कथं तर्हि कालमिमं न निर्णयथ?
\vakya कथञ्च स्वयमेव यथार्थं विचारं न कुरुथ?
\vakya शास्तुः समीपं गच्छंस्त्वं यावत् स्वप्रतिपक्षेण सार्धं व्रजसि, तावत् पथि तस्मादुद्धारं साधयितुं यतस्व, नोचेत् स त्वां बलेन प्राड्विवाकस्यान्तिकं नेष्यति, प्राड्विवाकश्च त्वां पदातौ समर्पयिष्यति, पदातिश्च त्वां कारायां क्षेप्स्यति।
\vakya त्वामहं ब्रवीमि, शेषोऽपि कपर्दको यावत् त्वया न शोधितस्तावत् तत्स्थानान्न निर्गमिष्यते\eoc