\adhyAya
\stitle{विधवाया दृष्टान्तः।}
\vakya अथ सर्वदा प्रार्थना कर्तव्या, निरुत्साहः परिहर्तव्य एतदधि स तेभ्यो दृष्टान्तकथामपि कथयन् बभाषे,
\vakya कस्मिंश्चिन्नगरे प्राड्विवाक एक आसीत्, स ईश्वरान्नाबिभेत्, मनुष्यमपि नामानयत्।
\vakya तस्मिन्नेव नगरे विधवा काचिदासीत्, सा तस्य समीपमुपस्थाय तमवदत्, अन्यायं प्रतीकृत्य मत्प्रतिपक्षान्मामुद्धर।
\vakya अत्र स दीर्घकालमसम्मतोऽतिष्ठत्। ततः परन्तु मनस्यवदत्, यद्यपीश्वरान्न बिभेमि मनुष्यमपि न मानयामि,
\vakya तथापीयं विधवा मां क्लिश्नातीतिहेतोरन्यायं प्रतीकृत्येमामुद्धरिष्यामि नोचेदियं चरममागत्य मुष्ट्याघातेन मां कलङ्कयिष्यति।
\vakya प्रभुः पुन र्बभाषे, तेनायाथार्थिकेन प्राड्विवाकेन यदुच्यते तद् युष्माभिः श्रूयतां।
\vakya ईश्वरेण तर्हि किं नोद्धारिष्यन्तेऽन्यायं प्रतीकृत्य तस्यात्मवरितनरा ये दिवारात्रं तमुद्दिश्य क्रोशन्ति? स किमुदासीनत्वात् तान् प्रति सहनशीलः?
\vakya युष्मानहं ब्रवीमि, स सत्वरमन्यायं प्रतीकृत्य तानुद्धरिष्यति। प्रत्युत मनुष्यपुत्र आगमनकाले किं मेदिन्यां विश्वासमवाप्स्यति?
\stitle{फरीशिनां शुल्कादायिनाञ्च दृष्टान्तः।}
\vakya अथ धार्मिका वयमित्यात्मनिष्ठा योऽपरान् तुच्छीकुर्वन्ति, तादृशान् कांश्चिन्मनुष्यानुद्दिश्य स दृष्टान्तकथामिमां कथयामास,
\vakya नरौ द्वौ प्रार्थयितुं धर्मधाम जग्मतुः। तयोरेकतरः फरीशी, अन्यतरः शुल्कादायी।
\vakya फरीशी स्थित्वा स्वगतमित्थं प्रार्थयत, भो ईश्वर, त्वां स्तौमि, यतो नास्म्यहं सदृशोऽन्यै र्मनुष्यैः परस्वापहारकैरन्यायाचारिभि र्व्यभिचारिभिश्च नापि वा सदृशोऽनेन शुल्कादायिना।
\vakya प्रतिसप्ताहमहं द्विकृत्व उपवसामि मदीयसर्वायस्य दशमांशमुपहरामि च।
\vakya स शुल्कादायी तु दूरे तिष्ठन् स्वर्गं प्रत्यूर्ध्वं निरीक्षितुमपि नैच्छत् अपि तु वक्षःस्थलमाहत्य व्याहत्, भो ईश्वर, प्रसीद मे पापिनः।
\vakya युष्मानहं ब्रवीमि, अमुमपहायैषो धार्मिकीकृतः स्वगृहमवततार। यतो यः कश्चिदात्मानमुच्चीकरोति स नीचीकारिष्यते, यस्त्वात्मानं नीचीकरोति स उच्चीकारिष्यते।
\stitle{शिशूनामधि शिक्षा।}
\vakya अथ शिशवोऽपि तस्यान्तिकमानिन्यिरे यत् तेन स्पृश्येरन्। तद् दृष्ट्वा शिष्यास्तानभर्त्सयन्।
\vakya यीशुस्तु तान् समीपमाहूय बभाषे, मत्समीपमागमिष्यतः शिशूननुमन्यध्वं मा वारयत, यत ईश्वरस्य राज्यमीदृशानामेव।
\vakya युष्मानहं सत्यं ब्रवीमि, यः कच्चिन्न शिशुरिवेश्वरस्य राज्यं गृह्णाति स तत्र नैव प्रवेक्ष्यति।
\stitle{धनासक्तिविषयकशिक्षा।}
\vakya अथ कश्चिदध्यक्षस्तं पप्रच्छ, भो सद्गुरो, किं कृत्वाहमनन्तजीवनस्याधिकारी भविष्यामि?
\vakya यीशुस्तु तमब्रवीत्, किमर्थं मां सन्तं वदसि? सत् कोऽप्यन्यो नास्ति, केवल ईश्वरः।
\vakya त्वमाज्ञा जानासि, व्यभिचारं मा कुरु, नरहत्यां मा कुरु, चौर्यं मा कुरु, मृणासाक्ष्यं मा देहि, स्वपितरं स्वमातरञ्च सम्मन्यस्व।
\vakya स जगाद, पालितं मया सर्वमेतदाबाल्यात्।
\vakya तच्छ्रुत्वा यीशुस्तमब्रवीत्, अधुनापि तवासम्पूर्णं किञ्चिदस्ति। सर्वं तव यद्यदस्ति देहि विक्रीय दरिद्रेभ्यः, तथा कृते स्वर्गे तव धनं स्थास्यति, ततः परमागत्य मामनुव्रज।
\vakya एतच्छ्रुत्वा स शोकार्तो बभूव, यतः सोऽतीव धनवानासीत्।
\vakya यीशुस्तु तं शोकार्तीभूतं दृष्ट्वा बभाषे, वित्तानां स्वामिनः कीदृगायासेनेश्वरस्य राज्यं प्रवेक्ष्यन्ति।
\vakya सुसाध्यं हि धनवत ईश्वरराज्यप्रवेशात् सूचीच्छिद्रेणोष्ट्रगमनं।
\vakya श्रोतारस्तदा जगदुः, कस्तर्हि तरितुं शक्नोति?
\vakya सोऽब्रवीत्, मनुष्याणां यद्यदसाध्यं तदीश्वरस्य साध्यं।
\vakya पित्रस्तदा व्याजहार, पश्यतु वयं सर्वं त्यक्त्वा भवन्तमनुव्रजितवन्तः।
\vakya स तान् जगद, युष्मानहं सत्यं ब्रवीमि, ईश्वरराज्यस्य कृते गृहं जनकौ वा भ्रातॄन् वा जायां वा सन्तानान् वा
\vakya त्यक्त्वा य इहकाले तद्बहुगुणम् आगामियुगे चानन्तं जीवनं न लप्स्यते तादृशः कोऽपि नास्ति।
\vakya अथ स द्वादशशिष्यान् स्वान्तिकमानीयाब्रवीत्, पश्यत वयं यिरूशालेमं गच्छामः, भाववादिभिश्च मनुष्यपुत्रमधि यद्यल्लिखितं तत् सर्वं सेत्स्यति।
\vakya यतः स परजातीयेभ्यः समर्पयिष्यत उपहासिष्यते न्यक्कारिष्यतेऽवनिष्ठीविष्यते च,
\vakya ते च कशाभिराहत्य तं घातयिष्यन्ति, तृतीये दिने च स पुनरुत्थास्यतीति।
\vakya तैस्त्वेतासां कथानां किमपि नाबोधि तद् वचनञ्च तेभ्यस्तिरोहितमासीत्, यद्यदकथ्यत च तत् तै र्नाज्ञायि।
\stitle{अन्धाय नेत्रदानम्।}
\vakya अथ तस्मिन् यिरीहोः समीपम् उपस्थिते कश्चिदन्धो भिक्षमाणः पथपार्श्व उपविष्ट आसीत्।
\vakya स सन्निध्या गच्छतो जननिवहस्य शब्दं श्रुत्वा पप्रच्छः किं भूतमिति।
\vakya जनास्तमूचुः, नासरतीयो यीशुरनेन मार्गेण गच्छति।
\vakya ततः स क्रोशन् जगाद, भो दायूदस्य पुत्र यीशो, मामनुकम्पतां।
\vakya अनेनाग्रगामिनस्तं तर्जयन्तोऽवागिषुः मौनी भव, स तु बहुगुणाधिकमुच्चैरवदत्, भो दायूदस्य पुत्र, मामनुकम्पतां।
\vakya यीशुस्तदा स्थित्वा स्वसमीपं तस्यानयनमादिदेश।
\vakya ततः परं तस्मिन्नुपागते स तं पप्रच्छ, त्वदर्थं मया कर्तव्यं किं वाञ्छसि? स जगाद, प्रभो, दृक्‌शक्तिं यथा प्राप्नुयां।
\vakya ततो यीशुस्तमाह, दृक्‌शक्तिं गृहाण, तव विश्वासस्त्वां तारयामास।
\vakya स तत्क्षणं दृक्‌शक्तिं लेभे, ईश्वरञ्च प्रशंसंस्तमनुजगाम। कृत्स्नो जजनिवहश्च तद् दृष्ट्वेश्वरं तुष्टुवे\eoc