\adhyAya
\stitle{मनःपरिवर्तयितुमावश्यकत्वमधि शिक्षा।}
\vakya तस्मिन् काले केचिदुपस्थाय तस्मै तेषां गालीलीयानां, संवादं ददु र्येषां निजशोणितं यज्ञीयेन सार्धं पीलातेन मिश्रितं।
\vakya यीशुस्तदा प्रतिभाषमाणस्तान् अब्रवीत्, यूयं किं मन्यध्वे यत् तेन दुर्भगत्वेन ते गालीलीयः सर्वेभ्यो गालीलीयेभ्योऽधिकपापिनः पतिपन्नाः?
\vakya युष्मानहं ब्रवीमि, न तथा। प्रत्युत यदि न प्रत्यावर्तध्वे सर्वे तर्हि तथैव विनंक्ष्यथ।
\vakya अथवा शीलोहस्थेनोच्चगृहेण पतता ये हतास्तेऽष्टादशजना यिरूशालेमनिवासिभ्यः सर्वमनुष्येभ्योऽपराधिनः प्रतिपन्ना इति किं मन्यध्वे?
\vakya युष्मानहं ब्रवीमि, न तथा, प्रत्युत यदि न प्रत्यावर्तध्वे, सर्वे तर्हि तथैव विनंक्ष्यथ।
\vakya अनन्तरं स दृष्टान्तकथामिमां कथयामास, कस्यचित् स्वद्राक्षाक्षेत्रे रोपित उडुम्बरवृक्ष आसीत्, स चागत्य तस्मिन् फलमन्वैष्यत्, न तु प्राप।
\vakya ततः स द्राक्षाक्षेत्ररक्षकं जगाद, पश्य त्रीन् वर्षानहमागत्यास्मिन्नुडुम्बरवृक्षे फलमन्वेष्यामि न तु प्रप्नोमि, त्वमेनमुच्छिन्धि, किमर्थमहं भूमिमपि विकारयति?
\vakya स तु तं प्रत्यब्रवीत्, प्रभो, वर्षमेनमपीमं स्थातुमनुजानातु, अहं तावत् तत्परितः खनिष्याम्यालवालं स्थापयिष्यामि च,
\vakya किंस्वित् स फलं फलिष्यति, नोचेत् ततः परं स भवतोच्छेत्स्यते।
\stitle{विश्रामवारप्रतिपालनाय शिक्षा।}
\vakya अथ स विश्रामवारे कस्मिंश्चित् समाजगृह उपादिशत्।
\vakya पश्य च तत्राष्टादशवर्षान् यावद् दौर्बल्यजनकात्मनाविष्टा काचिद् योषिदासीत्। सा भुग्ना सम्पूर्णरूपमृजू र्भवितुं नाशक्नोत्।
\vakya यीशुस्तां दृष्ट्वाजुहाव जगाद च, नारि, तव दौर्बल्यात् त्वं मुक्ता। इत्युक्त्वा स तस्यां हस्तावर्पयामास।
\vakya ततः सा तत्क्षणमृजूर्बभूवेश्वरं तुष्टुवे च। समाजाध्यक्षस्तु विश्रामवारे यीशुनारोग्यसाधने क्रुद्ध्वा जननिवहं जगाद, परिश्रमो येषु विधेयस्तादृशानि षट् दिनानि सन्ति, आरोग्यलाभाय तेष्येवागच्छत, न तु विश्रामवारे।
\vakya ततः प्रभुः प्रतिभाषमाणस्तमाह, रे कपटिनः, किं न युष्माकमेकैको विश्रामवारे स्वगां स्वगर्दभं वा गवादनीतो मुक्त्वा तोयं पाययितुं नयति?
\vakya इयन्त्वब्राहामस्य या तनयाष्टादशवर्षान् यावत् शैतानेन बद्धा, एतस्या विश्रामवारे बन्धनादस्मान्मोचनं किं न विधेयम्?
\vakya तस्मिन्नेतत् कथितवति तद्विपक्षाः सर्वे लज्जापन्नाः कृत्स्नो जननिवहस्तु तेन साधितेषु यशस्विषु कर्मस्वाननन्द।
\stitle{सर्षपबीजकिण्वयोर्दृष्टान्तौ।}
\vakya अथ स बभाषे, ईश्वरस्य राज्यं केन सदृशं? केन वा तदुपमास्ये?
\vakya तत् सर्षपबीजेन तेन सदृशं यदादाय कश्चिन्मुनुष्यः स्वोद्यान उवाप। तच्च वृद्ध्वा महातरु र्बभूव, विहायसो विहङ्गमाश्च तस्य शाखासु न्यवसन्।
\vakya पुनरपि स जगाद, ईश्वरस्य राज्यं केनोपमास्ये?
\vakya तत् किण्वेन तेन सदृशं यत् कयाचिद् योषितादाय गोधूमचूर्णानां द्रोणत्रयपरिमितानां मध्ये निह्नुतं येन च परिणामे सञ्जातं तत्साकल्यं किण्वभावितं।
\stitle{परित्रातुम् प्राणपणकरणमधि शिक्षा।}
\vakya अथ स नगरान्नगरं ग्रामाच्च ग्रामं गत्वोपादिशन् यिरूशालेममुद्दिश्य यात्रामकरोत्।
\vakya एकदा कश्चित् तम् अप्राक्षीत्, प्रभो, त्राणपात्राणि किमल्पानि?
\vakya स तं जगाद, सङ्गीर्णेन गोपुरेण प्रवेष्टुं प्राणपणेन चेष्टध्वे, यतोऽहं युष्मान् ब्रवीमि, बहवः प्रवेष्टुं यतिष्यन्ते, न तु शक्ष्यन्ति।
\vakya गृहस्वामिनोत्थाय द्वारे बद्धे बहिस्तिष्ठन्तो यूयं द्वारमाहन्तुमारभ्य यदा वक्ष्यथ, प्रभो प्रभो, अस्मदर्थं द्वारं मुञ्चत्विति, तेन त्विदं प्रतिवक्ष्यध्वे, कुत्रत्या यूयं तन्न जानामीति,
\vakya तदा यूयमिदं वक्तुं प्रवर्तिष्यध्वे, भवतः समक्षं वयं भोजनपाने कृतवन्तोऽस्मदीयचत्वरेषु च भवानुपदिष्टवानिति।
\vakya स तु व्याहरिष्यति, युष्मानहं ब्रवीमि, कुत्रया यूयं तन्न जानामि, मत्तोऽपसृत्य तिष्ठत रे अधर्माचारिणः सर्व इति।
\vakya तत्र रोदनं दन्तैर्दन्तघर्षणञ्च भविष्यते युष्माकं वीक्षमाणानाम् अब्राहामम् इस्‌हाकं याकोबञ्च भाववादिनश्च सर्वान् ईश्वरस्य राज्ये (सुखासीनान्), युष्मांस्तु बहिर्निक्षिप्तान्।
\vakya अपरञ्च पूर्वपश्चिमोत्तरदक्षिणदिग्भ्यो जना आगत्येश्वरस्य राज्ये भोजनायोपवेक्ष्यन्ति।
\vakya पश्यत चान्त्यैः कैश्चित् प्रथमै भवितव्यम् प्रथमैश्च कैश्चिदन्त्यैः।
\vakya तस्मिन् दिने फरीशिनः केचित् तदन्तिकमागत्य तमब्रुवन्, प्रस्थाय दूरं याहि, यतो हेरोदस्त्वां जिघांसति।
\vakya स तु तान् जगाद, यूयं गत्वा तं भूरिमायं वदत, पश्याहमद्य श्वश्च भूतान् निःसारयाम्यारोग्यदानानि साधयामि च, तृतीये दिने च सिद्ध्वकर्मा भविष्यामि।
\vakya प्रत्युताद्य श्वश्च परश्वश्च मया यात्रा कर्तव्या। यतो यिरूशालेमाद् बहि र्भाववादिनो विनाशो न सम्भवति।
\vakya हा यिरूशालेम, हा यिरूशालेम, हा भाववादिनां हन्त्रि नराणाञ्च त्वत्समीपं प्रहितानां प्रस्तराघातिनि पुरि, पक्षयोरधः स्वशावकान् सङ्गृह्णती कुक्कुटीवाहं कतिकृत्वस्तव शावकान् सङ्गृहीतुं वाञ्छितवान्, यूयन्तु न सम्मताः।
\vakya पश्यत युष्माकं भवनं युष्मत्कृत उत्सन्नं विहीयते। युष्मांश्चाहं सत्यं ब्रवीमि, धन्यः स यः प्रभो र्नाम्नायातीति यदा यूयं वक्ष्यत, स कालो यावदनुपस्थितस्तावन्मां पुन र्न द्रक्ष्यथ\eoc