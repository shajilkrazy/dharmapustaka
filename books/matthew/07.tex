\adhyAya
\stitle{परस्य विचारकरणकथनम्।}
\vakya मा विचारं कुरुत, तेनैव युष्माकं विचारो न कारिष्यते।
\vakya यतो यूयं यद्विचारेण परस्य दोषान् निर्णयथ, तद्विचारेणैव युष्माकं दोषा निर्णयिष्यन्ते।
\vakya येन परिमाणेन च मिमीध्वे, तेनैव युष्मादर्थमपि मायिष्यते।
\vakya कुतश्च तं स्वभ्रातुश्चक्षुःस्थं शूककणं निरीक्षसे, तव चक्षुःस्थन्तु गेहकाष्ठं नावधारयसि? तव चक्षुषः शूककणस्य मयोद्धरणमनुमन्यस्वेति वा स्वभ्रातरं कथं गदिष्यसि? तव चक्षुषि तु गेहकाष्ठमवतिष्ठते।
\vakya कपटिन प्रथमं स्वचक्षुषस्तद् गेहकाष्ठमुद्धर, ततः स्वभ्रातुश्चक्षुषः शूककणस्योद्धरणार्थं स्पष्टं द्रक्ष्यसि।
\vakya युष्माकं पवित्रं द्रव्यं सारमेयेभ्यो मा दत्त। मापि स्वमुक्ताः शूकरेभ्यो निक्षिपत, यतस्तेन ते चरणैस्ता मर्दिष्यन्ति परावृत्य च युष्मान् विदारयिष्यन्ति।
\stitle{प्रार्थनाकथनम्।}
\vakya याचध्वं, तेन युष्मभ्यं दायिष्यते। अन्विच्छत, तेनासादयिष्यथ। द्वारमाहत तेन युष्मभ्यम् उद्घाटयिष्यते।
\vakya यतो यः कश्चिद् याचते स लभते, यश्चान्विच्छति स आसादयति, यश्च द्वारमाहन्ति तदर्थमुद्घाट्यते।
\vakya युष्मत्सु तादृशः को वा मानवो विद्यते यः स्वपुत्रेण पूपं याचितस्तस्मै प्रस्तरं दास्यति,
\vakya मीनं वा याचितः सर्पं दास्यति?
\vakya तद् दुर्जना अपि यूयं चेत् स्वसन्तानेभ्यो हितदानानि वितरितुं जानीथ, तर्हि किमधिकं युष्माकं स्वर्गस्थः पिता स्वयाचकेभ्यो हितानि वितरिष्यति।
\vakya तत् सर्वस्मिन् विषये युष्मान् प्रति मनुष्यै र्यादृशमाचरितुं यूयमभिवाञ्छथ, यूयमपि तान् प्रति तादृशमेवाचरत। यद् एतदेव व्यवस्थाया भाववादिग्रन्थस्य च तत्वं।
\stitle{स्वर्गपथे गमनकथनम्।}
\vakya प्रविशत सङ्कीर्णेनैव गोपुरेण, यतो येन विनाशं नीयते विशालं तद् गोपुरं स मार्गः पृथुतरश्च, तेनैव च बहवः प्रविशन्ति।
\vakya जीवनं हि येन नीयते सङ्कीर्णं तद् गोपुरं मार्गश्च स सङ्कुचितः, स च यैः प्राप्यते तेऽल्पाः।
\vakya यूयं कूटभाववादिभ्यः सावधाना भवत, ते मेषवेशा युष्मत्समीपमागच्छन्ति, अन्तरे तु ते ग्रसिष्णवो वृकाः।
\vakya तत्फलैरेव यूयं तान् निश्चेष्यथ। कण्टकेभ्यः किं द्राक्षा गोक्षुरकेभ्यो वोडूम्बरफलानि सञ्चीयन्ते?
\vakya एवमेव सर्वः सुवृक्षः सुफलान्युत्पादयति, कुवृक्षस्तु कुफलान्येवोत्पादयति।
\vakya सुवृक्षः कुफनान्युत्पादयितुं न शक्नोति,कुवृक्षश्च सुफलान्युत्पादयितुं न शक्नोति।
\vakya येन सुफलं नोत्पाद्यते, तादृशः सर्वो वृक्ष उच्छिद्यते वह्नौ प्रक्षिप्यते च।
\vakya अतो यूयं तेषां फलैरेव तान् निश्चेष्यथ।
\vakya ये मां प्रभो प्रभो इत्यभिभाषन्ते न ते सर्वे स्वर्गराज्यं प्रवेक्ष्यन्ति; यस्तु मम स्वर्गस्थपितुरभीष्टमाचरति स एव प्रवेक्ष्यति।
\vakya अमुष्मिन् दिने बहवो मां वक्ष्यन्ति, प्रभो प्रभो न किं भवतो नाम्नास्माभि र्भावोक्ति र्घोषिता? नापि किं भवतो नाम्नास्माभि र्भूता निःसारिताः? न चापि किं भवतो नाम्नास्माभि र्बह्व प्रभावसिद्धाः क्रिया साधिताः?
\vakya तान् तु तदाहं स्पष्ट वदिष्यामि, युष्मानहं न कदाचित् ज्ञातवान्, अधर्माचारिणो यूयं मत्तोऽपसरत।
\vakya अतो यः कश्चिन्ममैतानि वाक्यानि निशम्याचरति तमह तेन बुद्धिमता नरेणोपमास्ये येन पाषाणोपरि स्वगेहं निरमायि,
\vakya परं वृष्ट्या पतित्वा प्रवाहैरागत्य वायुभिः प्रोह्य तद् गेहमाक्रान्तं, न तु पपात, यतस्तस्य भित्तिमूलं पाषाणे स्थापितमासीत्।
\vakya यः कश्चित् तु ममैतानि वाक्यानि निशम्य नाचरति स तेन मूढेन नरेण सदृशः प्रतिपत्स्यते, येन सिकतोपरि स्वगेहं निरमायि।
\vakya यत् परं वृष्ट्य पतित्वा प्रवाहैरागत्य वायुभिः प्रोह्य तद् गेहमाहतं, तदा तत् पपात्, अभूच्च तत्पतनं घोरतरम्।
\vakya अनन्तरं यीशुनैतेषु वाक्येषु समापितेषु जननिवहास्तस्य शिक्षामाश्चर्यां मेनिरे, यतः स क्षमतापन्न इव, न तु शास्त्राध्यापका इव तान् अशिक्षयत्\eoc