\adhyAya
\stitle{पर्वतोपरि प्रभु यीशोरुपदेशनम्।}
\vakya तान् मानवनिवहान् दृष्ट्वा स गिरिमारुरोह। तत्र तस्मिन्नुपविष्टे शिष्यास्तस्य समीपमागमन्।
\vakya तदा स वक्त्रमुद्घाट्य तेभ्यः शिक्षां दातुं प्रवृत्तोऽब्रवीत्-
\stitle{स्वर्गराज्यस्य प्रजानिर्णयः।}
\vakya “दीनात्मानो धन्याः, यतः स्वर्गराज्यं तेषामेव।
\vakya शोकार्त्ता धन्याः, यतस्ते सान्त्वयिष्यन्ते।
\vakya मृदुशीला धन्याः, यतस्ते दायांशवत् क्षितिं लप्स्यन्ते।
\vakya धार्मिकतां बुभुक्षवः पिपासवश्च धन्याः, यतस्ते परितर्प्स्यन्ति।
\vakya कृपावन्तो धन्याः, यतस्ते कृपामवाप्स्यन्ति।
\vakya शुचिहृदो धन्याः, यतस्त ईश्वरं द्रक्ष्यन्ति।
\vakya सन्धिकारिणो धन्याः, यतस्त ईश्वरस्य पुत्रा इत्यभिधायिष्यन्ते।
\vakya धर्महेतुनोपद्रुता धन्याः, यतः स्वर्गराज्यं तेषामेव।
\vakya धन्या यूयं यदा मनुष्या मदर्थं युष्मान् निन्दन्त्युपद्रवन्ति च युष्मद्विरुद्धं मृषा सर्वविधां कुकथां व्याहरन्ति च।
\vakya आनन्दतोल्लसत च, यतः सञ्चितं स्वर्गे युष्माकं प्रभूतं पारितोषिकं। वास्तवं युष्माकं पूर्वं ये भाववादिन आसन्, तांस्ते तथैवोपाद्रवन्।
\vakya यूयं पृथिव्या लवणं। लवणं तु यदि विस्वादं जायते तर्हि केनोपायेन तत् स्वादयुक्तं कारिष्यते? ततः परं तत् कास्मिंश्चित् कार्य न युज्यते, केवलं बहि र्निक्षेपणे मानवपदै र्मर्द्दने च युज्यते।
\vakya यूयं जगतो दीपः। पर्वतोपरि स्थितिं नगरं प्रच्छन्नमवस्थातुं न शक्नोति।
\vakya मनुष्याश्च दीपिकां प्रज्वाल्य न द्रोणस्याधस्ताद् अपि तु दीपाधारस्योपरि स्थापयन्ति सा च गृहेऽवस्थितानां सर्वेषां राजते।
\vakya तथैव युष्माकं दीप्ति र्मनुष्याणां समक्षं विराजतां, तथा कृते युष्माकं सत्क्रिया दृष्ट्वा ते युष्माकं स्वर्गस्थं पितरं स्तोष्यन्ति।”
\stitle{स्वर्गराज्यस्य व्यवस्थोत्कर्षः।}
\vakya “व्यवस्थाया भाववादिनां वा मोचनायाहमागत इति मानुमिमीध्वं।
\vakya न मोचनाय, प्रत्युत पूरणायाहमागतः, यतोऽहं युष्मान् सत्यं ब्रवीमि, यावत् व्योममेदिन्योरत्ययो न भविष्यति तावत् व्यवस्थाया एका मात्रैको विन्दु र्वा नैवात्येष्यति सर्वमेव हि सेत्स्यति।
\vakya अतो यः कश्चिदेतासां क्षोदिष्ठानामाज्ञानामेकां मुञ्चति मनुष्यांश्च तदनुरूपं शिक्षयति, स स्वर्गराज्ये क्षोदिष्ठोऽभिधायिष्यते।
\vakya यस्तु ता आचरति शिक्षयति च स स्वर्गराज्ये महानभिधायिष्यते। यतोऽहं युष्मान् ब्रवीमि, युष्माकं धार्मिकता यदि शास्त्राधापकानं फरीशिनाञ्च धार्मिकतातोऽधिकं प्रभूता न स्यात्, तर्हि यूयं स्वर्गराज्यं नैव प्रवेक्ष्यथ।
\vakya त्वं नरहत्यां माऽकार्षीः, यस्तु नरहत्यां करोति स धर्माधिकरणे शासनीयो भविष्यतीति प्राचीनेभ्यः कथितमासीत् युष्माभिस्तच्छ्रुतम्।
\vakya अहन्तु युष्मान् ब्रवीमि, यः कश्चिदकारणं स्वभ्रात्रे क्रुध्यति स धर्माधिकरणे शासनीयो भविष्यति। यश्च स्वभ्रातरं निर्बोधस्त्वमिति वदति स सभायां शासनीयो भविष्यति। यश्च वदति मूढस्त्वमिति, सोऽग्निमये नरके शासनीयो भविष्यति।
\vakya अतस्त्वया स्वकीयोपहारे यज्ञवेदिमानीते तव भ्रातु र्मनसि त्वद्विरुद्धा कथा विद्यत इति तत्र चेत् स्मरसि,
\vakya तर्हि तत्र स्वकीयोपहारं यज्ञवेद्याः सम्मुखे विहाय याहि, प्रथमं स्वभ्रात्रा सम्मिलितो भव, ततः परमागत्य स्वकीयोपहारं निवेदय।
\vakya स्वप्रतिपक्षेण सह यावन्मार्गे वर्त्तसे, तावत्, तूर्णं तस्यानुकूलो भव, नो चेत् प्रतिपक्षेण विचारयितु र्हस्ते समर्पितस्त्वं विचारयित्रा पदातिकस्य हस्ते समर्पितः कारायां निक्षेप्स्यसे।
\vakya त्वामहं सत्यं वदामि, शेषोऽपि कपर्दको यावत् त्वया न शोधितस्तावत् तत्स्थानान्न निर्गमिष्यते।
\vakya त्वं व्यभिचारं माऽकार्षीरिति प्राचीनेभ्यः कथितमासीत्, युष्माभिस्तच्छ्रुतम्।
\vakya अहन्तु युष्मान् ब्रवीमि यः कश्चित् कामुकभावेन योषितं प्रति दृक्‌पातं करोति, स तावता स्वहृदये तया सह व्यभिचारं कृतवान्।
\vakya यदि तु तव दक्षिणं नेत्रं तव स्खलनकारणं भवेत्, तर्हि तदुत्पाट्य दूरं निक्षिप। यतस्तवैकस्याङ्गस्य नाशः, न तु कृत्स्नशरीरस्य नरके निपातस्तद्धि तव हितं।
\vakya यदि वा तव दक्षिणो हस्तस्तव स्खलनकारणं भवेत्, तर्हि तं छित्त्वा दूरं निक्षिप। यतस्तवैकस्याङ्गस्य नाशः, न तु कृत्स्नशरीरस्य नरके निपातस्तद्धि तव हितम्।
\vakya पुनश्च कथितमासीत्, यः कश्चित् स्वभार्यां त्यजति, स तस्यै त्यागपत्रं ददातु।
\vakya अहन्तु युष्मान् वदामि, य कश्चिद् व्यभिचारदोषादन्येन हेतुना स्वभार्यां त्यजति, स तां व्यभिचारं कारयति। यश्च त्यक्तां योषितमुद्वहति स व्यभिचारं करोति।
\vakya पुनस्त्वं मृषा शपथं माऽकार्षीः, स्वशपथफलन्तु प्रभवे दास्यमीति प्राचीनेभ्यः कथितं, युष्माभिस्तच्छ्रुतम्।
\vakya अहन्तु युष्मान् ब्रवीमि, यूयं सर्वथा मा शपध्वं; न स्वर्गेण, यतः स ईश्वरस्य सिंहासनं;
\vakya न मेदिन्या, यतः सा तस्य पादपीठं, न यिरूशालेमेन, यतस्तन्महतो राज्ञो नगरं।
\vakya निजशिरसा वा मा शपस्व, यथः शिरोरुह एकोऽपि सितीकर्तुम् असितीकर्तुं वा त्वया न शक्यते।
\vakya युष्माकन्तु संलपने, यत् तथा तत् तथैव यन्न तन्नैव भवतु; यदेतदधिकं तत् पापात्मतो जातम्।
\vakya चक्षुषो विनिमये चक्षु र्दन्तस्य विनिमये च दन्त इति कथितमासीत् युष्माभिस्तच्छ्रुतम्।
\vakya अहन्तु युष्मान् ब्रवीमि, दुर्जनस्य प्रतिरोधो न कर्तव्यः, अपितु यः कश्चित् तव दक्षिणकपोले चपेटाघातं करोति, तं प्रत्यन्यतरं कपोलमपि व्याघोटय।
\vakya यश्च धर्माधिकरणे त्वया विवदमानस्तवाङ्गाच्छादकं जिहीर्षति, तस्य कृते प्रावारमपि त्यज।
\vakya यश्च विनावेतनं क्रोशमेकं गमनाय त्वां हरति, गच्छ तेन क्रोशद्वयं।
\vakya यश्च त्वत्तो याचते तस्मै देहि। यश्च त्वत्त ऋणमभिवाञ्छति तं प्रति विमुखो मा भव।
\vakya त्वं स्वनिकटस्थं प्रति प्रेम, स्वशत्रुं प्रति तु द्वेषं करिष्यसीति कथितमासीत् युष्माभिस्तच्छ्रुतम्।
\vakya अहन्तु युष्मान् ब्रवीमि, यूयं स्वशत्रून् प्रति प्रेम कुरुत, ये युष्मान् शपन्ति तान् आशिषं वदत, ये युष्मान् द्विषन्ति तेषां हितमाचरत, ये युष्मान् अपवदन्त्युपद्रवन्ति च तेषां कृते प्राथनां कुरुत।
\vakya तथा कृते युष्माकं यः स्वर्गस्थः पिता दुर्जनानां सुजनानाञ्चोपरि स्वसूर्यमुदाययति धार्मिकाणामधार्मिकाणाञ्चोपरि तोयं वर्षयति च, तस्य पुत्रा भविष्यथ।
\vakya यतो यदि केवलं युष्मत्प्रेमकारिणः प्रति प्रेम कुरुथ तर्हि किं पारितोषिकं लप्स्यध्वे?
\vakya शुल्कादायिनोऽपि किं न तदेव कुर्वन्ति? केवलं स्वभ्रातॄन् वा यद्यभिवदथ, तर्हि विशिष्टं किं कुरुथ? शुल्कादायिनोऽपि किं न तदेव कुर्वन्ति?
\vakya अतो युष्माकं स्वर्गस्थः पिता यथा सिद्धोऽस्ति, यूयं तथैव सिद्धा भवत”\eoc