\adhyAya
\stitle{स्नापकयोहनस्य प्रचारादिकार्यम्।}
\vakya तस्मिन् काले स्नापको योहन उपतिष्ठते। स यिहूदियाया मरौ घोषणां कुर्वन्नवदत्,
\vakya मनांसि परावर्तयत यतः स्वर्गराज्यं समीपमागतं।
\vakya वास्तवं स हि भाववादिना यिशायाहेन तिर्द्दिष्टो नरो यथा,
\begin{poem}
\startwithline “मरौ घोषयतः प्रोच्चैरस्त्ययं कस्यचिद् रवः।
\pline प्रभोः संस्कुरुताध्वानं विधद्धं तत्मृती ऋजूः॥”
\end{poem}
\vakya तस्य तु योहनस्योष्ट्रलोमनिर्मितं वसनं कटौ बद्धा चर्मनिर्मिता पटुका चास्तां पतङ्गा वन्यमधु च तस्य खाद्यान्यासन्।
\vakya तदानीं यिरूशालेमस्य कृत्स्नयिहूदियादेशस्य यर्द्दनान्तिकस्य
\vakya कृत्स्नजनपदस्य च मानवास्तत्समीपं गत्वा स्वपापानि स्वीकुर्वाणा यर्द्दने तेनास्नाप्यन्त।
\vakya अनन्तरं तदीयस्नानार्थिनः फरीशिनां सद्दूकिनाञ्च बहवो नरा आगच्छन्तीति दृष्ट्वा स तान् जगाद, “भो सर्पवंशाः, यूयं भाविक्रोधात् पलायितुं केनादिष्टाः?
\vakya अतो मनःपरावर्तनस्य योग्यानि फलानि फलत।
\vakya अस्माकं पिताब्राहामो विद्यत इति वाक्यप्रयोगं वा स्वान्तरे मैव कल्पयत। यतोऽहं युष्मान् ब्रवीमि, एतेभ्यः प्रस्तरेभ्योऽब्राहामस्य कृते सन्तानान्युत्पादयितुमीश्वरः शक्तिमान्।
\vakya परन्त्वधुनैव पादपानां मूलेषु कुठारो लगन्नास्ते, अतो यः कश्चित् पादपः सत्फलं न फलति, स उच्छिद्यते वह्नौ निक्षिप्यते च।
\vakya युष्मान् अहं मनःपरावर्तनाय तोये स्नापयामि; मत्पश्चात्तु य आगच्छति स मत्तो बलवान् तस्योपोनहौ वोढुमहं न योग्यः, स युष्मान् पवित्र आत्मनि वह्नौ च स्नापयिष्यति।
\vakya तस्य हस्ते च सूर्पो विद्यते, स स्वीयखलं सम्यक् संशोधयिष्यति निजगोधूमान् कुशूले सङ्ग्रहीष्यति च तुषांस्त्वनिर्वाणवह्निना दाहयिष्यति।”
\stitle{प्रभो र्यीशोः स्नापनं परीक्षा च।}
\vakya तदा योहनेनाहं स्नापयितव्य इति मत्वा यीशु र्गालीलतो यर्द्दने तत्समीपमाजगाम।
\vakya योहनस्तु तं वारयन्नब्रवीत्, “भवता मम स्नापनं प्रयोजनीयं, भवांस्तु मदान्तिकमागच्छति?”
\vakya ततो यीशुस्तं प्रत्यवादीत्, “आधुनानुमन्यस्व यस्मादित्थं सर्वधर्मसाधनमावयो र्युज्यते। तदा स तमनुमन्यते।”
\vakya स्नापितस्तु यीशुस्तूर्णं तोयादुत्थितः। पश्य तदा तस्य कृते स्वर्ग उद्घाटित ईश्वरस्यात्मा च कपोत इवावरोहन् तमाश्रयंश्च तेन दृष्टः
\vakya स्वर्गत सञ्जाता चैका वाणी बभाषे, “मम प्रिय पुत्रो ऽयम् अस्मिन्नेवाहं प्रीतः”\eoc