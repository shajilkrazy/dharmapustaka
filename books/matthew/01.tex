\adhyAya
\stitle{प्रभो र्यीशुख्रीष्टस्य वंशावलिपत्रम्।}
\vakya अथ यीशोः ख्रीष्टस्य जन्मपत्रम्। स तु दायूदस्य सन्तानः। दायूदोऽब्राहामस्य सन्तानः।
\vakya अब्राहाम इस्‌हाकं जनयामास। इस्‌हाको याकोबं जनयामास। याकोबो यिहूदां तस्य भ्रातॄंश्च जनयामास।
\vakya थामरो गर्भे यिहूदाः पेरसं सेरहञ्च जनयामास। पेरसो हिष्रोणं जनयामास। हिष्रोणोऽरामं जनयामास।
\vakya अरामोऽम्मीनादाबं जनयामास। अम्मीनादाबो नहशोनं जनयामास। नहशोनः सल्मोनं जनयामास।
\vakya सल्मोनो राहबो गर्भे बोयसं जनयामास। बोयसो रूतो गर्भे ओबेदं जनयामास। ओबदो यिशयं जनयामास।
\vakya यिशयो राजानं दायूदं जनयामास। राजा दायूदो मृतस्योरियस्य जायाया गर्भे शलोमानं जनयामास। शलोमा रहबियामं जनयामास।
\vakya रहबियामो ऽबियं जनयामास। अबिय आसां जनयामास।
\vakya आसा यिहोशाफटं जनयामस। यिहोशाफटो योरामं जनयामास। योराम उषियं जनयामास। उषियो योथमं जनयामस।
\vakya योथम आहसं जनयामास। आहसो हिष्कियं जनयामास।
\vakya हिष्कियो मनःशिं जनयामास। मनःशिरामोनं जनयामास।
\vakya आमोनो योशियं जनयामास। योशियो यिकनियं तस्य भ्रातॄंश्च जनयामास। तदा बाबिलीयप्रवासोऽभूत्। बाबिलीयप्रवासात् परं यिकनियः शल्टीयेलं जनयामास।
\vakya शल्टीयेलः सरुब्बाबिलं जनयामास। सरुब्बाबिलोऽबीहूदं जनयामास।
\vakya अबीहूद इलियाकीमं जनयामास। इलियाकीम आसोरं जनयामास। आसोरः सादोकं जनयामास। सादोक आखीमं जनयामास।
\vakya आखीम इलीहूदं जनयामास। इलीहूद इलियासरं जनयामास।
\vakya इलियासरो मत्तनं जनयामास। मत्तनो याकोबं जनयामास। याकोबश्च मरियमः पतिं योषेफं जनयामास।
\vakya तस्या एव गर्भे ख्रीष्टोऽर्थतोऽभिषिक्त इत्यभिधो यीशु र्जज्ञे।
\vakya इत्थमब्राहामाद् दायूदं यावत् साकल्येन चतुर्दशपुरुषाः, दायूदात् पुन र्बाबिलीयप्रवासं यावत् चतुर्दशपुरुषाः, बाबिलीयप्रवासात् पुनः ख्रीष्टं यावत् चतुर्दशपुरुषाश्च।
\stitle{प्रभो र्यीशोर्जन्मविवरणम्।}
\vakya यीशोः ख्रीष्टस्य तु जन्मेत्थमभूत्। तस्य मातरि मरियमि योषेफाय वाग्दत्तायां सत्यां तयोः सङ्गमात् प्राक् सा पवित्रेणात्मना गर्भवतीत्याविष्कृतं।
\vakya तस्याः पति र्योषेफस्तु धार्म्मिकोऽथच तां साधारणनिन्दाभाजनं कर्तुमनिच्छन् तां गुप्तं परित्यक्तुमकल्पयत्। तस्मै त्वेतच्चिन्तयते प्रभो र्दूतः स्वप्ने दर्शनं दत्त्वावादीत्,
\vakya भो दायूदस्य सुत योषेफ स्वभार्याया मरियमः परिग्रहणान्माऽभैषीः, यतस्तस्या गर्भफलं पवित्रादात्मनः सम्भूतम्।
\vakya सा हि पुत्रं प्रसविष्यते त्वञ्च तस्य नाम यीशुरर्थतस्त्रातेति करिष्यसि, यतः स एव स्वप्रजास्तासां पापेभ्यस्तारयिष्यति।
\vakya सर्वमेतत्तु तथा बभूव यथा भाववादिना व्याहृतं प्रभोरिदं वाक्यं सिद्धिं गच्छेत्, यथा,
\begin{poem}
\startwithvakya “कुमारीं गर्भिणीं पश्य सा पुत्रं प्रसविष्यते।
\pline इम्मानूयेल इत्येव तस्य नाम भविष्यति॥”
\end{poem}
\vakya नाम्नोऽस्य तात्पर्यम् अस्मत् सङ्गीश्वर इति। अनन्तरं निद्रोत्थितो योषेफो यथादिष्टं दूतेन तथैवाकार्षीत्, फलतः स्वभार्यां पर्यग्रहीत्,
\vakya सा परं यावत् प्रथमसुतं न सुषुवे तावत् स तां न प्राजानात्। तस्य नाम च यीशुरित्यकार्षीत्\eoc