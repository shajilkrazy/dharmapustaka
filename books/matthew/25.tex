\adhyAya

\stitle{दशकन्यानां दृष्टान्तकथनं।}
\vakya स्वर्गराज्यं तदा भविष्यति सदृशं दशकन्याभि र्याः स्वप्रदीपान् आदाय वरं प्रत्युद्गन्तुं निरिताः।
\vakya तासां मध्यादासन् पञ्च बुद्धिमत्यः पञ्च च बुद्धिहीनाः।
\vakya यास्तु बुद्धिहीनास्ताः स्वप्रदीपान् आदाय स्वसार्धं तैलं नाददिरे।
\vakya बुद्धिमत्यश्च पुनः पात्रेषु तैलं निधाय स्वप्रदीपैः सहाददिरे।
\vakya विलम्बमाने तु वरे निद्रालुतां गत्वा सर्वा एव प्रसुषुपुः।
\vakya अर्धरात्रे त्वयम् उच्चध्वनि र्बभूव, पश्यतायाति वरो यूयं तत्प्रत्युद्गमनाय निर्गच्छत।
\vakya कन्यास्तास्तदा सर्वा उत्थाय स्वप्रदीपान् संश्चक्रुः।
\vakya बुद्धिहीनास्तदा बुद्धिमती र्जगदुः, दत्तास्मभ्यं किञ्चिद् युष्मदीयतैलाद् यतो निर्वायन्त्यस्माकं प्रदीपाः।
\vakya बुद्धिमत्योऽनेन प्रतिजगदुः, नैवं कर्तव्यम्। न पर्याप्तं भविष्यति तदस्माकञ्च युष्माकञ्च कृते। क्रीणीत वरं यूयं विक्रेतॄणां समीपं गत्वा स्वार्थं।
\vakya तास्तु क्रयार्थं यावदपगच्छन्ति वरस्तावदुपतस्थे, याः सज्जीकृतास्ताश्च तेन सह विवाहोत्सवीयगृहं प्रविविशु र्द्वारञ्च रुरुधे।
\vakya ततः परम् अन्यास्ताः कन्या आगत्य जगदुः, प्रभो प्रभो, मोचयतु द्वारमस्मत्कृते।
\vakya स तु प्रतिबभाषे, युष्मानहं सत्यं ब्रवीमि नाहं जाने युष्मान्।
\vakya अतो यूयं जागृत, यतो यूयं दिवसं तं दण्डञ्च तं न जानीथ, आयाति यदा मनुष्यपुत्रः।
\stitle{दूरदेशगामिन उत्तमर्णस्य तदीयदासानां दृष्टान्तकथनं।}
\vakya फलतः स नरेण तेन सद़ृशः, यो विदेशगमनार्थं यात्राकाले निजदासान् आहूय स्वसम्पत्तिं तेषु समार्पयत्।
\vakya स एकस्मै गोणीः पञ्च मुद्रापूर्णा अपरस्मै गोण्यौ द्वे, अन्यस्मै गोणीमेकामित्थं यस्य यथा सामर्थ्यं तस्मै तदनुरूपमर्थं ददौ ततश्च तूर्णां प्रतस्थे।
\vakya ततो लब्ध्वा येन मुद्रागोण्यः पञ्च स ताभि र्बाणिज्यं कुर्वन्नपराः पञ्च गोणीरुपार्जयामास।
\vakya लब्ध्वे येन गोण्या द्वे स तथैवापरं गोणीद्वयमुपार्जयामास।
\vakya लब्धा तु येनैका गत्वा स भूमिं खनित्वा स्वप्रभो र्मुद्रास्तत्र न्यगूहत्।
\vakya काले दीर्घेऽतीते दासानां तेषां प्रभुरागत्य तैः सह गणयति।
\vakya लब्ध्वा येन गोण्यः पञ्च स तदोपस्थायापराश्च गोणीः पञ्चानीय व्याजहार, प्रभो समर्पिता भवता मयि गोण्यः पञ्च। पश्यतु ताभिर्मयापरा गोण्यः पञ्चोपार्जिताः।
\vakya तस्य स्वामी तं जगाद, साधु भद्र विश्वस्त दास, आसीस्त्वं स्वल्पेषु विश्वस्तः, बहुषु मयाधिकारिष्यसे, गत्वाभ्यन्तरं भव भागी स्वप्रभोरानन्दस्य।
\vakya अनन्तरं लब्ध्वे येन गोण्यौ द्वे सोऽप्युपस्थाय व्याजहार, प्रभो समर्पिते भवता मयि गोण्यौ द्वे, पश्यतु ताभ्यां मयापरं गोणीद्वयमुपार्जितं।
\vakya तस्य स्वामी तं जगाद, साधु भद्र विश्वस्त दास, आसीस्त्वं स्वल्पेषु विश्वस्तः बहुषु मयाधिकारिष्यसे। गत्वाभ्यन्तरं भव भागी स्वप्रभोरानन्दस्य।
\vakya ततः परं लब्ध्वा येन गोण्येका, सोऽप्युपस्थाय व्याजहार, प्रभो भवान् नरः कठोरः, नोप्तवान् यत्र बीजानि कृन्तति तत्र शस्यं, न यत्र विकीर्णवान् सञ्चिनोति तत्र,
\vakya अतोऽहं भीतो गत्वा भूमौ गोणीं भवतो गूढवान्, पश्यतु यद् भवतस्तद् गृह्णातु।
\vakya तस्य स्वामी तु प्रतिगदन् तमब्रवीत् रे दुष्ट मन्द दास, त्वयेदं भो न्वज्ञायि यन्मया नोप्तानि यत्र बीजानि कृत्यते तत्र शस्यं, न यत्र विकीर्णं सञ्चीयते तत्रेति?
\vakya तथा सति आसंस्त्वया मदीयमुद्रा बणिक्ष्वर्पयितव्याः, तथा कृतेऽहमेवोपस्थाय निजस्वं मम सवृद्धि आदायिष्यं।
\vakya अतो यूयं गोणीं तामस्माद् अपहृत्य दशगोणीनाम् अधिकारिणे दत्त।
\vakya यतो यस्य कस्यचिद् आस्ते तस्मै दायिष्यते तस्य च प्राचुर्यं भविष्यति, यस्य तु नास्ते तस्य यदस्ति तदपि तस्माद् अपहारिष्यते।
\vakya दासं परन्त्विमम् अनुपयोगिनं तिमिरे बहिःस्थे निक्षिपत, तत्र रोदनं दन्तै र्दन्तघर्षणञ्च सम्भविष्यतः।
\stitle{विचारदिनविवरणञ्च।}
\vakya यदा तु स्वप्रतापान्वितो मनुष्यपुत्रः सहितः सर्वै र्दूतै पवित्रैरागमिष्यति,
\vakya स तदा स्वप्रतापसिंहासन उपवेक्ष्यति जातयश्च सर्वास्तस्य समक्षम् एकत्रीकारिष्यन्ते। यथा च पालरक्षकः पृथक् करोति मेषेभ्यश्छागान्, स तथा विभज्य मनुष्यांस्तान् पृथक् करिष्यति,
\vakya स्थापयिष्यति च मेषान् स्वदक्षिणे त्वजान् स्ववामे।
\vakya ततः परं राजा वदिष्यति स्वदक्षिणे स्थितान्, भो मत्पितुराशीर्वादभाजनान्यायात, दायवद् भुङ्क्त राज्यमुपकल्पितं युष्मदर्थमाजगत्स्थापनात्।
\vakya यतो मयि क्षुत्क्षामे यूयं मह्यं भक्ष्यम् अदत्त, तृषातुरे माम् अपाययत, अतिथौ माम् अन्वगृह्णीत,
\vakya विवस्त्रे मां वासांसि पर्यधापयत, अस्वस्थे मां पर्यपश्यत, कारानिहिते माम् अभ्यागच्छत।
\vakya धार्मिकास्ते तदा तं प्रतिभाषिष्यन्ते, प्रभो दृष्टो वा कदा भवान् अस्माभिः क्षुत्क्षामो भक्ष्येण तर्पितश्च? दृष्टो वा कदा तृषातुरः पायितश्च?
\vakya दृष्टो वा कदा भवान् अतिथिरस्माभिरनुगृहीतश्च? दृष्टो वा कदा विवस्त्रो वासांसि परिधापितश्च?
\vakya लक्षितो वा कदा भवान् अस्वस्थः कारानिहितो वास्माभिः परिदृष्टश्च?
\vakya राजा तांस्तदा प्रतिवदिष्यति, युष्मानहं सत्यं ब्रवीमि, भ्रातॄणां ममैतेषां क्षोदिष्ठानाम् एकं प्रति युष्माभि र्यावद् अकारि तावन्मां प्रत्यकारि।
\vakya ततः परं स वामे स्थितानपीत्थम् आलपिष्यति, रे शप्ता यूयं मत्तो दूरमपगच्छतानन्तम् अग्निम् उपकल्पितं दियाबलस्य तदीयदूतानाञ्च कृते।
\vakya यतो मयि क्षुत्क्षामे यूयं मह्यं भक्ष्यं नादत्त, तृषातुरे मां नापाययत,
\vakya अतिथौ मां नान्वगृह्णीत, विवस्त्रे मां वासो न पर्यधापयत, अस्वस्थे कारानिहिते वा मां न पर्यपश्यत।
\vakya तेऽपि तदोत्तरं कृत्वा तं गदिष्यन्ति, प्रभो, कदा वा भवन्तं क्षुत्क्षामं वा तृषातुरं वातिथिं वा विवस्त्रं वास्वस्थं वा कारानिहितं वा दृष्ट्वास्माभि र्भवतः परिचर्या न कृत्वा?
\vakya स तांस्तदा प्रतिभाषिष्यते, युष्मानहं सत्यं ब्रवीमि, युष्माभिः क्षोदिष्ठानामेतेषाम् एकं प्रति यावन्नाकारि तावन्मामेव प्रति नाकारि।
\vakya ततोऽपगमिष्यन्तीमे दण्डं भजितुमनन्तं धार्मिकास्त्वनन्तं जीवनम्\eoc