\adhyAya
\stitle{अशौचमधि उपदेशः।}
\vakya तदा यिरूशालेमत आगताः शास्त्राध्यापकाः फरीशिनश्च यीशोरन्तिकमुपस्थाय जगदुः,
\vakya भवतः शिष्याः कथं प्राचीनानां परम्परागतां शिक्षां लङ्घयन्ति? यतस्ते भोजनकाले हस्तान् न प्रक्षालयन्ति।
\vakya स तु तान् प्रतिजगाद, यूयमपि कथं युष्माकं परम्परागतशिक्षायामनुरागवशाद् ईश्वरस्याज्ञां लङ्घयथ?
\vakya यत ईश्वर इदमाज्ञापयामास, त्वं स्वपितरं स्वमातरञ्च सम्मन्यस्वेति, अपि च, यः स्वपितरं मातरं वा शपेत स बध्यो भविष्यतीति।
\vakya यूयन्तु वदथ, यः स्वपित्रे स्वमात्रे वा कथयति मत्तो यद्दानेन तवोपकारः समभविष्यत् तदीश्वरायोपहृतमित्यादि;
\vakya स्वपितरं स्वमातरं वेत्थं स न कथञ्चन सम्मंस्यते। स्वशिक्षायां परम्परागतायामनुरागाद् इत्थं युष्माभिरीश्वरस्याज्ञा व्यर्थीकृता।
\vakya भो कपटिनः, युष्मानधि यिशायाहः समीचीनामिमां भावोक्तिं कथितवान्, यथा,
\begin{poem}
\startwithvakya “मनुष्याः सकला एत आयान्त्यास्यै र्मदन्तिकं।
\pline अधरैरेव सम्मानम् आचरन्ति च मां प्रति।
\pline किन्त्वन्तःकरणं तेषां मत्तो दूरमवस्थितं॥
\vakya अलीकार्थन्त्विमे सर्वे मम कुर्वन्ति सेवनं।
\pline धर्मशिक्षाच्छलेनैव शिक्षयन्तो नृणां विधीन्॥”
\end{poem}
\vakya अनन्तरं स जननिवहं स्वसमीपमाहूय जगाद, श्रुत्वा युष्माभि र्बुध्यतां।
\vakya यन्मुखं प्रविशति न तन्मनुष्यमशुचीकरोति। यत्तु मुखान्निःसरति तदेव मनुष्यमशुचीकरोति।
\vakya तस्य शिष्यास्तदोपागत्य जगदुः, अपि ज्ञायते भवता यद् वाक्यं तच्छ्रुत्वा फरीशिनः स्खलिता इति?
\vakya स तु प्रतिजगाद, उन्मूलयिष्यते यः कश्चिद् वृक्षो मम पित्रा न रोपितः।
\vakya यूयममून् परित्यजत, तेऽन्धानामन्धाः पथप्रदर्शकाः। अन्धश्चेदन्धं नयेत्, तर्ह्युभौ गर्ते पतिष्यतः।
\vakya पित्रस्तु तं प्रतिजगाद, भवानस्मानुपमां तां बोधयितुमर्हति।
\vakya यीशुस्तदाब्रवीत्, यूयं किमधुनाप्यबोधाः?
\vakya युष्माभिरधुनापि किं न बुध्यते यद् यत् किञ्चन मुखं प्रविशति तदुदरमाश्रयति शौचकूपे च निरस्यते,
\vakya यत्तु मखान्निःसरति, तद्धृदयान्निरेति, तदेव च मनुष्यमशुचीकरोति।
\vakya यतो हृदयात् कुतर्का नरहत्या परदारगमनं व्यभिचारश्चौर्यक्रिया मृषासाक्ष्यं धर्मनिन्दा च बहूशो निःसरन्ति।
\vakya एतान्येव मनुष्यमशुचीकुर्वन्ति, नाशुचीकरोति पुन र्मनुष्यमप्रक्षालितै र्हस्तैराहारकरणम्।
\stitle{यीशुना कस्यैचिद् भूतग्रस्तबालिकायै आरोग्यदानं, चतुःसहस्रलोकेभ्यो भोज्यदानञ्च।}
\vakya स्थानात् तस्मात् प्रस्थाय यीशु र्जगाम प्रदेशं सोरसीदोनयोः।
\vakya पश्य च कनानीया काचिन्नारी तत्परिसीमतो निर्गत्य तमुद्दिश्य क्रोशन्ती जगाद, प्रभो दायूदस्य सन्तान, मामनुकम्पतां। दुहिता मम भूतावेशवशाद् अतीव क्लिश्यते।
\vakya स तु तां प्रत्यवादीन्नैकमपि वाक्यम्। शिष्याः पुनस्तस्यान्तिकमुपस्थाय प्रार्थयमाना अब्रुवन्, विसृजत्वमूम्, यतः सास्माकं पश्चात् क्रोशति।
\vakya स तदा प्रतिजगाद, प्रहितोऽहं केवलमिस्रायेलकुलस्य हारितान् मेषान् प्रति नापरं प्रति।
\vakya सा तु तमुपागम्य प्रणिपपात जगाद च, प्रभो, ममोपकरोतु।
\vakya स प्रत्यवादीत्, न भद्रं भक्ष्यं बालकानां हृत्वा शुनामभिमुखं तत् प्रक्षेप्तु।
\vakya सानेन जगाद, सत्यं प्रभो, यत् श्वानः स्वामिनां भोज्यमञ्चाद् भ्रंशतीर्भक्ष्यफेलीः खादन्ति।
\vakya यीशुस्तदा तां प्रत्यवादीत्, भो नारि, महांस्तव विश्वासः, सम्भवतु तथा तव यथाभीष्टं। निरामयाभूच्च तस्या दुहिता तस्मादेव दण्डात्।
\vakya स्थानात् तस्मात् प्रस्थाय यीशु र्गालीलीयसमुद्रस्य तटमागत्य पर्वतमारुह्य तत्रोपविवेश।
\vakya महान्तो जननिवहाश्च खञ्जान् अन्धान् मूकान् हीनाङ्गान् अपरांश्च बहून् सार्धमानीय यीशोश्चरणसमीपे निचिक्षिपुः, स च तान् निरामयांश्चकार।
\vakya इत्थं मूकान् भाषमाणान्, हीनाङ्गान् स्वस्थान्, खञ्जान् पर्यटतोऽन्धांश्च प्राप्तदृष्टीन् निरीक्ष्य जननिवहा विस्मयं गत्वेस्रायेलस्येश्वरं तुष्टुवुः।
\stitle{सप्तपूपाल्पमीनेन चतुःसहस्रपुंसां भोजनम्।}
\vakya अनन्तरं यीशुः शिष्यान् स्वसमीपमाहूय जगाद, मनुष्येष्वेतेषु कारुण्यं ममोत्पद्यते, यतस्त आदिनत्रयान्मम सन्निधाववतिष्ठन्ते भक्ष्यन्तु तेषां किमपि नास्ति। मार्गे ते मूर्च्छां यास्यन्तीति भयादहं तान् अकृताहारान् विस्रष्टुं नेच्छामि।
\vakya तदा शिष्यास्तमूचुः, कुतोऽस्माभि र्लभ्याः स्थानेऽत्र निर्जने तावन्तः पूपा यैर्जननिवहमीदृङ्महान्तं तर्पयिष्यामः?
\vakya यीशुस्तान् पप्रच्छ, पूपाः कति युष्माकं विद्यन्ते? ते जगदुः सप्त स्वल्पे मत्स्याश्च क्षुद्राः।
\vakya ततः स जननिवहान् भुव्युपवेशनमाज्ञापयामास,
\vakya तांश्च सप्तपूपान् मत्स्यांश्चादाय धन्यवादवाचनपूर्वकं भङ्क्त्वा स्वशिष्येभ्योऽददात्, ते च मनुष्येभ्योऽददन्।
\vakya ततः सर्वे भुक्त्वा तृप्तवन्तो भग्नांशाना शेषेण च सप्तपेटकान् पूरयित्वाददिरे।
\vakya योषितो बालकांश्च विहाय भोक्तारस्ते पुरुषाश्चतुःसहस्राण्यासन्।
\vakya अनन्तरं स जननिवहान् विसृज्य नौकामारुह्य मग्दलायाः सीमानमागतम्\eoc