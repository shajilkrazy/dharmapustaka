\adhyAya
\stitle{पीलातस्य करे ख्रीष्टार्पणं।}
\vakya ततः परं जाते पुनः प्रातःकाले मुख्ययाजका जनानां प्राचीनाश्च सर्वे यीशो र्विरुद्धं तस्य बधार्थं मन्त्रणां चक्रुर्बद्ध्वा
\vakya च तमपनीय देशाधिपतौ पन्तीये पीलाते समर्पयामासुः।
\stitle{ईष्करियोतीययिहूदाकर्तृकः आत्मवधः।}
\vakya कृता यीशो र्दण्डाज्ञेति दृष्ट्वा तस्य समर्पयिता यिहूदास्तदानुतप्य तास्त्रिंशद् रौप्यमुद्रा मुख्ययाजकेषु प्राचीनेषु च प्रत्यर्पयन् बभाषे,
\vakya शोणितं निरपराधं समर्प्य पापं मयाकारीति। ते त्वब्रुवन्, किमनेनास्माकं? त्वयैवालोच्यतां।
\vakya स तदा रौप्यमुद्रास्ता मन्दिरे निक्षिप्य प्रतस्थे गत्वा चात्मानं प्रतिलम्ब्य ममार।
\vakya ततो मुख्ययाजकास्ता मुद्रा आदाय जगदुः, भाण्डागारे स्थापनमेतासामविधेयं यतः शोणितस्य मूल्यमेतत्।
\vakya अतस्ते मन्त्रणां कृत्वा ताभि र्विदेशिनां मृतानां समाधिस्थानं कुम्भकारस्य क्षेत्रम् अक्रैषुः।
\vakya तद्धेतुना क्षेत्रस्य तस्याद्यापि प्रबलं शोणितक्षेत्रमिति नाम सम्भूतम्।
\vakya इत्थं भाववादिना यिरमियाहेण व्याहृतमिदं वाक्यं सिद्धिं गतं, यथा, “इस्रायेलः सुतानाम् आदेशाद् यस्य मूल्यं निरूपितं, तस्य मूल्यवतो मूल्यं त्रिंशन्मुद्रामाणं रूप्यं
\vakya तैरादाय सदाप्रभु र्मां यथादिक्षत् तथा कुम्भकारस्य क्षेत्रेऽर्पितमिति।”
\stitle{देशाध्यक्षसमक्षं यीशोर्विचारः।}
\vakya अनन्तरं यीशु र्देशाधिपतेः समक्षम् अतिष्ठत्। स तु देशाधिपतिस्तं पप्रच्छ त्वं किं यिहूदीयानां राजा? यीशुस्तमवादीत्, भवान् व्याहरति।
\vakya मुख्ययाजकैस्तु प्राचीनैश्च क्रियमाणे तस्याभियोगे स किमपि न प्रत्यभाषत।
\vakya तदा पीलातस्तं ब्रवीति, न किं शृणोषि तव विरुद्धममी साक्ष्यवचांसि कति वदन्ति?
\vakya स तु तस्मै नैकस्यापि वचनस्योत्तरम् अदात्। अनेन देशाधिपतिरतीवाश्चर्यं मेने।
\vakya रीतिरियञ्चासीत् यद् देशाधिपतिः प्रतिवर्षं पर्वणि तस्मिन् जनानां कृते तैरभीप्सितम् नरमेकं कारागुप्तं मोचयतीति।
\vakya प्रसिद्धो नर एकस्तेषां तदा बारब्बा इतिनामकः कारागुप्त आसीत्।
\vakya तेष्वतः समागतेषु पीलातस्तान् अपृच्छत्, यूयं मत्तः कतरस्य मोचनं वाञ्छथ? बारब्बाः किंवा ख्रीष्ट इति विख्यातस्य यीशोः?
\vakya यतो देशाधिपतिनाज्ञायत स यन्मात्सर्यात् तैः समर्पितः।
\vakya अनन्तरं यदा स विचारासन उपविष्ट आसीत् तस्य पत्नी तदा प्रेष्यं प्रहित्य तं जगाद, धर्मवता तेन सह व्यापारः कोऽपि भवतो मा भवतु यतस्तस्य कृतेऽद्य स्वप्नदर्शने सम्भूता मम यातना बहुतरा।
\vakya यथा तु जननिवहो बारब्बां याचेत यीशुञ्च नाशयेत् मुख्ययाजकाः प्राचीनाश्च तांस्तथा प्रावर्तयन्। 
\vakya देशाधिपतिस्ततस्तान् पप्रछ युष्मदिच्छातस्तयो र्द्वयोः कतरो मया मोचयितव्यः? ते जगदुः, बारब्बाः।
\vakya पीलातस्तान् अब्रवीत्, खीष्टाभिधं तर्हि तं यीशुं प्रति मया किं कर्तव्यम्? सर्वे तमब्रुवन् स क्रुशमारोप्यताम्।
\vakya देशाधिपति र्बभाषे, कथमेतत्? किमपराद्धं तेन? अनेन तेऽधिकं क्रोशन्तोऽवदन्, स क्रुशमारोप्यताम्।
\vakya तदा यत्नो मम विफलः प्रत्युताधिकः कलहो जायत इति दृष्ट्वा पीलातस्तोयमादाय जननिवहस्य समक्षं हस्तौ प्रक्षाल्य बभाषे, धर्मवतोऽस्य शोणितपातने निर्दोषोऽहं युष्माभिरालोच्यतामिति।
\vakya कृत्स्नः प्रजानिवहस्तु तं प्रतिबभाषे, तस्य शोणितम् अस्माकम् अम्मत्सन्तानानाञ्च शिरःसु वर्ततां।
\vakya तेषां कृते स तदा बारब्बां मोचयामास यीशुन्तु कशाभिः प्रहार्य क्रुशारोपणार्थं समर्पयामास।
\vakya देशाधिपतेः सैनिकनरा यीशुं तदा राजहर्म्यस्याभ्यन्तरं नीत्वा सैन्यदलं कृत्स्नं तस्य परितः समागमयामासुः।
\vakya वस्त्राणि तस्य च मोचयित्वा तं लोहितवर्णं प्रावारं परिधापयामासुः
\vakya कण्टकैश्च स्रजं निर्माय तस्य शिरसि निदधुस्तस्य दक्षिणहस्तोपरि च नलं समर्पयामासुरनन्तरं तस्य समक्षं जानुपातं कुर्वन्त उपहासेन तमवदन्, यिहूदीयानां राजन् प्रणामः।
\vakya पुनश्च तस्मिन् निष्ठीव्य तं नलमादाय तस्य शिरोऽताडयन्।
\vakya इत्थं तम् उपहस्य तं प्रावारं मोचयित्वा तस्य स्ववासांसि परिधाप्य च क्रुशम् आरोपयितुं तम् अपनिन्युः।
\stitle{यीशोः क्रुशारोपणं मृत्युश्च।}
\vakya निर्गच्छन्तस्तु ते शिमोननामकं कुरीणीयं नरमेकमासादयामासु र्दध्रुश्च तं, कुशं तस्य वोढुं वेतनं विना।
\vakya अनन्तरं गल्गथार्थतः कपालस्थलमित्यभिधं स्थानमुपस्थाय पानीयार्थं ते तस्मै पित्तमिश्रितम् अम्लरसं ददुः।
\vakya स तु तमास्वाद्य पातुं नाङ्गीचक्रे।
\vakya अनन्तरं ते तं क्रुशमारोप्य गुटिकापातेन तस्य वासांसि मिथो विभेजिरे। इत्थं भाववादिनोक्तेयं कथा सिद्धिं गता, यथा,
\begin{poem}
\startwithline “मामकीनानि वस्त्राणि स्वमध्ये विभजन्ति ते।
\pline मत्परिच्छदलिप्सातो गुटिकां पातदन्ति च॥”
\end{poem}
\vakya अनन्तरं ते तत्रोपविश्य तम् अरक्षन्।
\vakya तस्य शिरसश्चोर्ध्वे ते बबन्धुरिदं तस्याभियोगलेख्यं यथा, यिहूदिनां राजा यीशुरयम्।
\vakya तदा च दस्यू द्वो तेन सार्धं क्रुशे आरोप्येताम्, एकतरस्तस्य दक्षिणे वामे चान्यतरः।
\vakya अनन्तरं ये मनुष्या मार्गेण तेनाव्रजंस्ते शिरांसि चालयन्तस्तमित्थमनिन्दन्, यथा,
\vakya मन्दिरभञ्जक दिनत्रये च तन्निर्मातरात्मानं तारय। त्वञ्चेदीश्वरस्य पुत्रोऽसि तर्हि क्रुशादवरोह।
\vakya शास्त्राध्यापकैः प्राचीनैश्च सार्धम् ईदृशमेव तमुपहसन्तो मुख्ययाजका अपि तमवदन्, सोऽपरान् अतारयत्, आत्मानं तारयितुं न शक्नोति।
\vakya स चेद् इस्रायेलस्य राजास्ति तर्हीदानीं क्रुशाद् अवरोहतु, तथा कृते वयं तस्मिन् विश्वसिष्यामः।
\vakya स ईश्वरे विश्वासं कृतवान् ईश्वरश्चेत् तम् अभिरोचयति तर्हि तमुद्धरतु, यतः स कथितवान् ईश्वरस्य पुत्रोऽहं।
\vakya तेन सार्धं क्रुशारोपितो दस्यू तावपि तथैव तमपवदताम्।
\vakya आमध्याह्नात्तु तृतीयप्रहरं यावत् कृत्स्ने भूतलेऽन्धकारोऽभूत्।
\vakya तृतीये प्रहरे च यीशुरुत्क्रोशन्नुच्चरवेण बभाषे, एली, एली, लाम्मा शबक्तानीति, अस्यार्थोऽयं, हे मदीश मदीश त्वं मां परित्यक्तवान् कुतः।
\vakya तच्छ्रुत्वा तत्र स्थितानां नराः केचिदवदन् असावेलियमाह्वयति।
\vakya तत्क्षणञ्च तेषामेको द्रुत्वा स्पञ्जमादायाम्लरसेन पूरयित्वा नलाग्रे बद्ध्वा तम् अपीप्यत्।
\vakya अपरे त्ववदन्, निवर्तस्व, अमुं तारयितुम् एलिय आगच्छति न वेत्यस्माभि र्दृश्यताम्।
\vakya ततः परं यीशुः पुनरुच्चैरुत्क्रुश्य प्राणांस्तत्याज।
\vakya पश्य च तदा मन्दिरस्य तिरस्करिण्यग्रतोऽधो यावद् विदद्रे भूश्चकम्पे शैला विचिच्छिदिरे
\vakya शवागाराण्युद्घाटयामासिरे निद्राणानाञ्च पवित्राणां कुणपानि बहून्युत्थापयाञ्चक्रिरे
\vakya तस्योत्थापनात् परञ्च निर्गत्य ते पुण्यनगरं प्रविश्यानेकेषां प्रत्यक्षीबभूवुः।
\vakya तांस्तु भूकम्पादीन् व्यापारान् दृष्ट्वा शतपति र्यीशो रक्षणे नियुक्तास्तस्य सङ्गिनश्चातीव भीत्वावदन्, सत्यम् आसीदसावीश्वरस्य पुत्रः।
\vakya अपि च गालीलतो यीशुम् अनुगतास्तत्परिचर्यायाञ्च नियुक्ता बह्व्यो योषितो दूरादवलोकमानास्तत्राविद्यन्त।
\vakya तासां मध्ये मग्दलीनी मरियम्, याकोबयोष्योश्च माता मरियम् सिबदियसुतयो र्माता चासन्।
\stitle{यीशोः समाधिः।}
\vakya सन्ध्यायान्तु जातायाम् अरिमाथियानिवासी योषेफनामक एको धनी नर आगमत्। सोऽपि यीशोः शिष्यः।
\vakya स पीलातस्य समीपं गत्वा यीशो र्देहं ययाचे। आदिष्टे तदा पीलातेन देहस्य दाने
\vakya योषेफो देहमादाय शुचिना क्षौमवस्त्रेण वेष्टयामास
\vakya तक्षिते शैलरन्ध्रे च स्वार्थं स यच्छवागारं निर्मितवांस्तस्मिन् नूतने शवागारे निदधे महाश्मानञ्च लोठयित्वा शवागारस्य मुखे दत्त्वा प्रतस्थे।
\vakya तत्र तु मग्दलीनी मरियम् अन्यतरा च मरियम् शवागारस्य सम्मुखम् उपविश्यावातिष्ठेताम्।
\vakya परदिने त्वर्थतः सज्जनदिनात् परं यद्दिनं तस्मिन् मुख्ययाजकाः फरीशिनश्च पीलातस्य समक्षं समेत्य जगदुः
\vakya प्रभो स प्रवञ्चको जीवत्काल उक्तवान् दिनत्रयात् परम् उत्थास्यामीत्यस्माभिः स्मर्यते।
\vakya अतस्तृतीयं दिवसं यावत् तच्छवागारस्य रक्षणं भवतादिश्यताम्। नोचेत् शिष्यास्तस्य नक्तमागत्य तं हृत्वा जनेभ्यः कथयिष्यन्ति स मृतानां मध्यादुत्थापित इति। एवं सत्यादिभ्रान्तितोऽन्तिमा भ्रान्तिरनिष्टतरा भविष्यति।
\vakya पीलातस्तान् जगाद, प्रहरिवर्गो युष्माकमास्ते, गत्वा तत् स्थानं यथाज्ञानं गोपयत। ते तदा गत्वा तमश्मानं मुद्रयाङ्कयित्वा प्रहरिवर्गेणापि शवागारम् अगोपयन्\eoc