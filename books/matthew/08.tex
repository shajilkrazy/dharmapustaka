\adhyAya
\stitle{यीशो र्विविधालौकिककर्माणि।}
\vakya पर्वताद् यीशोरवरोहणात् परं महान्तो जननिवहास्तमनुजग्मुः।
\stitle{यीशुना मनैककुष्ठिनो रोगमोचनम्।}
\vakya पश्य चापरं नर एकः कुष्ठी आगत्य तस्य समयं प्रणिपत्य जगाद, प्रभो भवोन् यदीच्छति तर्हि मां शुचीकर्तुं शक्नोति।
\vakya ततो यीशु र्हस्तं प्रसार्य तं स्पृष्ट्वा जगाद, इच्छामि, शुचि र्भव।
\vakya अनेनैव तस्य कुष्ठं शुचित्वं गतं। यीशुश्च तं जगाद, सावधानः, कमपि मा ज्ञापय, अपितु याहि याजकमात्मानं दर्शय, मोशिनादिष्टमुपहारञ्च त्भ्यः साक्ष्यदानार्थमुत्सृज।
\stitle{यीशो र्जनैकशतपतेर्दासाय निरामयत्वदानम्।}
\vakya अनन्तरं यीशौ कफरनाहूमं प्रविष्टे कश्चिच्छतपतिस्तस्यान्तिकमागत्य तं प्रसादयन् जगाद,
\vakya प्रभो, मम किङ्करो गेहेऽवशाङ्गस्तीव्रं पीड्यमानः शेते।
\vakya ततो यीशुस्तं ब्रवीति, अहमागत्य तं निरामयं करिष्यामि।
\vakya स शतपतिस्तु प्रत्यब्रवीत्, प्रभो, भवान् यन्मम वेश्म प्रविशेत्, नाहं तावता योग्यः। केवलं वाक्यं व्याहरतु, तेनैव मम किङ्करो निरामयो भविष्यति।
\vakya यतोऽहमपि क्षमताधीनो मनुष्यः, ममाधीनाश्च सैनिकाः सन्ति। तेषामेको गच्छेति मया कथिते गच्छति, अपरश्चायाहीति कथिते समायाति, मम दासश्चेदं कुर्विति कथिते तत् करोति।
\vakya इदं श्रुत्वा यीशुराश्चर्यं मन्यमानोऽनुगामिनोऽब्रवीत्, युष्मानहं सत्यं वदामि, इस्रायेलेऽपि मयेदृशो विश्वासो न लक्षितः।
\vakya युष्मांस्तु वदामि, पूर्वपश्चिमदिशो र्बहवो मनुष्याः स्वर्गराज्येऽब्राहामेणेस्‌हाकेन याकोबेन च सह भोज्ये समासिष्यन्ते,
\vakya राज्यस्य पुत्रास्तु बहिःस्थान्धकारे निक्षेप्स्यन्ते, तत्र च रोदनं दन्तै र्दन्तघर्षणञ्च सम्भविष्यतः।
\vakya अनन्तरं यीशुस्तं शतपतिं जगाद, याहि यथा च त्वया विश्वसितं तथैव तव सिध्यतु। तस्मिन्नैव च दण्डे तस्य किङ्करो निरामयीभूतः।
\stitle{यीशुना पितरस्य श्वश्र्वाः ज्वरमोचनम्।}
\vakya ततः परं यीशुना पित्रस्य गेहं प्रविश्य तस्य श्वश्रूः शयाना ज्वरातुरा च दृष्टा,
\vakya ततस्तेन तस्या हस्ते स्पृष्टे ज्वरस्तां तत्याज, सा चोत्थाय तं पर्यचरत्।
\vakya सन्ध्यायान्तूपस्थितायां भूताविष्टा बहवो मनुष्यास्तस्य समीपमानीताः। स च वाक्येन भूतान् निःसारयामास, अस्वस्थांश्च सर्वान् निरामयान् चकार।
\vakya इत्थं भाववादिना यिशायाहेन कथितमिदं वचनं सिद्धिं गतं, यथा,
\begin{poem}
\startwithline “सत्यमस्माकमेवार्तीः स गृहीत्वावहत् स्वयं।
\pline अस्मदीयव्यथाभारं स चाधार्षीत् स्वभारवत्॥”
\end{poem}
\vakya ततः परं यीशुः स्वपरितो जननिवहान् बहून् दृष्ट्वा स्वशिष्यान् ह्रदं तरिदुमाज्ञापयामास।
\vakya कश्चिच्छास्त्राध्यापकस्तदा तमागत्य जगाद, गुरो, यत्र कुत्रचिद् भवता गन्तव्यं तत्राहमपि भवन्तमनुगमिष्यामि।
\vakya यीशुस्तं जगाद, सन्ति गर्तानि शृगालानं नीडाश्च विहायसो विहङ्गमानां न स्थानं मनुष्यपुत्रस्य तु शिरः शाययितुं।
\vakya शिष्याणामन्यतमस्तं जगाद, प्रभो, प्रथमं स्वपितुः सत्कारार्थं गमिष्यन्तं मामनुमन्यस्व।
\vakya यीशुस्तु तं जगाद, मामनुगच्छ, मृतानेव मृतानां स्वकीयानां सत्कारायानुजानीही।
\stitle{यीशो र्झञ्झानिवारणम्।}
\vakya ततस्तस्मिन् नौकामारूढे तस्य शिष्यास्तमनुजग्मुः। पश्य च सागरे सञ्जातस्तुमुलसङ्क्षोभो नौस्तरङ्गैश्चाच्छाद्यत।
\vakya स तु निद्रामसेवत।
\vakya शिष्याः पुनस्तत्समीपमागत्य तं जागरयित्वा चावदन्, प्रभो, रक्षत्वस्मान्, वयं नश्यामः।
\vakya स तु तान् जगाद, भो स्तोकविश्वासिनः कुतो यूयं भीरवः? उत्थाय च स तदा वातान् समुद्रञ्च ततर्ज, भूतञ्च विशालं निःक्षोभं।
\vakya नराश्चाश्चर्यं मत्वा जगदुः; कीदृगसौ, यतो वाताः समुद्रश्चास्याज्ञां गृह्णान्ति?
\stitle{यीशुना लोकद्वयस्य भूतमोचनम्।}
\vakya अनन्तरं तस्मिन् सागरस्यापरपारस्थं गादारीयाणां देशमागते नरौ द्वौ भूताविष्टौ शवागारेभ्यो निर्गत्य तत्समक्षमुपस्थितौ, तयोरतिप्राचणडान्मार्गेण तेन गन्तु केनापि नाशक्यत।
\vakya पश्य च तावुत्क्रोशन्तौ जगदतुः, भो ईश्वरस्य सुत यीशो, भवता सहावयोः कः सम्बन्धः? भवान् किं समयात् प्रागावयो र्यातनां चिकीर्षुरत्रागतवान्?
\vakya तेषान्तु कियद्दूरे बहूनां शूकराणां व्रजोऽचरत्।
\vakya अतस्तौ भूतौ तं प्रसादयन्तौ जगदतुः, यद्यावां निःसारयेत् तर्ह्यावाममुष्मिन् शूकरव्रज अवेष्टुमनुजानातु।
\vakya स तौ जगाद, यातं। ततस्तौ निःसृत्य प्रविविशतुस्तं शूकरव्रजं, पश्य च तदा स कृत्स्नः शूकरव्रजो वेगेन धावन् पतित्वा शैलाग्रतस्तोये पञ्चत्वं जगाम।
\vakya ये च तमचारयंस्ते पलाय नगरं गत्वा निखिलवृत्तान्तं भूताविष्टयोस्तयोः कथाञ्च निवेदयामासुः।
\vakya पश्य च नगरस्य सर्वे यीशुं साक्षादुपस्थातुं निर्गत्य तं दृष्ट्वा तेषां स्वसीमभ्यस्तस्य स्थानान्तरगमनं प्रार्थयाञ्चक्रिरे\eoc