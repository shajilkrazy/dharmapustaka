\adhyAya
\stitle{यीशो र्यिरूशालेमगमनम्।}
\vakya अनन्तरं तेषु यिरूशालेमस्य समीपमागत्य जैतूनाख्यगिरौ स्थितं बैतफगीनामकं ग्राममुपस्थितेषु यीशुः शिष्यौ द्वौ प्रेषयन्निदमादिदेश,
\vakya युष्मत्सम्मुखस्थममुं ग्रामं गच्छतं, तत्र तत्क्षणमेवैकां बद्धां सवत्सां गर्धभीमासादयिष्यथः, तां मुक्त्वोभौ मदन्तिकमानयतं।
\vakya यदि तु कश्चिद् युवां किमपि वदेत्, तर्हि ब्रूतम्, आभ्यां प्रभोः प्रयोजनमस्ति। तेन स तत्क्षणं तौ विस्रक्ष्यति।
\vakya सर्वस्यैतस्य फलमिदं, यद् भाववादिनोक्तेयं कथा सफलाभूत्, यथा,
\begin{poem}
\startwithvakya “यूयं ब्रूत कथामेतां सियोन्‌दुहितरं प्रति।
\pline पश्य यस्तव राजासौ समायाति कृते तव।
\pline विनीतो गद्धभारूढो रासभीशिशुवाहनः॥”
\end{poem}
\vakya अनन्तरं शिष्यौ तौ गत्वा यीशुना यथादिष्टं तथैव कृत्वा
\vakya तां गर्धभीं तस्याः पोतञ्चानिन्यतुः स्ववस्त्राणि च तयो र्निदधाते स च तदुपर्युपविवेश।
\vakya जननिवहश्चाधिकांशो मार्गे स्ववसनानि विस्तारयामास, अन्ये च पादपानां शाखाश्छित्त्वा पथि व्यस्तारयन्,
\vakya अग्रपश्चाद्गामिनश्च बहव उच्चैःस्वरेणाब्रुवन् जय दायूदस्य पुत्र, प्रभो र्नाम्ना य आयाति स धन्यो भूयात, ऊर्धलोके जयध्वनि र्भवतु।
\vakya तस्मिंस्तु यिरूशालेमं प्रविष्टे नगरस्य जनाः सर्वे वेपमाना अपृच्छन्, कोऽसौ?
\vakya जननिवहास्त्ववदन्, नासरतनिवासी गालीलीयो भाववादी यीशुरसौ।
\vakya अनन्तरं यीशुरीश्वरीय धर्मधाम प्रविश्य धर्मधाम्नि क्रयविक्रयकारिणः सर्वान् बहिश्चका, बणिजां मुद्रासनानि कपोतविक्रेतॄणामासनानि च न्युब्जयामास,
\vakya जगाद च तान्, लिखितमास्ते, “प्रार्थनागृहमित्येव नाम्ना मे ख्यास्यते गृहं।” तत्तु युष्माभिरकारि दस्यूनां गह्वरः।
\vakya अनन्तरमन्धाः खञ्जाश्च धर्मधाम्नि तस्यान्तिकम् आगमन् स च तान् निरामयांश्चकार।
\vakya मुख्ययाजकाः शास्त्राध्यापकाश्च त्वालोक्याद्भुतकर्माणि तेन क्रियमाणानि बालकांश्च धर्मधाम्नि जय दायूदस्य पुत्रेति वाचम् उदीरयतः क्रुध्यन्तस्तमूचुः, श्रूयते किं त्वयामी यद् गदन्ति? 
\vakya यीशुस्तु तान् प्रत्यवादीत्, श्रूयत इति। यूयं किं कदापि नैतत् पठितवन्तो यथा, “शिशूनां स्तनपानाञ्च वक्त्रात् त्वं व्यदधाः स्तवम्।
\vakya अनन्तरं स तांस्त्यक्त्वा गत्वा च नगरस्य बहिःस्थं वैथनियाग्रामं रजनीं तत्र यापयामास।।
\vakya प्रभाते तु नगरं गच्चन् सोऽक्षुध्यत, तत् पथपार्श्व उडुम्बरवृक्षं दृष्ट्वा स तत्समीपं जगाम,
\vakya तस्मिंस्तु पत्रेभ्योऽन्यत् किमपि नादृक्षत्, सोऽतस्तं जगाद, इतः परं कदापि त्वत्तः फलं नोत्पाद्यतां। अनेनोडुम्बरवृक्षः स सद्यः शुष्को जातः।
\vakya तद् दृष्ट्वा शिष्या आश्चर्यं मत्वाऽब्रुवन्, वृक्षोऽसौ कियच्छीघ्रं शुष्कीभूतः।
\vakya यीशुस्तान् प्रतिजगाद, अहं युष्मान् सत्यं ब्रवीमि, यूयं यदि विश्वासमवलम्बा न संशेध्वे, तर्हि न केवलम् उडुम्बरवृक्षमधि युष्माभिरीदृशं कर्म कारिष्यते, परन्तु त्वमुत्पत्य समुद्रे निपतेत्यमुं गिरिं प्रति युष्माभिरुक्तेऽपि तत् सेत्स्यति।
\vakya लप्स्यध्वे च सर्वं प्रार्थनया विश्वसन्तो यद्यद् याचिष्यध्वे।
\stitle{यीशोरधिकारमधि शिक्षा।}
\vakya अनन्तरं धर्मधाय प्रविश्य स यदोपदेशमददात् मुख्ययाजका जनानां प्राचीनाश्च तदा तस्यान्तिकमागत्य पप्रच्छुः, केनाधिकारेण करोषि त्वं कर्माण्येतानि? को वा तुभ्यं तमधिकारं दत्तवान्?
\vakya यीशुस्तु तान् प्रत्यब्रवीत्, युष्मानहमपि कथामेकां प्रक्ष्यामि, मह्यं चेत् तदुत्तरं दत्थ, तर्हि केनाधिकारेण सर्वमेतत् करोमि, तदहमपि युष्मान् वदिष्यामि।
\vakya कुतो योहनस्य स्नापनमुत्पन्नम्? अपि स्वर्गादुत मनुष्येभ्यः? ते तदा परस्परं विचारयन्तोऽवदन् तत् स्वर्गोत्पन्नमित्युक्ते सोऽस्मान् प्रक्ष्यति, यूयं तर्हि तस्मिन् किमर्थं न विश्वसितवन्तः?
\vakya तत्तं मनुष्येभ्य उत्पन्नमित्युक्ते जननिवहाद् भेतव्यं, यतः सर्वे योहनं भाववादिनं मन्यन्ते।
\vakya तत् ते यीशुं प्रत्यवदन्, न जानीमः। सोऽप्यनेन तानवादीत्, नाहमपि युष्मान् ज्ञापयामि कोम्यहं सर्वमेतत् केनाधिकारेणेति।
\stitle{द्वयोः सुतयोर्दृष्टान्तकथनं।}
\vakya यूयं परं किं मन्यध्वे? कस्यचिन्नरस्य पुत्रौ द्वावास्तां। स तयोरेकस्यान्तिकं गत्वा तं जगाद, वत्स, याहि, कुरु चाद्य कर्म मम द्राक्षाक्षेत्रे।
\vakya स प्रत्यब्रवीत्, नेच्छामि, पश्चात्तु सोऽनुतप्य जगाम।
\vakya नरः सोऽन्यतरस्यान्तिकं गत्वा तथैव जगाद। स तं प्रत्यब्रवीत्, यथाज्ञापयतु प्रभो, न तु जगाम।
\vakya तयो र्द्वयोः कः पितुरभिमतमाचरितवान्? ते तं वदन्ति, प्रथमः। यीशुस्तदा तान् जगाद, युष्मानहं सत्यं ब्रवीमि, शुल्कादायिनो वेश्याश्चैश्वरराज्यं प्रवेष्टुं युष्माकमग्रगामिनो भवन्ति,
\vakya यतो योहनो धर्ममार्गेण युष्मदन्तिकमागतवान् यूयं पुनस्तस्मिन् न विश्वसितवन्तः, शुल्कादायिनस्तु वेश्याश्च विश्वासं तस्मिन् चक्रिरे। तद्दृष्ट्वा पश्चादपि यूयं नान्वतप्यध्वं तस्मिन् विश्वसितुं।
\stitle{गृहस्वामिकृषकाणां दृष्टान्तः।}
\vakya शृणुतापरामेकां दृष्टान्तकथां। गृहस्वामी कश्चिद् द्राक्षालता रोपयन् द्राक्षोद्यानं कृत्वा वृत्या परिवार्य तन्मध्ये द्राक्षामर्दनार्थककुण्डं खनित्वाट्टालं निर्माय च कृषकेषु करदायिषु समर्प्य देशान्तरं जगाम।
\vakya ततः परम् आसन्न फलकाले निजफलांशमादातुं कृषकाणां समीपं स स्वदासान् प्राहिणोत्।
\vakya कृषकास्तु तस्य दासान् धृत्वा तेषामेकं ताडयामासुरन्यं मारयामासुरपञ्च प्रस्तरैरभ्याजघ्नुः।
\vakya पुनः स प्रथमेभ्यस्तेभ्योऽधिकान् अन्यान् दासान् प्राहिणोत्, ते त्विमान् प्रत्यपि तादृशमाचारं चक्रुः।
\vakya अतो मत्पुत्रं प्रति ते त्रपिता भविष्यन्तीति बुद्ध्या स पश्चात् तेषां समीपं स्वपुत्रं प्राहिणोत्।
\vakya तन्तु दृष्ट्वा कृषकास्ते परस्परमब्रुवन्, दायहरोऽसौ, तदागच्छत, वयममुं हनिष्यामस्तस्य दायांशञ्चात्मसात् करिष्यामः।
\vakya एवमुक्त्वा ते तं धृत्वा द्राक्षोद्यानाद् बहिर्निक्षिप्य जघ्नुः।
\vakya अतो द्राक्षोद्यानस्य स्वामी यदायास्यति, स तदा कृषकांस्तान् प्रति किं करिष्यति?
\vakya ते तं वदन्ति, स तान् दुर्जनान् दुरन्तं विनाशयिष्यति, समर्पयिष्यति च स्वद्राक्षोद्यानं तेष्वेव कृषकेषु ये तस्मै यथाकालं फलानि दास्यन्ति।
\vakya यीशुस्तदा तानवादीत्, युष्माभिः कदापि किं शास्त्रे न पठितं, यथा, “गृहनिर्मातृभि र्लोकै र्यः पाषाणो निराकृतः। स एव गृहकोणस्थः प्रमुख्यः प्रस्तरोऽभवत्। परमेशस्य कर्मेदम् अस्मद्दृष्टौ तदद्भुतम्॥”
\vakya अतोऽहं युष्मान् ब्रवीमि, ईश्वरस्य राज्यं युष्मत्तोऽपहारिष्यते, दायिष्यते च तदन्यस्यै जात्यै या तदुपयुक्तैः फलैः फलवती भविष्यति।
\vakya प्रस्तरेऽस्मिन् यस्तु पतिष्यति स खण्डशो भग्नो भविष्यति, प्रस्तरश्चायं यस्मिन् पतिष्यति तं सञ्चूर्णयिष्यति।
\vakya तास्तस्य दृष्टान्तकथाः श्रुत्वा मुख्ययाजकैः फरीशिभिश्चाबोधि यदयमस्मानधि कथयतीति।
\vakya तं पुन र्धर्त्तुं यतमानास्ते जननिवहेभ्यो बिभ्युः यतस्ते तं भाववादिनममन्यन्त\eoc