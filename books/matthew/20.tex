\adhyAya
\stitle{कृषकाणां दृष्टान्तः।}
\vakya यतः स्वर्गराज्यं गृहस्वामिना तेन सदृशं, यः प्रत्यूष एव स्वद्राक्षाक्षेत्रे कृषकान् वेतनजीविनो नियोक्तुमिच्छन् निर्जगाम।
\vakya कृषकांश्च प्राप्य तैः सह दैनिकभृतिं मुद्रापादं निरूप्य तान् स्वद्राक्षाक्षेत्रं प्राहिणोत्।
\vakya ततः परं प्रायशः प्रहरे प्रथमेऽतीते स पुन र्निगत्यापरान् कतिपयान् निष्कर्मकान् हट्टे तिष्ठतो दृष्ट्वा तानपि जगाद,
\vakya यूयमपि मम द्राक्षाक्षेत्रं गच्छत, यच्च न्याय्यं तदहं युष्मभ्यं दास्यामि, तेन च ते जग्मुः।
\vakya द्वितीयस्य दृतीयस्य च पुनः प्रहरस्यान्ते निर्गत्य स तथैव चकार।
\vakya चतुर्थप्रहरस्य पुनस्तृतीयांशेऽवशिष्टे निर्गत्य सोऽन्यान् निष्कर्मकं तिष्ठतः कतिपयान् दृष्ट्वा पप्रच्छ, किमर्थं यूयमत्र कृत्स्नं दिनं निष्कर्मकास्तिष्ठथ?
\vakya ते तमब्रुवन्, कर्माणि केनापि वयं न नियुक्ताः। स तानब्रवीत्, यूयमपि मम द्राक्षाक्षेत्रं गच्छत, तेन यन्न्याय्यं तल्लप्स्यध्वे।
\vakya सन्ध्यायान्तूपस्थितायां द्राक्षाक्षेत्रस्य स्वामी विषयाध्यक्षं निजं जगाद, त्वं कृषकानाहूयान्त्येभ्यो दातुमारभ्य प्रथमान् यावत् तेभ्यो भृतिं देहि।
\vakya अनेन ये सन्ध्यायाः सार्धदण्डद्वयं प्रागागतवन्त उपस्थाय ते प्रत्येकं मुद्रापादं लेभिरे।
\vakya अनन्तरं प्रथमा उपस्थायास्माभिरधिकं लप्स्यत इत्यमन्यन्त।
\vakya तेषान्त्वप्येकैको मुद्रापादमेव लेभे।
\vakya लब्ध्वा तु ते गृहस्वामिनमपवदन्तो जगदुः,अन्त्या अमी कर्म कृतवन्तो मुहूर्तमेकमेव, वयन्तु दिनस्य भारमुत्तापञ्च सोढवन्तः, भवान् पुनरमूनस्माभिस्तुल्यान् कृतवान्।
\vakya स ततः प्रतिवदन् तेषामेकं जगाद, मित्र, नाहं तवापराध्नोमि, मया सह त्वया किं भृति र्मुद्रापादमाना न निरूपिता?
\vakya गृहीत्वापगम्यतां त्वया यल्लभ्यं। रोचये त्वमन्त्यायैतस्मै दातुं तुभ्यमिव।
\vakya न किं विधेयं मया स्वं मम निजस्वमधि यदहं कर्तुमिच्छामि? अथवा किं तव नेत्र दुष्यति। यदहं परहितैषीति?
\vakya इत्थमेवान्त्याः प्रथमा भविष्यन्ति प्रथमाश्चान्त्याः। यत आहूता बहवोऽल्पे तु वरिताः।
\stitle{यीशुना स्वमरणमधि तृतीयवारं भविष्यद्वाक्यकथनम्।}
\vakya ततः परं यीशुरूर्ध्वगमार्गेण यिरूशालेमं गच्छन् पथि द्वादशशिष्यान् निभृतं नीत्वा तेभ्यः कथयामास,
\vakya पश्यत वयं यिरूशालेमं गच्छामः, मनुष्यपुत्रश्च मुख्ययाजकेषु शास्त्राध्यापकेषु च समर्पयिष्यते।
\vakya ते च विचारेण तस्य प्राणदण्डाज्ञां कृत्वा पारुष्याय कशाघाताय क्रुशारोपणाय च तं परजातीयेषु समर्पयिष्यन्ति, तृतीये दिने तु स पुनरुत्थास्यति।
\stitle{तत्त्वतः महान् कः? एतद्विषयकशिक्षा।}
\vakya सिबदियस्य पुत्रयो र्जननी तदा स्वसुताभ्यां सह तस्यान्तिकमागत्य प्रणिपत्य वरं ययाचे।
\vakya स तां जगाद, किं वाञ्छसि? सा तमब्रवीत्, भवतो राज्ये यथा ममैतयोः पुत्रयोरेको भवतो दक्षिणेऽन्यतरश्च भवतो वामे समुपविशेत् तथैवादिशतु।
\vakya यीशुस्तु प्रत्यब्रवीत्, युवां यद् याचेथे तन्न जानीथः। येन पानपात्रेणाहं पास्यामि तेन किं युवां पातुं शक्नुथः? येन च स्नानेनाहं स्नास्ये, तेन स्नातुं किं युवाभ्यां शक्यं तौ तमूचतुः शक्यं।
\vakya तदा स तावब्रवीत्, मम पानपात्रेण युवां पास्यथः, येनाहं स्नास्ये तेन स्नानेन स्नास्येथे च, परन्तु मम दक्षिणे वामे वोपवेशनाधिकारस्य दानं मया न विधेयं स तु केवलेभ्यस्तेभ्यो दातब्यो येषां कृते मम पित्रा स निरूपितः।
\vakya श्रुत्वैतदन्ये दश शिष्यास्तयोः सहोदरयो रौक्ष्यमापन्नाः।
\vakya यीशुस्तु तान् स्वसमीपमाहूयोवाच, यूयं जानीश, परजातीयानां शास्तारस्तेषामुपरि तीक्ष्णप्रभुत्वं कुर्वते, महान्तश्च तेषां तीक्ष्णकर्तृत्वं कुर्वते।
\vakya न भव्यन्तु तादृशं युष्माकं मध्ये। प्रत्युत युष्मन्मध्ये यो महान् भवितुमिच्छति स युष्माकं परिचारको भविष्यति।
\vakya यश्च युष्मन्मध्ये प्रथमो भवितुमिच्छति, स युष्माकं दासो भविष्यति।
\vakya यतस्तथैवागतो मनुष्यपुत्रो न परिचर्यां भोक्तुम् अपि तु परिचरितुं दातुञ्च स्वप्राणान् निष्क्रयमूल्यं बहूनां विनिमयेन।
\stitle{अन्धाय चक्षुर्दानम्।}
\vakya अनन्तरं तेषु यिरीहूतो निर्गच्छत्सु महान् जननिवहस्तमन्वव्रजत्।
\vakya पश्य च पथपार्श्व उपविष्टौ द्वाबन्धौ यीशो र्गमनस्य कथां श्रुत्वा क्रोशन्ताववदतां, भो प्रभो दायूदस्य पुत्र, आवामनुकम्पतां।
\vakya जननिवहस्तावनेन तर्जयन्नब्रवीत्, तूष्णीम्भवतं। तौ त्वधिकतरं क्रोशन्ताववदतां, भो प्रभो दायूदस्य पुत्र, आवामनुकम्पतां।
\vakya यीशुस्तदा तिष्ठंस्तावाहूय पप्रच्छ, किं वाञ्छथः? युवयो र्मया किं कर्तव्यं?
\vakya तौ तमूचतुः, प्रभो, आवयो र्नेत्राणि यदुद्घाट्येरन् तद् वाञ्छावः।
\vakya तदा यीशुः कृपां कृत्वा तयो र्नेत्राणि पस्पर्श, तत्क्षणञ्च तौ दृक्शक्तिं लब्ध्वा तम् अन्वव्रजताम्\eoc