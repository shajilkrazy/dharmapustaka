\adhyAya
\vspace{25pt}
\vakya समापितो स्वीयद्वादशशिष्येभ्य अदेशप्रदाने यीशु र्जनानं नगरेषु शिक्षादानार्थं घोषणार्थञ्च तस्मात् स्थानात् प्रतस्थे।
\stitle{योहनस्य प्रश्नः यीशुख्रीष्टस्य उत्तरञ्च।}
\vakya कारायां योहनस्तु ख्रीष्टस्य क्रियाणां संवादं श्रुत्वा स्वशिष्याणां द्वौ प्रहित्य तं पप्रच्छ,
\vakya येनागन्तव्यं स कि भवान्, वान्यः कश्चिदस्माभिः प्रतीक्षितव्यः?
\vakya ततो यीशुस्तौ प्रतिजगाद, युवाभ्यं यद्यच्छ्रूयते दृश्यते च गत्वा तद् योहनाय निवेद्यतां।
\vakya अन्धा दृष्टिं लभन्ते, खञ्जाः परिव्रजन्ति, कुष्ठिनः शुचीभूयन्ते, वधिराः शृण्वन्ति, मृता उत्थाप्यन्ते, दरिद्राश्च सुसंवादं ज्ञाप्यन्ते,
\vakya यश्च मयि न स्खलति स धन्यः।
\stitle{योहनं प्रति ख्रीष्टस्य साक्षित्वं।}
\vakya तयोस्त्वपगच्छतो र्यीशु र्जननिवहेभ्यो योहनमधीदं कथयितुमारेभे, किं निरीक्षितुं यूयं मरुं निर्गतवन्तः? किं वायुना चाल्यमानं नलं किमथवा द्रष्टुं यूयं निर्गतवन्तः?
\vakya किं सूक्ष्मवेशपरिहितं मनुष्यं? पश्यत सूक्ष्मवेशा मनुष्या राजभवनेषु विद्यन्ते।
\vakya किमथवा द्रष्टुं यूयं निर्गतवन्तः? किं भाववादिनं? तथैव, युष्मास्त्वहं ब्रवीमि, भाववादितोऽपि श्रेष्ठतरं नरं।
\vakya यतोऽसौ स यमधीदं लिखितमास्ते,
\begin{poem}
\startwithline “पश्य त्वद्वदनस्याग्रे स्वदूतं प्रहिणोम्यहं।
\pline गन्तव्यं तव मार्गं स तवाग्रे संस्करिष्यति॥”
\end{poem}
\vakya युष्मानहं सत्यं ब्रवीमि, नारीप्रसूतेषु स्नापकाद् योहनान्महान् कोऽपि नोत्पन्नः। स्वर्गराज्ये तु यः क्षोदीयान् स तस्मादपि महत्तरः।
\vakya परन्तु स्नापकस्य योहनस्य दिनान्यारभ्याद्य यावत् स्वर्गराज्यं बलादाक्रम्यते बलवन्तश्च तद्धरन्ते।
\vakya यतो भाववादिनो व्यवस्था च योहनपर्यन्ताः,
\vakya यूयञ्च चेदेतद् ग्रहीतुं सम्मतास्तर्हि जानीत, येनागन्तव्यं स एलियोऽसौ।
\vakya शृणोतु यस्य श्रोतुं श्रोत्रे स्तः।
\vakya तदेत्कालीया मया कैरुपमेयाः? ते हट्टेषूपविष्टैस्तैर्बालकैः सदृशाः,
\vakya ये स्ववयस्यान् आह्वयन्तो वदन्ति, युष्मत्कृते वयं वंशीरवादयाम यूयन्तु नानृत्यत, युष्मत्कृते वयं व्यलपाम यूयन्तु स्ववक्षांसि नाताडयत।
\vakya योहनो हि न भोजनं न पानं वा सेवमान आगतः, जनैस्तूच्यते स भूताविष्टः।
\vakya मनुष्यपुत्रो भोजनं पानञ्च सेवमान आगतः, जनैस्तूच्यते पश्यासौ भोक्ता मद्यपश्च मनुष्यः, शुल्कादायिनां पापिनाञ्च बन्धुः। प्रज्ञा तु स्वसन्तानानामाचारतो निर्दोषीकृता।
\stitle{अविश्वासिनः प्रति भर्त्सनम्; भाराक्रान्तलोकान् प्रति निमन्त्रणवाक्यम्।}
\vakya तदानीं येषु नगरेषु कर्माणि भूयिष्ठानि प्रभावसिद्धानि तेन कृतानि, तेषां निवासिभि र्मनांसि न परावर्तितानीति हेतोः स तानि भर्त्सयितुमारभ्य जगाद, 
\vakya हा कोरासिन, हा बैत्सैदे, युवां सन्तापार्हे, यतो युवयो र्मध्ये कृतानि यानि प्रभावसिद्धानि कर्माणि, तानि चेत् सोरे सीदोने चाकारिष्यन्त, प्रागेव तन्निवासिस्तर्हि शाणं परिधाय भस्मन्युपविश्य च मनांसि परावर्तयिष्यन्।
\vakya अहन्तु वां ब्रवीमि, विचारदिने युवको र्दशातः सोरस्य सीदोनस्य च दशा सह्यतरा भविष्यति।
\vakya त्वञ्च हा स्वर्गं यावदुन्नमिते कफरनाहूमपुरि, पातालं यावदवरोहयिष्यसे, यतस्त्वयि कृतानि प्रभावसिद्धानि यानि कर्माणि, तानि चेत् सदोमेऽकारिष्यन्त, तर्ह्यद्य यावत् तदस्थास्यत्।
\vakya युष्मांस्त्वहं ब्रवीमि, विचारदिने सदोमदेशस्य दशा तव दशातः सह्यतरा भविष्यति।
\stitle{भाराक्रान्तजनाह्वानञ्च।}
\vakya तस्मिन् काले यीशुरनुबभाषे, भो स्वर्गमर्त्ययोः स्वामिन् पितः, त्वामहं साधु वदामि, यतस्त्वया विज्ञेभ्यस्तीक्ष्णबुद्धिभियश्चेमानि निगुह्य शिशूनामाविष्कृतानि।
\vakya अतः किं पितः, यदित्थं तव दृष्टौ यत् प्रीतिकरं तदेव सिद्धिं।
\vakya मम पित्रा मयि सर्वमेव समर्पितं; पितरं विनापरः कोऽपि पुत्रं न विजानाति, पुत्रञ्च विनापरः कोऽपि पितरं न विजानाति, यस्मै च तं प्रकाशयितुं पुत्राय रोचते सोऽपि तं विजानाति।
\vakya भो परिश्रान्ता भाराक्रान्ताश्च सर्वे, मम समीपमायात, तेनाहं युष्मान् विश्रमयिष्यामि।
\vakya मम युगं स्वेष्वर्पयत मत्तः शिक्षाञ्च गृह्णीत, यतो मृदुशीलो नम्रहृदयश्चाहं। तथानुष्ठिते यूयं स्वमनसां कृते विश्राममवाप्स्यथ।
\vakya यतः सुसह्यं मम युगं लघुश्च मम भारः\eoc