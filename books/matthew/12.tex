\adhyAya
\stitle{विश्रामवारमधि यीशोरुपदेशः।}
\vakya यीशुस्तदा विश्रामवारे शस्यक्षेत्रेणाव्रजत्, तस्य शिष्यास्त्वक्षुध्यन् शस्यमञ्जरी र्भङ्क्त्वा च भोजने प्रावर्तन्त।
\vakya तद् दृष्ट्वा फरीशिनस्तमूचः, पश्यतु, विश्रामवारे यदविधेयं भवतः शिष्यैस्तत् क्रियते।
\vakya स तु तान् जगाद, न किं पठितं भो दायूदस्तस्य सङ्गिनश्च यदाक्षुध्यन् तदा स किमकार्षीत्?
\vakya स हीश्वरस्य गेहं प्रविश्य तानेव दर्शनीयपूपान् अभुङ्क्त यान् भोक्तुं याजकैः केवलै र्विधेयमासीत्, न तु तेन न तत्सङ्गिभि र्वा।
\vakya नापि किं व्यवस्थायां पठितं भो यद् धर्मधान्नि याजका विश्रामवारञ्च लङ्घयन्ति निर्दोषास्तु सन्ति?
\vakya युष्मांस्त्वहं ब्रवीमि, धर्मधान्नोऽपि महत्तरेण केनाप्यत्रोपस्थितं।
\vakya परन्तु “दयामेवाभिवाञ्छामि न तु यज्ञक्रियामहम्।” एतस्य तत्वं यदि यूयमज्ञास्यत, न निर्दोषान् दाषिणस्तदाकरिष्यत।
\vakya प्रभु र्हि मनुष्यपुत्रो विश्रामवारस्यापि।
\vakya अनन्तरं स स्थानात् तस्मात् प्रगत्य तेषां समाजगृहं प्रविवेश।
\vakya पश्य च तदासीदुपस्थितो नर एकः शुष्कहस्तः। ते च तस्मिन्नभियोगमारोपयितुकामास्तं पप्रच्छुः, किमामयस्य प्रतीकारो विधेयो विश्रामवारे?
\vakya स तु तान् जगाद,अस्ति वः कोऽप्येकस्य मेषस्य स्वामि? तस्य मेषश्चेद् विश्रामवारे गर्ते पतति, स तर्हि किं तं धृत्वा नोद्धरिष्यति?
\vakya मेषात्तु मनुष्यः किं न सुदूरं विशिष्यते? अतो विधेया सत्क्रिया विश्रामवारे।
\vakya स तदा नरं तं जगाद, तव हस्तं प्रसारय। हस्तश्च तेन प्रसारितः स्वस्थीभूतश्च सोऽन्यतर इव।
\vakya अनन्तरं फरीशिनो निर्गत्य मिथो मन्त्रयाञ्चक्रिरे कथं स विनाशयितव्यः।
\vakya यीशुस्तु तज्‌ज्ञात्वा स्थानं तत् तत्याज, महान्तो जननिवहाश्च तमन्वव्रजन्, स च सर्वान् निरामयांश्चकार, तान् निर्भर्त्सयंश्च जगाद,
\vakya मां साधारणगोचरं माऽकार्ष्ट।
\vakya इत्थं भाववादिना यिशायाहेन कथितमिदं वाक्यं सिद्धं, यथा,
\vakya “सेवको मम पश्यासौ वरयित्वा मया धृतः। मामकप्रेमपात्रं स मन्मनस्तुष्टिकारकं॥
\vakya मया तस्योपरिष्टाच्च स्वात्माधिष्ठापयिष्यते। विचारस्य च संवादं जातिभ्यः स प्रदास्यति॥ नैव कोलाहलस्तेन नोच्चशब्दः करिष्यते। राजमार्गेषु वा कोऽपि नैव श्रोष्यति तद्रवं॥
\vakya प्रक्षुणोऽपि नलस्तेन नैव खण्डीकरिष्यते। निस्तेजा वर्तिका तेन न च निर्वापयिष्यते॥
\vakya इत्थं शेषे जथप्राप्तं विचारं सोऽभिनेष्यति। करिष्यन्ति च तस्यैव नाम्न्याशां परजातयः॥”
\stitle{यीशुना सूतग्रस्ताय आरोग्यदानं लोकेभ्य उपदेशदानाञ्च।}
\vakya तदानीमन्धो मूकश्चैको भूताविष्टो मनुष्यस्तस्य समीपमानिन्ये, स च तं स्वस्थं चकार, तेनान्धो मूकश्च स वाक्‌शक्तिं दृष्टिञ्च लेभे।
\vakya ततः सर्वे जननिवहा आश्चर्यं मत्वाब्रुवन्, किमसौ दायूदस्य पुत्रः?
\vakya फरीशिनस्तु तच्छ्रुत्वा जगदुः, निःसारयत्यसौ भूतान् केवलं भूतराजस्य बेलसबूबस्य साहाय्येन नान्येन केनाप्युपायेन।
\vakya यीशुस्तु तेषां चिन्ता विज्ञाय तान् जगाद, उत्सीदति सकलं तद् राज्यं यद् भिन्नं स्वविरुद्धं, नाप्यवस्थास्यते सकलं तन्नगरं कुलं वा यद् भिन्नं स्वविरुद्धं।
\vakya शैतानश्चेत् शैतानं निःसारयति स तर्हि स्वविरुद्धं भिन्नः सञ्जातः, तेन तस्य राज्यं वा कथमवस्थास्यते?
\vakya तदहं यदि बेलसबूबबलेन भूतान् निःसारयामि, युष्माकं पुत्रास्तर्हि केन तान् निःसारयन्ति? अतस्ते युष्माकं विचारयितारो भविष्यन्ति।
\vakya यदि त्वीश्वरस्यात्मनाहं भूतान् निःसारयामि, तर्हीश्वरस्य राज्यं युष्मत्समीपमुपस्थितं।
\vakya अथवा तस्य बलिष्ठस्य गेहं प्रविश्य तस्य द्रव्याणि लोठयितुं मनुष्येण कथं शक्यते? बलिष्ठं तम् अग्रे न बद्ध्वा तत् कर्तुं तेन न शक्यं बद्ध्वा तु तस्य गेहस्थं धनं तेन लोठयिष्यते।
\vakya यो न मम सहायः स मम विरोधी, यश्च मया सार्धं न सञ्चिनोति स विकिरति।
\vakya तद् युष्मानहं ब्रवीमि, क्षमिष्यते तु मनुष्याणां सर्वपापं सर्वनिन्दा च आत्मनस्तु निन्दा मनुष्याणां न क्षमिष्यते।
\vakya यस्तु मनुष्यपुत्रस्य प्रतिकूलं वाक्यं व्याहरति तस्य क्षमिष्यते; यस्तु पवित्रस्यात्मनः प्रतिकूलं वाक्यं व्याहरति तस्य नास्मिन् युगे न भविष्यति युगे वा क्षमिष्यते।
\vakya युष्माभिः सुवृक्षे कल्पिते सुफलान्यपि तस्य कल्पयितव्यानि, कुवृक्षे च कल्पिते कुफलान्यपि तस्य कल्पयितव्यानि यतो फलेनाभिज्ञायते वृक्षः।
\vakya भो कालसर्पान्वयाः, यूयं दुष्टाः, युष्माभिः सद्वाक्यानि व्याहर्तुं कथं शक्यते? यतो हृदयस्यातिपुरणमेव वक्त्रं व्याहरति।
\vakya सुजनो हृदयात् सञ्चितसुधनात् सुवाक्यानि निःसारयति, कुजनश्च कुधनात् कुवाक्यानि नःसारयति।
\vakya युष्मांस्त्वहं ब्रवीमि, वचनस्यानर्थकस्यैकैकस्य मनुष्यैः कथितस्योत्तरं विचारदिने दातव्यं।
\vakya यतस्त्वं स्ववाक्येभ्यो निर्दोषः स्ववाक्येभ्य एव दोषी वा कारिष्यसे।
\vakya तदानीं शास्त्राध्यापकानां फरीशिनाञ्च केचित्तं प्रत्यवदन्, भो गुरो, भवतो दूरेऽभिज्ञानं द्रष्टुमिच्छामः।
\vakya स तु तान् प्रतिजगाद, दुष्टो व्यभिचारी च वंशोऽभिज्ञानमनुसन्धत्ते, तस्मै तु भाववादिनो योनाहस्याभिज्ञानादन्यदभिज्ञानं न दायिष्यते।
\vakya यतो योनाहो यथा दिनत्रयं रात्रित्रयञ्च यावत् तिमेरुदरेऽवस्थितवान्, तथा मनुष्यपुत्रो दिनत्रयं रात्रित्रयञ्च यावत् पृथिव्या उदरेऽवस्थास्यते।
\vakya नीनवीयनरा विचारे वंशेनैतेन सार्धमुत्थास्यन्ति तं दोषीकरिष्यन्ति च, यतो योनाहस्य घोषणे तै र्मनांसि परावर्तितानि, पश्य त्वत्र योनाहान्महत्तरेण केनाप्युपस्थितं।
\vakya दक्षिणदिशो राज्ञी विचारे वंशेनैतेन सार्धमुत्थापयिष्यते तं दोषीकरिष्यति च, यतः सा शलोमनो विज्ञानोक्तीः श्रोतुं पृथिव्याः प्रान्तेभ्य आगतवती, पश्य त्वत्र शलोमनो महत्तरेण केनाप्युपस्थितं।
\vakya आत्मा परन्त्वशुचिर्मनुष्यान्निर्यानात् परं निरुदकानि स्थानानि पर्यटन् विश्रामं मृगयते न तु तं प्राप्नोति।
\vakya स तदा वदति निर्गतोऽहं मामकाद् गेहाद् यस्मात् तत् पुनर्गच्छामि। तत्रोपस्थाय तु स तत् शून्यं मार्जितं शोभितञ्च पश्यति।
\vakya गत्वा च स तदापरान् स्वतो दुष्टतरान् सप्तात्मनः स्वसङ्गिनः करोति, सर्वे च ते तत्र प्रविश्य निवसन्ति। अनेन मनुष्यस्य तस्यान्तिमदशादिदशातो निकृष्टा भवति। दुष्टस्य वंशस्यैतस्यापि तादृशी दशा सम्भविष्यति।
\vakya स यावत् जननिवहेभ्योऽकथयत्, पश्य तावत् तस्य माता भ्रातरश्च बहिस्तिष्ठन्तस्तेन संलपितुमैच्छन्।
\vakya कश्चिच्च तं जगाद, पश्यतु, भवतो माता भ्रातरश्च भवता संलपितुमिच्छन्तो बहिस्तिष्ठन्ति।
\vakya स तु तं संवाददातारं प्रतिजगाद, का मम माता? के वा मम भ्रातरः?
\vakya स्वशिष्यानुद्दिश्य हस्तं प्रसार्य च स जगाद, पश्यामी मम माता मम भ्रातश्च।
\vakya यतो यः कश्चिन्मम स्वर्गस्थस्य पितुरिच्छां पालयति, स एव मम भ्राता भगिनी माता च\eoc