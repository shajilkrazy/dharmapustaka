\adhyAya
\stitle{यीशोरुज्ज्वलमूर्तिधारणम्।}
\vakya ततः परं दिनेषु षट्ष्वतीतेषु यीशुः पित्रं याकोबं तस्य सहोदरं योहनञ्च सार्धं नीत्वा निभृतं गिरिमुच्चं कञ्चिदारुरोह,
\vakya रूपान्तरीबभूव च तत्र तेषां समक्षं, तस्यास्यं तदा सूर्यवद् व्यराजत वसनानि च तस्य दीप्तिवत् सितान्यभवन्।
\vakya पश्यापि च मोशिरेलियश्च तैः संलपन्तौ तेभ्यो दर्शनमददतुः।
\vakya पित्रस्तदा यीशुं प्रत्यब्रवीत्, प्रभो, भद्रमस्माकमत्रावस्थानं। भवांश्चेदभिमन्यते, तर्ह्यस्माभिरत्रोटजानि त्रीणि निर्मायिष्यन्ते एकं भवदर्थम् एकं मोश्यर्थम् एकमेलियार्थम्।
\vakya पश्य च तस्मिन्नित्थं ब्रुवाणे दीप्र एको मेघस्तेषामुपरि छायां ददौ, पश्यापि च मेघस्य तस्य मध्याद् वाणीयमुदभूत्; अयं मम प्रियः पुत्रः, अस्मिन्नहं प्रीतः, अस्य वचांसि युष्माभिः श्रूयन्तामिति।
\vakya एतच्छ्रुत्वा शिष्या अधोमुखाः पतित्वातीव बिभ्युः।
\vakya तदा यीशुरुपागत्य तानब्रवीत्, उत्तिष्ठत, मा भैष्ट।
\vakya ततस्ते लोचनान्युन्मील्य केवलं यीशुं विहाय नापरं किमपि ददृशुः।
\vakya ततः परं तेषु पर्वतादवरोहत्सु यीशुस्तानिदमादिदेश, मनुष्य-पुत्रो यावन्‌मृतानां मध्यान्नोत्तिष्ठति, तावद् दर्शनमिदं यूयं कस्मैचिन्मा कथयतेति।
\vakya तदा तस्य शिष्यास्तमपृच्छन्, प्रथमम् एलियेनागन्तव्यमिति शास्त्राध्यापकैः कथमुच्यते?
\vakya यीशुस्तु तान् प्रतिजगाद, एलियः प्रथममागत्य सर्वं प्रतिसंस्करिष्यतीति सत्यं,
\vakya युष्मांस्त्वहं ब्रवीमि, इतिपूर्वमेलिय आगतवान्, मनुष्यास्तु तं नाभिज्ञाय तं प्रति तत्तत् कृतवन्तो यद्यदैच्छंस्ते। तादृशं मनुष्यपुत्रेणापि तेभ्यो दुःखभोगो लप्स्यते।
\vakya शिष्यैस्तदाबोधि यत् सोऽस्मान् स्नापकस्य योहनस्य कथामुक्तवानिति।
\stitle{यीशुना कस्यैचिद् अपस्मारग्रस्तबालकाय आरोग्यदानम्।}
\vakya ततस्तेषु जननिवहस्यान्तिकमागतेषु नरः कश्चित् तत्समीपमुपस्थाय तं सजानुपातं जगाद,
\vakya प्रभो, मम पुत्रमनुकम्पतां यतः स चन्द्राहतोऽतीव क्लिश्यते च। स हि भूयो वह्नौ भूयश्च तोये पतति।
\vakya स च मया भवतः शिष्याणां समीपमानीतः, ते तु तं स्वस्थं कर्तुं नाशक्नुवन्।
\vakya यीशुरनेन प्रत्यवादीत्, रे अविश्वासिन् उन्मार्गगामिंश्च वंश, कियन्तं वा कालं स्थातव्यं मया युष्माभिः सार्धम्? कियन्तं वा कालं सोढव्या यूयं मया? बालकं तं मदन्तिकमत्रानयत।
\vakya ततो यीशुस्तं भर्त्सयामास भूतश्च स तस्मान्निःससार, दण्डाच्च तस्माद् यावद् बालकः स स्वस्थोऽभूत्।
\vakya ततः परं शिष्या निभृतं यीशोः समीपमागत्य तं पप्रच्छुः, किमतः कारणं यदस्माभिः स भूतो निःसारयितुं नाशक्यत?
\vakya यीशुस्तान् प्रतिजगाद, युष्माकमविश्वासस्तत्र हेतुः। यतोऽहं युष्मान् सत्यं ब्रवीमि, सति विश्वासे युष्माकं सर्षपवीजमिते यूयं चेदमुं पर्वतं वदथ, त्वमतोऽपसरन्नदो याहीति, तर्हि सोऽपसरन् यास्यति किमपि च न भविष्यति युष्माभिरशक्यं।
\vakya परन्त्वेतज्जातीयो भूतो न निःसार्यते प्रार्थनोपवासाभ्यामन्येन केनाप्युपायेन।
\stitle{यीशुना स्वमरणमधि द्वितीयवारं भविष्यद्वाक्यकथनम्।}
\vakya ततः परं गालीले तेषां परिभ्रमणकाले यीशुस्तान् अब्रवीत्, मनुष्यपुत्रो मनुष्याणां हस्तेषु समर्पयिष्यते ते च तं हनिष्यन्ति,
\vakya तृतीये दिने स पुनरुत्थास्यति च। अनेन तेऽतीव शोकाकुला बभूवः।
\stitle{मीनमुखे मुद्रा।}
\vakya तेषु तु कफरनाहूममागतेष्वर्धशेकलमानानां मुद्राणामादायिनः पित्रस्य समीपमुपस्थाय जगदुः, न दीयते किं युष्माकं गुरुणा करो मन्दिरार्थकोऽर्धशेकलमानः?
\vakya सोऽब्रवीत्, दीयते। गृहं प्रविष्टे तु तस्मिन् यीशुस्तस्य कथनमप्रतीक्ष्य जगाद, भो पित्र, त्वया किमनुमीयते? मेदिन्या राजानः केभ्यः शुल्कं करं वा गृह्णन्ति? किं स्वपुत्रेभ्योऽथवान्येभ्यः?
\vakya पित्रस्तं जगाद, अन्येभ्यः। यीशुस्तमब्रवीत्, तथा ते पुत्राः सुतरां निष्कराः सन्ति।
\vakya किन्त्वमी यदस्माभि र्न स्खाल्येरन्, तत् त्वं समुद्रतटं गत्वा वडिशं क्षिप, प्रथमश्च यो मीन उदेष्यति, तं धर, तस्य मुखमुद्घाट्य च त्वं शेकलमानां मुद्रामेकां द्रक्ष्यसि, तामेवादाय मम तव च कृतेऽमीभ्यो देहि\eoc