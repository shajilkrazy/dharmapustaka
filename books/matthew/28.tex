\adhyAya
\stitle{यीशोरुत्थानम्।}
\vakya विश्रामवारे त्वतीते सप्ताहस्य प्रथमदिवसस्य प्रातःकाले मग्दलीनी मरियम् सा चान्यतरा मरियम् तच्छवागारं द्रष्टुमागच्छताम्।
\vakya पश्य च तदा महान् भूकम्पोऽभूत्, यतः प्रभोरेको दूतः स्वर्गाद् अवरुह्य समीपमागत्य प्रस्तरं तं द्वाराल्लोठयित्वा तदुपर्युपविवेश।
\vakya तस्य तु विद्युदिवाभा तुषारवच्छुक्लश्च परिच्छद आस्तां
\vakya तस्य भयाच्च रक्षकाः समुद्विग्ना मृतकल्पाश्च बभूवुः।
\vakya स दूतस्ते योषिते प्राह, युवां मा भैष्टं यतोऽहं जाने युवां क्रुशारोपितं यीशुम् अन्विष्यथः।
\vakya स त्वत्र न विद्यते यतः स यथोक्तवांस्तथैवोत्थापितः।
\vakya आयातं यत्र प्रभुरशेत स्थानं तन्निरीक्षेथां गत्वा च सत्वरं तस्य शिष्यान् वदतं स मृतानां मध्यादुत्थापितः पश्यत च स युष्मदग्रतो गालीलं याति, तत्रैव तं द्रक्ष्यथ।
\vakya पश्यतमहं युवाभ्यां कथितवान्। ततस्ते सभयं महानन्देन च सत्वरं शवागारात् प्रस्थाय तस्य शिष्येभ्यः संवादं दातुम् अद्रवताम्।
\vakya पश्य च शिष्येभ्यस्तयोः संवादं दातुं गच्छत्योर्यीशुस्ते साक्षात्कृत्याब्रवीत्, युवयो र्मङ्गलं भूयात्।
\vakya ते चोपागत्य तस्य चरणौ धृत्वा भजनाम् अकार्ष्टाम्। तदा यीशुस्ते अवादीत्, मा भैष्टं गत्वा तु मम भ्रातॄन् ममैनमादेशं ज्ञापयतं यत् तै र्गालीलं गन्तव्यं, तत्रैव ते मां द्रक्ष्यन्तीति।
\vakya यावत् ते गच्छतः पश्य तावद् रक्षकवर्गस्य केचिन्नरा नगरमुपस्थाय यद्यद् वृत्तं तत् सर्वं मुख्ययाजकेभ्यो निवेदयामासुः।
\vakya ते तु प्राचीनैः सह समेत्य मन्त्रणाञ्च कृत्वा सैनिकनरेभ्यो यथेष्टा रौप्यमुद्रा ददु र्बभाषिरे
\vakya च यूयं वदत, शिष्यास्तस्य रात्रावागत्य निद्रितेष्वस्मासु तम् अचूचुरन्निति।
\vakya यदि च देशाधिपतेः समक्षम् अस्य श्रवणं भवेत्, वयं तर्हि तम् अनुनेष्यामो युष्मांश्च निःशङ्कान् करिष्यामः।
\vakya ततस्ते ता मुद्रा गृहीत्वा यथा शिक्षितास्तथैव चक्रुः। सा च कथा यिहूदिनां मध्ये व्याप्याद्याप्यवतिष्ठते।
\stitle{ख्रीष्टस्य शिष्यैर्दर्शनम् आदेशश्च।}
\vakya एकादश शिष्यास्तु गालीलं गत्वा यीशुना निर्दिष्टे गिरावुपतस्थिरे।
\vakya तञ्च दृष्ट्वा तस्य भजनां चक्रुः। केचित्तु समशेरत।
\vakya तदा यीशुः समीपमागत्य तैः संलपन् बभाषे, स्वर्गमेदिन्यो र्निखिलं सामर्थ्यं मह्यम् अदायि।
\vakya अतो यूयं गत्वा यावती र्जातीः शिष्यान् कुर्वन्तः पितुः पुत्रस्य च पवित्रस्यात्मनश्च नामोद्दिश्य तान् स्नापयत,
\vakya शिक्षयत च तांस्तत्सर्वस्य रक्षणं युष्मानहं यद्यदादिष्टवान्। अधिकन्तु पश्यत, युगान्तं यावत् सर्वदिनान्यहं युष्माभिः सार्धमासे\eoc