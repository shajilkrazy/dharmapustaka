\adhyAya
\stitle{स्त्रीत्यागमधि शिक्षा।}
\vakya वाक्यान्येतानि समाप्य यीशु र्गालीलात् प्रस्थाय यर्द्दनस्य पारे स्थितां यिहूदियायाः सीमानमागमत्।
\vakya महान्तो जननिवहाश्च तमन्वव्रजन् स च तत्र तान् निरामयान् चकार।
\vakya अनन्तरं फरीशिनस्तस्यान्तिकमागत्य परीक्षमाणास्तं पप्रच्छुः, विधेयः किं मनुष्येण भार्यात्यागो येन केनचिद्धेतुना?
\vakya स तु तान् प्रत्यब्रवीत्, युष्माभिः किमेतन्न पठितं भो यदादौ निर्माता नरं नारीञ्च तौ निर्मितवान् कथितवांश्च,
\vakya एतद्धेतो र्मनुष्यः पितरं मातरञ्च परित्यज्य स्वजायायामासंक्ष्यते तावुभौ चैकाङ्गीभविष्यत इति?
\vakya अनेन तौ पुन र्न द्वौ स्तः, तावेकाङ्गीभूतौ। अतो यद् ईश्वरेण संयोजितं, तन्मनुष्येण मा वियोज्यतां।
\vakya ते तं पप्रच्छुः, तर्हि त्यागपत्रदानपूर्वकं तस्यास्त्यागो मोशिना कथमादिष्ट?
\vakya स तान् जगाद, युष्माकं हृत्काठिन्यमुद्दिश्य युष्मदीयभार्याणां त्यागो मोशिनान्वज्ञायि न त्वादितो भूतमित्थम्।
\vakya अहन्तु युष्मान् वदामि व्यभिचारादन्यहेतुना यः कश्चित् स्वभार्यां त्यक्त्वापरामुद्वहति, स व्यभिचारं करोति, यश्च स्वामित्यक्तां स्त्रियमुद्वहति स व्यभिचारं करोति।
\vakya तस्य शिष्मास्तमूचुः, भार्यया सह यदि नरस्येदृशः सम्बन्धस्तर्ह्युद्वाहो न हितावहः।
\vakya स तु तान् जगाद, वचनमेतन्न सर्वै र्गृह्यते, तैरेव केवलं गृह्यते सामर्थ्यं येभ्योऽदायि।
\vakya फलतो विद्यन्ते च क्लीवा आमातृजठरात् तथाविधा जाताः, नरकृतक्लीवाश्च विद्यन्ते, स्वर्गराज्यार्थम् आत्मकृतक्लीवाश्च विद्यन्ते। गृह्णातु यो ग्रहणे समर्थः।
\stitle{शिशुविषयिणी शिक्षा।}
\vakya शिशवः केचित् तदानीं तस्यान्तिकमानिन्यिरे यत् तेन तेषु हस्तावर्पयित्वा प्रार्थना क्रियेत। शिष्यास्तु तान् अभर्त्सयन्।
\vakya यीशुः पुनरवादीत्, मत्समीपमागमिष्यतः शिशून् अनुमन्यध्वं मा वारयत यतः स्वर्गराज्यम् ईदृशानामेव। 
\vakya स तदा तेषु हस्तावर्पयित्वा प्रतस्थे।
\stitle{धनसम्बन्धिशिक्षा।}
\vakya पश्य चानन्तरं नरः कश्चिन्निकटमागत्य तं पप्रच्छ, भो सद्गुरो, सत् किं कर्तव्यं मयानन्तजीवनलाभार्थम्?
\vakya स तु तमब्रवीत्, किमर्थं मां सतः कथां पृच्छसि? सन्नेक एव। यदि तु जीवनं प्रवेष्टुमिच्छसि तर्ह्याज्ञाः पालय।
\vakya स ब्रूते, का आज्ञाः? यीशुरनेनाब्रवीत्, नरहत्यां मा कुरु, व्यभिचारं मा कुरु, चौर्यं मा कुरु, मृषासाक्ष्यं मा देहि, स्वपितरं स्वमातरञ्च सम्मन्यस्व।
\vakya अपि च स्वनिकटस्थं प्रत्यात्मवत् प्रेम कुर्विति।
\vakya युवा स त जगाद, पालितं मया सर्वमेतदाबाल्यात्, अधुना मम किमसम्पूर्णम्?
\vakya यीशुस्तं जगाद, यदि सिद्धो भवितुमिच्छसि, तर्हि गत्वा सर्वं तव यद्यदस्ति देहि विक्रीय दरिद्रेभ्यः। तथा कृते स्वर्गे तव धनं स्थास्यति।ततः परमागत्य मामनुव्रज।
\vakya वचनमेतच्छ्रुत्वा युवा स शोचन्नापजगाम, यतस्तस्य वसूनि प्रचुराण्यासन्।
\vakya यीशुस्तदा स्वशिष्यानुवाच, युष्मानहं सत्यं ब्रवीमि, दुष्करो हि धनवतः स्वर्गराज्यप्रवेशः।
\vakya पुनश्च युष्मान् ब्रवीमि, धनवतः स्वर्गराज्यप्रवेशात् सूचीच्छिद्रेणोष्ट्रगमनं सुसाध्यं।
\vakya एतच्छ्रुत्वा शिष्याश्चमत्कारमतीव गत्वा जगदुः, कस्तर्हि तरितुं शक्नोति?
\vakya यीशुस्तदा तान् समालोक्याब्रवीत्, तन्मनुष्याणामसाध्यम्, ईश्वरस्व तु सर्वं साध्यम्।
\vakya पित्रस्तदानीं तं प्रत्यब्रवीत्, पश्यतु वयं सर्वं त्यक्त्वा भवन्तम् अनुव्रजितवन्तः, किमस्माभि र्लप्स्यते?
\vakya यीशुस्तान् जगाद, युष्मानहं सत्यं ब्रवीमि, पुनर्जन्मनि यदा मनुष्यपुत्रः स्वप्रतापसिंहासन उपविष्टो भविष्यति, तदा मामनुव्रजितवन्तो यूयमपि द्वादशसिंहासनेषूपविष्टा इस्रायेलस्य द्वादशवंशानां विचारं करिष्यथ।
\vakya अपरं यः कश्चिन्मम नाम्नः कृते गृहाणि वा भ्रातॄन् वा भगिनी र्वा पितरं वा मातरं जायां वा सन्तानान् वा भूम्यधिकारान् वा त्यक्तवान् स तच्छतगुणं लप्स्यते, दायांशरूपमनन्तजीवनञ्च प्राप्स्यति।
\vakya अपि तु प्रथमा बहवोऽन्त्या भविष्यन्ति, अन्त्याश्च बहवः प्रथमा भविष्यन्ति\eoc