\adhyAya
\stitle{विवाहभोजनदृष्टान्तः।}
\vakya तदुत्तरं यीशुरिदं पुनस्तेभ्यो दृष्टान्तैः कथयामास, स्वर्गराज्यं राज्ञा तेन सदृशं यः स्वपुत्रस्योद्वाहोत्सवं चकार,
\vakya नरांश्च निमन्त्रितान् उद्वाहोत्सवायाह्वातुं स्वदासान् प्राहिणोत्।
\vakya ते त्वागन्तुमभवन्न सम्मताः।
\vakya स पुन र्दासानपरान् प्रेषयन्निदमादिदेश, यूयं नरांस्तान् निमन्त्रितान् वदत, पश्यत, पूर्वाह्णिकं भोज्यं मम सज्जीकृतं, पशवश्च पुष्टा गवादयो मारिताः, सर्वमेव प्रस्तुतं। यूयमुद्वाहोत्सवमायातुमर्हथ।
\vakya ते त्ववहेलयन्तः स्वक्षेत्रमेकः स्वबाणिज्यं वापरो जगाम,
\vakya शेषाश्च दूतांस्तस्य धृत्वा न्यक्कारपूर्वकं जघ्नुः।
\vakya तच्छ्रुत्वा राजा स प्रकुप्य स्वबलानि प्रहित्य घातकांस्तान् नाशयामास नगरश्च तेषां दाहयामास।
\vakya ततः स स्वदासान् व्याजहार, उद्वाहः सज्जीकृतो निमन्त्रितास्ते पुनरयोग्या आसन्।
\vakya यूयमतस्तावत उद्वाहोत्सवायाह्वयत राजमार्गाणां मंयोगस्थानानि गत्वा यावतो यूयं द्रक्ष्यथ।
\vakya ततो दासास्ते राजमार्गान् गत्वा दुर्जनान् सुजनांश्च यावत आसादयन् तावतः समानिन्युः। तेनोद्वाहगृहं जनैः समासीनैः पूर्णमभूत्।
\vakya राजा तदा भोज्योपविष्टान् मनुष्यान् निरीक्षितुं प्रविश्योद्वाहवसनहीनं जनमेकं तत्र दृष्ट्वा पप्रच्छ,
\vakya मित्र उद्वाहवसनविहीनस्त्वं कथमेतत् स्थानं प्रविष्टः? अनेन स निर्वाक् जातः।
\vakya राजा तदा परिचारकान् जगाद, यूयमस्य चरणौ हस्तौ च बन्धीत धृत्वा चैनं बहिःस्थे तिमिरे निक्षिपत तत्र रोदनं दन्तै र्दन्तघर्षणञ्च भविष्यतः।
\vakya वास्तवं ह्याहूता बहवः स्वल्पे तु वरिताः।
\stitle{यीशुना शत्रूणां कतिपयप्रश्नानां उत्तरदानम्।}
\vakya फरीशिभि र्गत्वा तदा मन्त्रयाञ्चक्रे कथं स वाक्‌पाशेन निबध्यत इति।
\vakya ततस्ते स्वशिष्यान् हेरोदीयैः सहितान् तस्यान्तिकं प्रेषयामासुः। त ऊचुः, भो गुरो, वयं जानीमो यद् भवान् सत्यवान् सत्येनेश्वरस्य पन्थानं शिक्षयति च, कस्मादपि न बिभेति, यतो भवान् मनुष्याणां मुखापेक्षां न करोति,
\vakya अतो भवानस्मान् वक्तुमर्हति किं मन्यते भवान् कैसराय करदानं विधेयं न वेति।
\vakya यीशुस्तु तेषां दुष्टतां ज्ञात्वा जगाद, किमर्थं मां परीक्षध्वे कपटिनः?
\vakya करदानस्य मुद्रामेकां मां दर्शयत।
\vakya तैस्त्वेकस्मिन्नानीते दीनारे स तान् जगाद, कस्येयं मूर्तिरिदं लेखनञ्च? ते तं वदन्ति, कैसरस्य।
\vakya तदा स तानब्रवीत्, दत्त तर्हि कैसराय यद्यत् कैसरस्य, दत्त चेश्वराय यद्यदीश्वरस्य।
\vakya श्रुत्वैतत्त आश्चर्यं मेनिरे तं विहाय च प्रतस्थिरे।
\vakya दिने तस्मिन् पुनरुत्थानमनङ्गीकुर्वाणाः सद्दूकिनस्तस्यान्तिकमागत्य तमूचुः,
\vakya भो गुरो, मोशिनादिष्टं यो निःसन्तानो म्रियते, तस्य भ्रात्रा तद्भार्यामुदुह्य स्वभ्रात्रे वंशमुत्पादयिष्यतीति।
\vakya अस्माकं मध्ये तु सप्त भ्रातर आसन्, प्रथमो विवाहं कृत्वा ममार, निःसन्तानत्वाच्च स्वभार्यां भ्रातुः कृते तत्याज।
\vakya ततः परं द्वितीयस्य तृतीयस्य चैवं सप्तानामेव तादृशी गति र्बभूव।
\vakya सर्वेषां पश्चाद् योषित् सापि ममार।
\vakya अतः पुनरुत्थाने सा तेषां सप्तानां कस्य भार्या भविष्यति? यतस्ते सर्व एव तामुदूढवन्तः।
\vakya यीशुस्तदा तान् प्रत्यवादीत्, यूयं शास्त्राणीश्वरस्य शक्तिञ्च न जानीथ हेतोरतो भ्राम्यथ।
\vakya यतः पुनरुत्थाने मनुष्या नोद्वहन्ति नोदुह्यन्ते वा, प्रत्युतेश्वरस्य दूता इव स्वर्गे वर्तन्ते।
\vakya न पठितं भो मृतानां पुनरुत्थानमधि युष्मभ्यमीश्वरेण कथितमिदं वाक्यं, यथा,
\vakya अब्राहामस्येश्वर इस्‌हाकस्य चेश्वरो याकोबस्य चेश्वरोऽहमिति। ईश्वरो न मृतानाम्, अपि तु जीवताम् ईश्वरोऽस्ति। श्रुत्वेदं जननिवहास्तस्य शिक्षां चमत्कारं मेनिरे।
\vakya फरीशिनस्तु सद्दूकिनां तेन निरुत्तरीभवनम् अवगत्यैकत्र समाजग्मुः,
\vakya तेषां मध्ये चैको व्यवस्थावेत्ता परीक्षमाणस्तं पप्रच्छ,
\vakya भो गुरो, व्यवस्थायाः काज्ञा महती?
\vakya यीशुस्तमब्रवीत्, त्वं कृत्स्नान्तःकरणेन कृत्स्नप्राणैः कृत्स्नचित्तेन च स्वेश्वरे प्रभौ प्रेम कुर्वितीयं,
\vakya प्रथमा महती चाज्ञा।
\vakya द्वितीया चास्याः सदृशी, यथा त्वम् स्वनिकटस्य आत्मवत् प्रेम कुर्विति।
\vakya आज्ञयोर्द्वयोरेतयोः कृत्स्ना व्यवस्था भाववादिनश्च समालम्बन्ते।
\stitle{यीशोः शत्रूणां निर्वाक्‌त्वम्।}
\vakya ततः समेतेषु फरीशिषु यीशुस्तान् पप्रच्छ, ख्रीष्टमधि युष्माभिः किं मन्यते? कस्य पुत्रः सः?
\vakya ते तं वदन्ति दायूदस्येति।
\vakya स तान् ब्रवीति, आत्मन आवेशाद् दायूदेन कथं स तर्हि प्रभुरित्यभिधीयते? यथा,
\begin{poem}
\startwithvakya “मम प्रभुमिदं वाक्यं बभाषे परमेश्वरः। 
\pline त्वच्छत्रून् पादपीठं ते यावन्नहि करोम्यहं।
\pline अवतिष्ठस्व तावत् त्वम् आसीनो मम दक्षिणे॥”
\end{poem}
\vakya तद् यदि स दायूदेन प्रभुरित्यभिधीयते, कथं तर्हि स तस्य पुत्रो भवेत्?
\vakya एतस्योत्तरे केनापि वाक्यमेकं वक्तुं नाशक्यत, नाभवच्च तद्दिनादारभ्य साहसं कस्यापि तं किमपि प्रष्टुं\eoc
