\adhyAya
\stitle{योहनस्नापकस्य वधः।}
\vakya तस्मिन् काले चतुर्थांशाधिपति र्हेरोदो यीशोः ख्यातिं श्रुत्वा स्वदासान् जगाद,
\vakya असौ स्नापको योहनः, मृतानां मध्यादुत्थितः सः। अतो हि प्रभाववैभवं तस्मिन् स्वकार्यं कुरुते।
\vakya तद्यथा हेरोदः स्वभ्रातुः फिलिपस्य भार्याया हेरोदियाभिधायाः कृते योहनं धृत्वा बद्ध्वा च कारायां निहितवान्।
\vakya यतो योहनस्तमवदत्, न विध्यं भवता तस्याः पतित्वं।
\vakya तं हन्तुकामस्तु स जननिवहादबिभेत्, यतः सर्वे तं भाववादिनममन्यन्त।
\vakya हेरोदस्य जन्मदिनोत्सवे तु हेरोदियाया दुहिता सभामध्ये नृत्यन्ती हेरोदायारोचत।
\vakya स च सशपथं प्रत्यज्ञासीत्, किमपि यत् त्वं याचिष्यसे तदेवाहं तुभ्यं दास्यामि।
\vakya तदा सा स्वमात्रा प्रवर्तिता जगाद, स्नापकस्य योहनस्य शिरः स्थाले निधायात्र मह्यं ददातु।
\vakya अनेन राजाशोचीत् शपथानां कृते सहभोजिनश्चापेक्ष्य
\vakya तु तस्य दानमादिदेश प्रेष्यञ्च प्रहित्य कारायां योहनस्य शिरश्छेदयामास।
\vakya स्थाले निहितमानीतञ्च तस्य शिरो तत्तञ्च तद् बालायै, उपहृतञ्ज तया स्वमातुः समीपम्।
\vakya ततः परं शिष्यास्तस्यागत्य शवमादाय शवागारे निदधुः, यीशोः समीपमागत्य च तस्मै संवादं निवेदयामासुः।
\stitle{यीशुना पञ्चसहस्रलोकेभ्यो भोजनदानम् पदव्रजेन जलसञ्चरणञ्च।}
\vakya तदाकर्ण्य यीशुस्तस्मात् स्थानात् प्रस्थाय नौकया निर्जनं स्थानं किञ्चन निभृत जगाम, जननिवहास्तु तच्छ्रुत्वा नगरेभ्यो निर्गत्य स्थलमार्गेण तमनुवव्रजुः।
\vakya यीशुस्तदा बहिरागत्य महान्तं जननिवहं दृष्ट्वानुचकम्पे चकार च निरामयान् तेषां मध्ये रोगिणः।
\vakya सन्ध्यायान्तूपस्थितायां तस्य शिष्यास्तत्समीपमागत्य जगदुः, स्थानमेतन्निर्जनम् आगतञ्च दिनावसानं, तद् विसृज्यन्तां भवता जननिवहा यथामि ग्रामान् गत्वा खाद्यानि स्वार्थं क्रीणीयुः।
\vakya यीशुस्तु तान् जगाद, तेषामपगमनमनावश्यकं, यूयं वितरत तेभ्यो भक्ष्याणि।
\vakya ते तमब्रुवन् पूपान् पञ्च मीनौ च द्वी विहायास्माकमत्र किमपि नास्ति।
\vakya स तान् जगाद, तानेवात्र मदन्तिकमानयत।
\vakya ततः स जननिवहानां शष्योपर्युपवेशनमादिदेश, तान् पूपान् पञ्च मीनौ च द्वौ चादाय स्वर्गं प्रत्यूर्ध्वदृष्टिं कृत्वाशीर्वाचनं चकार, ततः पूपान् भङ्क्त्वा शिष्येभ्यो ददौ, ते च जननिवहेभ्यो ददुः।
\vakya भुक्त्वा च सर्वे ततृपुः। भग्नानामंशानां शेषेण च ते द्वादशडल्लकान् पूरयित्वाददुः।
\vakya योषितो बालकांश्च विहाय भोक्तारस्ते प्रायेण पुरुषाः पञ्चसहस्राण्यासन्।
\vakya अनन्तरं यीशुः शिष्यांस्तत्क्षणं नौकामारोह्य तावदात्मनः प्राक् पारं गन्तुं साग्रहमादिदेश, स यावत् जननिवहं विसृजति।
\vakya जननिवहं विसृज्य स निभृतं प्रार्थनार्थं गिरिमारुरोह च। तत् सन्ध्यायाः परं स तत्रासीदेकाकी,
\vakya तदा तु मध्यसमुद्रं नौरूर्मिभिराहन्यत, यतो वायुरासीत् प्रतिकूलः।
\vakya यामे तु यामिन्याश्चतुर्थे यीशुस्तेषां समीपं गन्तुकामः समुद्रोपर्यव्रजत्,
\vakya शिष्यास्तु तं समुद्रोपरि व्रजन्तं दृष्टोदविजन्त त्रासाच्चुक्रुशुश्चोक्त्वा प्रतिच्छायासाविति।
\vakya यीशुस्तु तत्क्षणं तानालप्याब्रवीत्, आश्वसित, अहमेषः, मा भैष्ट।
\vakya पित्रस्तदा तमन्वब्रवीत्, प्रभो भवानेव चेत्, तर्हि तोयोपरि भवतोऽन्तिकं मम गमनमनुमन्यतां।
\vakya स जगाद, एहि। ततः पित्रो नौकातोऽवरुह्य यीशोः समीपगमनार्थं तोयोपर्यव्रजत्।
\vakya वायुन्तु बलवन्तं दृष्ट्वाबिभेत्, निमज्जितुमारभ्य च चुक्रोश, प्रभो मां तारयतु।
\vakya यीशुस्तत्क्षणं हस्तं प्रसार्य तं दधार जगाद च, भो स्तोकविश्वासिन्, किमर्थं समशेथाः?
\vakya ततस्तयो र्नौकामारूढयो र्वायुः शशाम
\vakya नौकास्थजनागत्य प्रणिपत्य च तं जगदुः,सत्यं भवान् ईश्वरस्य पुत्रः।
\stitle{भूरिलोकानां निरामयत्वञ्च।}
\vakya तरित्वेत्थं ते प्रदेशे गिनेषरताख्य उपतस्थिरे। स्तानस्य तस्य नराश्च तमभिज्ञाय
\vakya कृत्स्नं जनपदं जनान् प्रहित्यास्वस्थान् मनुष्यान् सर्वांस्तस्य समीपमानाययामासुः
\vakya प्रार्थयाञ्चक्रिरे च ते तेषां कृते तदीयवस्त्रप्रलम्बकस्य स्पर्शनानुमतिमेव, तेरुश्च तावन्तो यावन्तःस्पर्शनं चक्रुः\eoc