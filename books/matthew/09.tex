\adhyAya
\stitle{यीशुना जनैकपक्षाघातिने आरोग्यदानं तस्य पापमर्षणञ्च।}
\vakya अनन्तरं स नौकामारूह्य ह्रदं तरित्वा च स्वनगरमाजगाम।
\vakya पश्य च तदा खट्वायां शयानो नर एकोऽवशाङ्गस्तत्समीपमानीयत। यीशुश्च तेषां विश्वासं दृष्ट्वा तमवशाङ्गं जगाद, आश्वसिहि, वत्स, तव पापानि मोचितानि।
\vakya पश्य च तदा शास्त्राध्यापकानां केचित् स्वान्तर्जगदुः, असावीश्वरं निन्दति।
\vakya यीशुस्तु तेषां मनोगतान्यवलोक्य जगाद, कुतो युष्माभिः स्वहृदयेषु कुचिन्ताश्चिन्त्यन्ते?
\vakya ब्रूत, तव पापानि मोचितानीति वा त्वमुत्थाय परिव्रजेत्येतयोः कथनयोः
\vakya कतरम् अनायासं? पृथिव्यान्तु पापानि मोचयितुं मनुष्यपुत्रस्य सानर्थ्यमस्तीति यथा युष्माभि र्ज्ञायते, तदर्थं -तमवशाङ्गमभिभाष्य स उवाच- त्वमुत्थाय स्वखट्वामादाय स्वगेहं याहि।
\vakya स तूत्थाय स्वगेहं जगाम।
\vakya तद् विलोक्य जननिवहा भयं गत्वा तुष्टुवुरीश्वरं मनुष्येभ्यस्तादृशसामर्थ्यं दत्तवन्तम्।
\stitle{मथ्याह्वानं। यीशोस्तंद्विषयकशिक्षा।}
\vakya अनन्तरं यीशुस्तस्मात् स्थानात् प्रगत्य शुल्कादायस्थाने मथिरित्यभिधं नरं दृष्ट्वा जगाद, मामनुगच्छ, स चोत्थाय तमन्वगच्छत्,
\vakya ततः पश्य गेहमध्ये भोजनार्थमासीने तस्मिन् शुल्कादायिनो बहवः पापिनश्चागत्य यीशुना तदीय शिष्यैश्च सार्धम् उपविवशुः।
\vakya तद्दृष्ट्वा फरीशिनस्तस्य शिष्यान् जगदुः, किमर्थं भुङ्क्ते गुरु र्युष्माकं शुन्कादायिभिः पापिभिश्च सार्धम्?
\vakya तच्छ्रुत्वा यीशुस्तु तान् जगाद, चिकित्सको न बलवताम् अपि त्वस्वस्थानामावश्यकः।
\vakya यूयन्तु गत्वा शिक्षध्वं वचनस्यैतस्य तात्पर्यम्, “दयामेवाभिवाञ्छामि न तु यज्ञक्रियामहं।” यतो न धार्मिकान् अपि तु मनःपरावर्तनाय पापिन आह्वातुमहमागतः।
\vakya योहनस्य शिष्यास्तदा तस्य समीपमागत्य जगदुः, कथं वा फरीशिनो वयञ्च भूयोभूय उपवसामः, भवतः शिष्या वा कथं नोपवसन्ति?
\vakya यीशुस्तान् जगाद, वरो यावद् वरसखैः सार्धं वर्तते, तावत् किं तैः शोचितुं शक्यते? दिनानि तावदुपस्थास्यन्ते यदा वरस्तेभ्योऽपहारिष्यते तदैव च त उपवत्स्यन्ति।
\vakya जीर्णवसनेन न कोऽप्यनाहतवस्त्रस्य खण्डं सिव्यति, यतस्तादृशपूरणेन जीर्णवसनात् कियदपह्रियते, महत्तरं तेन च कुछिद्रं सञ्जायते।
\vakya नापि नवो द्राक्षारसो जीर्णकुतूषु निधीयते, यतस्तथाकृते कुत्वो विदीर्यन्ते, तेन द्राक्षारसश्च विस्रवति, कुत्वोऽपि नश्यन्ति। प्रत्युत नवो द्राक्षारसो नवकुतूष्वेव निधीयते, तेन चोभयोरक्षा सम्भवति।
\stitle{यीशुना प्रदररोगिण्वाः आराग्यकरण मृतकन्यायै जिवनदानञ्च।}
\vakya स यावत् तानिमानि वाक्यान्यदत्, पश्य तावदागत्य कश्चिदध्यक्षस्तत्समक्षं प्रणिपतन् जगाद, मम दुहितेदानीमेव मृतवती, तथापि भवान् आगत्य तस्यां हस्तमर्पयतु, तेन सा जीविष्यति।
\vakya ततो यीशुस्तस्य शिष्याश्चोत्थाय तमनुजग्मुः।
\vakya पश्य चानन्तरं द्वादशवर्षान् यावद् रक्तस्रावातुरा काचिन्नारी पश्चाद्दिश्युपागत्य तदीयवस्त्रप्रलम्बकं पस्पर्श,
\vakya यतः सा स्वान्तरवदत्, केवलं चेत् तस्य वस्त्रं स्पृशेयं, तर्हि तरिष्यामि।
\vakya यीशुस्तु परावृत्य दृष्ट्वा च तां जगाद, आश्वसिहि, वत्से, तव विश्वासस्त्वां तारितवान्। ततार च सा नारी तस्मिन्नेव दण्डे।
\vakya अनन्तरं यीशुस्तस्याध्यक्षस्य गेहमागत्य वंशीवादकान् शब्दायमानं महान्तं जननिवहञ्च दृष्ट्वा जगाद,
\vakya अपगच्छत् यतो बालिकासौ न मृता, सा निद्राणा।
\vakya ते तु तमपजहसुः। बहिष्कृते तु जननिवहे तेनान्तः प्रविश्य तस्या हस्तो धृतः, बालिका चोत्तस्थे।
\vakya अनेन तस्य ख्यातिः कृत्स्नं तं प्रदेशं व्यानशे।
\stitle{यीशुना अन्धद्वाय नेत्रदानं मूकाय आरोग्यदानञ्च।}
\vakya तस्मात् प्रगच्छति यीशौ द्वावन्धौ तमनुव्रजन्तौ प्रोच्चरवेणावदतां, भो दायूदस्य पुत्र भवानावामनुकम्पतां।
\vakya गेहं प्रविष्टे च तस्मिंस्तावन्धौ तत्समीपमाजग्मतुः। यीशुस्तदा तौ जगाद, एतत् कर्तुं मया शक्यत इति विश्वासो युवयोः किमस्ति?
\vakya तावूचतुः, अस्ति, प्रभो। स तदा तयो र्नेत्राणि स्पृष्ट्वा कथयामास, विश्वासानुरूपं वां स्ध्यतु।
\vakya अभवंश्चोद्घाटितानि तावत् तयो र्नेत्राणि। यीशुस्ततस्तौ प्रत्यन्यथावृत्तिं गत्वा जगाद, यतेथां, कोऽप्येतन्मा जानातु।
\vakya तौ तु निर्गत्य कृत्स्ने प्रदेशे तस्मिन् तत्ख्यातिं कीर्तयामासतुः।
\vakya पश्य च तयो र्निर्गमनकाले नर एको मूको भूताविष्टस्तत्समीपमानीतः।
\vakya ततो निःसारिते च भूते मूकः सोऽभाषत। अनेन जननिवहा आश्चर्यं मत्वा जगदुः, इस्रायेले नेदृशं कदापि प्रतिभातं।
\vakya फरीशिनस्तूचुः, निःसारयत्यसौ भूतान् भूतराजस्य साहाय्येन।
\stitle{यीशुना द्वादशशिष्याणां प्रेरितपदे नियोगः।}
\vakya ततः परं यीशु र्नगराणि सर्वाणि ग्रामांश्च सर्वान् परिभ्राम्यन् तेषां समाजगेहेषूपादिशत् राज्यस्य सुसंवादञ्चाघोषयत् प्रत्यकरोच्च तथा प्रजावृन्देषु निखिलरोगं निखिलामयञ्च।
\vakya निरीक्ष्य च स जननिवहान् तेषु सकरुणो बभूव, यतस्ते व्याकुला अवसन्नाश्चासन् मेषा इव रक्षकविहीनाः।
\vakya स तदा स्वशिष्यानवादीत्, प्रचुरं तावच्छस्यं कर्तनीयं कार्यकारिणस्त्वल्पे, तत् तदेव प्रार्थयध्वं शस्यक्षेत्रस्य स्वामिनं यत् स स्वक्षेत्रे कार्यकारिणः प्रहिणुयात्\eoc