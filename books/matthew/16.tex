\adhyAya
\stitle{यीशो र्नानाविधशिक्षा।}
\vakya तदा फरीशिनः सद्दूकिनश्च तस्यान्तिकमागत्य तं परीक्षमाणा गगनात् अभिज्ञानस्य कस्यचित् प्रदर्शनं ययाचिरे।
\vakya स तु तान् प्रतिजगाद, सन्ध्यायामुपस्थितायां यूयं वदथ, सुदिनं, यतो भात्याकाशो रक्तवर्णः।
\vakya प्रातश्च वदथ, वात्याद्य, यतो भात्याकाशो मलिनो रक्तवर्णश्च। भो कपटिनः, यूयम् आकाशस्य रूपं निर्णेतुं जानीथ, न शक्नुथ किं कालानामभिज्ञानानि निर्णेतुं?
\vakya दुष्टो व्यभिचारी च वंशोऽभिज्ञानमनुसन्धत्ते, तस्मै तु भाववादिनो योनाहस्याभिज्ञानादन्यदभिज्ञानं न दायिष्यते। अनन्तरं स तान् विहाय प्रतस्थे।
\vakya ततः परं शिष्याः पारमागत्य पूपानादातुं विस्मृताः।
\vakya यीशुस्तु तानब्रवीत्, यूयमालोच्य फराशिनां सद्दूकिनाञ्च किण्वतः सावधाना भवतः।
\vakya अनेन ते मिथो विचारयन्तो जगदुः, अस्माभि र्यत् पूपा नादत्ता इति।
\vakya यीशुस्तु तज्‌ज्ञात्वा तानब्रवीत्, भो स्तोकविश्वासिनः, युष्माभिः पूपा नादत्ता इत्येवं कथं मिथो विचारयथ?
\vakya अधुनापि किं न बुध्यते? अपि न स्मरथ किं पञ्चसहस्रेषु तान् पञ्चपूपान् लब्धडल्लकसङ्ख्याञ्च,
\vakya चतुःसहस्रेषु च तान् सप्तपूपान् लब्धपेटकसङ्ख्याञ्च?
\vakya युष्माभिः कथं न बुध्यते यत् फरीशिनां सद्दूकिनञ्च किण्वतो युष्माभिः सावधावै र्भवितव्यमित्येतदधि मया कथितं न पूपानधीति?
\vakya तदा तैस्तस्योक्तेस्तात्पर्यमित्येवमबोधि यदस्माभिः फरीशिनां सद्दूकिनाञ्च शिक्षातः सावधानै र्भवितव्यं न तु पूपस्य किण्वत इति।
\stitle{यीशुरेव स ख्रीष्टः जीवनमयस्येश्वरस्य पुत्रः।}
\vakya ततः परं यीशुः कैसरिया-फिलिप्याः सन्निकटं प्रदेशमागत्य स्वशिष्यानप्रच्छत्, मनुष्यपुत्रो योऽहं सोऽहं क इत्यधि किं वदन्ति मानवाः?
\vakya ते प्रत्यूचुः, केचिद् वदन्ति, भवान् स्नापको योहनः, अपरे वदन्ति, भवान् एलियः, अन्ये च वदन्ति, भवान् यिरमियाहोऽथवा भाववादिनामन्यतमः कश्चिदिति।
\vakya स तान् जगाद, कोऽहमित्यधि यूयमेव वा किं वदथ?
\vakya शिमोनः पित्रस्तदा प्रतिजगाद, भवान् सोऽभिषिक्तः पुरुषो जीवनमयस्येश्वरस्य पुत्रः।
\vakya यीशुः प्रतिवदन् तं जगाद, भो योनाःपुत्र शिमोन, धन्यस्त्वं यतो न रक्तमांसाभ्यं तुभ्यमिदं व्यक्तीकृतम्, अपि तु मम स्वर्गस्थेन पित्रैव।
\vakya अहञ्च त्वां ब्रवीमि, त्वं पित्रः (पाषाणः), पाषाणस्यैतस्योपरि चाहं मम मण्डलीं प्रतिष्ठापयिष्यामि, पातालस्य गोपुराणि च न तस्याः प्रभविष्यन्ति।
\vakya तुभ्यञ्चाहं स्वर्गराज्यस्य कुञ्चिकां दास्यामि, किमपि यत् त्वं मेदिन्यां भन्त्स्यसि स्वर्गे तद् बद्धं स्थास्यति, यदपि च त्वं मेदिन्यां मोक्ष्यसि स्वर्गे तन्मुक्तं स्थास्यति।
\vakya स तांस्तदादिदेश, कस्मैचिदपि युष्माभि र्न कथयितव्यं यदहं सोऽभिषिक्तः पुरुष इति।
\stitle{यीशोः स्वमरणमधि भविष्यद्वाक्यकथनम्।}
\vakya ततो यावद् यीशुः स्वशिष्येभ्यो दर्शयितुमारेभे यदहं यिरूशालेमं गत्वा प्रचीनै र्मुख्ययाजकैः शास्त्राध्यापकैश्च प्रचुरं क्लेशं भोक्ष्ये घानिष्ये च तृतीये दिने तु पुनरुत्थास्यामीत्युपयुक्तं।
\vakya पित्रोऽनेन तं निभृतं नीत्वा भर्त्सयितुमारभ्याब्रवीत्, ईश्वरो भवन्तमनुकम्पतां, प्रभो, दशा सा भवतः कदापि न भविष्यति।
\vakya स तु परावृत्य पित्रं जगाद, मत्तोऽपसर, शैतन, त्वं मम स्खलनहेतुः, यतस्त्वया मानवं चिन्त्यते नेश्वरीयम्।
\vakya यीशुस्तदा स्वशिष्यान् जगाद, कश्चिच्चेन्मामनुगन्तुमिच्छति, स तर्ह्यात्मानं प्रत्याख्यातु, स्वक्रुशमाददातु मामनुव्रजतु च।
\vakya यतो यः कश्चित् स्वप्राणान् रिरक्षिषति स तान् हारयिष्यति, यस्तु मदर्थं स्वप्राणान् हारयति स तान् लप्स्यते।
\vakya यतो मनुष्यः कृत्स्नं जगत् लब्ध्वा यदि स्वप्राणै र्वञ्च्यते, तर्हि तस्य हितं वा किं भवति?
\vakya स्वप्राणानां कं वा निष्क्रयं मनुष्यो दास्यति? यतः स्वीयदूतैः सह मनुष्यपुत्रः स्वपितुः प्रतापेनागमिष्यति, तदा च स प्रत्येकं तदाचारानुरूपं फलं दास्यति।
\vakya युष्मानहं सत्यं वदामि, विद्यन्तेऽत्र तिष्ठतां मध्ये केऽपि जना यै र्मनुष्यपुत्रं स्वराज्यान्वितमागच्छन्तं न दृष्ट्वा मृत्यु र्नास्वादयितव्यः\eoc