\adhyAya
\stitle{प्रभो र्यीशोः शैशवविवरणम्।}
\vakya राज्ञो हेरोदस्य काले यिहूदियादेशस्ये बैतलेहमे यीशो र्जन्मनः परं, पश्य, प्राच्यदेशात् केचिज्ज्योतिर्विदो यिरूशालेममागत्य जगदुः, कुत्र स यिहूदीयानां नवजातो राजा?
\vakya यस्मादुदयदिशि तस्य नक्षत्रं दृष्टवन्तो वयं तं प्रणिपतितुमागताः।
\vakya श्रुते त्वस्मिन् राज्ञो हेरोदस्य तेन च सह कृत्स्नस्य यिरूशालेमस्योद्वेगः मञ्जातः।
\vakya ततः स सर्वान् मुख्ययाजकान् प्रजानां शास्त्राध्यापकांश्चैकत्र कृत्वा पप्रच्छ ख्रीष्टेन कुत्र जन्म ग्रहीतव्यं?
\vakya ते तं जगदुः, यिहूदियादेशस्ये बैतलेहमे, यतो भाववादिनेत्थं लिखितमास्ते यथा,
\begin{poem}
\startwithvakya “यिहूदीयप्रदेशस्य त्वन्तु भो बैतलेहम।
\pline यिहूदानायकश्रेण्यां न क्षोदिष्ठं कथञ्चन॥
\pline त्वत्त एव यतो हेतोः स उदेष्यति नायकः।
\pline मत्प्रजावृन्दमिस्रायेल् येन सम्पालयिष्यते॥”
\end{poem}
\vakya अनन्तरं हेरोदस्तान् ज्योतिर्विदो गुप्तमाहूय तन्नक्षत्रं कति कालं प्रकाशितमिति पृष्ट्वा
\vakya सूक्ष्ममवगत्य तान् बैतलेहमं प्राहिणोदिदमब्रवीञ्च, यूयं गत्वा सूक्ष्मं तस्य शिशोः तथ्यानुसन्धानं कुरुत, प्राप्ते तूद्देशे मां ज्ञापयत, तेनाहमपि गत्वा तं प्रणिपतिष्यामि।
\vakya राज्ञो वाक्यं श्रुत्वा तै र्गमने कृते, पश्य, पूर्वदिशि यन्नक्षत्रं तै र्दृष्टं तत् तेषामग्रतोऽगच्छत् शिशुश्च स यत्राविद्यत तत् स्थानं प्राप्य तदूर्ध्वमवतस्थे।
\vakya नक्षत्रं तद् दृष्ट्वा तेऽतीव महतानन्देन प्रफुल्लिताः।
\vakya अनन्तरं प्रविश्य तै र्गृहं मात्रा मरियमा सह शिशुराविश्चक्रे।
\vakya ततस्ते प्रणिपत्य तस्य भजनं चक्रुः स्वधनकोषानुन्मोच्य तस्मै स्वर्णकुन्दुरुगन्धरसान् दर्शनीयान्युपजह्रुश्च। ततः परं ते हेरोदं प्रति मा प्रत्यावर्तध्वमित्यादेशं स्वप्नेनेश्वराल्लब्ध्वान्य मार्गेण स्वदेशं प्रतस्थिरे।
\vakya प्रस्थितेषु तेषु, पश्य, प्रभो र्दूतः स्वप्ने योषेफाय दर्शनं दत्त्वा जगाद, उत्तिष्ठ शिशुं तन्मातरञ्च गृहीत्वा मिसरं पलायस्व च, यावच्चाहं तुभ्यं नान्यत् कथयामि तावत् तत्रावतिष्ठस्व, यतो हेरोदः शिशो र्हत्यायै तं मृगयितुमुद्यतः।
\vakya अतः स रात्रावुत्थाय शिशुं तन्मातरञ्च गृहीत्वा मिसर प्रतस्थे हेरोदस्य मृत्युं यावत् तत्रावतस्थे च।
\vakya इत्थं भाववादिनोक्ता प्रभोरियं कथा संसिद्धा यथा, “मिसर्देशत एवाहं स्वीयपुत्रं समाह्वयम्।”
\vakya तदा ज्योतिर्विद्भिरहं प्रवञ्चित इति दृष्ट्वा हेरोदो भृशं चुकोप, ज्योतिर्विदां वचनाच्च सूक्ष्मं निर्णीतकालानुरूपं वर्षद्वयवयस्कास्तन्न्यूनवयस्काश्च यावन्तः पुंशिशवो बैतलेहमो तत्कृत्स्नपरिसीमनि चाविद्यन्त जनान् प्रहित्य स तान् सर्वान् घातयामास।
\vakya तदा भाववादिना यिरमियाहेणोक्तमिदं वचनं सिद्धिं गतं यथा,
\begin{poem}
\startwithvakya “रामायां श्रूयते रावः शोकजं परिदेवनं।
\pline रोदनं बहुरूपस्य हाहाकारस्य च ध्वनिः॥
\pline स्वसुतान् अनुशोचन्ती राहेल् करोति रोदनं।
\pline सान्त्वनां सा न गृह्णाति यतो हेतो र्न सन्ति ते॥”
\end{poem}
\vakya अनन्तरं पश्य मृतो हेरोदे प्रभो र्दूतो मिसरे योषेफाय स्वप्ने दर्शनं दत्त्वा जगाद,
\vakya “उत्तिष्ठ शिशुं तन्मातरञ्च गृहीत्वेस्रायेलदेशं गच्छ च, यस्माच्छिशोः प्राणनाशार्थिनो मृतवन्तः।”
\vakya ततः स उत्थाय शिशुं तन्मातरञ्च गृहीत्वेस्रायेलदेशं प्रविवेश।
\vakya किन्त्वार्खिलायः स्वपितुर्हेरोदस्य पदं प्राप्य यिहूदियाया राजास्तीति निशम्य स तत्र गन्तुं शङ्कितः।
\vakya ततः परं स्वप्न ईश्वरीयादेशं लब्ध्वा स गालीलीयजनपदं प्रतस्थे, तत्रागत्य च नासरताभिधे नगरे वसितं चक्रे। इत्थं स नासरीयोऽभिधायिष्यत इति भाववादिभिरुक्ता कथा सिद्धिं गता\eoc