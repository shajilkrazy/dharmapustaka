\adhyAya
\stitle{दयावलेन ख्रीष्टस्य परीक्षणं।}
\vakya तदानीं यीशु र्दयावलेन परीक्षासहनार्थमात्मना मरुं नीतः।
\vakya तत्र स चत्वारिंशद्दिनानि चत्वारिंशद्रात्रीश्चोपोषितः, ततः परं चुक्षोध।
\vakya तदा परीक्षकस्तमुपागत्य जगाद, “भवांश्चेदीश्वरस्य पुत्रस्तर्हीमे प्रस्तरा यथा पुपा भवेयुस्तथाज्ञापयतु।”
\vakya स तु प्रतिजगाद, “लिखितमास्ते, न केवलेन पूपेन मनुष्यो जीविष्यति, किन्त्वीश्वरस्य मुखान्निर्गच्छता येन केनचिद् वचनेन।”
\vakya तदा दियावलस्तं पुण्यनगरं नीत्वा धर्मधाम्नः शिखरे स्थापयित्वा जगाद,
\vakya “भवांश्चेदीश्वरस्य पुत्रस्तर्ह्यधस्तात् प्रपततु यतो लिखितमास्ते,
\begin{poem}
\startwithline त्वत्कृते निजदूतेभ्यः स आदेशं प्रदास्यति।
\pline यन्नाहन्याः पदं शैले तत् त्वां वक्ष्यन्ति ते करैः॥”
\end{poem}
\vakya यीशुस्तं जगाद, “इदमपि लिखितमास्ते, त्वं स्वेश्वरस्य प्रभोः परीक्षां माऽकार्षीः।”
\vakya पुनश्च दियावलस्तमतीवोच्चं गिरिं नीत्वा जगतो निखिलराज्यानि तेषां प्रतापञ्च दर्शयन्नवादीत्,
\vakya त्वं चेत् प्रणिपत्य मम भजनां कुर्यास्तर्ह्येतानि सर्वाणि तुभ्यं दास्यामि।”
\vakya तदा यीशुस्तं जगाद, “अपसर शैतान, यतो लिखितमास्ते, निजेश्वरस्य प्रभो र्भजना त्वया कर्तव्या,एकश्च स एव त्वयाराधयितव्यः।”
\vakya तदा दियावलस्तमत्याक्षीत्, पश्य चापरं स्वर्गदूता उपागत्य तं परिचरितुं प्रावर्तन्त।
\stitle{यीशोः प्रकाश्यकार्यारम्भः।}
\vakya ततः परं योहनः कारायां समर्पित इति निशम्य यीशु र्गालीलं प्रतस्थे,
\vakya नासरतं त्यक्त्वा च सबूलूनस्य नप्ताल्श्च सीमनि सागरतीरस्थे कफरनाहूम उपस्थाय वसतिं चक्रे।
\vakya इत्थं भाववादिना यिशायाहेन कथितमिदं वाक्यं सिद्धिं गतं यथा,
\begin{poem}
\startwithvakya “सबूलूनस्य यो देशो नप्तालेरपि या क्षितिः।
\pline पद्धतिः सागरासन्ना यर्द्दनस्योत्तरे स्थिता॥
\pline सेविते परजातीयैस्तस्मिन् गालीलमण्डले।
\pline अन्धकारे समासीनै र्ज र्दृष्टा महाद्युतिः॥
\vakya मृत्युच्छायावृते देश उपविष्टाश्च ये पुरा।
\pline ज्यातिस्तेषां मनुष्याणामुदितं दृष्टिगोचरे॥”
\end{poem}
\vakya ततः परं यीशु र्घोषणां कर्त्तुमारभ्येदमब्रवीत्, मनांसि परावर्तयत, यतः स्वर्गराज्यं सन्निकटमागतं।
\vakya अपरं यीशु र्गालीलीयसागरश्च तटे विहरन् द्वौ भ्रातरावर्थतः पित्राभिधं शिमोनं तस्य सहोदरमान्द्रियञ्च सागरे जालं क्षिपन्तौ ददर्श, यतस्तौ धीवरावास्तां।
\vakya स तौ जगाद, “मम पश्चाद् आगच्छतम्, अहञ्च युवां मनुष्यधारिणौ धीवरौ विधास्यामि।”
\vakya ततस्तौ तूर्णं जालानि विहाय तमन्वगच्छतां।
\vakya तस्मात् स्थानात् प्रगत्य सोऽपरौ द्वौ भ्रातरावर्थतः सिबदियस्य पुत्रं याकोबं तदीयसहोदरं योहनञ्च स्वजनकेन सिबदियेन सह नौकायां स्वजालानां जीर्णोद्धारे नियुक्तौ दृष्ट्वा तावाजुहाव।
\vakya तौ च तत्क्षणमेव नौकां स्वपितरञ्च विहाय तमन्वगच्छतां।
\vakya आनन्तरं यीशुः कृत्स्नं गालीलं परिभ्राम्यन् तत्रत्यानां समाजगृहेषु शिक्षामददात् राज्यस्य सुसंवादञ्चाघोषयत् जनानां यावतीयरोगं यावतीयदौर्बल्यञ्चोपाशमयत्।
\vakya एवं तस्य ख्यातिः कृत्स्ने सुरियादेशे व्यानशे, अस्वस्था विविधै रोगै र्व्यसनैश्च क्लिष्टा भूताविष्टा अपस्मारिणः पक्षाघातिनश्च सर्व मनुष्यैस्तस्यान्तिकमानीतास्तेन स्वस्थीकृताञ्च।
\vakya अपरं गालील-दिकापलि-यिरूशालेम-यिहूदियाप्रदेशेभ्यो यर्द्दनः पाराञ्च महान्तो मानवनिवहा तमन्वगच्छन्\eoc