\adhyAya
\stitle{स्वर्गराज्ये महान् कः एतदधि शिक्षा।}
\vakya तस्मिन् दण्डे शिष्या यीशोरन्तिकमागत्य तं पप्रच्छुः, स्वर्गराज्ये कस्तावन्महत्तरः?
\vakya यीशुस्तदा क्षुद्रं बालकमेकं स्वसमीपमाहूय तेषां मध्ये स्थापयित्वा तान् जगाद,
\vakya युष्मानहं सत्यं ब्रवीमि, यूयं यदि न परावर्तध्वे न च बालका इव भवथ, स्वर्गराज्यं तर्हि कथञ्जन न प्रवेक्ष्यथ।
\vakya अतः यः कश्चिदात्मानं बालकमिममिव नम्रीकरोति, स एव स्वर्गराज्ये महत्तरः।
\vakya यश्च मम नाम्नेदृशमेकं बालकं गृह्णाति, स मां गृह्णाति।
\vakya यस्तु मयि विश्वसतामेषां क्षुद्राणामेकं स्खालयति, तत्कण्ठे बृहत्‌पेषणीबन्धनं गभीरसमुद्रे च तस्य निमज्जनं श्रेयः।
\vakya स्खलनहेतुभ्यो जगतः सन्तापो भविष्यति। स्खलनहेतवो ह्यवश्यम्भाविनः। येन तु स्खलनहेतुरायाति मनुष्यस्य तस्य सन्तापो भविष्यति।
\vakya यदि तु तव हस्तश्चरणो वा तव स्खलनहेतु र्भवेत् तर्हि तं छित्त्वा दूरं निक्षिप; तव हस्तद्वयसम्पन्नस्य चरणद्वयसम्पन्नस्य वानन्ते वह्नौ निपातात् श्रेयांस्तव खञ्चस्य हीनाङ्गस्य वा जीवने प्रवेशः।
\vakya तव चक्षु र्वा यदि तव स्खलनहेतुकं भवेत् तर्हि तदुत्पाट्य दूरं निक्षिप; तव नेत्रद्वयसम्पन्नस्य वह्निमये नरके निपातात् श्रेयांस्तवैकनेत्रस्य जीवने प्रवेशः।
\vakya सावधानास्तिष्ठत, क्षुद्राणामेषाम् एकमपि एकमपि मावमन्यध्वं।
\vakya यतो युष्माहं ब्रवीमि, स्वर्गे तेषां दूता निरन्तरं मम स्वर्गस्थस्य पितुरास्यं निरीक्षन्ते। वास्तवं हि हारितस्य त्राणार्थं मनुष्यपुत्र आगतवान्।
\vakya युष्माभिः किमनुमीयते? मनुष्यस्य कस्यचित् सत्सु शतमेषेतु यदि तेषामेको भ्रान्तो भवेत् तर्हि स किमन्यान् एकोनशतं मेषांस्त्यक्त्वा गिरिमयाणि स्थानानि च गत्वा हारितं तं मेषां न मृगयते?
\vakya अहञ्च युष्मान् सत्यं ब्रवीमि, यदि स कृतार्थस्तमासादयति, तर्ह्यभ्रान्तेषु तेष्वेकोनशतमेषेषु यस्तस्यानन्दोऽस्ति तस्मादपि महत्तरस्तस्मिन्नेवानन्दो जनिष्यते।
\vakya इत्थमेव क्षुद्राणामेषामेकस्यापि विनाशो मम स्वर्गस्थस्य पितु र्दृष्ट्यामवाञ्छितः।
\vakya त्वां प्रति तु तव भ्रात्रा पापे कृते त्वं याहि तं भर्त्सय च केवलं त्वयि तस्मिंश्च वर्तमानयोः। स चेत् तव वाक्यं गृह्णीयात् तर्हि त्वया स्वभ्रातोपार्जितः।
\vakya यदि तु न गृह्णीयात् तर्ह्यात्मना सार्धमपरमेकमपरौ वा द्वौ गृहाण, तथानुष्ठिते द्वयोस्तिसृणां वा साक्षिणां मुखेन सर्वकथा संस्थास्यते।
\vakya स चेत् तेषां वाक्यं न गृह्णीहात् तर्हि मण्डलीं ज्ञापय। यदि स मण्डलीवाक्यमपि न गृह्णीयात् तर्हि तव सकाशं स परजातीयेन शुल्कादायिना च समानो भवतु।
\vakya युष्मानहं सत्यं वदामि, किमपि यद् यूयं मेदिन्यां भन्‌त्स्यथ स्वर्गे तद् बद्धं स्थास्यति। यदपि च यूयं मेदिन्यां मोक्ष्यथ स्वर्गे तन्मुक्तं स्थास्यति।
\vakya पुनरपि युष्मान् ब्रवीमि, याचयितव्यं कञ्चिद् वरमधि यदि युष्मन्मध्ये नरौ द्वावेकमतौ भवेतां तर्हि मम स्वर्गस्थात् पितुः स वरस्ताभ्यां लप्स्यते।
\vakya वास्तवं हि यत्र कुत्रापि जनौ द्वौ त्रयो वा जना मन्नामोद्दिश्य समागतास्तेषां मध्येऽहं तत्र विद्ये।
\stitle{क्षमाशीलत्वमधि शिक्षा।}
\vakya तदा पित्रस्तस्यान्तिकमागत्य जगाद, प्रभो, कतिकृत्वो मम भ्रात्रा मत्प्रतिकूलं पापे कृते तस्याहं क्षमिष्ये? किं यावत् सप्तकृत्व?
\vakya यीशुस्तं प्रत्यब्रवीत्, सप्तकृत्वो यावदिति त्वां न वदामि, अपि तु सप्ततिकृत्वः सप्तकृत्वो यावदिति।
\vakya अनेन स्वर्गराज्यं स्वदासैः सह जिगणयिषुणा नराधिपमनुष्येणोपमेयं।
\vakya तस्मिन् गणयितुम् आरब्धवत्ययुतमुद्रागोणीनामृण्येको दासस्तत्समीपमानीतः।
\vakya तस्य तु परिशोधोपायाभावात् प्रभुस्तस्य तदीयभार्यासन्तानानां तत्सर्वस्वस्य च विक्रयं तन्मूल्येन चर्णशोधमाज्ञापयामास।
\vakya ततः स दासः प्रणिपत्य पूजयित्वा च जगाद, प्रभो, सहतां मे, तेन तुभ्यं सर्वं प्रत्यर्पयिष्यामि।
\vakya प्रभुस्तदा दासं तमनुकम्प्य विससर्ज चक्षमे च तस्य तदृणं।
\vakya दासः स तु निर्गतः सहदासानां मध्ये यस्तस्य शतं मुद्रापादान् अधारयत् तमासाद्य धृत्वा च तस्य कण्ठदेशं पीडयन् जगाद, प्रत्यर्पय मम यद्यद् धारयसि।
\vakya सहदासः स तदा तस्य चरणयो र्निपत्य सविनयमुवाच, सहतां मे, तेन तुभ्यं सर्वं प्रत्यर्पयिष्यामि।
\vakya स त्वसम्मतोऽभूत्, गत्वा च तमृणशोधं यावत् कारायां निचिक्षेप।
\vakya तमाचारं दृष्ट्वा सहदासस्तस्यातीव शुशुचु र्गत्वा च स्वप्रभवे सर्ववृत्तान्तं निवेदयामासुः।
\vakya तस्य प्रभुस्तदा तं स्वसमीपमाहूय जगाद, रे दुष्ट दास तव सविनयां प्रार्थनां गृहीत्वाहं तव तत् सर्वमृणं क्षमितवान्।
\vakya कृपा यथा मया त्वयि कृता तव सहदासेऽपि कि न तथा कृपा त्वयापि कर्तव्यासीत्?
\vakya अनन्तरं तस्य प्रभुः क्रुद्ध्वा तेन यदधार्यत, तत्सर्वस्य परिशोधं यावत् यन्त्रणादायिषु तं समर्पयामास।
\vakya यूयञ्च प्रत्येकं यदि हृदयेण स्वभ्रातुरपराधान् न क्षमध्वे, तर्हि मम स्वर्गस्थः पितापि युष्मान् प्रति तथैव करिष्यति\eoc