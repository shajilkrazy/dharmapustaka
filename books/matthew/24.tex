\adhyAya
\stitle{मन्दिरविनाशस्य भविषद्वाक्यं।}
\vakya अनन्तरं यीशु र्यदा निष्क्रम्य धर्मधामतोऽपागच्छत्, तस्य शिष्यास्तदा तं धर्मधाम्नो निर्माणप्रकारान् दर्शयितुं समीपम् आगमन्।
\vakya यीशुस्तु तान् जगाद, न दृश्यन्ते किं युष्माभिरेतानि सर्वाणि? युष्मानहं सत्यं ब्रवीमि, अत्र प्रस्तरोपरि प्रस्तर एकोऽप्यनिपातयितव्यो न विहायिष्यते।
\vakya अनन्तरं शिष्यास्तस्य जैतुनगिरावुपविष्टस्य समीपं गुप्तमागत्य जगदुः, वक्तुमर्हत्यस्मान् भवान्, कदा तत् सम्भविष्यति किं वाभि ज्ञानं भवतश्चोपस्थित्या युगान्तस्य चेति।
\stitle{विनाशकालस्य घटनाः।}
\vakya यीशुस्तदा तान् प्रत्यवादीत्, सावधानास्तिष्ठत, युष्मान् कोऽपि मा प्रतारयतु,
\vakya यतो बहवो मन्नामध्वजिन आगत्य वदिष्यन्त्यहं ख्रीष्ट इति प्रतारयिष्यन्ति च बहून्।
\vakya कथाश्च यूयं युद्धानां युद्धस्य किंवदन्तीश्च श्रोष्यथ। सावधानास्तिष्ठत, मैवोद्विजध्वं, यतः सर्वेणैतेन भवितव्यं, परिणामस्तु नापि तदा।
\vakya वस्तुतो जाति जाते र्विरुद्धं राज्यञ्च राज्यस्य विरुद्धमुत्थास्यति, भविष्यन्ति च स्थाने स्थाने दुर्भिक्षमारीभूकम्पाः।
\vakya उपक्रमः पुनर्यातनानां सर्वमेतत्।
\stitle{शिष्याणां दुःखं लोकानां पलायनं।}
\vakya मानवास्तदा क्लेशभोगाय युष्मान् समर्पयिष्यन्ति युष्मान् मारयिष्यन्ति च, द्वेक्ष्यध्वे च यूयं सर्वजातिभि र्मन्नामकृते।
\vakya तदानीं पुन र्बहवः स्खलिष्यन्ति, एकोऽन्यं समर्पयिष्यति विद्वेक्ष्यति च।
\vakya कूटभाववादिनो बहवश्चोत्थास्यन्ति बहून् प्रतारयिष्यन्ति च।
\vakya अनुष्णं भविष्यति च बहुतराणां प्रेमाधर्मप्राचुर्यात्।
\vakya यस्त्वन्तं यावत् स्थिरः स्थास्यति स तरिष्यति।
\vakya सुसंवादश्च राज्यस्यायं घोषयिष्यते कृत्स्ने भूमण्डले साक्ष्यार्थं सर्वजातीनां कृते। तदानीं पुनरुपस्थास्यते परिणामः।
\vakya अतो यूयं यदा भाववादिना दानीयेलेन कथितं ध्वंसकारि घृण्यवस्तु स्थाने पवित्रे संस्थितं द्रक्ष्यथ -पाठको बुध्यतां!-
\vakya तदा ते पलाय्य गिरीन् आश्रयन्तां ये यिहूदियायां विद्यन्ते,
\vakya नावरोहतु च स स्वगेहात् किमप्यादातुं यो गृहपृष्ठे विद्यते,
\vakya स च न स्ववासांस्यादातुं प्रत्यावर्ततां यः क्षेत्रे विद्यते।
\vakya भविष्यन्ति च दिनेषु तेषु गर्भिण्यः स्तन्यदायिन्यश्च सन्तापभाजनानि।
\vakya तथा च प्रार्थयध्वं युष्माकं पलायनं यथा शीतकाले विश्रामवारे वा न भवेत्।
\vakya यतस्तदा महाक्लेशो यादृशः सम्भविष्यति, आजगदारम्भाद् अद्य यावत् तादृशो न सम्भूतो नापि कदा सम्भविष्यति।
\vakya दिनानि तानि च चेन्न न्यून्यकारिष्यन्त मर्त्यः कोऽपि तर्हि नातरिष्यत्। न्यूनीकारिष्यन्ते तु दिनानि तानि वरितजनानां कृते।
\vakya तदा च मा विश्वसित यूयमुक्ता अपि केनचित् पश्य ख्रीष्टोऽत्र वामुत्र विद्यत इति।
\vakya यतः कूटख्रीष्टाः कूटभाववादिनश्चोत्थास्यन्ति, प्रदर्शयिष्यन्ति च ते महाभिज्ञानान्यद्भुतलक्षणानि च तादृशानि यैः साध्ये सति ते वरितानपि मनुष्यान् पथभ्रष्टान् करिष्यन्ति।
\vakya पश्यत युष्मानहं पूर्वमेवोक्तवान्।
\vakya तत् पश्य स मरौ विद्यत इति कथिते जनै र्युष्मभ्यं यूयं मा निर्गच्छत, पश्य वा सोऽन्तःपुरे विद्यत इति कथिते मा विश्वसित।
\vakya यतो मनुष्यपुत्रस्यागमनं तथैव भविष्यति क्षणप्रभा यथा पूर्वदिश उदेत्य पश्चिमदिशं यावत् प्रकाशते।
\vakya यत्र हि कुणपं गृध्राश्च तत्र समागमिष्यन्ति।
\stitle{मनुष्यपुत्रस्य पुनरागमनं।}
\vakya तेषान्तु दिनानां क्लेशाद् अव्यवहितपरं सूर्यः सान्धकारो भविष्यति, चन्द्रश्च स्वज्योत्स्नं न प्रदास्यति, नक्षत्राणि च नभसः पतिष्यन्ति गगनस्य बलानि च विचलिष्यन्ति।
\vakya तदा व्योम्नि मनुष्यपुत्रस्याभिज्ञानं प्रकाशिष्यते। तदा गोष्ठ्यश्च सर्वा देशस्थाः शोकेन स्ववक्षांस्याहनिष्यन्ति, निरीक्षिष्यन्ते च पराक्रमेण महाप्रतापेन च परीतमाकाशीयमेघरथेनागच्छन्तं मनुष्यपुत्रम्।
\vakya स च तूर्यध्वनिना महानिनादेन सहितान् स्वदूतान् प्रहेष्यति, ते च गगनस्यैकप्रान्तमारभ्यापरप्रान्तं यावच्चतुर्भ्यो वायुभ्यो मनुष्यांस्तस्य वरितान् समानेष्यन्ति।
\stitle{उडुम्बरवृक्षस्य दृष्टान्तः।}
\vakya उडुम्बरवृक्षाद् दृष्टान्तं शिक्षितुमर्हथ। स्पष्टं यदा तस्य शाखा जायते कोमला पत्राणि च विकाशन्ते, जानीथ यूयम् आसन्नस्तदा ग्रीष्मकाल इति।
\vakya तथैव यदा यूयं सर्वमेतद् द्रक्ष्यथ, ज्ञास्यथ तदा स समीपस्थो द्वार उपस्थितश्चेति।
\vakya युष्मानहं सत्यं ब्रवीमि, यावत् सर्वमेतन्न सम्भवति, तावन्न व्यत्येष्यन्ति मानवा एतत्कालिकाः।
\vakya द्यावापृथिव्यावत्येष्यतः, मम वाक्यानि तु नैवात्येष्यन्ति।
\vakya दिनस्य तस्य तु दण्डस्य च तस्य तत्त्वं कोऽपि न जानाति, स्वर्गीयदूता अपि न जानन्ति, जानाति तन्मम पिता केवलः।
\vakya वास्तवं हि नोहस्य काले यादृशमासीत् मनुष्यपुत्रस्यागमनेऽपि तादृशं भविष्यति।
\vakya फलतो महावन्यायाः प्राक्तनेषु दिवसेषु पोते नोहस्य प्रवेशदिनं यावद् आसन्निविष्टा यथा मानवा भोजने पाने च विवहने विवाहने च नाजानंश्च
\vakya तावद् वन्ययोपस्थाय यावत् सर्वे न समह्रियन्त, तादृशमेव भविष्यति मनुष्यपुत्रस्यागमने। 
\vakya नरयो र्द्वयोस्तदा क्षेत्रे तिष्ठतोरेकतरो ग्राहिष्यतेऽन्यतरः परित्यक्ष्यते।
\vakya योषितो र्द्वयोः पेषणीनियुक्तयोरेकतरा ग्राहिष्यतेऽन्यतरा परित्यक्ष्यते।
\vakya अतो यूयं जागृत, यतो यूयं न जानीथ दण्डे कस्मिन् युष्माकं प्रभुरायातीति।
\vakya जानीत परन्त्विदं यद् यदि गृहस्वाम्यज्ञास्यद् आयाति चौरः कस्मिन् याम इति, स तर्ह्यजागरिष्यत्, कुड्यभेदञ्च स्वगेहस्य नासहिष्यत।
\vakya तद् यूयमपि ससज्जास्तिष्ठत, यत आयास्यति मनुष्यपुत्रस्तस्मिन्नेव दण्डे दण्डो यो युष्माभिर्नानुभूयते।
\stitle{विश्वास्याविश्वास्यदासयो र्दृष्टान्तः।}
\vakya को नु खलु स विश्वस्तो बुद्धिमांश्च दासो यो यथासमये भृत्येभ्यो भक्ष्यवितरणाय स्वामिना स्वभृत्येष्वधिकृतः?
\vakya धन्यः स दासः स्वामी यमागमनकाले तथैवाचरन्तमासादयिष्यति।
\vakya अहं युष्मान् सत्यं ब्रवीमि स तं कृत्स्ने सर्वस्वेऽधिकरिष्यति।
\vakya मम स्वामी त्वागन्तुं विलम्बत इति मनसि ध्यात्वा दासः स दुष्टो
\vakya यदि सहदासान् ताडयितुं प्रमत्तैश्च सह भोक्तुं पातुञ्च प्रवर्तते
\vakya स तर्हि दिने यस्मिन् नापेक्षते दण्डञ्च यं स न जानाति तदैव दासस्य स्वामी तस्योपस्थाय
\vakya तं द्विधाकरिष्यति भाग्यञ्च तस्य कपटिभिः सार्धं निरूपयिष्यति। तत्र रोदनं दन्तै र्दन्तघर्षणञ्च सम्भविष्यतः\eoc