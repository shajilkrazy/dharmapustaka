\adhyAya
\stitle{फरीश्यध्यापकान् प्रति यीशोराक्षेपः।}
\vakya यीशुस्तदा जननिवहेभ्यः स्वशिष्येभ्यश्च कथयामास,
\vakya शास्त्राध्यापकाः फरीशिनश्च मोशेरासनोपविष्टाः, अतः पालयतानुतिष्ठत च यद्यद् युष्मांस्ते पालनीयं वदन्ति।
\vakya मा पुनः कुरुत तेषां क्रियानुरूपं, यतस्ते वदन्ति न त्वनुतिष्ठन्ति।
\vakya वस्तुतस्ते भारान गुरून् दुर्वहांश्च बद्ध्वा स्कन्धेषु मनुष्याणामर्पयन्ति स्वयं तांस्त्वङ्गुल्यापि सारयितुं नाङ्गीकुर्वते।
\vakya स्वकर्माणि सर्वाणि ते लोकदर्शनायैव कुर्वन्ति, कुर्वन्ति च पृथून् स्वपट्टबन्धान् सुदीर्घाणि च स्ववसनानां प्रलम्बकानि।
\vakya आकाङ्क्षन्ति च ते भोज्येषु श्रेष्ठस्थानानि, समाजगृहेषु श्रेष्ठासनानि,
\vakya हट्टेष्वभिवन्दनानि, मनुष्यैश्च भो रब्बिन् भो रब्बिन्नित्यभिभाषणम्।
\vakya यूयन्तु रब्बिन्निति माभिभाष्यध्वं, यतो युष्माकं गुरुरेक एव ख्रीष्टः, यूयं सर्वे मिथो भ्रातरः।
\vakya किमपि च मेदिन्यां भो पितरिति माभिभाषध्वं, यतः पिता युष्माकमेक एव स स्वर्गवासी।
\vakya मनुष्यैराचार्या इति माभिभाष्यध्वं, यतो युष्माकमाचार्य एक एव ख्रीष्टः।
\vakya युष्माकं मध्ये यश्च महत्तमः स युष्माकं परिचारको भविष्यति।
\vakya यस्त्वात्मानम् उच्चीकरिष्यति स नीचीकारिष्यते, यश्चात्मानं नीचीरिष्यति स उच्चीकारिष्यते।
\vakya अरे कपटिनः शास्त्राध्यापकाः फरीशिनश्च, यूयं सन्तापभाजनानि, यतो यूयं मनुष्येभ्यः स्वर्गराज्यस्य द्वारं रुन्ध, यूयं पुन र्न प्रविशथ, प्रविशतस्तु प्रवेशान्निवारयथ।
\vakya अरे कपटिनः शास्त्राध्यापकाः फरीशिनश्च, यूयं सन्तापभाजनानि, यतो यूयं विधवानां गृहाणि ग्रसथ छलाच्च सुदीर्घं प्रार्थयध्वे। तल्लप्स्यध्वे विचारे गुरुतरं दण्डं।
\vakya अरे कपटिनः शास्त्राध्यापकाः फरीशिनश्च, यूयं सन्तापभाजनानि, यतो यूयं सागरं स्थलञ्च परिभ्राम्यथ कर्तुं मनुष्यमेकं यिहूदिमतावलम्बिनम्, कुरुथ च तं नारकिणं युष्मत्तो द्विगुणं यस्तदवलम्बी भवति।
\vakya अरे अन्धाः पथप्रदर्शका, यूयं सन्तापभाजनानि, यूयं हि वदथ, यो मन्दिरेण शपते नानुष्ठेयं तेन किमपि, यस्तु सुवर्णेन मन्दिरस्थेन शपते स ऋणी।
\vakya अरे मूढा अन्धाश्च, महत् कतरं, सुवर्णम् अथवा सुवर्णस्य पावकं मन्दिरं?
\vakya वदथ च, यो यज्ञवेद्या शपते नानुष्ठेयं तेन किमपि, यस्तूपहारेण तदुपरिस्थेन शपते स ऋणी।
\vakya अरे मूढा अन्धाश्च, महत् कतरं, उपहारोऽथवोपहारस्य पाविका यज्ञवेदी?
\vakya यो हि यज्ञवेद्या शपते, शपते स तया सर्वेण च तेन तदुपरि यद्यदास्ते।
\vakya यश्च मन्दिरेण शपते, शपते हि स तेन तन्निवासिना च।
\vakya यश्च स्वर्गेण शपते, शपते च स ईश्वरस्य सिंहासनेन तदासीनेन च।
\vakya अरे कपटिनः शास्त्राध्यापकाः फरीशिनश्च, यूयं सन्तापभाजनानि, यतो यूयं पोदिनायाः सितच्छत्राया जीरकस्य च दशमांशान् उपहरथ, परित्यक्तवन्तस्त्वेतान् व्यवस्थाया गरीयसोऽंशान् विचारो दया विश्वासश्च। आसन्निमे युष्माभिरनुष्ठातव्या अमी च न त्यक्त्वाः।
\vakya अरे अन्धाः पथप्रदर्शका यूयं पानीय परिस्रावयन्तो मशकमपसारयथ महाङ्गन्तु निगिलथ।
\vakya अरे कपटिनः शास्त्राध्यापकाः फरीशिनश्च, यूयं सन्तापभाजनानि, यतो युष्माभिः पानपात्रस्य स्थालस्य च बहिर्देशः शुचीक्रियते, अन्तस्तु ते परस्वापहारेणाजितेन्द्रियत्वेन च परिपूर्णे स्तः।
\vakya अरे अन्ध फरीशिन् प्रथमं पानपात्रस्य स्थालस्य चान्तर्देशं शुचीकुरु, तथा कृते बहिर्देशोऽपि तयोः शुचि र्भविष्यति।
\vakya अरे कपटिनः शास्त्राध्यापकाः फरीशिनश्च, यूयं सन्तापभाजनानि, यतो यूयं सुधाधवलितैः शवागारैः सदृशा यानि बहिश्चारूणि प्रतिभान्ति, सन्त्यन्तस्तु शवानाम् अस्थिभिः सर्वविधैरपद्रव्यैश्च परिपूर्णानि।
\vakya तानीव यूयमपि बहि र्मनुष्येभ्यो धार्मिकाः प्रतिभाथ, अन्तस्तु कापट्येनाधर्मेण च पूर्णाः स्थ।
\vakya अरे कपटिनः शास्त्राध्यापकाः फरीशिनश्च, यूयं सन्तापभाजनानि, यतो यूयं भाववादिनां शवागाराणि निर्मिमीध्वे धार्मिकाणाञ्च, समाधिस्थानान्युपशोभयथ,
\vakya वदथ च, वयं चेत् स्वपूर्वपुरुषाणां कालेऽवेत्स्यामहि, नाभविष्याम वयं तर्हि तैः सहभागिनो भाववादिनां शोणितपातने।
\vakya एवमेव यूयं यद् भाववादिघातकानां पुत्राः स्थ युष्मानधि तस्य साक्ष्यं यूयमेव दत्थ।
\vakya यूयमपि स्वपूर्वपुरुषाणां परिमाणं पूरयत।
\vakya अरे सर्पाः कालसर्पाणामात्मजाः, विचारे कथं यूयं नरकदण्डान्निस्तरीतुं शक्यथ?
\vakya तत् पश्यत यूष्मदन्तिकमहं भाववादिनो विज्ञवरान् शास्त्राध्यापकांश्च प्रेषयितुमुद्यतोऽस्मि। तेषां मध्यात् केचिद् युष्माभिर्धानिष्यन्ते क्रुशमारोपयिष्यन्ते च, केचिच्च तेषां युष्माभिः स्वसमाजगृहेषु कशाभिराहनिष्यन्ते केचिच्च नगरान्नगरं प्रद्रावयिष्यन्ते।
\vakya इत्थं धार्मिकस्य हेबलस्य शोणितमारभ्य मन्दिनवेद्योरन्तराले युष्मद्घतस्य बेरिखियसुतस्य सखरियस्य शोणितं यावद् धार्मिकशोणितं भूतले यावदेव सिक्तमभूत्, युष्मासु तत्साकल्येन वर्तितव्यं।
\vakya युष्मानहं सत्यं ब्रवीमि, जनेष्वधुनातनेषु तत् सर्वं वर्तिष्यते।
\vakya हा यिरूशालेम, हा यिरूशालेम, हा भाववादिनां हन्त्रि नराणां त्वत्समीपं प्रहितानां प्रस्तराघातिनि पुरि, पक्षयोरधः स्वशावकान् सङ्गृह्णन्ती कुक्कुटीवीहं कतिकृत्वस्तव शावकान् सङ्ग्रहीतुं वाञ्छितवान्, यूयन्तु न सम्मताः।
\vakya पश्यत युष्माकं भवनं युष्मत्कृत उत्सन्नं विहीयते।
\vakya यतोऽहं युष्मान् ब्रवीमि, यावद् यूयं न वक्ष्यथ, धन्यः स यः प्रभोर्नाम्नायातीति, इतः परं तावन्मां न पुन र्द्रक्ष्यथ\eoc