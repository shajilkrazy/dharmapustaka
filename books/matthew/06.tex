\adhyAya
\stitle{दानप्रार्थनादिधर्मकर्मणां कथा।}
\vakya “यूयं स्वधर्मानुष्ठाने सावधाना भवत, जनावलोकनार्थाय तन्मानवानां समक्षं माऽकार्ष्ट। अन्यथा युष्माकं स्वर्गस्थपितृतो लभ्यं किमपि पारितोषिकं युष्माकं न भविष्यति।
\vakya अतस्त्वं यदा भिक्षां ददासि तदा मनुषेभ्यः प्रशंसालिप्सुकामेन समाजगृहेषु रथ्यासु च कपटिनो यथा कुर्वन्ति तथा आत्मनोऽग्रो तूर्यध्वनिं मा कुरु।
\vakya अहं युष्मान् सत्यं वदामि ते स्वपारितोषिकं लब्धवन्तः। त्वयि तु भिक्षां ददति तव दक्षिणहस्तेन यत् क्रियते तत् तव वामहस्तेन मा ज्ञायतां।
\vakya तथानुष्ठिते तव भिक्षादानं निभृतं भविष्यति, तव निभृतदर्शी पिता तु तत्फलं तुभ्यं प्रकाशं दास्यति।
\vakya यदा पुनः प्रार्थनां करोषि, तदा कपटिनां सदृशो मा भव, यतस्ते मनुष्याणां गोचरीभवितुकामाः समाजगृहेषु चत्वरास्रेषु च तिष्ठन्तः प्रार्थयितुमनुरक्ताः। युष्मानहं सत्यं ब्रवीमि, ते स्वपारितोषिकं लब्धवन्तः।
\vakya त्वन्तु यदा प्रार्थयसे तदा स्वान्तरागारं प्रविश्य द्वारञ्च रुद्ध्वा निभृतम् उपस्थितं तव पितरमुद्दिश्य प्रार्थनां कुरु, निभृतदर्शी तव पिता तु तत्फलं तुभ्यं प्रकाशं दास्यति।
\vakya प्रार्थनकाले च यूयं परजातीयजनवद् वृथा पुनरुक्तिं मा कुरुत, ते हि स्ववाक्यबाहुल्यादुत्तरलाभः सम्भविष्यतीति मन्यन्ते, तद् यूयं तेषां सदृशा मा भवत।
\vakya युष्माकं हि यद्यदेव प्रयोजनं तत्तद् युषामाभि र्याचनात् प्राग् युष्मत्पित्रा ज्ञायते।
\vakya अतो यूयमित्थं प्रार्थयध्वं, भो अस्माकं स्वर्गस्थ पितः, तव नाम पवित्रं पूज्यतां।
\vakya तव राज्यमायातु। यथा स्वर्गे तथा मेदिन्यामपि तवेच्छा सिध्यतु।
\vakya श्वस्तनं भक्ष्यमद्यास्मभ्यं देहि।
\vakya वयञ्च यथास्मदपराधिनां क्षमामहे, तथा त्वमस्माकमपराधान् क्षमस्व।
\vakya अस्मांश्च परीक्षां मा नय, अपि तु दुरात्मत उद्धर। यतो राज्यं पराक्रमः प्रतापश्च युगे युगे तवैव। आमेन्।
\vakya वास्तवं हि यूयं चेन्मनुष्याणामपराधान् क्षमध्वे, तर्हि युष्माकं स्वर्गस्थः पिता युष्माकमपि क्षमिष्यते।
\vakya यदि तु मनुष्याणामपराधान् न क्षमध्वे, तर्हि युष्माकं पिता युष्माकमप्यपराधान् न क्षमिष्यते।
\vakya यदा चोपवासं कुरुथ, तदा कपटिन इव विषणवदना मा भवत, यतस्ते मनुष्याणाम् उपवासिनः प्रत्यक्षीभवितुकामाः स्ववदनानि मलिनीकुर्वते। युष्मानहं सत्यं वदामि ते स्वपारितोषिकं लब्धवन्तः।
\vakya त्वन्तूपवालकाले स्वशिरसि तैलं सिञ्च स्ववदनं प्रक्षालय च।
\vakya तथा कृते न मनुष्याणाम् अपि तु निभृतमास्थितस्य तव पितुः त्वमुपोषितः प्रत्यक्षीभविष्यसि। निभृतदर्शी तव पिता च तत्फलं तुभ्यं प्रकाशं दास्यति।
\stitle{स्वर्गे धनसञ्चयकरणकथनम्।}
\vakya यूयमात्मकृतेऽत्र मेदिन्यां धनानि मा सञ्चिनुत, यस्मादत्र कीटकलङ्कौ क्षयं नयतः, चौराश्च कुड्यं छित्त्वा मुष्णन्ति।
\vakya स्वर्गे त्वात्मकृते धनानि सञ्चिनुत, तत्र कीटः कलङ्को वा क्षयं न नयति, चौराश्च नापि कुड्यं छिन्दन्ति न वा मुष्णन्ति।
\vakya यतो युष्माकं यत्र वित्तं तत्र युष्माकं चित्तमपि स्थास्यति।
\vakya नयनं देहस्य दीपकं, अतस्तव नयनं चेत् सरलं तर्हि तव कृत्स्नो देहो दीप्तिमयो भविष्यति।
\vakya तव नयनं तु चेद् दुष्टं तर्हि तव कृत्स्नो देहस्तिमिरमयो भविष्यति। अतस्तवान्तर्ज्योतिश्चेत् तिमिरं भवेत्, तर्हि तत् तिमिरं कियन्महत्।
\vakya द्वयोः स्वामिनो र्दास्यं कर्तुं केनापि न शक्यं, यतः स एकतरं द्विषन्नन्यतरस्मिन् प्रेष्यते, न चेदेकतरस्मिन्नासज्जमानोऽन्यतरमवमंस्यते। ईश्वरस्य धनस्य चोभयो र्दास्यं कर्तुं युष्माभि र्न शक्यते।
\stitle{ईश्वरे विश्वासस्थापनकथा।}
\vakya अतोऽहं युष्मान् ब्रवीमि, किं भक्षिष्यामः किं वा पास्याम इति विचिन्त्य स्वप्राणानधि, किं वसिष्यामह इति विचिन्त्य स्वदेहमधि वा माकुलीभवत। किं हि प्राणा न भक्ष्याच्छ्रेष्ठाः? वसनाद् वा नापि देहः श्रेष्ठः?
\vakya विहायसो विहङ्गमान् निरीक्षध्वं, तै र्नोप्यते नापि कृत्यते न चापि कुशूलेषु सञ्चीयते, युष्माकं स्वर्गस्थः पिता च तान् पुष्णाति। किं यूयं तेभ्यो नाधिकं विशिष्यध्वे?
\vakya चिन्तयित्वा वा युष्माकं केन स्ववयो हस्तमेकं वर्धयितुं शक्यते?
\vakya कस्माद् वा वस्त्राण्यधि चिन्ताकुलीभवथ? शूशनाख्यानि क्षेत्रपुष्पाणि पर्यालोचयत, कथं तानि वर्धन्ते?
\vakya न तानि श्रमं कुर्वते नापि सूत्राणि तन्वन्ति। युष्मान् त्वहं ब्रवीमि, शलोमापि स्वकृत्स्नप्रतापे तेषासेकमिव न पर्यधीयत।
\vakya यदि त्वद्य वर्तमानं श्वश्चुल्ल्यां निक्षेप्तव्यं क्षेत्रस्थं तृणमीदृशमीश्वरः परिधापयति, तर्हि भो स्तोकविश्वासिनः, स किं युष्मान् नाधिकप्रचुरं परिधापयिष्यति?
\vakya अतः किं भक्षिष्यामः किं वा पास्यामः किं वापि वसिष्यामह इति विचिन्त्य माकुलीभवत, यतः परजातीयजना हि सर्वाण्येतान्यनुसन्दधते।
\vakya वास्तवं यदिमानि सर्वाणि युष्माकमावश्यकानि तद् युष्माकं स्वर्गस्थः पिता जानीते।
\vakya प्रथमतस्त्वीश्वरस्य राज्यं धार्मिकताञ्चान्विच्छत, तथा कृते सर्वाणीमान्यपि युष्मभ्यं प्रदायिष्यन्ते।
\vakya अतः श्वःकृते चिन्ताकुला मा भवत, यतः श्वस्तनं दिनमात्मनो विषयं चिन्तयिष्यति। दिनस्य निजकष्टं तस्य कृते पर्याप्तं”\eoc