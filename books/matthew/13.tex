\adhyAya
\stitle{स्वर्गराज्यमधि सप्तदृष्टान्ताः।}
\vakya तस्मिन्नेव दिने यीशु र्गेहान्निर्गत्य समुद्रतीर उपविवेश।
\vakya समीपागतेषु तु महत्सु जननिवहेषु स नौकामारूह्य तत्रोपविवेश, कृत्स्नो जननिवहस्तु समतले तीरेऽतिष्ठत्।
\vakya स तदा दृष्टान्तकथाभिर्बहुश आलपंस्तानब्रवीत्,
\stitle{बीजवापकस्य दृष्टान्तः।}
\vakya पश्य वप्ता बीजानि वप्तुं निर्जगाम; वपनकाले तु पथपार्श्वे कतिपयबीजानि पतितानि विहङ्गमैश्चागत्य तानि खादितानि।
\vakya पतितानि चापराणि कतिपयानि स्थनेषु पाषाणमयेषु, तत्र नाविद्यत मृत्तिका प्रचुरं तेषां कृते, मृत्तिकाया गभीरत्वस्याभावात् तान्याशु प्ररूढाणि,
\vakya सूर्ये तूच्चसंस्थे दग्धानि मूलाभावाच्च शोषितानि।
\vakya पतितान्यपराणि तु कण्टकेषु सम्पीडितानि च कण्टकै र्वृद्ध्वा।
\vakya अपराणि तूत्तममृत्तिकायां पतित्वा फलं फलितवन्ति, कानिचिच्छतगुणं, कानिचित् षष्टिगुणं, कानिचिद् वा त्रिंशद्गुणं।
\vakya शृणोतु यस्य श्रोतुं श्रोत्रे स्तः।
\vakya अनन्तरं शिष्याः समीपमागत्य तं पप्रच्छुः, कुतो भवान् दृष्टान्तकथाभिरमूनालपति?
\vakya स तान् प्रतिजगाद, युष्मभ्यं स्वर्गराज्यस्य निगूढाणां विषयानां ज्ञानमदायि, अमुभ्यस्तु नादायि।
\vakya यतो यस्यास्ति, तस्मै दायिष्यते, तेन तस्योपचायिष्यते। यस्य तु नास्ति तस्य यदस्ति तदपि तस्माद्धारिष्यते।
\vakya अतोऽहं दृष्टान्तकथाभिरमूनालपामि, यतस्ते पश्यन्तोऽपि न पश्यन्ति शृण्वन्तोऽपि च न शृण्वन्ति न वा बुध्यन्ते।
\vakya तेषु च यिशायाहस्येयं भाववाणी सिध्यति, यथा,
\begin{poem}
\startwithline “श्रोत्रैः संश्रुत्य युष्माभि र्बोधो नैवोपलभ्यंते।
\pline सम्यग् दृष्ट्वा च युष्माभि र्नैव दर्शनमाप्स्यते॥
\vakya जातेरस्या यतो हेतो र्हृदयं स्थूलतां गतं।
\pline मन्दमेव स्वकर्णैश्च वाचं शृण्वन्ति तज्जनाः॥
\pline विनिमीलितवन्तश्च स्वनेत्राणि नरा अमि।
\pline सर्वे ते लोचनै र्द्रष्टुं कर्णैः श्रोतुमनिच्छवः॥
\pline यस्मात् ते हृदयै र्बुद्ध्वा परावर्त्य मनांसि च।
\pline मत्तो रोगप्रतीकारं प्रग्रहीतुमसम्मताः॥”
\end{poem}
\vakya युष्माकन्तु नेत्राणि धन्यानि यतस्तानि पश्यन्ति, कर्णाश्च धन्या यतस्ते शृण्वन्ति।
\vakya वस्तुतोऽहं युष्मान् सत्यं ब्रवीमि, बहवो भाववादिनो धोर्मिकाश्च द्रष्टुं वाञ्छितवन्तो युष्माभि र्यद्यद्दृश्यते, श्रोतुञ्च युष्माभि र्यद्यच्छ्रूयते, न तु तानि दृष्टवन्तो न वा श्रुतवन्तः।
\vakya तद् युष्माभिः श्रूयतां तस्य वप्तु र्दृष्टान्तकथा।
\vakya यदा कश्चिद् राज्यस्य वाक्यं शृण्वन् न बुध्यते दुष्टात्मा तदागत्य तस्य हृदये यदुप्तं तदपहरति। मनुष्यः स एव लब्धबीजः पथपार्श्वे।
\vakya स्थलेषु पाषाणमयेषु लब्धबीजः स एव यो वाक्यं शृणोति तूर्णञ्च सानन्दं गृह्णाति,
\vakya स त्वन्तरे निर्मूलोऽतः क्षणस्थिरः, तत् तत्क्षणं स्खलति पुन र्जाते क्लेशे वाक्यकृत उपद्रवे वा।
\vakya कण्टकेषु लब्धबीजः स एव यो वाक्यं शृणोति परन्त्वेतस्य युगस्य भावनया धनस्य मायया च वाक्यं तत् सम्पीड्यते निष्फलञ्च जायते।
\vakya उत्तममृत्तिकायां लब्धबीजस्तु स एव यो वाक्यं शृणोति बुध्यते च। स च फलवान् भवति, कोऽपि शतगुणं, कोऽपि षष्टिगुणं, कोऽपि वा त्रिंशद्गुणं फलमुत्पादयति।
\stitle{श्यामाकद्दृष्टान्तः।}
\vakya अनन्तरं स तेभ्योऽन्यां दृष्टान्तकथां कथयामास, स्वर्गराज्यं तेनैव मनुष्येणोपमेयं यः स्वक्षेत्रे सुबीजमुवाप।
\vakya निद्राणेषु तु जनेषु तस्य शत्रुरागत्य गोधूमानां मध्ये श्यामाकबीजान्युप्त्वा प्रस्थितः।
\vakya ततस्तृणानि यदा प्ररूह्य फलवन्त्यभवन् तदा श्यामाका अपि प्रत्यक्षा अभवन्।
\vakya दासास्ततो गृहस्वामिनः समीपमागत्य जगदुः, प्रभो, अपि क्षेत्रे भवान् किं सुबीजानि नोप्तवान्? तदस्य श्यामाकाः, कथमेतत्?
\vakya स तान् जगाद, केनचिच्छत्रुणा मनुष्येणेदं कृतं। दासास्तं पुन र्जगदुः, अभिमन्यते भवता यद् वयं गत्वा तान् सङ्ग्रहीष्यामः?
\vakya स त्वब्रवीत्, मैवं यतः श्यामाकान् सङ्ग्रह्णन्तस्तैः सह गोधूमानुन्मूलयिष्यथेति बिभोमि।
\vakya शस्यच्छेदनं यावदुभयोरेकत्र वृद्धिं सहध्वं। शस्यच्छेदनकालेऽहं तच्छेत्तॄनिदमाज्ञापयिष्यामि, श्यामाकान् प्रथमं सङ्गृह्य दाहनार्थं कूर्च्चै र्बध्नीत, गोधूमांस्तु मम कुशुले निधत्तेति।
\stitle{सर्षपवीजकिण्वयोर्द्दृष्टान्तौ।}
\vakya स तेभ्योऽपरां दृष्टान्तकथां कथयामास, स्वर्गराज्यं शर्षपबीजेन तेन सदृशं यदादाय कश्चिन्मनुष्यः स्वक्षेत्र उवाप।
\vakya क्षुद्रतमं तत् सर्ववीजेषु, वृद्ध्वा पुनः शाकेभ्यो बृहद् भवति वृक्षो जायते च, ततो विहायसो विहङ्गमा आगत्य तस्य शाखासु निवसन्तीति।
\vakya स तानपरां दृष्टान्तकथां जगाद, स्वर्गराज्यं किण्वेन तेन सदृशं यत् कयाचिद् योषितादाय गोधूमचूर्णानां द्रोणत्रयपरिमितानां मध्ये निह्नुतं येन च परिणामे सञ्जातं तत्सकल्यं किण्वभावितं।
\vakya सर्वमेतद् यीशु र्दृष्टान्तकथाभि र्जननिवहान् जगाद, दृष्टान्तं बिना च स तेभ्यः किञ्चिन्नाकथयत्।
\vakya इत्थं भाववादिन इयमुक्तिः सिद्धिं गता, यथा,
\begin{poem}
\startwithline “अहं दृष्टान्तवाक्यै र्हि व्यादास्यामि निजं मुखं।
\pline आजगत्सृष्टितो गूढा व्याहरिष्याम्यहं कथाः॥”
\end{poem}
\stitle{श्यामाकद्दृष्टान्ततात्पर्यम्।}
\vakya तदानीं यीशु र्जननिवहान् विसृज्य गृहमाजगाम, तस्य शिष्याश्च तत्समीपमागत्य जगदुः, भवानस्मान् क्षेत्रस्थानां श्यामाकानां दृष्टान्तकथां बोधयितुमर्हति।
\vakya स ततस्तान् प्रत्यब्रवीत्, सुबीजानां वप्ता मनुष्यपुत्रः, क्षेत्रञ्च जगत्, सुबीजानि च राज्यस्य सन्तानाः।
\vakya श्यामाकास्तु दुष्टात्मनः सन्तानाः, तद्वप्ता च शत्रु र्दियाबलः।
\vakya शस्यच्छेदनकालश्च युगान्तः, छेत्तारश्च स्वर्गदूताः।
\vakya अतो यथा श्यामाकाः सङ्गृह्यन्ते वह्नौ च निर्दह्यन्ते, तथैव युगान्ते सम्भविष्यति,
\vakya मनुष्यपुत्रः स्वदूतान् प्रहेष्यति, ते च तस्य राज्यस्य मध्यात् सर्वाणि विघ्नान्यधर्माचारिणश्च सङ्गृह्य वह्निमयचुल्ल्यां निक्षेप्स्यन्ति,
\vakya तत्र रोदनं दन्तैर्दन्तघर्षणाञ्च सम्भविष्यतः।
\vakya धार्मिकास्तदा स्वपितूराज्ये सूर्यवद् विराजिष्यन्ते। शृणोतु यस्य श्रोतुं श्रोत्रे स्तः।
\stitle{गुप्तवित्तद्दृष्टान्तः उत्तममौक्तिकद्दृष्टान्तश्च।}
\vakya पुनः स्वर्गराज्यं क्षेत्रे निह्नुतेन तेन धनेन सदृशं यदासाद्य कश्चिन्मनुष्यो निह्नुत्य निजानन्दवशाद् गत्वा सर्वस्वं विक्रिय क्षेत्रं तत् क्रीणीते।
\vakya पुनः स्वर्गराज्यं मुक्ताः समीचीनाः अनुसन्दधानेन बणिजा तेन सदृशं,
\vakya येन महामूल्यैका मुक्तासादिता, प्रस्थाय च सर्वस्वं विक्रीय सा क्रीता।
\stitle{आनायनिक्षेपद्दृष्टान्तः।}
\vakya पुनः स्वर्गराज्यं समुद्रे निक्षिप्तेन सर्वविधान् तोयचरान् सङ्गृह्णतानायेन सदृशं।
\vakya पुर्णे च सति तस्मिन्नानायिभिस्तं तीरमानीय तत्रोपविश्य सुमीनाः पात्रेषु सङ्गृहीताः कुत्सिताश्च बहि र्निक्षिप्ताः।
\vakya युगान्ते तादृशं सम्भविष्यति, स्वर्गीयदूता आगत्य धार्मिकाणां मध्यतो दुष्टान् पृथक्कृत्य वह्निमयचुल्ल्यां निक्षेप्स्यन्ति,
\vakya तत्र रोदनं दन्तैर्दन्तघर्षणाञ्च सम्भाविष्यतः।
\vakya यीशुस्तान् पप्रच्छ, बुद्धं किं युष्माभिरेतत् सर्वं? ते तं जगदुः, बुद्धं प्रभो।
\vakya तदा स तानब्रवीत्, अतो यः कश्चिच्छास्त्राध्यापकः स्वर्गराज्यस्य कृते शिक्षितः स स्वभाण्डागाराद् द्रव्याणि पुराणानि नवीनानि चाहरता गृहस्वामिना सदृशः।
\stitle{यीशोः स्वनगरे असम्मानम्।}
\vakya समाप्यैता दृष्टान्तकथा यीशुः स्थानात् तस्मात् प्रतस्थे।
\vakya ततः स स्वदेशमागत्य तत्रत्यजनानां समाजगृहे तेभ्यः शिक्षां दातुं प्रावर्तत, तेन च त आश्चर्यं मत्वावदन्, कुतोऽसावीदृशज्ञानम् ईदृशप्रभाववैभवञ्च लब्धवान्?
\vakya नासावस्मदीयतक्ष्णः पुत्रः? नामुष्य मातु र्नामि मरियमिति? भ्रातॄणाञ्च नामानि याकोबो योषिः शिमोनो यिहूदाश्चेति?
\vakya नामुष्य भगिन्योऽपि सर्वा विद्यन्तेऽत्रास्मन्मध्ये? तत् कुतोऽमुनालम्भि सर्वमेतत्?
\vakya इत्थं ते तस्मिन् स्खलिताः। यीशुस्तु तान् जगाद, नान्यत्र कुत्रापि भाववादिनोऽसम्माननं, केवलं तस्य स्वदेशे स्वकुले च।
\vakya तेषामविश्वासवशाच्च तेन तत्र प्रभावसिद्धानि कर्माणि बहूनि न चक्रिरे\eoc