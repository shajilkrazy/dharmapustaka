\adhyAya
\stitle{ख्रीष्टस्य विरुद्धं कुमन्त्रणं।}
\vakya समाप्य वाक्यानि सर्वाण्येतानि यीशुः स्वशिष्यान् जगाद,
\vakya जानीथ यूयं यदुपस्थास्यते निस्तारपर्व दिनद्वयात् परं, मनुष्यपुत्रस्तदा क्रुशारोपणाय समर्पयिष्यते।
\vakya काले तस्मिन् मुख्ययाजकाः शास्त्राध्यापकाः प्रजानां प्राचीनाश्च महायाजकस्य कायाफाभिधस्य हर्म्ये समाजग्मुः
\vakya सम्मन्त्रयामासिरे च यथा यीशुं छलेन धृत्वा हन्युः।
\vakya ते त्ववदंश्च मैव पर्वाणि, चेत् तथा भविष्यति कलहो जनानां मध्ये।
\stitle{यीशोरभिषेकः।}
\vakya बैथनियायान्तु यीशौ शिमोनस्य कुष्ठिनो गेहे विद्यमाने
\vakya योषिदेका श्वेतोपलभाजनेन महार्हं सुगन्धितैलमानीय तस्यान्तिकं गत्वा स्रावयामास तच्छिरसि तस्य भोजनायासीनस्य।
\vakya एवं दृष्ट्वा शिष्यास्तस्यासन्तुष्टा भूत्वा अगदन्, कथमयमपब्ययः?
\vakya तैलेऽस्मिन् विक्रीते मूल्यं महल्लब्धुं दरिद्रेभ्यो दातुञ्चाशक्ष्यत।
\vakya यीशुस्तु तज्ञात्वा तान् बभाषे, कुरुथ कथं दुःखितां योषामिमां? कृतवतीयन्तु मयि कर्म सत्।
\vakya यतो दरिद्राः सततं युष्मत्सङ्गिनोऽहन्तु न सततं युष्मत्सङ्गी।
\vakya देहे मम तैलामिदं निषिच्येयं तावत् कृतवती कर्म मम समाध्युपयोगि।
\vakya युष्मानहं सत्यं ब्रवीमि, कृत्स्नस्य जगतो यत्र कुत्रचिद् घोषयिष्यते सुसंवादोऽयं, तत्र सर्वत्र कथयिष्यते कर्माप्येतत् कृतमनया स्मरणार्थमस्याः।
\stitle{यिहूदाविश्वासघातित्वं।}
\vakya तदा द्वादशशिष्याणाम् एक ईष्करियोतीयो यिहूदा इत्यभिधो मुख्ययाजकानां समीपं गत्वा जगाद,
\vakya अहं यत् तं युष्मासु समर्पयेयं तद् यूयं मह्यं किं दातुं सम्मताः? ततस्ते त्रिंशद् रौप्यमुद्रास्तोलयामासुस्तस्य कृते।
\vakya तदाप्रभृत्यगवेषयत् स तत्समर्पणस्य समयम्।
\stitle{निस्तारपर्वोयभोज्यं।}
\vakya अनन्तरं निष्किण्वपूपानां प्रथमे दिने यीशोः समीपम् उपस्थाय शिष्यास्तं जगदुः, भवत इच्छातः कुत्रास्माभिरुपकल्पयितव्यं भोजनं निस्तारपार्वणं?
\vakya स बभाषे, यूयं नगरममुकस्य समीपं गत्वा तं ब्रूत, गुरु र्वक्ति, आसन्नो मम कालस्तव गृहेऽहं स्वशिष्यैः सह निस्तारपर्व सम्पालयिष्यामि।
\vakya अतो यीशुना यथादिष्टं शिष्यास्तथैव कुर्वन्तो भोजनं निस्तारपार्वणमुपकल्पयामासुः।
\vakya सन्ध्यायान्तु जातायां यीशु र्द्वादशशिष्यैः सह भोजनार्थम् उपविवेश।
\vakya तेषु च भक्षमाणेषु स बभाषे, अहं युष्मान् सत्यं ब्रवीमि, युष्माकम् एको मां समर्पयिष्यति।
\vakya अनेन तेऽतीव शोचन्तः प्रत्येकं तं प्रष्टुम् आरेभिरे, प्रभो स किमहं?
\vakya स तु प्रतिबभाषे मया सार्धं यः सूपपात्रे पाणिममज्जयत् स एव मां समर्पयिष्यति। समर्प्यते स सन्तापपात्रं । तस्य नरस्य जन्म चेन्नाभविष्यत् तर्हि तेनैव तस्य क्षेमम् अभविष्यत्।
\vakya तस्य समर्पयिता यिहूदास्तदा प्रतिजगाद, भो रब्बिन् स किमहं? स बभाषे, त्वं व्याहार्षीः।
\stitle{प्रभो र्भोज्यनिरूपणं।}
\vakya अतस्तेषां भोजनकाले यीशुः पूपमादायाशीर्वादं कृत्वा तं भङ्क्त्वा शिष्येभ्यो ददौ बभाषे च, गृह्णीत भुंग्ध्वम्, एतन्मम शरीरं।
\vakya ततः स पानपात्रमादाय धन्यवादं कृत्वा तेभ्यस्तद् ददौ बभाषे च, पिबतानेन सर्वे,
\vakya यत एतन्मम शोणितं पापमोचनाय बहूनां कृते विस्राव्यमाणं नूतननियमस्य शोणितं।
\vakya युष्मांस्त्वहं ब्रवीमि, गोस्तनीलतोत्पन्नोऽयं रसो दिने यस्मिन् मम पितूराज्ये नवीनो मया युष्माभिः सह पायिष्यते अद्यारभ्य तद्दिनं यावत् स मया नैव पायिष्यते।
\vakya ततस्ते स्तोत्रं गीत्वा जैतूनपर्वतं जग्मुः।
\stitle{पित्रस्य ख्रीष्टानङ्गीकरणमधि भविष्यद्वाक्यं।}
\vakya यीशुस्तांस्तदा जगाद, रात्र्यामस्यां सर्वे यूयं मयि स्खलिष्यथ, यतो लिखितम् आस्ते, “अहं व्रजरक्षकमाहनिष्यामि, व्रजस्य मेषाञ्च विकीर्णा भविष्यन्ति।”
\vakya परन्तु ममोत्थानात् परम् अहं युष्मदग्रे गालीलं यास्यामि।
\vakya पित्रस्तं प्रतिजगाद, सर्वेऽपि चेद् भवति स्खलेयुस्तथाप्यहं कदाचन न स्खलिष्यामि।
\vakya यीशुस्तं जगाद, त्वामहं सत्यं ब्रवीमि, रात्र्यामस्यां कुक्कुटश्च रवात् प्राक् त्वं मां त्रिकृत्वः प्रत्याख्यास्यसि।
\vakya पित्रस्तं ब्रवीति, यद्यपि भवता सार्धं मम मरणम् अवश्यं स्यात्, तथाप्यहं भवन्तं नैव प्रत्याख्यास्यामि। एवमेव शिष्याः सर्वेऽप्यभाषन्त।
\stitle{गेत्शिमान्युद्याने यीशोर्प्रार्थनः।}
\vakya यीशुस्तदा तैः सह गेत्शिमानीत्यभिधं स्थानमुपस्थाय शिष्यान् जगाद यूयमत्रोपविशत, यावदहममुत्र गत्वा प्रार्थनां कुर्वे।
\vakya अनन्तरं स पित्रं सिबदियस्य पुत्रौ च सङ्गिनः कृत्वा शोचितुं विषत्तुञ्जारेभे तांश्च जगाद,
\vakya प्राणा मम मृत्यवे शोकापन्नाः, यूयमत्रावतिष्ठमाना मया सह जागृत।
\vakya ततः स किञ्चिद् अग्रं गत्वाधोमुखो निपत्य प्रार्थयमानोऽब्रवीत्, भो मम पितः, यदि शक्यं स्यात् तर्हीदं पानीयपात्रं मत्तोऽपसरतु, तथापि न मया, अपि तु त्वया यथा वाच्छ्यते तथैव भवतु।
\vakya अनन्तरं स तेषां शिष्याणां समीपमागत्य तान् निद्राणान् दृष्ट्वा पित्रं जगाद, एवमेव यूयं घटिकाम् एकां मया सार्धं जागर्तुं नाशक्नुत?
\vakya जागृत प्रार्थयध्वञ्च नो चेत् परीक्षां निवेक्ष्यध्ये, यत आत्मोद्यतः शरीरन्तु दुर्बलम्।
\vakya द्वितीयवारं पुन र्गत्वा स प्रार्थयाञ्चक्रे भो मम पितः, पानीयपात्रमिदं यावन्मया न पीयते तावद् यदि मत्तोऽपसरितुं न शक्नुयात्, तवेच्छा तर्हि सिध्यतु।
\vakya सोऽनन्तरमागत्य पुनस्तान् निद्राणान् ददर्श, यतस्तेषां नेत्राण्यवशान्यासन्।
\vakya ततः स तांस्त्यक्त्वा पुनरपगत्य तृतीयवारं वाक्यं तदेव व्याहरन् प्रार्थयाञ्चक्रे।
\vakya ततः परं स्वशिष्याणां समीपमागत्य तान् जगाद, एवमेव यूयं निद्राणा विश्रामं सेवध्वे किम्? पश्यतोपस्थितः स दण्डो मनुष्यपुत्रश्च पापिनां करेषु समर्प्यते।
\vakya उत्तिष्ठत वयं गच्छामः। पश्यत मम समर्पयिता समीपमागतः।
\stitle{यीशोः शत्रुहस्ते समर्पणम्।}
\vakya पश्य स यावदित्थं भाषत आगमत् तावद् द्वादशानाम् एकोऽर्थतो यिहूदास्तेन सार्धञ्च मुख्ययाजकानां जनप्राचीनानाञ्च सकाशाद् असियष्टिधारी महान् जननिवहः।
\vakya तस्य समर्पयिता तेभ्यः सङ्केतं दत्तवान्, यथा, अहं यं चुम्बिष्यामि स एव सोऽस्ति यूयं तमेव धरतेति।
\vakya अतः स तत्क्षणं यशोरन्तिकमुपस्थाय, रब्बिन् प्रणाम इत्युक्त्वा तं चुचुम्ब।
\vakya यीशुस्तु तमब्रवीत्, मित्र किमर्थमुपतिष्ठसे? जनास्ते तदोपागत्य यीशौ हस्तार्पणं कृत्वा तं दध्रुः।
\vakya पश्य नरस्तदा यीशोः सङ्गिनामेको हस्तं प्रसार्य कोषान्निजासिं निष्कृष्य महायाजकस्य दासमाहत्य तस्य कर्णमेकं परिचिच्छेद।
\vakya यीशुस्तदा तमब्रवीत्, तवासिं पुनः स्वस्थाने निधेहि, यतोऽसिग्राहिणः सर्वेऽसिना विनङ्क्ष्यन्ति।
\vakya इदानीमेवाहं स्वपितरं याचितुं शक्तोमि, स च मह्यं द्वादशभ्यो वाहिनीभ्योऽप्यधिकान् स्वर्गदूतान् वितरिष्यतीदं त्वया किम् असम्भवं मन्यते?
\vakya परन्त्वेतेनैव भवितव्यमिति शास्त्रीयोक्तयः कथं सेत्स्यन्ति?
\vakya दण्डे तस्मिन् यीशु र्जननिवहान् जगाद, दस्युमिव मां धर्तुमसीन् यष्टींश्चादाय किं यूयं वहिरागताः? अहन्तु प्रत्यहं धर्मधाम्नि युष्मत्सकाशमासीनोऽशिक्षयं यूयं तदा न मां धृतवन्तः।
\vakya सम्भूतन्तु सर्वमेतद् यथा भाववादिनां शास्त्रीयोक्तयः सिद्धिं गच्छेयुः। शिष्यास्तदा सर्वे तं त्यक्त्वा पलायाञ्चक्रिरे।
\stitle{महायाजकसमक्षं यीशोर्विचारः।}
\vakya ये तु यीशुं धृतवन्तस्तैः स महायाजकस्य कायाफाः समापं निन्ये। तत्र शास्त्राध्यापकाः प्राचीनाश्च समागमन्।
\vakya पित्रस्तु दूरान्महायाजकस्य हर्म्यं यावत् तमन्वव्रजत, परिणामं दिदृक्षुश्च तत्र प्रविश्य पदातिभिः सहोपविवेश।
\vakya तदा मुख्ययाजकाः प्राचीनाः कृत्स्ना सभा च यीशुं हन्तुकामास्तस्य प्रतिकूलं मृषासाक्ष्यम् अमृगयन्त न तु लेभिरे,
\vakya बहुषु मृषासाक्षिषूपागतेष्वपि न लेभिरे। पश्चात्तु द्वौ मृषासाक्षिणावुपस्थायोचतुः,
\vakya असौ व्याहृतवान् अहम् ईश्वरस्य मन्दिरं भङ्क्त्वा दिनत्रये निर्मातुं शक्नोमीति।
\vakya महायाजकोऽनेनोत्थाय तमब्रवीत्, किमपि त्वं किं न प्रतिभाषसे? इमौ तव प्रतिकूलं किं साक्ष्यं दत्तः?
\vakya यीशुस्तु मौनम् अवलम्ब्यातिष्ठत्। अनन्तरं महायाजकस्तं प्रतिवभाषे, अहं त्वां जीवनमयेनेश्वरेण शपयामि, त्वमीश्वरस्य पुत्रः खीष्टो न वेत्यस्मान् वद।
\vakya यीशुस्तं जगाद, त्वं व्याहार्षीः। अधिकन्तु युष्मानहं ब्रवीमि, इतः प्रभृति यूयं मनुष्यपुत्रं प्रभावस्य दक्षिण आसीनम् आकाशीयमेघरथेनागच्छन्तञ्च वीक्षिष्यध्वे।
\vakya महायाजकस्तदा स्ववासांसि विदीर्य बभाषे, स ईश्वरनिन्दां चकार, साक्षिणामधिकानां किं प्रयोजनम्? पश्यताधुनाश्रावि युष्माभिरीश्वरनिन्दा।
\vakya युष्माभिः किं मन्यते? ते प्रत्यवदन्, स प्राणदण्डमर्हति।
\vakya ते तदा तस्यास्ये न्यष्ठीवन् मुष्टिभिश्च तमाघ्नन्।
\vakya केचित्तु तं प्राहरन्नवदंश्च, ख्रीष्ट भावोक्त्यास्मान् वद कस्त्वां ताडितवान्?
\stitle{पितरेण यीशुः वारत्रयमस्वीकृतः।}
\vakya पित्रस्तु बहिः प्राङ्गण आसीनोऽतिष्ठत्। दासी तदैका तस्यान्तिकमागत्याब्रवीत् त्वमपि गालीलीयस्य यीशोः संग्यासीः।
\vakya स तु सर्वेषां समक्षम् अनःङ्गीकुर्वन् जगाद, न बुध्यते मया त्वया किं गद्यते।
\vakya तस्मिंश्च पुन र्गोपुरं गते काचिदन्या तं दृष्ट्वा जनांस्तत्र विद्यमानानवदत्, अयमपि नासरतीयस्य यीशोः संग्यासीत्।
\vakya स ततः शप्त्वा पुनरनङ्गीकुर्वन् जगाद मनुष्यं तं न जानामीति।
\vakya क्षणात् परं नरास्तत्र स्थिता उपागत्य पित्रमब्रुवन्, नूनं त्वमपि तेषाम् एकतमोऽसि यतस्तव भाषा त्वां व्यञ्जयति।
\vakya स तदाभिशप्तुं दिव्यञ्च कर्तुमारभ्याब्रवीत्, मनुष्यं तं न जानामीति। सद्यस्तु कुक्कुटो रुराव।
\vakya अनन्तरं कुक्कुटरावात् प्राक् त्वं मां त्रिकृत्वः प्रत्याख्यास्यसीति वाक्यं यत् स यीशुनोक्तस्तत् स्मृत्वा पित्रो बहि र्गत्वा तीव्रं रुरोद\eoc