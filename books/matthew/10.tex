\adhyAya
\stitle{द्वादशशिष्याणां प्रेषणं।}
\vakya अनन्तरं स शिष्यान् द्वादश निजान् स्वसमीपमाह्वय ददौ तेभ्यः सामर्थ्यं निःसारणायात्मनाम् अशुचिनां प्रतीकाराय च सर्वरोगस्य सर्वामयस्य च।
\vakya तेषां द्वादशशिष्याणां नामानीमानि, पित्राभिधः शिमोनः प्रथमस्तस्य भ्रातान्द्रियश्च, सिबदियस्य पुत्रो याकोबस्तस्य भ्राता योहनश्च,
\vakya फिलिपो बर्थलमयश्च, थोमा शुल्कादायी च मथिः, आल्फेयस्य पुत्रो याकोबो थद्देयाभिधश्च लिब्बेयः,
\vakya कानानी शिमोन ईष्कारियोतीयश्च यिहूदाः, असौ शत्रुषु तस्य समर्पयिता।
\vakya इमान् द्वादश यीशुः प्रेषयामास, प्रेषणकाले तु स तान् जगाद, परजातीयानां पन्थानं मा गच्छत, शमरीयाणां वा नगरं मा प्रविशत, वरमिस्रायेलकुलस्य हारितान् मेषाननुगच्छत।
\vakya गच्छन्तश्चेदं घोषयत, स्वर्गराज्यं सन्निकटं जातमिति।
\vakya अस्वस्थान् निरामयान् कुरुत, कुष्ठिनः शुचीकुरुत, मृतानुत्थापयत, भूतान् निःसारयत।
\vakya लब्धवन्तो यूयं मूल्यं विनैव, प्रदत्त मूल्यं विना।
\vakya मोपार्जयतु कोऽपि युष्माकं स्वकटिबन्धे स्वर्णं वा रूप्यं वा ताम्रं
\vakya वा गमनस्य कृते चेलसम्पुटकं वाङ्गरक्षके द्वे वस्त्रे वा पादुके वा यष्टिं, यतः कार्यकारी स्ववृत्तिमर्हति।
\vakya प्रविशथ तु यदा किञ्चन नगरं कञ्चन ग्रामं वा, अनुसन्दद्ध्वं तदा नरस्तत्र को योग्यः, यावच्च न प्रस्थास्यध्वे तावत् तत्रैवावतिष्ठध्वं।
\vakya गेहं प्रविशन्तश्च तस्मै मङ्गलं वदत।
\vakya गेहं तद् यदि योग्यं स्यात् युष्मदुक्ता शान्तिस्तर्हि तदाश्रयतु, यदि त्वयोग्यं पुन र्युष्मदुक्ता शान्तिस्तर्हि युष्मासु परावर्ततां।
\vakya कश्चिच्च यदि युष्मान् न गृह्णाति युष्माकं वाक्यानि च न शृणोति, तर्हि यूयं तस्माद् गेहान्नगराद् वा निर्गच्छन्तः स्वचरणेभ्यो धूलिमवधूनुत।
\vakya युष्मानहं सत्यं ब्रवीमि, भविष्यति विचारदिने सह्यतरा सदोमघोमरयो र्देशस्य दशा नगरस्य तस्य दशातः।
\stitle{तेषां दुःखज्ञापनं।}
\vakya पश्यत, प्रहीयध्वे मया यूयं मेषा इव वृकाणां मध्ये, तद्भवत यूयं भुजङ्गा इव सतर्काः कपोता इव चामायिकाः।
\vakya भवत तु सावधाना मनुष्येभ्यः, यतस्ते युष्मान् विचारसभासु समर्पयिष्यन्ति स्वसमाजगृहेषु च कशाभिः प्रहरिष्यन्ति।
\vakya नायिष्यध्वेऽपि च यूयं समक्षं देशाधिपानां राज्ञाञ्च, साक्ष्यन्तु सम्भविष्यत्यनेन तेषां परजातीनाञ्च कृते।
\vakya यदा तु समर्पयिष्यध्वे, कथं किं वा तदा वक्तव्यं तच्चिन्तया माकुलीभवत। यतो यद् वक्तव्यं तत् तस्मिन्नेव दण्डे युष्मभ्यं दायिष्यते।
\vakya यतो न यूयं वक्तारः, वक्ता तु युष्मदन्तर्भाषमाणो युष्मत्पितुरात्मा।
\vakya भ्राता पुन र्भ्रातरं जनकश्च सुतं मृत्यवे समर्पयिष्यति, सन्तानाश्च पित्रोः प्रातिकूल्येनोत्थाय तौ घातयिष्यन्ति।
\vakya मम नाम्मो हेतुना च यूयं सर्वै र्द्वेक्ष्यध्वे। यस्त्वन्तं यावत् स्थिरः स्थास्यति स एव त्रायिष्यते।
\vakya यदा त्वेकस्मिन् नगरे जना युष्मानुपद्रवन्ति, तदान्यन्नगरं गन्तुं पलायध्वं। यतो युष्मानहे सत्यं ब्रवीमि, इस्रायेलस्य नगरेषु युष्माकं कार्यसमाप्तेः प्राक् मनुष्यपुत्र अगमिष्यति।
\vakya न श्रेयान् शिष्यो गुरुतो न च दासः स्वामितः।
\vakya स्वगुरुणा समानत्वं शिष्यस्य कृते पर्याप्तं, दासस्य कृते च स्वामिना समानत्वं। गेहपति र्यदि तै र्बेल्‌सबूब इत्यभिहितस्तर्हि तद्गृह्याः किं वाधिकं नाभिधायिष्यन्ते?
\vakya यूयन्तु तेभ्यो मा बिभीत, यतो नास्ति किमपि तादृशं तिरोहितं यन्नाविर्हितं भविष्यति, नापि तादृशं निगूढं यन्न ज्ञायिष्यते।
\vakya यूयं मया तिमिरे यदुच्यध्वे तद् दीप्तौ वदत, कर्णे च कथितं यच्छृणुथ तद् गृहोपरिष्टाद् घोषयत।
\vakya ये च देहं घ्नन्ति, नात्मानन्तु हन्तुं शक्नुवन्ति, तेभ्यो मा बिभीत, तस्मादेव वरं बिभीत यस्त्वात्मानञ्च देहञ्च नरके नाशयितुं शक्नोति।
\vakya चटकौ द्वौ किं नैकेन ताम्रखण्डेन विक्रीयेते? युष्माकं पितुरनुमतिं विना तु तेषाम् एकोऽपि भूमौ न पतिष्यति।
\vakya सन्ति च गणिताः सर्वेऽपि युष्मच्छिरसां कचाः।
\vakya अतो मा बिभीत। बहुभ्यश्चटकेभ्यो यूयं विशिष्यध्वे।
\vakya तद् यः कश्चिन्मनुष्याणां समक्षं मामङ्गीकुरुते, तमहमपि मम स्वर्गस्थस्य पितुः समक्षमङ्गीकरिष्ये।
\vakya यस्तु मनुष्याणां समक्षं मां प्रत्याख्याति, तमहमपि मम स्वर्गस्थस्य पितुः समक्षं प्रत्याख्यास्यामि।
\stitle{ख्रीष्टोपदेशफलं।}
\vakya मानुमिमीध्वं यदहं पृथिव्यामैक्यमवतारयितुमागत इति। नैक्यम् अपि त्वसिम् अवतारयितुमागतोऽस्मि।
\vakya यतो मनुष्यं पितु र्विरुद्धं दुहितरं मातु र्विरुद्धं पुत्रबंधू श्वश्र्वा विरुद्धं भेदयितुमहमागतः,
\vakya स्वगृह्याश्च मनुष्यस्य शत्रवो भविष्यन्ति।
\vakya यः पितरि मातरि वा मत्तोऽधिकं प्रेम कुरुते स मम न योग्यः। यश्च स्वक्रुशमादाय मां नानुव्रजति, स मम न याग्यः।
\vakya यः स्वप्राणानासादयति स तान् हारयिष्यते, यश्च मत्कृते स्वप्राणान् हारयते स तानासादयिष्यति।
\vakya यो युष्मान् गृह्णाति स मामेव गृह्णाति, माञ्च यो गृह्णाति स मत्प्रेषयितारमेव गृह्णाति।
\vakya यो भाववादिप्रत्ययाद् भाववादिनं गृह्णाति स भाववादिनः पारितोषिकं लप्स्यते। यश्च धार्मिकप्रत्ययाद् धार्मिकं गृह्णाति स धार्मिकस्य पारितोषिकं लप्स्यते।
\vakya शिष्यप्रत्ययाद् यश्चैतेषां क्षुद्राणामेकतमं चषकमेकं शीतलसलिलं पाययति, युष्मानहं सत्यं ब्रवीमि स कथञ्चन स्वपारितोषिकेण नोपवञ्चिष्यते\eoc