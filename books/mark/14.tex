\adhyAya
\stitle{ख्रीष्टविरुद्धा कुमन्त्रणा}
\vakya दिनद्वयात् परं निस्तारपर्वणा निष्किण्वपूपानां समयेन चोपस्थातव्यमितिकाले मुख्ययाजकाः शास्त्राध्यापकाश्चायतन्त यथा तं छलेन धृत्वा हन्युः।
\vakya ते त्ववदन्, मैव पर्वणि, चेत् तथा भविष्यति कलहो जनानां मध्ये।
\stitle{यीशोरभिषेकः।}
\vakya बैथनियायान्तु तस्मिन् विद्यमाने शिमोनस्य कुष्ठिनो गेहे भोजनायासिने च योषिदेका श्वेतोपलभाजनेन बहुमूल्यं प्रकृतं जटामांसीरसयुक्तं सुगन्धितैलमानीय श्वेतपलभाजनं तद् भित्त्वा तस्य शिरसि तैलं स्रावयामास।
\vakya तत्रोपस्थितास्तु केचिद् अन्तरे रौक्ष्यापन्ना जगदुः, किमर्थं तैलस्यायमपव्ययः सम्भूतः?
\vakya तस्मिन् विक्रीते त्रिशतमुद्रापादेभ्योऽधिकं मूल्यं लब्धुं दरिद्रेभ्यो दातुञ्चाशक्ष्यत। एवं ते तस्यामधैर्यमकुर्वन्।
\vakya यीशुस्तूवाच, इमां त्यजत, कुरुथ कथं दुःखितामिमाम्? कृतवतीयं मयि कर्म सत्।
\vakya यतो दरिद्राः सततं युष्मत्सङ्गिनः, यदा चेच्छथ तदैव तानुपकर्तुं शक्नुथ, अहन्तु न सततं युष्मत्सङ्गी।
\vakya यद् साध्यमनया तदकारि। समयात् प्रागियं मम देहं सुगन्धितैलाक्तं कृतवती मत्समाध्यर्थम्।
\vakya युष्मानहं सत्यं ब्रवीमि, कृत्स्नस्य जगतो यत्र कुत्रचिद् घोषयिष्यते सुसंवादोऽयं तत्र कुत्रचित् कथयिष्यते कर्माप्येतत् कृतमनया स्मरणार्थमस्याः।
\vakya ततो द्वादशानां मध्ये गणित ईष्करियोतीयो यिहूदा मुख्ययाजकानां समीपं जगाम तेष्येव तं समर्पयितुकामः।
\vakya तच्छ्रुत्वा ते हर्षं गत्वा तस्मै रूप्यं दातुम् अङ्गीचक्रिरे स च सुसमये तं समर्पयितुम् अयतत।
\stitle{निस्तारपर्वपालनं प्रभोर्भोज्यस्थापनञ्च।}
\vakya अनन्तरं निष्किण्वपूपानां प्रथमे दिनेऽर्थतो निस्तारपर्वीयमेषशावकहत्यादिने शिष्यास्तं जगदुः, भवत इच्छातः कुत्रास्माभि र्गत्वोपकल्पयितव्यं भवदर्थं भोज्यं निस्तारपार्वणम्?
\vakya तदा स शिष्यौ द्वौ प्रेषयन् जगाद, नगरं यातं तत्र मनुष्य एको जलपूर्णं कुम्भं वहन् युष्मत्सम्मुखवर्ती भविष्यति, तमेवानुगच्छतं स च यत्र प्रवेख्यति तत्र गृहस्वामिनं ब्रूतं,
\vakya गुरु र्वक्ति कुत्र सातिथिशाला यत्र शिष्यैः सार्धं मया भोज्यं निस्तारपार्वणं सम्पालयितव्यमिति।
\vakya स च वां भूमित ऊर्ध्वस्थामासनादिसज्जया सज्जितां बृहतीमेकां शालां दर्शयिष्यति तत्रैवास्मदर्थमुपकल्पयतम्।
\vakya अतः शिष्यौ तौ प्रस्थाय नगरमागत्य तेन यदुक्तं तदनुरूपं स्थानं प्राप्य निस्तारपर्वीयभोज्यमुपकल्पयामासतुः।
\vakya सन्ध्यायान्तु जातायां स द्वादशभिः सार्धमाजगाम।
\vakya तेषु तूपविश्य भक्षमाणेषु यीशुर्बभाषे, युष्मानहं सत्यं ब्रवीमि, युष्माकमेको मां समर्पयिष्यति यो मत्सहभोजी।
\vakya अनेन ते शोचितुमारभ्य प्रत्येकं तं पप्रच्छुः, स किमहं? स किमहं?
\vakya स तु प्रतिभाषमाणस्तानाह, द्वादशानामेको यो मया सार्धं सूपपात्रे हस्तं मज्जयति, स एव।
\vakya मनुष्यपुत्रमधि यथा लिखितमास्ते स तथैव प्रयाति। नरेण येन तु मनुष्यपुत्रः समर्प्यते स सन्तापपात्रं। तस्य नरस्य जन्म चेन्नाभविष्यत् तर्हि तेनैव तस्य क्षेमम् अभविष्यत्।
\vakya अतस्तेषां भोजनकाले यीशुः पूपमादायाशीर्वादं कृत्वा तं भङ्क्त्वा तेभ्यो ददौ बभाषे च, गृह्णीत भुंग्ध्वम्, एतन्मम शरीरं।
\vakya ततः परं स पानपात्रमादाय धन्यवादं कृत्वा तेभ्यो ददौ सर्वे च तेन पपुः
\vakya स च तानाह, एतन्मम शोणितं, बहूनां कृते विस्राव्यमाणं नूतननियमस्यैव शोणितं।
\vakya युष्मानहं सत्यं ब्रवीमि, गोस्तनीलतोत्पन्नोऽयं रसो दिने यस्मिन् ईश्वरस्य राज्ये नवीनो मया पायिष्यते तद्दिनं यावत् स मया पुन र्न पायिष्यते।
\vakya ततस्ते स्तोत्रं गीत्वा जैतुनपर्वतं जग्मुः।
\stitle{पित्रस्यानङ्गीकारे भविष्यद्वाक्यम्।}
\vakya यीशुस्तांस्तदा जगाद, रात्रावस्यां सर्वे यूयं मयि स्खलिष्यथ, यतो लिखितमास्ते, “अहं व्रजरक्षकमाहनिष्यामि मेषाश्च विकीर्णा भविष्यन्ति।”
\vakya परन्तु ममोत्थानात् परमहं युष्मदग्रे गालीलं यास्यामि।
\vakya पित्रस्तु तं जगाद, सर्वेऽपि चेत् स्खलेयुस्तथाप्यहं न स्खलिष्यामि।
\vakya यीशुस्तदा तमाह, त्वामहं सत्यं ब्रवीमि, अद्यैव रात्रावस्यां कुक्कुटस्य द्वितीयरवात् प्राक् त्वं मां त्रिकृत्वः प्रत्याख्यास्यसि।
\vakya स तु दृढतरमवोचत्, यद्यपि भवता सार्धं मम मरणम् अवश्यं स्यात्, तथाप्यहं भवन्तं नैव प्रत्याख्यास्यामि। एवमेव सर्वेऽप्यभाषन्त।
\stitle{गेत्‌शिमानीत्यभिधे उद्याने यीशोः प्रार्थना।}
\vakya ततः परमुपस्थितेषु तेषु गेत्‌शिमानीत्यभिधे स्थाने स स्वशिष्यान् जगाद, यूयमत्रोपविशत यावदहं प्रार्थनां कुर्वे।
\vakya अनन्तरं स पित्रं याकोबं योहनञ्च सङ्गिनः कृत्वा क्षोब्धुं विषत्तुञ्चारेभे तांश्च जगाद,
\vakya प्राणा मम मृत्यवे शोकापन्नाः, यूयमत्रावतिष्ठमाना जागृत।
\vakya ततः स किञ्चिदग्रं गत्वा भूमौ निपत्येदं प्रार्थयाञ्चक्रे, साध्ये सति यथा सा घटिका तस्मादपगच्छेत्।
\vakya स चाब्रवीत्, आब्बा तात, सर्वं साध्यं भवता, पानीयपात्रमिदं मत्तोऽपसारयतु। तथापि न मया, अपि तु त्वया यद् वाञ्छ्यते तदेव भवतु।
\vakya अनन्तरं स आगत्य तान् निद्राणान् दृष्ट्वा पित्रं जगाद, शिमोन निद्रासि किम्? घटिकामेकां जागर्तुं किं नाशक्नोः?
\vakya यूयं जागृत प्रार्थयध्वञ्च नो चेत् परीक्षां निवेक्ष्यध्वे, यत आत्मोद्यतः शरीरन्तु दुर्बलम्।
\vakya द्वितीयवारं गत्वा स वाक्यं तदेव व्याहरन् प्रार्थयाञ्चक्रे।
\vakya प्रत्यावृत्य तु तान् पुन र्निद्राणान् ददर्श, यतस्तेषां नेत्राण्यवशान्यासन् तस्मै च किमुत्तरं दातव्यं तत् तै र्नाबुध्यत।
\vakya तृतीयवारं पुनरागत्य स तान् जगाद, एवमेव यूयं निद्राणा विश्रामं सेवध्वे किम्? अलमनेन। उपस्थितः स दण्डः। पश्यत मनुष्यपुत्रः पापिनां करेषु समर्प्यते।
\vakya उत्तिष्ठत। वयं गच्छामः। पश्यत मम समर्पयिता समीपमागतः।
\stitle{यीशोः शत्रुहस्ते समर्पणम्।}
\vakya तथैव तस्मिन् भाषमाणे द्वादशानां मध्ये गणितो यिहूदा उपतस्थे, तेन सार्धञ्च मुख्ययाजकानां शास्त्राध्यापकानां प्राचीनानाञ्च सकाशाद् असियष्टिधारी महान् जननिवहः।
\vakya स समर्पयिता तेभ्यः सङ्केतं दत्तवान्, यथा, अहं यं चुम्बिष्यामि स एव सोऽस्ति, यूयं तमेव धृत्वा यत्नेन रक्षन्तोऽपनयत।
\vakya अतः स आगत्य तूर्णं तदन्तिकमुपस्थाय रब्बिन्नित्युक्त्वा तं चुचुम्ब।
\vakya तदा ते तस्मिन् हस्तार्पणं कृत्वा तं दध्नुः।
\vakya नराणान्तु तत्र तिष्ठतामेकः कोषादसिं निष्कृष्य महायजकस्य दासमाहत्य तस्य कर्णं परिचिच्छेद।
\vakya यीशुस्तु प्रतिभाषमाणस्ता् जगाद, दस्युमिव मां धर्तुम् असीन् यष्टींश्चादाय किं यूयं बहिरागताः?
\vakya अहन्तु प्रत्यहं धर्मधाम्नि शिक्षयन् युष्मत्सकाशमासं, यूयं तदा न मां धृतवन्तः। परन्तु शास्त्रीयोक्तिभिः सिद्धिर्गन्तव्या।
\vakya तदा सर्वे तं त्यक्त्वा पलायाञ्चक्रिरे।
\vakya युवा कश्चित् तु नग्नकाये सूक्ष्मं वस्त्रं परिधाय तमन्वगच्छत्।
\vakya युवभिर्ध्रियमाणस्तु स तत् सूक्ष्मं वस्त्रं विहाय नग्नः पलायाञ्चक्रे।
\stitle{महायाजकसमक्षं यीशोर्विचारः।}
\vakya यीशुस्तु महायाजकस्यान्तिकमानीतः। तदर्थञ्च सर्वे मुख्ययाजकाः प्राचीनाः शास्त्राध्यापकाश्च समाजग्मुः।
\vakya पित्रस्तु दूरान्महायाजकीयहर्म्यस्याभ्यन्तरं यावत् तमनुगत्य पदातिभिः सार्धमुपविश्याग्नितापमसेवत।
\vakya मुख्ययाजकाः कृत्स्ना महासभा च यीशुं हन्तुकामास्तस्य प्रतिकूलं साक्ष्यममृगयन्त, न तु लेभिरे।
\vakya बहवस्तद्विरुद्धं मृषासाक्ष्यमकथयन्, तेषां साक्ष्याणि तु समानानि नाभवन्।
\vakya तदा केचिदुत्थाय तस्य विरुद्धं मृषासाक्ष्यं कथयन्तोऽवदन्,
\vakya अस्य मुखादस्माभि र्वाक्यमिदमश्रावि यथा, हस्तकृतमिदं मन्दिरं भङ्क्त्वाहं दिनत्रयेऽपरमेकमहस्तकृतं निर्मास्ये।
\vakya इत्थमपि तेषां साक्ष्यं समानं नाभवत्।
\vakya अनेन महायजको मध्यस्थान उत्थाय यीशुं पप्रच्छ, किमपि त्वं किं न प्रतिभाषसे? इमे तव प्रतिकूलं किं साक्ष्यं ददति? स तु मौनमवलम्ब्यातिष्ठत्, किमपि न प्रत्यवदत्।
\vakya महायाजकः पुनस्तं पृच्छन्नवादीत्, त्वं किं परमधन्यस्य पुत्र ख्रीष्टः?
\vakya यीशुस्तदा जगाद, सोऽहमस्मि। यूयञ्च मनुष्यपुत्रं प्रभावस्य दक्षिण आसीनम् आकाशीयमेघरथेनागच्छन्तञ्च वीक्षिष्यध्वे।
\vakya महायाजकस्तदा स्ववासांसि विदीर्य बभाषे, साक्षिणामधिकानां किं प्रयोजनम्?
\vakya अश्रावि युष्माभिरीश्वरनिन्दा। युष्माभिः किं मन्यते? तदा सर्वे तं प्राणदण्डार्हमुक्त्वा दोषिण चक्रुः।
\vakya ततः केचित् तद्वपुषि निष्ठीवितुं तस्यास्यमाच्छाद्य मुष्टिभिराहन्तुम् इदं वक्तुञ्चारेभिरे, भावोक्तिं व्याहरेति। अनन्तरं पदातिनः प्रहारैस्तं जगृहुः।
\stitle{पितरेण यीशुः त्रिकृत्वः अस्वीकृतः।}
\vakya पित्रस्तु यदाधः प्राङ्गणेऽविद्यत, तदा महायाजकस्यैका दास्यागत्य
\vakya तापसेवने निविष्टं पित्रं दृष्ट्वा सुनिरीक्ष्य च बभाषे त्वमपि नासरतीययीशोः सङ्ग्यासीः।
\vakya अनेन सोऽनङ्गीकुर्वन् जगाद, न ज्ञायते न बुध्यते वा मया त्वया किं गद्यते। ततः स बहिःप्राङ्गणं जगाम तदैव कुक्कुटो रराव च।
\vakya तत्रत्या दासी च तं दृष्ट्वा पुनः समीपे संस्थितान् जनान् वक्तुमारेभे, असौ तेषामेकः। स तु पुनर्वारमनङ्गीचकार।
\vakya क्षणात् परं समीपे संस्थिता जनाः पुनर्वारं पित्रमूचुः, सत्यं त्वं तेषामेकतममः, त्वं हि गालीलीयोऽसि तव भाषापि तादृशी प्रतिभाति।
\vakya तदा सोऽनिष्टं प्रार्थयितुं शप्तुञ्चारभ्य व्याजहार, न जानेऽहं तं नरं यो युष्माभिरभिधीयते।
\vakya तदा द्वितीयवारं कुक्कुटो रुराव। अनन्तरं कुक्कुटस्य द्वितीयरवात् प्राक् त्वं मां त्रिकृत्वः प्रत्याख्यास्यसीति वाक्यं यत् स यीशुनोक्तस्तत् स्मृत्वा पित्रश्चिन्ताकुलो भूत्वा क्रन्दितुं प्रवृत्तः\eoc