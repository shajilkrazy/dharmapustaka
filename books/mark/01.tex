\adhyAya
\stitle{प्रभो र्यीशुख्रीष्टस्य स्नापनं।}
\vakya ईश्वरपुत्रस्य यीशोः ख्रीष्टस्य सुसंवादारम्भः। भाववादिनां ग्रन्थे लिखितम् आस्ते,
\begin{poem}
\startwithvakya “पश्य त्वद्गमनस्याग्रे स्वदूतं प्रहिणोम्यहं।
\pline तवाग्रे तव पन्थानं स हि सज्जीकरिष्यति॥
\vakya मरौ घोषयतः प्रोच्चैरस्त्ययं कस्यचिद् रवः।
\pline प्रभोः संस्कुरुताध्वानं विधद्धं तत्‌सृती ऋजूः॥”
\end{poem}
\vakya तथैव योहनः प्रान्तर उपस्थायास्नापयत् पापमोचनाय च मनःपरावर्तनार्थकम् स्नापनम् अघोषयत्।
\vakya अपि च यिहूदियाया निखिलो जनपदो यिरूशालेमनिवासिनश्च तदन्तिकम् अपागच्छन् सर्वे च स्वपापानि स्वीकुर्वाणास्तेन यर्दननद्याम् अस्नाप्यन्त।
\vakya स तु योहन उष्ट्रलोमजपरिधानश्चर्मपटुकया बद्धकटिश्चासीत् पतङ्गान् वन्यमधु चाभक्षयत्।
\vakya घोषणकाले तु सोऽवदन्, मत्तो बलवत्तरो नरो मत्पश्चाद् आयाति, नत्वा तदीयोपानहो र्बन्धन्यौ मोक्तुमप्यहं न योग्यः।
\vakya अहं तावद् युष्मान् तोये स्नापितवान्, स तु युष्मान् पवित्र आत्मनि स्नापयिष्यति।
\vakya तेष्वेव दिनेषु यीशु र्गालीलस्थनासरताद् आगत्य योहनेन यर्दनि स्नापयाञ्चक्रे
\vakya सद्यश्च तोयान्निर्गच्छन्नेव स गगनं भिद्यमानम् आत्मानञ्च कपोतमिव स्वोपर्यवतरमाणम् अदृक्षत्,
\vakya स्वर्गाच्चेयं वाणी समभूत् त्वं मम प्रियः पुत्रो यस्मिन्नह प्रीत इति।
\vakya तत्क्षणञ्चात्मा तं मरुमपनिनाय तत्र मरौ तिष्ठंश्च
\vakya स चत्वारिंशद् दिनानि शैतानेन पर्यैक्ष्यत सत्त्वैः सार्धञ्चावर्तत स्वर्गदूताश्च तं पर्यचरन्।
\stitle{प्रभो र्यीशोः प्रकाश्यकार्यारम्भः।}
\vakya समर्पिते तु योहने यीशु र्गालीलमागत्येश्वरराज्यस्य सुसंवादं घोषयन्नब्रवीत्,
\vakya सम्पूर्णः कालः समुपस्थितञ्चेश्वरस्य राज्यं। यूयं मनांसि परावर्तयत सुसंवादे च विश्वसित।
\vakya ततः परं गालीलीयसमुद्रस्य तटे विहरन् स शिमोनं तस्य भ्रातरम् आन्द्रियञ्च समुद्रे जालं क्षिपन्तौ ददर्श यतस्तौ धीवरावास्ताम्।
\vakya यीशुस्तानब्रवीत्, मम पश्चाद् आयातम् अहञ्च युवां मानवधारिणौ धीवरौ विधास्यामि।
\vakya तदा तौ सपदि स्वजालानि त्यक्त्वा तम् अन्वव्रजन्।
\vakya ततश्च स्तोकम् अग्रं गत्वा स सिबदियस्य पुत्रं याकोबं तत्सहोदरं योहनञ्च ददर्श, जालानां जीर्णोद्धारं कुर्वाणौ तावपि नौकायामास्माम्।
\vakya तदा सपदि तेनाहूतौ तौ स्वपितरं सिबदियं वेतनजीविभिः सार्धं नौकायां त्यक्त्वा तमनुजग्मतुः।
\stitle{भूतग्रस्ताद् भूतबहिष्करणम्।}
\vakya ततः परं ते कफरनाहूमं प्रविष्टाः। सद्यश्च विश्रामवारे स समाजगृहं प्रविश्याशिक्षयत्।
\vakya जनाश्च तस्य शिक्षायां चमत्कारम् अमन्यन्त। यतः स सामर्थ्यविशिष्ट इवाशिक्षयत् न तु शास्त्राध्यापका इव।
\vakya तेषां समाजे चाशुच्यात्माविष्ट एको नरोऽविद्यत स उत्क्रोशन् जगाद,
\vakya भो नासरतीय यीशो, अस्माकं भवतश्च किं? भवान् किमस्मान् नाशयितुमागतः? भवान् कस्तदहं जानामि, ईश्वरस्य पवित्रो नरो भवानिति।
\vakya यीशुस्तु तं निर्भर्त्सयन् बभाषे मौनीभव, अस्माच्च निःसर।
\vakya तदा सोऽशुचिरात्मा नरस्य तस्याङ्गान् सङ्कोच्य प्रोच्चरवेण क्रोशंस्तस्मान्निःसृतः।
\vakya अनेन सर्वे चमत्कारं मत्वा मिथ आलोचमाना अवदन्, किमिदं? का नूतनेयं शिक्षा? असौ तावत् सामर्थ्येनाशुचीनात्मनोऽप्याज्ञापयति ते चामुष्याज्ञावर्तिनो भवन्ति।
\vakya अनन्तरं तस्य ख्यातिः सत्वरं गालीलस्य कृत्स्नं परिसरं व्यानशे।
\stitle{पित्रस्य श्वश्र्वा आरोग्यकरणम्।}
\vakya ततः परं समाजगृहान्निष्क्रम्य स सत्वरं याकोबेन योहनेन च सार्धं शिमोनस्यान्द्रियस्य च गेहं प्रविवेश।
\vakya शिमोनस्य श्वश्रू तु ज्वरातुराशेत। ते च तस्मै सत्वरं तस्या अवस्थाम् अकथयन्।
\vakya तदा स उपस्थाय तस्या हस्तं धृत्वा ताम् उत्थापयामास। सद्यश्च ज्वरस्तां तत्याज सा च तान् पर्यचरत्।
\stitle{बहूनां लोकानामारोग्यकरणम्।}
\vakya सन्ध्यायान्तु जातायां यदा सूर्योऽस्तमगमत् तदास्वस्या भूताविष्टाश्च सर्वे तस्यान्तिकम् आनीताः
\vakya कृत्स्नं नगरञ्च द्वारि समेत्यातिष्ठत्।
\vakya स च नानाविधव्याधिभिरस्वस्थान् बहून् मानवान् स्वस्थीचकार भूतांश्च बहून् निःसारयामास, भूतांस्तु भाषितुं नानुजज्ञे यतस्ते तम् अभ्यजानन्।
\vakya प्रातस्त्वतीव प्रत्यूषं स उत्थाय निष्क्रान्तो निर्जनं स्थानमपगत्य प्रार्थयत।
\vakya शिमोनस्तु स्वसङ्गिभिः सार्धं तम् अनुदुद्राव।
\vakya तमासाद्य च तेऽब्रुवन्, सर्वे भवन्तम् अन्विष्यन्ति।
\vakya स तु तान् बभाषे, आयात, वयं समीपस्थान् ग्रामान् गच्छामः, तत्रापि मया घोषणं कर्तव्यं, यतस्तदर्थमहं निर्गतोऽस्मि।
\vakya ततः परं स कृत्स्ने गालीले जनानां समाजगृहेष्वघोषयत् भूतांश्च निरसारयत्।
\vakya एकः कुष्ठी तु तस्यान्तिकमागत्य प्रसादयन् जानू पातयंश्च तम् अब्रवीत्, भवांश्चेदिच्छति तर्हि मां शुचीकर्तुं शक्नोति।
\vakya तदा यीशुः करुणाविष्टो हस्तं प्रसार्य तं पस्पर्श बभाषे च, इच्छामि शुचीभव।
\vakya एवमुक्ते तेन सपदि कुष्ठं ततोऽपससार स च शुचीभूतः।
\vakya स तु तस्मिन्नधैर्यं गत्वा तत्क्षणं विसृज्य तं जगाद,
\vakya सावधानो भव कमपि किमपि मा वद, अपि तु गत्वा याजकायात्मानं दर्शय तेषाञ्च साक्ष्यार्थं स्वशुचित्वलाभनिमित्तं मोशिना यद्यदादिष्टं तदुपहर।
\vakya स तु निर्गत्य बहूशो घोषयितुं तां कथां विज्ञापयितुञ्च प्रावर्तत, तस्मात् पुनः प्रकाशं नगरं प्रवेष्टुं तेन नाशक्यत, अपि तु स बहि र्निर्जनेषु स्थानेष्ववर्तत सर्वतश्च जनास्तत्समीपमागच्छन्\eoc