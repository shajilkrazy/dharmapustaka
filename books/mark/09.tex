\adhyAya
\stitle{यीशोरुज्ज्वलमूर्तिधारणम्।}
\vakya षड्भ्यो दिनेभ्यः परं यीशुः पित्रं याकोबं योहनञ्च गृहीत्वा विजनं केवलं तानेव प्रोच्चमेकं गिरिम् आरोहयामास तेषां समक्षं रूपान्तरीबभूव च।
\vakya तस्य परिच्छदश्चोज्ज्वलो हिमवदतीव शुभ्रश्चाभूत्, तादृक् शुभ्रीकर्तुं पृथिवीस्थेन रजकेण न शक्यं।
\vakya अपि च मोशिरेलियश्च तेभ्यो दर्शनं ददतुः।
\vakya तौ यीशुना सह समलपतां।
\vakya तदा पित्रो यीशुं सम्बोध्यावादीत् रब्बिन् भद्रमस्माकमत्रावस्थानम् अतोऽस्माभिस्त्रीण्युटजानि निर्मीयन्तां भवतः कृत एकं मोशेः कृत एकम् एलियस्य कृते चैकं।
\vakya वास्तवं किं वक्तव्यं तत् तेन नाज्ञायत यतस्ते त्रासापन्ना आसन्।
\vakya ततः परं तेषामुपरि छायाकरो मेघः सञ्जातस्तस्मान्मेघाद् वाणीयमुदभूत् अयं मम प्रियः पुत्रोऽस्य वचांसि युष्माभिः श्रूयन्तामिति।
\vakya अकस्मात्तु परितो दृक्‌पातं कुर्वन्तस्ते पुन र्नापरं कमपि ददृशुः केवलं तैः सहितं यीशुं।
\vakya गिरितस्त्ववरोहत्सु तेषु यीशुस्तानिदमादिदेश, मनुष्यपुत्रो यावन्मृतानां मध्यान्नोत्तिष्ठति तावद् यूयं यद्यद् दृष्टवन्तस्तत् कमपि मा ज्ञापयत।
\vakya अतस्ते तद् वचनं स्वार्थम् अरक्षन् अपि तु मृतानां मध्यादुत्थानं किं भवेदेतदधि मिथो व्यचारयन्।
\vakya ततस्ते यीशुं पप्रच्छुः प्रथमम् एलियेनागन्तव्यमिति शास्त्राध्यापकैः कथमुच्यते?
\vakya स तान् प्रतिजगाद, सत्यं प्रथमम् एलिय आगत्य सर्वं पुनः संस्करिष्यति। मनुष्यपुत्रमप्यधि कथमिदं लिखितमास्ते यत् तेन बहुदुःखं सोढव्यं नगण्यता च गन्तव्या?
\vakya युष्मांस्त्वहं सत्यं ब्रवीमि, एलियोऽप्यागतवान् तमधि च यल्लिखितमास्ते तदनुरूपं ते तं प्रति तत्तत् कृतवन्तो यद्यदैच्छंस्ते।
\stitle{यीशुना भूतग्रस्तबालकस्यारोग्यकरणम्।}
\vakya ततः परं यदा स शिष्याणामन्तिकमागमत् तदा महान्तं जननिवहं तेषां परितः संस्थितं शास्त्राध्यापकांश्च तै र्विवदमानान् ददर्श।
\vakya तत्क्षणन्तु स कृत्स्नो जननिवहस्तं दृष्ट्वा चमत्कारं मेने मनुष्याश्च समाद्रवन्तस्तमभ्यनन्दन्।
\vakya स तु शास्त्राध्यापकान् पप्रच्छ, एतैः किं विवदध्वे?
\vakya तदा जननिवहमध्यान्नर एकः प्रतिबभाषे, गुरो, भवत्समीपमहं स्वपुत्रमानीतवान् स मूकेनात्मनाविष्टः।
\vakya स चात्मा यत्र तमाक्राम्यति तत्र तं कर्षति स च फेनमुद्गमयति दन्तैश्च दन्तान् घर्षति यष्टिवत् कठिनो जायते च। भवतः शिष्या यत् तं निःसारयेयुस्तदर्थमहं तेभ्यो न्यवेदयं ते तु नाशक्नुवन्।
\vakya स तदा तान् प्रतिभाषमाणो जगाद, रे विश्वासहीन वंश युष्मत्समीपं मया कियत्कालमवस्थातव्यम्? कियत्कालं यूयं मया सोढव्याः? स मदन्तिकमानीयताम्।
\vakya ततस्तस्मिंस्तदन्तिकमानीते तं दृष्ट्वैव स आत्मा तं बालं कर्षन् क्षोभयामास स च भूमौ निपत्य फेनमुद्गमयन्नलुठत्।
\vakya तदा स तत्पितरं पप्रच्छ, कियत्कालमेतत् तस्य सम्भूतम्? स जगाद, आबाल्यात्।
\vakya स च बहुकृत्वस्तं नाशयितुकामो वह्नौ तोये च निक्षिप्तवान्।
\vakya यीशुस्तु तमब्रवीत् यदि शक्यमिति कथं वदसि? सर्वमेव शक्यं विश्वसता।
\vakya ततस्तस्य बालकस्य पिता तत्क्षणमुच्चैःस्वरेण साश्रुनेत्रो जगाद, विश्वसिमि प्रभो भवान् ममाविश्वासमुपकरोतु।
\vakya तदा यीशु र्जननिवहं समाद्रवन्तं दृष्ट्वा तमशुचिमात्मानं तर्जयन् बभाषे, रे मूक बधिर आत्मन् त्वामहमाज्ञापयामि, अस्मान्निःसर मैव पुनरिममाविश।
\vakya तदा स क्रोशित्वा भृषं तं कर्षित्वा च निःससार। स बालस्तु मृतकल्पो बभूव तेन बहवोऽवादिषुः स मृतः।
\vakya यीशुस्तु तस्य हस्तं धृत्वा तमुत्थापयामास ततः स उत्तस्थौ।
\vakya ततः परं तस्मिन् गृहं प्रविष्टे तस्य शिष्या निर्जनं तं पप्रच्छुः, कथं वयं तमात्मानं निःसारयितुं नाशक्नुम?
\vakya स तान् जगाद, जातेरस्या निःसारणं नान्येन केनाप्युपायेन सिध्यति, केवलं प्रार्थनोपवासाभ्याम्।
\stitle{यीशोः स्वमरणमधि द्वितीयवारं कथाकथनम्।}
\vakya स्थानात् तस्मात् प्रस्थाय ते गालीलदेशेन प्रागच्छन्, तत्तु केनापि ज्ञायेत तत् स नैच्छत्,
\vakya यतः स स्वशिष्यान् शिक्षयन्नकथयत्, मनुष्यपुत्रो मनुष्याणां हस्तेषु समर्पयिष्यते तै र्घानिष्यते च, हतस्तु स तृतीये दिने पुनरुत्थास्यति।
\vakya ते तु तद् वचनं नाबुध्यन्त तं प्रष्टुमबिभयुश्च।
\stitle{श्रेष्ठः कः।}
\vakya ततः परं ते कफरनाहूमम् आजग्मुः। गृहं प्रविष्टश्च स तान् पप्रच्छ पथि किमधि मिथो वादानुवादावकुरुत?
\vakya ते तु मौनिनो बभूवु र्यतः को महत्तम एतदधि ते पथि मिथो वादानुवादौ कृतवन्तः।
\vakya स चोपविश्य द्वादश शिष्यानाहूय जगाद, यदि कश्चित् प्रथमो भवितुं वाञ्छति तर्हि स सर्वेषामन्त्यः सर्वेषां परिचारकश्च भवतु।
\vakya अपि च स बालकमेकमादाय तेषां मध्ये स्थापयामास तं क्रोडे गृहीत्वा च तान् अब्रवीत्,
\vakya यः कश्चिदीदृशमेकं बालकं मम नाम्ना गृह्णाति स मां गृह्णाति, माञ्च यो गृह्णाति स न माम् अपि तु मत्प्रेरकमेव गृह्णाति।
\stitle{ख्रीष्टस्य नानोपदेशः।}
\vakya तदा योहनस्तं प्रतिबभाषे, गुरो, भवतो नाम्ना यो भूतान् निःसारयत्यस्मांस्तु नानुगच्छति तादृशो नरः कश्चिदस्माभि र्दृष्टः, स त्वस्मान् नानुगच्छतीतिहेतो र्वयं तं न्यवारयाम।
\vakya यीशुस्तु जगाद, मैव तं निवारयत, यतो नास्ति कोऽपि तादृशो यो मन्नाम्ना प्रभावसिद्धं कर्म कृत्वा सहसा मां निन्दितु शक्नुयात्।
\vakya यतो योऽस्माकं विपक्षो नास्ति सोऽस्माकं स्वपक्षः।
\vakya यूयं ख्रीष्टस्येति हेतो र्मम नाम्ना यः कश्चिद् युष्मान् चषकमेकं तोयं पाययति, युष्मान् सत्यं ब्रवीमि, स कथञ्चन स्वपारितोषिकेण नापवञ्चिष्यते।
\vakya यः कश्चित्तु मयि विश्वसतामेतेषां क्षुद्राणामेकं स्खालति तत्कण्ठे बृहत्पेषणीबन्धनं समुद्रे च तस्य निपातनं श्रेयः।
\vakya तव हस्तश्च चेत् तव स्खलनहेतु र्भवेत् तर्हि तं छिन्धि।
\vakya त्वदर्थमिदं श्रेयो यत् त्वं हीनाङ्गो जीवनं प्रविशेः, न त्विदं यद् द्विहस्तो नरकमनिर्वाणञ्च वह्निं गच्छे र्यत्र निवासिनां कीटो न म्रियते वह्निश्च न निर्वायति।
\vakya तव चरणो वा चेत् तव स्खलनहेतु र्भवेत् तर्हि तं छिन्धि।
\vakya त्वदर्थमिदं श्रेयो यत् त्वं खञ्जो जीवनं प्रविशेः, न त्विदं यद् द्विचरणो नरके वह्नौ चानिर्वाणे निक्षिप्येथा यत्र निवासिनां कीटो न म्रियते वह्निश्च न निर्वायति।
\vakya तव नेत्रं वा चेत् तव स्खलनहेतु र्भवेत् तर्हि तदुत्पाटय।
\vakya त्वदर्थमिदं श्रेयो यत् त्वमेकनेत्र ईश्वरराज्यं प्रविशेः, न त्विदं यद् द्विनेत्रो वह्निमये नरके निक्षिप्येथा यत्र निवासिनां कीटो न म्रियते वह्निश्च न निर्वायति।
\vakya वास्तवमेकैको नरो वह्निरूपेण लवणेन मिश्रयिष्यते, एकैको यज्ञीयोपहारश्च लवणेन मिश्रयिष्यते। लवणं भद्रं।
\vakya लवणन्तु यदि लवणत्वहीनं जायते तर्हि केनोपायेन तत् स्वादयुक्तं करिष्यथ? स्वान्तरे लवणं रक्षत मिथ ऐक्यमाचरत च\eoc