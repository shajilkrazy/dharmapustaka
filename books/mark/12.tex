\adhyAya
\stitle{द्राक्षाक्षेत्रस्य कृषकाणां दृष्टान्तः।}
\vakya तदा स उपमाभिस्तेभ्यः कथयितुमारेभे, नरः कश्चिद् द्राक्षालता रोपयन् द्राक्षोद्यानं कृत्वा वृत्या परिवार्य तन्मध्ये द्राक्षामर्दनार्थककुण्डं खनित्वाट्टालं निर्माय च कृषकेषु करदायिषु तत् समर्प्य देशान्तरं जगाम।
\vakya ततः परं फलकाले कृषकेभ्यो द्राक्षोद्यानस्य फलानामंशमादातुं स कृषकाणां समीपं दासमेकं प्राहिणोत्।
\vakya ते तु तं धृत्वा ताडयामासूरिक्तहस्तं विससृजुश्च।
\vakya ततः स पुनस्तेषां समीपमन्यमेकं दासं प्रेषयामास। ते प्रस्तरक्षेपणेन तस्यापि मूर्धानमाजघ्नुर्न्यक्कृत्य विससृजुश्च।
\vakya ततः परमन्यस्मिन् प्रहिते तेन ते तं घातयामासुः। अपरांश्चानेकान् अहिंसिषुः, कांश्चित् प्राहारेण कांश्चिद् वा वधेन।
\vakya तदैक एव तस्य प्रियः पुत्रोऽवशिष्टः। अतो मत्पुत्रं प्रति ते त्रपिता भविष्यन्तीति बुद्ध्या शेषे स तमपि तेषां समीपं प्राहिणोत्।
\vakya ते कृषकास्तु परस्परमब्रुवन्, दायहरोऽसौ, तदागच्छत वयममुं हनिष्यामस्तेन दायांशोऽस्माकं भविष्यति।
\vakya अनेन ते तं धृत्वा जघ्नु र्द्राक्षोद्यानाद् बहि र्निचिक्षिपुश्च।
\vakya अतस्तस्योद्यानस्य स्वामी किं करिष्यति? स आगत्य तान् कृषकान् विनाशयिष्यति, द्राक्षोद्यानञ्चान्येषु समर्पयिष्यति।
\vakya युष्माभिः कदापि किं शास्त्रोक्तिरियं न पठिता, यथा,
\begin{poem}
\startwithline “गृहनिर्मातृभि र्लोकै र्यः पाषाणो निराकृतः।
\pline स एव गृहकोणस्थः प्रमुख्यः प्रस्तरोऽभवत्।
\vakya परमेशस्य कर्मेदम् अस्मद्दृष्टौ तदद्भुतम्॥”
\end{poem}
\vakya ते तदा तं धर्तुमयतन्त जननिवहात्त्वबिभयुः। यतस्तैरबोधि यत् स तानधि तां दृष्टान्तकथामवदत्। तं त्यक्त्वा तु ते प्रस्थिताः।
\stitle{करदाने फरीशिनां निरुत्तरताकरणम्।}
\vakya ततः परं वाक्‌पाशेन तं निबद्धुमिच्छन्तस्ते कांश्चित् फरीशिनो हेरोदीयांश्च तस्यान्तिकं प्रेषयामासुः।
\vakya त उपागत्य तमूचुः भो गुरो वयं जानीमो यद् भवान् सत्यवान् कस्मादपि न बिभेति यतो भवान् मनुष्याणां मुखापेक्षां न करोत्यपि तु सत्येनेश्वरस्य पन्थानं शिक्षयति। कैसराय करदानं विधेयं न वेति? दास्यामोऽथवा न दास्यामः किं?
\vakya स तु तेषां कापट्यं बुद्ध्वा तान् जगाद, कथं मां परीक्षध्वे? मुद्रापादमेकमानयताहं तं निरीक्षिष्ये।
\vakya तस्मिंस्तैरानीते स तान् पप्रच्छ कस्येयं मूर्तिरिदं लेखनञ्च? ते तं वदन्ति कैसरस्य।
\vakya तदा यीशुस्तान् प्रतिबभाषे, दत्त कैसराय यद्यत् कैसरस्य, दत्त चेश्वराय यद्यदीश्वरस्य। अनेन ते तस्मिन्नाश्चर्यं मेनिरे।
\stitle{परकालविषयिणी शिक्षा।}
\vakya तदा पुनरुत्थानमनङ्गीकुर्वाणाः सद्दूकिनस्तस्यान्तिकमागत्य तं पप्रच्छुः,
\vakya गुरो, मोशिरस्मदर्थमिममादेशं लिखितवान्, यस्य भ्राता भार्यां विहाय निःसन्तानो म्रियते, स तां भ्रातृभार्यामुदुह्य स्वभ्रात्रे वंशमुत्पादयत्विति।
\vakya आसंस्तु सप्त भ्रातरः। प्रथमः योषितमुदुह्य निःसन्तानो ममार।
\vakya द्वितीयस्तदा तामुदुवाह, सोऽपि निःसन्तानो ममार। तृतीयस्य तादृश्येव गति र्बभूव।
\vakya इत्थं ते सप्त तामुदूहु र्निःसन्ताना मम्रुश्च। सर्वेषां पश्चात् सा योषिदपि ममार।
\vakya अतः पुनरुत्थान उत्थितानां तेषां कस्य भार्या सा भविष्यति? यतस्ते सप्तैव तामुदूढवन्तः।
\vakya यीशुस्तदा तान् प्रत्यवादीत्, यूयं किं न हेतोरतो भ्राम्यथ यच्छास्त्राणीश्वरस्य शक्तिञ्च न जानीथ?
\vakya यतः पुनरुत्थानात् परं ते नोद्वहन्ति नोदुह्यन्ते वापि तु स्वर्गनिवोसिनो दूता इव वर्तन्ते।
\vakya मृतानधि, ते यदुत्तिष्ठन्त्येतदधि किं न पठितं युष्माभि र्मोशे र्ग्रन्थे स्तम्बवृत्तान्ते तं प्रति कथितमीश्वरस्येदं वाक्यं, यथा, अब्राहामस्येश्वर इस्‌हाकस्य चेश्वरो याकोबस्य चेश्वरोऽहमिति।
\vakya ईश्वरो न मृतानाम् अपि तु जीवितामीश्वरोऽस्ति, अतो यूयं वाढं भ्राम्यथ।
\stitle{सर्वप्रधानादेशमधि शिक्षा।}
\vakya शास्त्राध्यापक एकस्तेषां तं विवादं श्रुत्वा, स तेभ्य उत्तमं प्रत्युत्तरं दत्तवानिति बुद्ध्वा चोपागत्य तं पप्रच्छ, सर्वप्रथमाज्ञा का?
\vakya यीशुस्तं प्रत्यवादीत् सर्वाज्ञानामियं प्रथमा, भो इस्रायेल, शृणु, अस्माकमीश्वरः प्रभुरेकः प्रभुरेव।
\vakya त्वञ्च सर्वान्तःकरणेन, सर्वप्राणैः, सर्वचित्तेन, सर्वशक्त्या च स्वेश्वरे प्रभौ प्रेम कुर्वितीयं प्रथमाज्ञा।
\vakya द्वितीया चास्याः सदृशी, यथा, त्वं स्वनिकटस्थ आत्मवत् प्रेम कुर्विति। एतयो र्महत्तरापराज्ञा नास्ति।
\vakya स शास्त्राध्यापकस्तदा तं जगाद, वाढं, गुरो, सत्यमुक्त भवता, यत ईश्वर एक एव तदन्यो नास्ति,
\vakya सर्वान्तःकरणेन सर्वबुद्ध्या सर्वप्राणैः सर्वशक्त्या च तस्मिन् प्रेमाचरणं निकटस्थ आत्मवत् प्रेमाचरणञ्च सर्वहव्येभ्यः सर्वयज्ञेभ्यश्चाधिकम्।
\vakya स च सविवेचनमुत्तरं कृतवानिति दृष्ट्वा यीशुस्तमुवाच, ईश्वरराज्यात् त्वं न दूरवर्तीति। ततः परं नाभवदन्यस्य कस्यापि साहसं तं किमपि प्रष्टुं।
\vakya ततः परं यीशुः प्रतिभाषमाणो धर्मधाम्नि शिक्षयन्नब्रवीत्, शास्त्राध्यापकाः कथं वदन्ति ख्रीष्टो दायूदस्य पुत्र इति।
\vakya स्वयं दायूदस्तु पवित्रस्यात्मन आदेशादिदं कथितवान्,
\begin{poem}
\startwithline “मम प्रभुमिदं वाक्यं बभाषे परमेश्वरः।
\pline त्वच्छत्रून् पादपीठं ते यावन्नहि करोम्यहं।
\pline अवतिष्ठस्व तावत् त्वम् आसीनो मम दक्षिणे॥”
\end{poem}
\vakya अनेन स्वयं दायूदस्तं प्रभुं वदति, कथं तर्हि स तस्य पुत्रो भवेत्? तदा समागतो महाजननिवहः सानन्दं तस्य वाक्यान्याकर्णयत्।
\vakya स्वशिक्षायाञ्च स तान् जगाद, यूयं शास्त्राध्यापकेभ्यः सावधाना भवत, आकाङ्क्षन्ति ते महापरिच्छदै र्विहारं हट्टेष्वभिनन्दनानि समाजगृहेषु श्रेष्ठासनानि भोज्येषु च श्रेष्ठस्थानानि।
\vakya ते विधवानां गृहाणि ग्रसन्ति छलाच्च सुदीर्घं प्रार्थयन्ते।
\vakya विचारे ते घोरतरं दण्डं लप्स्यते।
\vakya तदा यीशु र्धनागारस्य समक्षमुपविश्य तत्र धनागारे जनैः कथं मुद्रा निक्षिप्यन्ते तन्निरैक्षत। बहवो धनिनश्च बहुमुद्रा न्यक्षिपन्।
\vakya विधवा तु काचिद् दरिद्रागत्य क्षोधिष्ठमुद्राद्वयं न्यक्षिपत् तत् ताम्रमुद्रापादेन समं।
\vakya यीशुस्तदा स्वशिष्यान् अन्तिकमाहूयावादीत्, युष्मानहं सत्यं ब्रवीमि, यावन्तोऽत्र धनागारे मुद्रा निक्षिप्तवन्तस्तावतां मध्येऽधिकतमं विधवासौ दरिद्रा निक्षिप्तवती।
\vakya यतः सर्वे प्रयोजनातिरिक्तं किञ्चिन्निक्षिप्तवन्तः, सा तु प्रयोजनीयस्याभावेऽपि सर्वस्वं निक्षिप्तवती जीविकां कृत्स्नामेव\eoc