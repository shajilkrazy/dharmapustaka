\adhyAya
\stitle{स्वदेशीयलोकै र्यीशोरसम्मानम्।}
\vakya ततः परं स तस्मात् स्थानान्निर्गत्य स्वदेशमागत् तस्य शिष्याश्च तम् अन्वगच्छन्।
\vakya उपस्थिते तु विश्रामवारे स समाजगृहे शिक्षयितुम् आरेभे बहवश्च श्रोतार आश्चर्यं मन्यमाना अवदन्, कुतोऽमुना सर्वमेतदलम्भि? अमुष्मै दत्तञ्चेदं किं ज्ञानं? अमुष्य कराभ्यामीदृशानि प्रभावस्य कर्माणि कथं साध्यन्ते?
\vakya असौ किं नास्मदीयतक्षा मरियमः सुतो याकोबस्य योषे र्यिहूदाः शिमोनस्य च भ्राता? अमुष्य भगिन्योऽपि किं नात्रास्मन्निकटे वर्तन्ते? इत्थं ते तस्मिन् स्खलिताः।
\vakya यीशुस्तु तान् अब्रवीत्, नान्यत्र कुत्रापि भाववाद्यसम्मानितः केवलं स्वदेशे स्वकुले स्वगृहे च।
\vakya अतः स तत्र प्रभावसिद्धं किमपि कर्म कर्तुं नाशक्नोत् केवलं स्वल्पान् अस्वस्थान् नरान् हस्तार्पणेन निरामयान् अकार्षीत्। तेषाम् अविश्वासे चातीवाश्चर्यं मेने।
\vakya ततः परं स परितः स्थितेषु ग्रामेषु पर्यटन्नशिक्षयत्।
\stitle{शिष्यान् प्रति यीशोरुपदेशः।}
\vakya अपि च स पूर्वोक्तान् द्वादश शिष्यान् स्वसमीपम् आहूय द्वौ द्वौ कृत्वा प्रेषयितुम् आरेभे, तेभ्यश्चाशुचीनाम् आत्मनां कर्तृत्वम् अदात्।
\vakya तांश्चेदमादिदेश, यात्रार्थं केवलं यष्टिमादायान्यत् किमपि मादद्ध्वं, भिक्षाधारं, पूपं, पटुकायां वा मुद्रा मादद्ध्वं,
\vakya अपि तु काष्ठोपानहौ बन्धीत, द्वे त्वङ्गाच्छादने मा परिधद्ध्वं।
\vakya स तान् इदमपि जगाद, यत्र कुत्रचिद् गृहं प्रविशथ, स्थानान्तरगमनं यावत् तत्रावतिष्ठध्वं।
\vakya ये च सर्वे मानवा युष्मान् नानुगृह्णन्ति युष्मद्वाक्यानि च न शृण्वन्ति, तेषां स्थानान्निर्गत्य तेषां विरुद्धं साक्ष्यार्थं स्वचरणाधःसिथितां मृत्तिकामप्यवधुनुत। युष्मानहं सत्यं ब्रवीमि, विचारदिने सदोमस्य घमोराया वा दशा तस्य नगरस्य दशातः सह्यतरा भविष्यति।
\vakya अतस्ते निर्गत्य मनःपरावर्तनस्य कर्तव्यताम् अघोषयन्,
\vakya बहून् भूतांश्च निरसारयन् अस्वस्थांश्च बहून् तैलेनाभिषिच्य निरामयान् अकुर्वन्।
\stitle{योहनस्नापकस्य वधः।}
\vakya तदा तस्य नाम्नि प्रसिद्धतां गते राजा हेरोदस्तस्य ख्यातिं श्रुत्वाब्रवीत्, स्नापको योहनो मृतानां मध्याद् उत्थितस्ततो हेतोः प्रभावास्तस्मिन् स्वकार्यं साधयन्ति।
\vakya अन्ये त्ववदन् स एलियः। अपरे चावदन् स भाववादी किंवा भाववादिनाम् एकेन सदृशः।
\vakya हेरोदस्तु श्रुत्वा जगाद, यस्य योहनस्य शिरो मया छिन्नं स एवासौ। स मृतानां मध्यादुत्थितः।
\vakya वास्तवं हेरोद एव स्वभ्रातुः फिलिपस्य भार्यां यां हेरोदियामुदूढवांस्तस्याः कारणात् प्रेष्यान् प्रहित्य योहनं धृत्वा कारायां बद्धवान्।
\vakya यतो योहनो हेरोदमगदत्, भ्रातृभार्यायाः स्वामित्वं त्वया न विधेयमिति।
\vakya हेरोदिया च तस्मै पर्यकुप्यत् तञ्च हन्तुम् ऐच्छत् न त्वशक्नोत्।
\vakya यतो हेरोदो योहनं धार्मिकं पवित्रञ्च नरं ज्ञात्वा तस्मादबिभेत् तञ्चारक्षत् तस्यादेशं श्रुत्वा च बहूनि कर्माण्यकरोत् प्रीतश्च तस्य वाक्यान्याकर्णयत्।
\vakya अपरम् उपस्थिते शुभदिने यदा हेरोदः स्वीयजन्मोत्सवे स्वीयमहल्लोकानां सहस्रपतीनां गालीलीयप्रथमानाञ्च निमित्तं रात्रिभोज्यम् अकरोत्,
\vakya तदा तस्या हेरोदियाया दुहितरि सभां प्रविश्य नृत्येन हेरोदस्य तत्रोपविष्टानां तत्सङ्गिनाञ्च प्रीतिमुत्पादितवत्यां राजा तां बालिकामब्रवीद् यद् वाञ्छसि तन्मां याचस्व तदेव तुभ्यं दास्यामि।
\vakya अपि च स शप्त्वा तां जगाद, यद् वाञ्छसि तत् तुभ्यं दास्यामि, अर्धराज्यपर्यन्तं तद् दास्यामि।
\vakya तदा सा निर्गत्य स्वजननीम् अप्राक्षीत्, किं याचिष्ये? सा जगाद, स्नापकस्य योहनस्य मस्तकमिति।
\vakya ततः सा क्षिप्रं सोत्साहं प्रविश्य रोज्ञोऽन्तिकं गत्वायाचिष्ट यथा भवान् सपदि मह्यं स्थाले स्नापकस्य योहनस्य मस्तकं ददात्वियं मम वाञ्छा।
\vakya अनेन राजा शोकार्तो जातोऽपि शपथानां सङ्गिनाञ्च भयात् तां प्रत्याख्यातुं नैच्छत्।
\vakya अतो राजा तूर्णं प्रतिहारिणं प्रहित्य तस्य शिरस आनयनमादिदेश।
\vakya स च गत्वा कारायां तस्य शिरश्छित्वा स्थाले निधायानीय बालिकायै ददौ बालिका च स्वमात्रे तद् ददौ। सर्वमेतच्छ्रुत्वा तस्य शिष्या आगत्य तस्य देहम् आदाय शवागारे निदधुः।
\stitle{यीशुना पञ्चसहस्र लोकेभ्य आश्चर्यरूपेण भोजनदानम्।}
\vakya ततः परं प्रेरिता यीशोः समक्षं समाजग्मुः, यच्च कृतवन्तः शिक्षितवन्तश्च तत् सर्वं तस्मै निवेदयामासुः।
\vakya स च तान् जगाद, यूयमेव गुप्तं निर्जनं स्थानमागत्य किञ्चिद् विश्राम्यत।
\vakya यत आगच्छतामपगच्छताञ्च महती सङ्ख्यासीत्, आहारायाप्यवकाशस्तेषां नाभवत्।
\vakya अतस्ते गुप्तं नावमारूह्य निर्जनं स्थानं जग्मुः।
\vakya जननिवहास्तु गच्छतस्तान् अपश्यन् बहवश्चाभ्यजानन्, अतः सर्वनगरेभ्यो मानवाः स्थलमार्गेण तत्र धावन्तस्तान् अतिचक्रमुस्तस्य समीपं समाजग्मुश्च।
\vakya यीशुस्तु वहिरेत्य माहान्तं जननिवहं दृष्ट्वा तान् अनुचकम्पे, यतस्ते हीनरक्षकै र्मेषैः सदृशा आसन्। स च तेभ्यः प्रभृतां शिक्षां दातुमारेभे।
\vakya इत्थं दिनस्य महाभोगेऽतीते शिष्या उपागत्य तमूचुः, निर्जनमिदं स्थानं गतञ्च बहुतरं दिनं।
\vakya भवान् अमून् विसृजतु यथा ते परितः स्थिताः पल्ली र्ग्रामांश्च गत्वा स्वार्थं पूपान् क्रीणीयु र्यतस्तेषां खाद्यं नास्ति।
\vakya स तु तान् प्रतिबभाषे, यूयमेव तेभ्य आहारं दत्त। अनेन ते तम् अब्रुवन्, वयं किं गत्वा द्विशतमुद्रापादैः पूपान् क्रीत्वामीभ्य आहारं दास्यामः?
\vakya स तु तान् पप्रच्छ, युष्मदन्तिकं कति पूपा विद्यन्ते? गत्वा पश्यत। ते तदवगत्याब्रुवन्, पञ्च पूपा द्वौ मत्स्यौ च।
\vakya ततः स तान् हरति तृणे पङ्क्तिभिः सर्वेषाम् उपवेशनमादिदेश।
\vakya तदा ते आहारार्थं शतं शतं पञ्चाशत् पञ्चाशच्च नराश्चतुरस्रपङ्क्तिभिरुपविष्टाः
\vakya स च तान् पञ्च पूपान् द्वौ मत्सौ चादाय स्वर्गं प्रत्युच्चदृष्टि र्भूत्वाशिषमवादीत्, पूपांश्च भङ्क्त्वा तेभ्यः परिवेषणार्थं स्वशिष्येभ्योऽददात् तौ द्वौ मत्स्यावपि सर्वेषां कृते विभेजे।
\vakya इत्थं सर्वे भुङ्क्त्वा ततृपुः।
\vakya भग्नांशान् सङ्गृह्य च द्वादशडल्लकान् पूरयामासुः, मत्स्योरपि भग्नांशान् आददिरे।
\vakya ते पूपभोक्तारः प्रायः पञ्चसहस्रपुरुषा आसन्।
\stitle{यीशुना पदव्रजेन जलसञ्चरणम्।}
\vakya ततः परं स सपदि सोत्साहं स्वशिष्यान् नावं प्रवेष्टुं यावच्च स जनौघं विसृजति तावद् अग्रतो बैत्सैदामुद्दिश्य गन्तुमादिदेश।
\vakya अनन्तरं स जनान् अनुज्ञाय प्रार्थयितुं गिरिमारुरोह।
\vakya सन्ध्यायान्तूपस्थितायां नौका मध्यसागरं स चैकाकी स्थल आसीत्।
\vakya नौकावाहने च ते क्लिश्यन्ते यतो वायुस्तेषां सम्मुख इति तेनादर्शि। प्रायो यामिन्याश्चतुर्थे यामे तु स समुद्रोपरि व्रजंस्तेषां समीपमागमत् तानतिक्रमितुञ्च प्रावर्तत।
\vakya ते च समुद्रोपरि व्रजन्तं तं दृष्ट्वापच्छायां मत्वा चुक्रुशुः, यतः सर्व एव तं पश्यन्त उद्विविजिरे।
\vakya स तु तूर्णं तानालपन्नब्रवीत्, आश्वसित, सोऽहं, मा भैष्ट।
\vakya ततः परं स नौकामारुह्य तेषां समीपम् उपतस्थे वायुश्च शशाम।
\vakya अनेन ते मनःस्वतिमात्रं विस्मिता आश्चर्यं मेनिरे, यतः पूपानां वृद्ध्या तै र्न बुद्धं, तेषां हृदयन्तु जडीभूतमासीत्।
\vakya तरित्वा ते गिनेषरताख्यां जनपदम् उपस्थाय नावं कूले बबन्धुः।
\vakya नावस्तु निर्गतेषु तेषु मानवाः सपदि तमभिज्ञाय धावन्तो निकटस्थं कृत्स्नं प्रदेशं गत्वा पर्यङ्कैरस्वस्थान् वहन्तो यत्र कुत्रापि तस्योपस्थितेः संवादम् अशृण्वंस्तत्तत् स्थानं निन्युः।
\vakya तस्मिंश्च ग्रामान् नगराणि वा पल्ली र्वा प्रविशति मानवा अस्वस्थान् हट्टस्थानेषु निधाय तैस्तदीयवस्त्रप्रलम्बकस्पर्शनमेव तमयाचन्त, यावन्तश्चास्पृशंस्ते सर्वे तेरुः\eoc