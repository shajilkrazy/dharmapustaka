\adhyAya
\stitle{बीजवापकस्य दृष्टान्तः।}
\vakya पुनः स समुद्रतटे शिक्षयितुम् आरेभे, तदा तु महति जनौघे तत्समीपं समागते स नौकां प्रविश्य समुद्र उपाविशत् कृत्स्नो जनौघश्च समुद्रतटस्थे स्थलेऽविद्यत।
\vakya स च तेभ्यो दृष्टान्तकथाभिः प्रचुरशिक्षाम् अददात्, विशेषतः स्वशिक्षायां तान् अब्रवीत्,
\vakya श्रूयतां, पश्य वप्ता बीजानि वप्तुं निर्गतः।
\vakya तस्य वपनकाले तु कतिपयानि बीजानि पथपार्श्वेऽपतन् तत आकाशस्था विहङ्गमा आगत्य तान्यभक्षयन्।
\vakya अपराणि तु पाषाणमये स्थलेऽपतन् यत्र तत्कृते प्रभूतमृत्तिका नासीत्।
\vakya मृत्तिकाया अगभीरत्वात् तानि क्षिप्रं प्रारोहन् उदिते तु सूर्येऽदह्यन्त मूलहीनत्वाच्चाशोष्यन्त।
\vakya अपराणि च कण्टकानां मध्येऽपतन् तैः कण्टकैश्च वृद्ध्वावपीडितानि फलं नाफलन्।
\vakya अपराणि तूत्तमायां भूम्यां पतित्वा प्ररोहि वर्धिष्णु च फलम् अफलन्, कानिचित् त्रिंशद्गुणं, कानिचित् षष्टिगुणं, कानिचिच्छतगुणञ्च।
\vakya स च तान् जगाद, शृणोतु यस्य श्रोतुं श्रोत्रे स्तः।
\stitle{बीजवापकस्य दृष्टान्तस्तस्य तात्पर्यं।}
\vakya यदा तु स निर्जने स्थान आसीत्, तदा द्वादशभिः सह तस्य सङ्गिनस्तं तस्य दृष्टान्तस्य तात्पर्यं पप्रच्छुः।
\vakya स च तानब्रवीत्, ईश्वरराज्यस्य गूढविषयस्य ज्ञानं युष्मभ्यं दत्तं बहिःस्थानान्त्वमीनां कृते सर्वं तथोपमारूपं जायते यथा
\vakya ते पश्यन्तो द्रक्ष्यन्ति न त्वनुभविष्यन्ति शृण्वन्तश्च श्रोष्यन्ति न तु भोत्स्यन्ते, अन्यथा परावृत्तानां तेषां पापक्षमा सम्भविष्यति।
\vakya पुनः स तानाह, किं न जानीथ दृष्टान्तकथामिमां कथञ्च सर्वा दृष्टान्तकथा युष्माभि र्ज्ञातव्याः?
\vakya स वप्ता वाक्यं वपति।
\vakya पथपार्श्वस्थाश्च ते यत्र वाक्यमुप्यते श्रुते तु तस्मिन् शैतान आगत्य तेषां हृदयेषूप्तं वाक्यम् अपहरति।
\vakya पुनः पाषाणमये स्थले लब्धबीजास्ते ये वाक्यं श्रुत्वैव तत्क्षणं सानन्दं गृह्णन्ति
\vakya तेषां त्वन्तर्मूलं नास्ति ततश्च तेऽल्पकालस्थायिनो भवन्ति ततः परं वाक्यहेतुना क्लेशं उपद्रवे वोत्पन्ने त सद्यः स्खलन्ति।
\vakya कण्टकानाञ्च मध्ये लब्धबीजास्ते ये वाक्यं शृण्वन्ति
\vakya ततः परं संसारसम्बन्धीयाश्चिन्ता धनस्य माया चान्यविषयाणामभिलाषाश्च प्रविश्य वाक्यमवपीडयन्ति ततस्तन्निष्फलं जायते।
\vakya उत्तमायान्तु भूम्यां लब्धबीजास्ते ये वाक्यं शृण्वन्ति गृह्णन्ति फलं फलन्ति च, केचित् त्रिंशद्गुणं, केचित् षष्टिगुणं, केचिच्च शतगुणं।
\stitle{प्रदीपस्य दृष्टान्तः।}
\vakya अपि च स तान् जगाद, दीपो यद् द्रोणस्याधः किंवा खट्वाया अधः स्थाप्येत तदर्थं किमायाति? स यद् दीपाधारोपरि निधीयेत किं न तदर्थम् आयाति?
\vakya वास्तवं नास्ति किमपि गूढं यन्न प्रत्यक्षं भविष्यति। तच्च येन सप्रकाशं स्थानं गच्छेत् तदर्थमेव प्रच्छन्नीभूतं।
\vakya शृणोतु यस्य श्रोतुं श्रोत्रे स्तः।
\vakya पुनः स तान् जगाद, युष्माभिः किं श्रूयते तद् आलोच्यतां। येन परिमाणेन यूयं मिमीध्वे तेनैव युष्मदर्थं मायिष्यते, शृण्वद्भ्यश्च युष्मभ्यम् अधिकं दायिष्यते।
\vakya यतो यस्यास्ते तस्मै दायिष्यते यस्य तु नास्ते तस्य यदस्ति तदपि तस्माद् अपहारिष्यते।
\stitle{बीजवर्धनस्य दृष्टान्तः।}
\vakya पुनः स बभाषे, ईश्वरराज्यं तथा यथा कश्चिन्नरो भूमौ बीजानि वपति
\vakya ततः प्रभृति यावत् स प्रतिरात्रं प्रतिदिनञ्च निद्रात्युत्तिष्ठति च तावत् तदलक्षितरूपं बीजानि प्ररोहन्ति वर्धन्ते च,
\vakya यतो भूमिः स्वयमेव फलन्ती प्रथमं तृणं ततः परं मञ्जरीं पश्चाच्च मञ्जर्यां सिद्धं शस्यम् उत्पादयति।
\vakya परिणते तु फले स तूर्णं लवित्रं प्रयुंक्ते यतः शस्यकर्तनकाल उपस्थितः।
\stitle{सर्षपस्य दृष्टान्तः।}
\vakya पुनः स बभाषे, वयम् ईश्वरराज्यम् केन सदृशं वदिष्यमः? कयोपमया वा तद् उपमास्यामहे?
\vakya तत् सर्षपबीजेन सदृशं, यतो भूमौ वपनकाले तत् सर्वबीजेभ्यः क्षुद्रतरम्,
\vakya उप्तन्तु प्ररुह्य सर्वशाकेभ्यो महत्तरं जायते महतीः शाखाश्च प्रसारयति, तस्य छायायाञ्चाकाशीयपक्षिणो निवस्तुं शक्नुवन्ति।
\vakya ईदृशैरेव बहुभि र्दृष्टान्तैः स तैषां श्रवणशक्त्यनुरूपं तेभ्यो वाचमकथयत् दृष्टान्तन्तु विना तेभ्यः किमपि नाकथयत्।
\vakya निर्जनन्तु स्वशिष्येभ्यः सर्वं व्याख्यत्।
\stitle{समुद्रस्य निस्तरङ्गकरणम्।}
\vakya तस्मिन् दिने सन्ध्यायां जातायां स तान् जगाद, पारं गम्यतां।
\vakya अतस्ते जनौघं त्यक्त्वा तथाभूतं तं नौकायाम् अग्रहिषुः। तत्सह त्वपराः काश्चिन्नौका आसन्।
\vakya तदा महावात्या जाता तरङ्गाश्च नावमाहनन् अनेन सा तोयेनापूर्यत।
\vakya स तु नावः पश्चाद्भाग उपविश्योपधानमालम्ब्यानिद्रात्। अतस्ते तं प्राबोधयन् व्याहरंश्च भो गुरो वयं नश्यामोऽत्र भवान् किं निश्चिन्तः?
\vakya तदा स जागरितो वायुं तर्जयामास समुद्रञ्च जगाद मौनं कुरु तूष्णीम्भव। अनेन वायुः शशाम सम्पूर्णं निर्वातो बभूव च।
\vakya स च तान् जगाद, किमर्थमीदृशं भीरवः स्थ? यूयं विश्वासहीनाः कथमेतत्?
\vakya तदा ते महाभयग्रस्ता मिथो जगदुः, को न्वयं यद् वायुः समुद्रस्चाप्याज्ञा गृह्णीतः\eoc