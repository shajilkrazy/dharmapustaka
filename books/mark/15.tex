\adhyAya
\stitle{देशाध्यक्षस्य समक्षं यीशोर्विचारः।}
\vakya जाते पुनः प्रातःकाले प्राचीनैः शास्त्राध्यापकैश्च सार्धं मुख्ययाजकाः कृत्स्ना महासभा च तूर्णं मन्त्रणां चक्रु र्बद्ध्वा च यीशुमपनीय पीलाते समर्पयामासुः।
\vakya पीलातस्तदा तं पप्रच्छ, त्वं किं यिहूदियानां राजा? स तं प्रत्यवादीत्, भवान् व्याहरति।
\vakya मुख्ययाजकाश्च बहुशस्तस्याभियोगमकुर्वन्।
\vakya पीलातस्तदा पुनस्तं पप्रच्छ, त्वं किं किमपि न प्रतिभाषसे? पश्य तव विरुद्धममी साक्ष्यवचांसि कति वदन्ति।
\vakya यीशुस्तु ततः प्रभृति किमपि न प्रत्यभाषत, पीलातश्च तेनाश्चर्यं मेने।
\vakya तस्मिन् पर्वणि तु प्रतिवर्षं स तेषां कृते तै र्याचितं नरमेकं कारागुप्तममोचयत्।
\vakya तदा च यैरुपप्लवकारिभिरुपप्लवे नरहत्याकारि तैः सार्धं बारब्बा इतिनामको नर एकः काराबद्ध आसीत्।
\vakya अतः स तान् प्रति नित्यं यादृशमकरोत् जननिवह उच्चैःस्वरेण क्रोशंस्तादृशमनुग्रहं याचितुमारेभे।
\vakya पीलातस्तदा तान् प्रत्यवादीत्, युष्मदर्थमहं यद् यिहूदीयानां राजानं मोचयेयं तत् किं वाञ्छथ?
\vakya यतस्तेनाज्ञायत स यन्मात्सर्यान्मुख्ययाजकैः समर्पितः।
\vakya यथा तु वरं बारब्बास्तेन मोच्येत तदर्थं मुख्ययाजका जननिवहं प्राचोदयन्।
\vakya पीलातस्तदा प्रतिभाषमाणः पुनस्तानब्रवीत्, यूयं यं यिहूदीयानां राजानं वदथ तं प्रति तर्हि मया कर्तव्यं किमिच्छथ?
\vakya ते पुनः क्रोशन्त ऊचुः, स क्रुशमारोप्यतां भवता।
\vakya पीलातस्तान् जगाद, किं नु तेनापराद्धम्? ते त्वधिकं क्रशन्तोऽवदन् स क्रुशमारोप्यतां भवता।
\vakya पीलातस्तदा जननिवहस्य प्रीतिजनकं कर्म कर्तुमभिरोच्य तेषामिच्छातो बारब्बां मोचयामास, यीशुन्तु कशाभिः प्रहार्य क्रुशारोपणार्थं तेषु समर्पयामास।
\stitle{यीशोः क्रुशारोपणं।}
\vakya सैनिकनरास्तदा तं हर्म्यस्यार्थतो राजभवनस्याभ्यन्तरं नीत्वा सैन्यदलं कृत्स्नमेकत्राहूय च
\vakya कृष्णलोहितं वस्त्रं परिधापयामासुः कण्टकैश्च स्रजं निर्माय तस्य शिरसि निदधुस्तमित्थमभिवादयितुञ्चारेभिरे,
\vakya यिहूदीयानां राजन् प्रणामः।
\vakya नलेन तस्य शिरश्चाताडयन् तस्य वपुषि न्यष्ठीवन् जानुपातं कुर्वन्तश्च तं प्राणमन्,
\vakya इत्थं तमुपहस्य तत् कृष्णलोहितवर्णं वस्त्रं मोचयित्वा तस्य स्ववासांसि परिधाप्य च क्रुशमारोपयितुं तं बहि र्निन्युः।
\vakya तेन मार्गेण तदा सिकन्दर-रूफयोः पिता शिमोननामा नर एकः कुरीणीयो ग्रामत आगच्छत्, तमेव ते दध्रुः क्रुशं तस्य वोढुं वेतनं विना।
\vakya इत्थञ्च तं गल्गथानामकं स्थलं निन्युः। नाम्नोऽस्यार्थः कपालस्थलमिति।
\vakya तत्र ते पानीयार्थं तस्मै गन्धरसमिश्रितं द्राक्षारसं दातुमुपचक्रमिरे स तु तं न जग्राह।
\vakya अनन्तरं ते तं क्रुशम् आरोप्य तस्य वासांसि मिथो विभजनमानाः केन किं लभ्यं तज्ज्ञातुं गुटिकापातमकार्षुः।
\vakya प्रथमे प्रहरे स तैः क्रुशमारोपयाञ्चक्रे।
\vakya तस्येदमभियोगपत्रञ्च लिखितमासीत्, यिहूदीयानां राजेति।
\vakya तेन सार्धञ्च तै र्दस्यू द्वौ क्रुशे आरोप्येताम्, एकतरस्तस्य दक्षिणे वामे चान्यतरः। 
\vakya इत्थं शास्त्रीयोक्तिरियं सिद्धिं गता,
\begin{poem}
\startwithline “दुष्क्रियाकारिणां मध्य एकः स गणितोऽभवत्।”
\end{poem}
\vakya अनन्तरं ये मनुष्या मार्गेण तेनाव्रजंस्ते शिरांसि चालयन्तस्तमित्थमनिन्दन्, यथा, हा मन्दिरभञ्जक दिनत्रये च तन्निर्मातरात्मानं तारय क्रुशादवरोह च।
\vakya शास्त्राध्यपकैः सार्धं मुख्ययाजका अपीदृशमेव मिथो रहस्यं कुर्वन्तोऽवदन्,
\vakya सोऽपरानतारयत्, आत्मानं तारयितुं न शक्नोति।
\vakya इस्रायेलस्य राजा ख्रीष्ट इदानीं क्रुशादवरोहतु, तथा कृते वयं दृष्ट्वा विश्वसिष्यामः। तेन सार्धं क्रुशारोपितौ तौ द्वौ नरावपि तमपावदताम्।
\stitle{यीशो र्मृत्युः।}
\vakya आमध्याह्नात्तु तृतीयप्रहरं यावत् कृत्स्ने भूतलेऽन्धकारोऽभूत्।
\vakya तृतीये प्रहरे च यीशुरुत्क्रोशन्नुच्चरवेण बभाषे, एलोही, एलोही, लाम्मा शबक्तानीति, अस्यार्थोऽयम्,
\begin{poem}
\startwithline “हे मदीश मदीश त्वं मां परित्यक्तवान् कुतः।”
\end{poem}
\vakya तच्छ्रुत्वा तत्र स्थितानां नराः केचिदवदन्, पश्यासावेलियमाह्वयति।
\vakya एकश्च द्रुत्वा स्पञ्जमम्लरसेन पूरयित्वा नलाग्रे बद्ध्वा च तमपीप्यत् प्राह च, निवर्तध्वम्, अमुमवरोहयितुम् एलिय आगच्छति न वेत्यस्माभि र्दृश्यताम्।
\vakya ततः परं यीशुरुच्चरवमुदीर्य प्राणांस्तत्याज। तदा मन्दिरस्य तिरस्करिण्यग्रतोऽधो यावद् विदीर्णा द्विखण्डा बभूव।
\vakya स चेदमुत्क्रोशन् प्राणांस्तत्याजेति दृष्ट्वा तत्सम्मुखं संस्थितः शतपतिरवादीत्,
\vakya सत्यं नरोऽयमीश्वरस्य पुत्र आसीत्।
\vakya अपि च योषितः काश्चिद् दूरादवलोकमानास्तत्राविद्यन्त।
\vakya तासां मध्ये मग्दलीनी मरियम् कनिष्ठयाकोबस्य योषेश्च माता मरियम् शालोमी चासन्, स यदा गालील आसीत् तदाप्येतास्तमनुगतवत्यः परिचरितवत्यश्च। तेन सार्धं यिरूशालेममागता बह्वोऽपरा अप्युपस्थिता आसन्।
\stitle{यीशोः समाधिः।}
\vakya तदा सन्ध्या सम्भूता। तद्दिनञ्च सज्जीकरणदिनमर्थतो विश्रामवारस्य पूर्वदिनम्,
\vakya अतोऽरिमाथियानिवासी योषेफ इति नामको यः सुशीले मन्त्रीश्वरराज्यं प्रत्यैक्षत स उपस्थाय पुरुषत्वमवलम्ब्य पीलातस्य गृहं प्रविवेश तञ्च यीशो र्देहं ययाचे।
\vakya तस्याशुमरणे पीलात आश्चर्यं मत्वा तं शतपतिमाहूय पप्रच्छ स दीर्घकालं मृतो न वेति।
\vakya शतपते र्मुखात् ज्ञातव्यमवगत्य स विनामूल्येन योषेफाय देहं ददौ।
\vakya स च सूक्ष्मं वस्त्रं क्रीत्वा तमवरोह्य तेन वस्त्रेण वेष्टयामास शैलमध्ये तक्षिते शवागारे निदधे च, प्रस्तरमेकं लोठयित्वा च तच्छवागारमुखं रुरोध।
\vakya स तु कुत्र निधीयते तन्मग्दलीनी मरियम् योषे र्माता चान्यतरा मरियम् ददृशतुः\eoc