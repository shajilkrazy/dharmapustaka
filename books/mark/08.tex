\adhyAya
\stitle{चतुःसहस्रलोकेभ्य आश्चर्यरूपेण भोजनदानम्।}
\vakya तेषु दिनेष्वेकदातीव महति जनारण्ये सति तेषाञ्च खाद्याभावे जाते यीशुः शिष्यान् स्वसमीपमाहूय जगाद,
\vakya अनुकम्पेऽहममून् मानवान्, यतस्त्रीण्येव दिनानि ते मत्समीपम् अवस्थिताः खाद्यञ्च किमपि तेषां नास्ति।
\vakya अनशनास्तु ते चेन्मया स्वगेहेषु निसृज्यन्ते तर्हि पथ्यवसत्स्यन्ति, यतस्तेषां केचित् सुदूरादागताः।
\vakya तस्य शिष्यास्तं प्रतिजगदुः, अत्र निर्जने स्थाने कः कस्मात् पूपान् लब्ध्वामून् तर्पयितुं शक्नुयात्?
\vakya स तान् पप्रच्छ, युष्माकं कति पूपाः सन्ति? तेऽब्रुवन्, सप्तैव।
\vakya ततः स जननिवहं भूमावुपवेष्टुमादिदेश, तान् सप्तपूपांश्चादाय भङ्क्त्वा च परिवेषणार्थं स्वशिष्येभ्योऽददात्, ते च जनौघाय पर्यवेषयन्।
\vakya स्वल्पाः क्षुद्रा मत्स्या अपि तेषामासन्, स आशिषं वदित्वा तेषामपि परिवेषणमाज्ञापयामास।
\vakya इत्थं ते बुभुजिरे ततृपुश्च, अपशिष्टस्य भग्नांशैः पूर्णान् सप्त पेटकांश्चाददिरे।
\vakya अनन्तरं स तान् विससर्ज।
\vakya तत्क्षणञ्च स स्वशिष्यैः सह नावं प्रविश्य दाल्मानूथेतिनामके प्रदेश उपतस्थे।
\stitle{यीशो र्नानाविधोपदेशाः कर्माणि च।}
\vakya फरीशिनस्तु बहिरागत्य तेन सह वादानुवादौ कर्तुमारेभिरे परीक्षमाणाश्च तं गगनात् किमप्यभिज्ञानं ययाचिरे।
\vakya स त्वात्मना निश्वस्य जगाद, एतत्कालस्य मानवाः किमर्थमभिज्ञानं याचन्ते? युष्मानहं सत्यं ब्रवीमि, वंशायास्मै किमप्यभिज्ञानं न दायिष्यते।
\vakya इत्युक्त्वा स तांस्त्यक्त्वा पुन र्नावं प्रविश्यापरपारं प्रतस्थे।
\vakya तैस्तु विस्मृत्य पूपा नानीता नावि च तैः साकं केवलमेकः पूप आसीत्।
\vakya स च तानादिशत्, सतर्का भवत, फरीशिनां किण्वतो हेरोदस्य च किण्वतः सावहिता भवत।
\vakya अनेन ते मिथो विचारयन्तोऽवदन् अस्माकं पूपा न सन्ति तदेतत् कथयति।
\vakya यीशुस्तु ज्ञात्वा तान् जगाद, युष्माकं पूपा न सन्तीति किं विचारयथ? इदानीमपि किं न जानीथ न वा बुध्यध्वे? अधुनापि किं जडहृदो यूयं?
\vakya सनेत्राः किं न पश्यथ? सकर्णाः किं न शृणुथ नापि स्मरथ?
\vakya पञ्चसहस्रेष्वहं यदा पञ्च पूपान् अभाङ्क्षं तदा भग्नांशैः पूर्णाः कति डल्लका युष्माभिरादत्ताः? ते तं वदन्ति द्वादशेति।
\vakya चतुःसहस्रेषु च यदा सप्त पूपानभाङ्क्षं, तदा भग्नांशैः पूर्णाः कति पेटका आदत्ताः? तेऽवदन्, सप्तेति।
\vakya ततः स तान् बभाषे, युष्माभि र्न बुध्यते कथमेतत्?
\stitle{अन्धाय नेत्रदानम्।}
\vakya बैत्सैदायान्तूपस्थिते तस्मिन् मानवा अन्धमेकमानीय प्रसादयन्तस्तं तस्य स्पर्शनं ययाचिरे।
\vakya ततः स तस्यान्धस्य हस्तं धृत्वा तं ग्रामाद् बहि र्निनाय तन्नेत्रयो र्निष्ठीव्य च तस्मिन् करावर्पयित्वा तं पप्रच्छ, त्वया किमपि दृश्यते किं?
\vakya अनेन स दृक्‌पातं कृत्वाब्रवीत्, मनुष्यान् निरीक्षे यतः पादपानिव तान् विहरतः पश्यामि।
\vakya तदा स पुनस्तन्नेत्रयोः करावर्पयित्वा तं दृक्‌पातं कारयामास। अनेन तस्यान्धत्वप्रतीकारे जाते स सर्वान् सुस्पष्टमपश्यत्।
\vakya स तु तं गेहाय विसृज्य जगाद, ग्राममपि मा प्रविश्य ग्रमस्थं वा कमपि मा ज्ञापय।
\stitle{यीशोः स्वमरणपुनरुत्थानेऽधि कथाकथनम्।}
\vakya ततः परं यीशुस्तच्छिष्याश्च कैसरियाफिलिप्या ग्रामान् जग्मुः। पथि च स शिष्यान् पप्रच्छ, मानवा मां कं ज्ञात्वा वदन्तीति।
\vakya ते प्रत्यूचुः, स्नापकं योहनम्, अपरे त्वेलियम् अपरे च भाववादिनामेकतममिति। स पुनस्तान् बभाषे,
\vakya यूयन्तु मां कं ज्ञात्वा वदथ? अनेन पित्रस्तं प्रत्युवाच, भवान् ख्रीष्ट इति।
\vakya ततः स तांस्तर्जयन्नब्रवीत्, मम कथां कमपि मा वदथ।
\vakya अनन्तरं स तानिदं शिक्षयितुमारेभे यथा, मनुष्यपुत्रस्यावश्यं यत् स बहुदुःखं भुज्जीत प्राचीनै र्मुख्ययाजकैः शास्त्राध्यपकैश्च निराक्रियेत घात्येत च त्रिभ्यो दिनेभ्यस्तु परं पुनरुत्तिष्ठेत्।
\vakya स च वाक्यमेतत् स्पष्टमभाषत। तदा पित्रस्तं निर्जनमादाय तर्जयितुमारेभे।
\vakya स तु परावृत्य स्वशिष्यान् दृष्ट्वा च पित्रं तर्जयन् जगाद, मत्तोऽपसर शैतान, यतो नैवेश्वरस्यापि तु मनुष्याणामेव विषयास्त्वया चिन्त्यन्ते।
\vakya ततः परं स स्वशिष्यैः सह जननिवहमपि स्वसमीपमाहूय तान् जगाद, यः कश्चिन्माम् अनुगन्तुमिच्छति स आत्मानं प्रत्याख्यातु स्वक्रुशमादाय च मामनुव्रजतु।
\vakya यतो यः स्वप्राणान् रिरक्षिषति स तान् हारयिष्यति। यस्तु मदर्थं सुसंवादार्थञ्च स्वप्राणान् हारयति स तान् रक्षिष्यति।
\vakya यतो मनुष्यः कृत्स्नं जगल्लब्ध्वा यदि स्वप्राणै र्वञ्च्यते तर्हि तस्य हितं वा किं भवति?
\vakya स्वप्राणानां वा किं निष्क्रयं मनुष्यो दास्यति?
\vakya यतो यः कश्चिद् व्यभिचारिषु पापिष्ठेषु चैतत्कालिकमानवेषु मत्तो मदीयवाक्येभ्यश्चापत्रपते, पवित्रै र्दूतै र्वृतो मनुष्यपुत्रो यदा स्वपितुः प्रतापेनागमिष्यति तदा सोऽपि तस्मादपत्रपिष्यते।
\vakya अपि च स तान् जगाद, युष्मानहं सत्यं ब्रवीमि, अत्र ये तिष्ठन्ति तेषां मध्ये केचिद् यावदीश्वरराज्यं प्रभावेनागतं न द्रक्ष्यन्ति तावन्मृत्योरास्वादं न लप्स्यन्ते\eoc