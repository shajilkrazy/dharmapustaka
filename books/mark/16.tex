\adhyAya
\stitle{यीशोः पुनरुत्थानं स्वर्गारोहणञ्च।}
\vakya व्यतीते तु विश्रामदिने मग्दलीनी मरियम् याकोबस्य माता मरियम् शालोमी च सुगन्धिद्रव्याणि चिक्रियु र्यतस्ता गत्वा तं संलेपयितुमैच्छन्।
\vakya ततः सप्ताहस्य प्रथमदिवसस्य प्रत्यषे ताः सूर्योदयात् परं शवागारस्यान्तिकमागच्छन्त्यो मिथोऽवदन्, शवागारस्य द्वारात् कोऽस्मदर्थं तं प्रस्तरं लोठयिष्यतीति।
\vakya तत् ऊर्ध्वदृष्टिं कृत्वा ताभिः स प्रस्तरो द्वाराल्लोठितो लक्षितो यतः सोऽतीव महानासीत्।
\vakya अनन्तरं ताः शवागारं प्रविश्य सितप्रावारपरिहितमेकं युवानं दक्षिणपार्श्व उपविष्टं ददृशुश्चुक्षुभिरे च।
\vakya स तु ता जगाद, मा क्षुभ्यत। यूयं क्रुशारोपितं नासरतीयं यीशुम् अन्विष्यथ।
\vakya पुनरुत्थितः स नात्र विद्यते। पश्यतेदं तत् स्थानं यत्र स शायितः।
\vakya प्रत्युत यूयं गत्वा तस्य शिष्यान् पित्रञ्च ब्रूत, स युष्मदग्रतो गालीलं याति, यूयं तेन यथोक्तास्तदनुरूपं तत्र तं द्रक्ष्यथेति।
\vakya ततस्ता निर्गत्य सत्वरं शवागारात् पलायाञ्चक्रिरे, कम्पविस्मयाभ्यामाक्रान्तास्तु कमपि किमपि नावदन् यतस्ता अबिभयुः।
\vakya सप्ताहस्य प्रथमे दिवसे प्रत्यूषे पुनरुत्थितो यीशुः प्रथमं मग्दलीन्यै मरियमे दर्शनं ददौ यस्याः स सप्त भूतान् निःसारितवान्।
\vakya सा च गत्वा शोचद्भ्यो रुदद्भ्यश्च तस्य सङ्गिभ्यः संवादं ददौ।
\vakya स तु जीवति तयादर्शि चेति श्रुत्वा ते न व्यश्वसिषुः।
\vakya ततः परं स रूपान्तरमादाय तेषां द्वयोः पदव्रजेन ग्रामं गच्छतोः प्रत्यक्षीबभूव।
\vakya तौ च गत्वान्येभ्यः संवादं ददतुः। तयो र्वाक्येष्यवपि ते न व्यश्वसिषुः।
\vakya शेष एकादशसु भोजनायोपविष्टेषु स तेषामेव प्रत्यक्षीबभूव, ये च तं पुनरुत्थितं दृष्टवन्तस्तेषां वाक्येषु तै र्विश्वासो नाकारीति हेतोः स तेषाम् अविश्वासं मनःकाठिन्यञ्च निनिन्द।
\vakya अपि च स तान् जगाद, यूयं कृत्स्नं जगद् गत्वा कृत्स्नसृष्टे र्ज्ञानार्थं सुसंवादं घोषयत।
\vakya यो विश्वस्य स्नापयिष्यते स तारयिष्यते, यस्तु न विश्वसिष्यति स विचारे दण्डपात्रं भविष्यति।
\vakya ये च विश्वसिष्यन्ति, अभिज्ञानानीमानि ताननुवर्तिष्यन्ते। मम नाम्ना ते भूतान् निःसारयिष्यन्ति। नूतना भाषा वदिष्यन्ति।
\vakya सर्पान् उपादास्यन्ते। पीतमपि तैः प्राणान्तकं पानीयं तान् नैव हिंसिष्यति। कृते तैः पीडितेषु हस्तार्पणे त आरोग्यं लप्स्यन्ते।
\vakya तैः संलपनात् परं प्रभुः स्वर्गमारोहित ईश्वरस्य दक्षिण उपविष्टश्च।
\vakya ते तु प्रस्थाय सर्वत्र सुसंवादमघोषयन् प्रभुश्च सहकर्माभूत्, अनुवर्तमानैरभिज्ञानैश्च वाक्यं सप्रमाणमकरोत्\eoc