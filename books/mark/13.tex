\adhyAya
\stitle{यिरूशालेमस्य विनाशः।}
\vakya स यदा धर्मधामतो निरगच्छत् तदा तच्छिष्णामेकस्तं जगाद, गुरो पश्यतु, कीदृशा प्रस्तराः कीदृशानि हर्म्याणि च।
\vakya यीशुस्तदा तं प्रतिजगाद, त्वमेतानि महान्ति हर्म्याणि किं पश्यसि? अत्र प्रस्तरोपरि प्रस्तर एकोऽप्यनिपातयितव्यो न विहायिष्यते।
\vakya ततः परं यदा स जैतूनपर्वते धर्मधाम्नः समक्षमासीन आसीत्, तदा पित्रो याकोबो योहन आन्द्रियश्च निर्जनं तस्मै निवेदयमासुः,
\vakya वक्तुमर्हत्यस्मान् भवान्, कदा सर्वं सम्भविष्यति? किं वाभिज्ञानं यदा सर्वं तत् सिद्धिमवाप्स्यति?
\vakya तदा यीशुः प्रतिभाषमाणस्तेभ्यः कथयितुमारेभे, सावधनास्तिष्ठत, युष्मान् कोऽपि मा प्रतारयतु,
\vakya यतो बहवो मन्नामध्वजनि आगत्य वदिष्यन्त्यहं स इति प्रतारयिष्यन्ति च बहून्।
\vakya कथाश्च युद्धानां युद्धस्य किंवदन्तीश्च श्रुत्वा मैवोद्विजध्वं, यत एतेन भवितव्यं, परिणामस्तु नापि तदा।
\vakya वस्तुतो जाति र्जाते र्विरुद्धं राज्यञ्च राज्यस्य विरुद्धमुत्थास्यति। भविष्यन्ति च स्थाने स्थाने भूकम्पाः। पुनः पुन र्दुर्भिक्षमुपप्लवश्चापि सम्भविष्यतः। उपक्रमो यातनानां सर्वमेतत्।
\vakya यूयन्तु स्वेषु सतर्का भवत, यतो मानवा युष्मान् विचारसभासु समर्पयिष्यन्ति यूयञ्च समाजेषु प्रहारिष्यध्वे मदर्थञ्च देशाध्यक्षाणां राज्ञाञ्च समक्षं स्थापयिष्यध्वे तेभ्यः साक्ष्यदानार्थम्।
\vakya प्रथमञ्च सर्वजातिषु सुसंवादो घोषयितव्यः।
\vakya मानवास्तु यदा युष्मान् समर्पयिष्यन्तो नेष्यन्ति तदा किं वक्तव्यमिति मालोचयत मा चिन्ताकुला भवत वा। तस्मिन् दण्डे तु युष्मभ्यं यद् दायिष्यते तदेव कथयत, यतो न यूयं वक्तारः, वक्ता तु पवित्र आत्मा।
\vakya भ्राता पुन र्भ्रातरं जनकश्च सुतं मृत्यवे समर्पयिष्यति, सन्तानाश्च पित्रोः प्रातिकूल्येनोत्थाय तौ घातयिष्यन्ति। मम नाम्नो हेतुना च यूयं सर्वै र्द्वेक्ष्यध्वे।
\vakya यस्त्वन्तं यावत् स्थिरः स्थास्यति स एव तरिष्यति।
\vakya अतो यूयं भाववादिना दानीयेलेन कथितं ध्वंसकारि घृण्यवस्त्वनुपयुक्ते स्थाने संस्थितं द्रक्ष्यथ -पाठको बुध्यतां- तदा ते पलाय्य गिरीनाश्रयन्तां ये यिहूदियायां विद्यन्ते,
\vakya यश्च गृहपृष्ठे विद्यते स गृहं मावरोहतु स्वगृहात् किमप्यादातुं वा तत्र मा प्रविशतु।
\vakya यश्च क्षेत्रे विद्यते स स्ववस्त्रमादातुं मा प्रत्यावर्ततां। भविष्यन्ति च दिनेषु तेषु गर्भधारिण्यः स्तन्यदायिन्यश्च सन्ताभाजनानि।
\vakya तथा च प्रार्थयध्वं युष्माकं पलायनं यथा शीतकाले न भवेत्।
\vakya यतस्तेषु दिनेषु क्लेशो यादृशः सम्भविष्यति न सम्भूतस्तादृश ईश्वरकृतसृष्टेरारम्भाद् अद्य यावत्, नापि सम्भविष्यति पुनः।
\vakya दिनानि तानि च प्रभुना चेन्न न्यून्यकारिष्यन्त मर्त्यः कोऽपि तर्हि नातरिष्यत्।
\vakya स तु यान् वरितवान् तेषां वरितजनानां कृते तानि दिनानि न्यूनीचकार।
\vakya तदा च मा विश्वसित यूयमुक्ता अपि केनचित् पश्य ख्रीष्टोऽत्र वामुत्र विद्यते।
\vakya यतो कूटख्रीष्टाः कूटभाववादिनश्चोत्थास्यन्ति प्रदर्शयिष्यन्ति च तेऽभिज्ञानान्यद्भुतलक्षणानि च तादृशानि यैः साध्ये सति ते वरितानपि मनुष्यान् पथभ्रष्टान् करिष्यन्ति।
\vakya यूयन्तु सावधानास्तिष्ठत। पश्यत युष्मानहं सर्वं पूर्वमेवोक्तवान्।
\vakya तेषु दिनेषु तस्मात् क्लेशात् परं सूर्यः सान्धकारो भविष्यति, चन्द्रश्च स्वज्योत्स्नां न प्रदास्यति,
\vakya नक्षत्राणि च नभसः पतिष्यन्ति गगनस्थबलानि च विचलिष्यन्ति।
\vakya मानवाश्च निरीक्षिष्यन्ते महापराक्रमप्रतापाभ्यां परीतं मेघरथेनागच्छन्तं मनुष्यपुत्रम्।
\vakya तदा च स स्वदूतान् प्रहित्य मेदिनीप्रान्तमारभ्य गगनप्रान्तं यावच्चतुर्भ्यो वायुभ्यो मनुष्यान् स्ववरितान् सङ्ग्रहीष्यति।
\vakya उडुम्बरवृक्षाद् दृष्टान्तं शिक्षितुमर्हथ। स्पष्टं यदा तस्य शाखा जायते कोमला पत्राणि च प्रकाशन्ते जानीथ यूयमासन्नस्तदा ग्रीष्मकाल इति।
\vakya तथैव यूयं यदा द्रक्षथ यत् सर्वमेतत् सम्भवति ज्ञास्यथ तदा स समीपस्थो द्वार उपस्थितश्चेति।
\vakya युष्मानहं सत्यं ब्रवीमि, यावत् सर्वमेतन्न सम्भवति तावन्न व्यत्येष्यन्ति मानवा एतत्कालिकाः।
\vakya द्यावापृथिव्यावत्येष्यतः, मम वाक्यानि तु नैवात्येष्यन्ति।
\vakya दिनस्य तस्य तु दण्डस्य च तस्य तत्त्वं केनापि न ज्ञायते, नापि स्वर्गस्थदूतै र्नापि पुत्रेण, केवलः पिता तज्जानाति।
\vakya यूयं सावधानास्तिष्ठत, जागृत प्रार्थयध्वञ्च यतो न जानीथ समयः कदोपतिष्ठते।
\vakya युष्मत्प्रभु र्देशान्तरगतेन नरेण सदृशो यः स्वगृहं त्यक्त्वा स्वदासेषु विषयमेकैकस्मिंश्च स्वं स्वं कर्म समर्पितवान् प्रतिहारिणञ्च जागरणमादिष्टवान्।
\vakya अतो जागृत यतो यूयं न जानीथ गृहस्वामी कदायाति, सायम् अर्धरात्रे वा कुक्कुटरावकाले वा प्रात र्वा।
\vakya अकस्मादागत्य स यथा युष्मान् सुप्तान् नासादयेत् तत् चिन्तयत।
\vakya युष्मांस्तु यद् ब्रवीमि तत् सर्वानेव ब्रवीमि, जागृतेति\eoc