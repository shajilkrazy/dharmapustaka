\adhyAya
\stitle{यीशो र्यीरूशालेमे प्रवेशः।}
\vakya अनन्तरं तेषु यिरूशालेमस्य समीपमागत्य जैतुनाख्यगिरौ स्थितयो र्बैत्फगी बैथनिया चेत्यभिधग्रामयोरुपस्थितेषु यीशुः शिष्यौ द्वौ प्रेषयन्निदमादिदेश,
\vakya युष्मत्सम्मुखस्थममुं ग्रामं गच्छतं तन्न प्रविश्यैव बद्धमेकं गर्दभशावकमासादयिष्यथो यत्पृष्ठं मनुष्यः कोऽपि नारूढवान्, तमेव मुक्त्वानयतं।
\vakya यदि च कोऽपि युवां पृच्छेत्, किमर्थमिदं कुरुथ इति, तर्हि वदतम्, अनेन प्रभोः प्रयोजनमास्त इति, ततः स तत्क्षणं तमत्र विस्रक्ष्यति।
\vakya तदा तौ गत्वा बहिः क्षुद्ररथ्यायाः पार्श्वे गृहद्वारे बद्धं तं गर्दभशावकमासाद्य मोक्तुं प्रावर्ततां।
\vakya तत्र तिष्ठन्तः केचिन्नरास्तु ताववदन्, किमर्थं तं शावकं मुञ्चथः?
\vakya तदा यीशु र्यदादिष्टवांस्तथैव ताभ्यां कथिते तौ तैरनुज्ञातौ।
\vakya शावकं तञ्च यीशोः समीपमानिन्यतुस्तत्पृष्ठे च स्ववासांसि निदधाते। ततः स तमारुरोह।
\vakya बहवश्च मार्गे स्ववसनानि विस्तारयामासुरन्ये च पादपानां पल्लवांश्छित्त्वा पथि व्यस्तारयन्
\vakya अग्रपश्चाद्गामिनो नराश्चोच्चरवेणाब्रुवन्, जय, प्रभो र्नाम्ना य आयाति स धन्यो भूयात्,
\vakya प्रभो र्नाम्नास्मत्पितु र्दायूदस्य यद् राज्यमायाति तद् धन्यं भूयात्, ऊर्ध्वलोके जयध्वनि र्भवतु।
\vakya अनन्तरं यीशु र्यिरूशालेमं धर्मधाम च प्रविष्टः, परितः सर्वं निरीक्ष्य च दिनावसानहेतो र्द्वादशशिष्यैः सार्धं बैथनियां प्रतस्थे।
\stitle{उडुम्बरतरौ शापः।}
\vakya परप्रातःकाले तेषु बैथनियातः प्रस्थितेषु सोऽक्षुध्यत् दूराच्च सपत्रमेकम् उडुम्बरवृक्षं निरीक्ष्य तत्र किमपि फलं लप्स्यते न वेति द्रष्टुं तदन्तिकं जगाम।
\vakya समीपमुपस्थाय तु पत्रेभ्योऽन्यत् किमपि नासादयामास। यतस्तदोडुम्बरफलानां समयो नासीत्।
\vakya अतो यीशुस्तं सम्बोध्यावादीत्, इतः परं कदापि कोऽपि त्वदुत्पन्नं फलं न भक्षयतु। तस्य शिष्याश्च तद् वचनमश्रौषुः।
\stitle{मन्दिराद् वणिजां बहिष्कृतिः।}
\vakya ततः परं ते यिरूशालेम उपतस्थिरे, यीशुश्च धर्मधाम प्रविश्य तत्र धर्मधाम्नि क्रयविक्रयकारणो बहिष्कर्तुमारेभे,
\vakya वणिजां मुद्रासनानि कपोतविक्रेतॄणामासनानि च न्युब्जयामास धर्मधाम्नो मध्येन च कस्यापि पात्रवहनं नानुमेने,
\vakya शिक्षयंश्च तान् जगाद, किं न लिखितमास्ते,
\begin{poem}
\startwithline “ख्यास्यते सर्वजातीनां प्रार्थनाधाम मद्गृहम्।”
\end{poem}
@V\eoc