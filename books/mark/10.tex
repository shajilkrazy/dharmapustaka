\adhyAya
\stitle{स्त्रीत्यागमधि शिक्षा।}
\vakya तस्मात् स्थानात् प्रस्थाय स यिहूदियाया यर्दनपारस्थप्रदेशस्य च सीमायामुपतस्थे पुनश्च जननिवहास्तदन्तिकं समाजग्मुः। स च स्वरीत्यनुसारेण पुनस्तानशिक्षयत्।
\vakya तदा फरीशिन उपागत्य परीक्षमाणास्तं पप्रच्छुः, भार्यात्यागः किं भर्तु र्विधेयः?
\vakya स तान् प्रतिबभाषे, मोशिर्युष्मान् किमादिष्टवान्?
\vakya तेऽवदन् त्यागपत्रं लिखित्वा तां विस्रष्टुं मोशिरनुजज्ञे।
\vakya तदा यीशुस्तान् प्रत्यवादीत्, युष्माकं हृत्काठिन्यमवेक्ष्य मोशि र्युष्मदर्थमिममादेशं लिखितवान्।
\vakya सृष्टेरादौ त्वीश्वरो नरं नारीञ्च तौ निर्मितवान्।
\vakya “एतद्धेतो र्मनुष्यः पितरं मातरञ्च परित्यज्य स्वजायायामासंक्ष्यते तावुभौ चैकाङ्गीभविष्यतः।”
\vakya अनेन तौ पुन र्न द्वौ स्तः, तावेकाङ्गीभूतौ।
\vakya अतो यदीश्वरेण संयोजितं तन्मनुष्येण मा वियोज्यतां।
\vakya ततः परं गृहे शिष्याः पुनस्तं तां कथां पप्रच्छुः।
\vakya तदा स तान् जगाद, यः कश्चित् स्वभार्यां त्यक्त्वापरामुद्वहति स तस्या विरुद्धं व्यभिचारं करोति,
\vakya या च जाया स्वपतिं त्यक्त्वान्येनोदुह्यते सा व्यभिचारं करोति।
\stitle{शिशुविषयिणी शिक्षा।}
\vakya ततः परं शिशवः केचित् तस्यान्तिकमानीयन्त यत् स तान् स्पृशेत्। शिष्यास्तु तदानेतॄन् अभर्त्सयन्।
\vakya तद् दृष्ट्वा यीशुः क्रुद्ध्वा तान् जगाद, मत्समीपमागमिष्यतः शिशून् अनुमन्यध्वं मा वारयत यतः स्वर्गराज्यमीदृशानामेव।
\vakya युष्मानहं सत्यं ब्रवीमि यः कश्चिन्न शिशुरिव स्वर्गराज्यं गृह्णाति स न कथञ्चन तत् प्रवेक्ष्यति।
\vakya ततः परं स तान् शिशून् क्रोडे गृहीत्वा तेषु हस्तावर्पयित्वा चाशिषं बभाषे।
\stitle{धर्माचरणविषयिणी शिक्षा।}
\vakya तस्मिंस्तु गृहाद् बहिः पन्थानं गते कश्चिद् द्रुतगमनेनागत्य जानुनी पातयित्वा च तं पप्रच्छ, भो सद्गुरो अनन्तजीवनस्याधिकारी यद् भवेयं तदर्थं मया किं कर्तव्यं?
\vakya यीशुस्तं जगाद, कथं मां सन्तं वदसि? सदेक एवेश्वरो नापरः कोऽपि।
\vakya त्वमाज्ञा जानासि, व्यभिचारं मा कुरु, नरहत्यां मा कुरु, चौर्यं मा कुरु, मृषासाक्ष्यं मा देहि, वञ्चनां मा कुरु, स्वपितरं स्वमातरञ्च सम्मन्यस्वेति।
\vakya स तं प्रत्युवाच, गुरो पालितं मया सर्वमेतदाबाल्यात्।
\vakya तदा यीशुस्तं समालोक्य तस्मिन् स्नेहमनुभूय च तं जगाद, त्वं गुणैकहीनोऽसि, गत्वा सर्वं तव यद्यदस्ति तद् विक्रयी दरिद्रेभ्यो देहि, तथा कृते स्वर्गे तव धनं स्थास्यति, तत परमेहि क्रुशमादाय मामनुगच्छ।
\vakya वचनेऽस्मिन् विषादं गत्वा स शोचन् प्रतस्थे यतस्तस्य वसूनि प्रचुराण्यासन्।
\vakya तदा यीशुः परितो दृक्‌पातं कृत्वा स्वशिष्यान् अब्रवीत्, वसूनां स्वामिनः कियतायासेनेश्वरराज्यं प्रवेक्ष्यन्ति?
\vakya तस्यैतेषु वाक्येषु शिष्याश्चमत्कारं मेनिरे। पुनश्च यीशुः प्रतिभाषमाणस्तान् जगाद, वत्साः, वित्तेषु विश्वसतामीश्वरराज्यप्रवेशः कियदायासयुक्तः।
\vakya धनवत ईश्वरराज्यप्रवेशात् सूचीच्छिद्रेणोष्ट्रगमनं सुसाध्यं।
\vakya अनेन तेऽत्यन्तमाश्चर्यं मन्यमाना मिथोऽवदन्, कस्तर्हि तरितुं शक्नुयात्?
\vakya यीशुस्तदा समालोक्य तानब्रवीत्, तदसाध्यं मनुष्याणां न त्वीश्वरस्य। सर्वं हि साध्यमीश्वरस्य।
\vakya तदा पित्रस्तमिदं वक्तुं प्रावर्तत, पश्यतु वयं सर्वं त्यक्त्वा भवन्तमनुव्रजितवन्तः।
\vakya यीशुः प्रतिबभाषे, युष्मानहं सत्यं ब्रवीमि, मदर्थं सुसंवादार्थञ्च गृहं भ्रातॄन् वा भगिनी र्वा पितरं वा मातरं वा भार्यां वा सन्तानान् वा क्षेत्राणि वा त्यक्त्वा
\vakya य इदानीमस्मिन् युग उपद्रवैः सार्धं शतगुणं गृह्णाति भ्रातॄन् भगिनी र्मातॄ सन्तानान् क्षेत्राणि च भावियुगेऽनन्तं जीवनञ्च न लप्स्यते तादृशः कोऽपि नास्ति।
\vakya बहवः प्रथमास्त्वन्या भविष्यन्त्याश्च प्रथमाः।
\stitle{यीशोः स्वमरणमधि तृतीयवारं कथाकथनम्।}
\vakya तदा ते यिरूशालेमं गच्छन्तः वर्त्मन्यासन् यीशुस्तेषामग्रेऽव्रजत् ते च विस्मिता आसन् तमनुगच्छन्तोऽबिभयुश्च। ततः स पुनस्तान् द्वादश शिष्यान् स्वसमीपं गृहीत्वा स्वं प्रति यद्यत् सम्भविष्यति तत् तेभ्यः कथयितुमारेभे, यथा, पश्यत वयं यिरूशालेमं गच्छामो मनुष्यपुत्रश्च मुख्ययाजकेषु शास्त्राध्यापकेषु च समर्पयिष्यते।
\vakya ते च विचारेण तस्य प्राणदण्डमादेक्ष्यन्ति परजातीयानां करेषु तं समर्पयिष्यन्ति च,
\vakya ते पुनस्तमुपहसिष्यन्ति, कशाभिः प्रहरिष्यन्ति, तद्वपुषि निष्ठीविष्यन्ति, तं घातयिष्यन्ति च। तृतीये दिने तु स पुनरुत्थास्यति।
\stitle{ईश्वरराज्ये महान् कः, एतदधि शिक्षा।}
\vakya तदा सिबदियसुतौ याकोबयोहनौ तस्यान्तिकमागत्य जगदतुः, गुरो आवां यद् याचिष्यावहे भवान् अस्मदर्थं तत् करोत्वेतद् वाञ्छावः।
\vakya स तौ पप्रच्छ युष्मदर्थं मया कर्तव्यं किं वाञ्छथः?
\vakya तौ तं जगदतुः, भवतः प्रताप आवयोरेकतरो भवतो दक्षिणेऽन्यतरो भवतो वामे समुपविशत्विमं वरमावाभ्यां दातुमर्हति।
\vakya यीशुस्तु तौ प्रत्यवादीत्, युवां यद् याचेथे तन्न जानीथः। येन पानपात्रेणाहं पास्यामि तेन किं युवां पातुं शक्नुथः? येन स्नापनेनाहं स्नास्ये तेन स्नातुं किं युवाभ्यां शक्यं?
\vakya तौ तमूचतुः, शक्यं। यीशुस्तावब्रवीत्, येनाहं पास्यामि तेन पानपात्रेण युवां पास्यथः, येनाहं स्नास्ये तेन स्नापनेन स्नास्येथे च,
\vakya परन्तु मम दक्षिणे वामे वोपवेशनाधिकारस्य दानं मया न विधेयं स येषां कृते निरूपितस्तेभ्य एव दातव्यः।
\vakya श्रुत्वैतदन्ये दश शिष्या याकोबे योहने च रौक्ष्यमापन्नाः।
\vakya यीशुस्तु तान् स्वसमीपमाहूयोवाच, यूयं जानीथ, ये परजातीयानां शास्तारः प्रतिभान्ति ते तेषामुपरि तीक्ष्णप्रभुत्वं कुर्वते, महान्तश्च तेषां तीक्ष्णकर्तृत्वं कुर्वते।
\vakya न भव्यन्तु तादृशं युष्माकं मध्ये। प्रत्युत युष्मन्मध्ये यो महान् भवितुमिच्छति स युष्माकं परिचारको भविष्यति,
\vakya यश्च युष्मन्मध्ये प्रथमो भवितुमिच्छति स सर्वेषां दासो भविष्यति।
\vakya यतो मनुष्यपुत्रोऽप्यागतो न परिचर्यां भोक्तुम् अपि तु परिचरितुं दातुञ्च स्वप्राणान् निष्क्रयमूल्यं बहूनां विनिमयेन।
\stitle{अन्धबरतीमयाय चक्षुर्दानम्।}
\vakya अनन्तरं ते यिरीहुमागताः। ततः परं तस्य तच्छिष्याणां महतो जननिवहस्य च यिरीहुतो निर्गमनकाले तीमयस्य पुत्रो बरतीमय इति नामकोऽन्धो भिक्षमाणः पथपार्श्व उपविष्ट आसीत्।
\vakya नासरतीयो यीशुरुपस्थित इति श्रुत्वा स क्रोशन् गदितुमारेभे, भो दायूदस्य पुत्र यीशो मामनुकम्पताम्।
\vakya अनेन बहवस्तं भर्त्सयन्तो मौनावलम्बनमादिदिशुः। स तु बहुगुणैरधिकमक्रोशत्। भो दायूदस्य पुत्र मामनुकम्पताम्।
\vakya यीशुस्तदा तिष्ठंस्तस्यानयनमाज्ञापयामास। अतस्ते तमन्धमाह्वयन्तोऽब्रुवन्, आश्वसिहि, उत्तिष्ठ, स त्वामाह्वयति।
\vakya तदा स स्ववसनं त्यक्त्वोत्पत्य यीशोरन्तिकमागतः।
\vakya यीशुस्तं संबोध्यावादीत्, त्वदर्थं मया कर्तव्यं किं वाञ्छसि? सोऽन्धस्तं जगाद, भो गुरो, दृष्टिशक्तिं लिप्से।
\vakya ततो यीशुस्तमुवाच, याहि तव विश्वासस्त्वां तारयामास। स च तत्क्षणं दृष्टिशक्तिं लेभे मार्गे यीशोरनुगामी बभूव च\eoc