\adhyAya
\stitle{अवशाङ्गिन आरोग्यकरणम्।}
\vakya कतिपयदिनेभ्यः परं स पुनः कफरनाहूमं प्रविष्टः स च गृहेविद्यत इति जनैः संश्रुतम्,
\vakya तत्क्षणञ्च बहवः समागतास्ततो द्वारसमीपेऽपि स्थानं नाशिष्यत। स च तेभ्यो वाचमकथयत्।
\vakya तदा केचिन्नराश्चतुर्भिरुह्यमानम् एकमवशाङ्गं नयन्तस्तम् अभ्यगच्छन्
\vakya जनौघहेतोस्तु तदन्तिकम् उपस्थातुमशक्यत्वात् स यत्रासीत् तत्र गृहपृष्ठं भित्तोत्खन्य च सोऽवशाङ्गो यत्राशेत तं पर्यङ्कम् अवारोहयन्।
\vakya तदा यीशुस्तेषां विश्वासं दृष्ट्वा तमवशाङ्गं जगाद, वत्स तव पापानां क्षमा जातेति।
\vakya तत्र तु केचिच्छास्त्राध्यापका आसीना आसन्, ते स्वचित्तेष्वित्थम् अतर्कयन्,
\vakya किमर्थमसावीदृशं भाषते? स ईश्वरं निन्दति। एकस्माद् ईश्वरादन्यः कः पापानि क्षमितुं शक्नोति?
\vakya ते तु स्वान्तर ईदृशं तर्कयन्तीति यीशुस्तत्क्षणम् आत्मन्यभिज्ञाय तान् अब्रवीत्, किमर्थं यूयं स्वचित्तेष्विमानि तर्कयथ?
\vakya अवशाङ्गमिमं कतरं वक्तुं सुकरं, तव पापानां क्षमा जातेत्यथवा त्वमुत्थाय स्वपर्यङ्कं वहन् विहरेति?
\vakya भेदिन्यान्तु पापानि क्षमितुं मनुष्यपुत्रस्य सामर्थ्यमस्तीति तेन यूयं जानीयात, तदर्थं (स तमवशाङ्गमाह)
\vakya त्वामहम् आदिशामि, उत्तिष्ठ, स्वपर्यङ्कमादाय वहन् स्वगृहं गच्छ।
\vakya स च सपद्युत्थाय पर्यङ्कमादाय वहन् सर्वेषां समक्षं निष्क्रान्तः। अनेन सर्वे चमत्कारं मत्वेश्वरं प्रशंसन्तोऽवदन्, ईदृशं किमप्यस्माभिः कदापि नादर्शि।
\stitle{लेव्याह्वानं।}
\vakya ततः परं स पुनः समुद्रतटमपजगाम। कृत्स्नो जननिवहश्च तत्समीपमागच्छत् स च तान् अशिक्षयत्।
\vakya गच्छंश्च शुल्कादायस्थान उपविष्टम् आल्फेयस्य पुत्रं मथिं दृष्ट्वा जगाद, मामनुव्रजेति। ततः स उत्थाय तमनुजगाम।
\vakya ततः परं तस्य गृहे तस्मिन् भोजनार्थमुपविशति बहवः शुल्कादायिनः पापिनश्चापि यीशुना तच्छिष्यैश्च सार्धम् उपाविशन्, यतस्ते बहुसङ्ख्या आसन् तञ्चान्वव्रजन्।
\vakya तन्तु शुल्कादायिभिः पापिभिश्च सार्धं भक्षयन्तं दृष्ट्वा शास्त्राध्यापकाः फरीशिनश्च तस्य शिष्यान् जगदुः, स किमर्थं शुल्कादायिभिः पापिभिश्च सह भक्षयति पिबति चेति?
\vakya तच्छ्रुत्वा यीशुस्तान् अब्रवीत्, न बलवताम् अपि त्वस्वस्थानां चिकित्सकेन प्रयोजनम्। न धार्मिकान् अपि तु पापिनो मनःपरावर्तनायाह्वातुम् अहम् आगतोऽस्मि।
\stitle{योहनस्य शिष्याणां फरीशिनाञ्चोपवासे निरुत्तरताकरणम्।}
\vakya तदा तु योहनस्य शिष्याः फरीशिनश्चोपावसन्, अतस्त आगत्य तम् अपृच्छन् योहनस्य फरीशिनाञ्च शिष्या उपवसन्ति, भवतः शिष्यास्तु नोपवसन्त्येतस्य कारणं किं?
\vakya यीशुस्तु तान् जगाद, कन्याया वरो यावत् सखिभिः सह वर्तते तावत् ते किमुपवस्तुं शक्नुवन्ति? यावत्कालं वरस्तैः सह विद्यते तावत्कालं त उपवस्तुं न शक्नुवन्ति।
\vakya आयास्यन्ति तु दिनानि यदा वरस्तेषां सान्निध्याद् अपहारिष्यते, तेष्वेव दिनेषु त उपवत्स्यन्ति।
\vakya जीर्णे वाससि कोऽप्यनाहतवस्त्रस्य खण्डं न सीव्यति, अन्यथा नूतनेन तेन पूरणोपायेन जीर्णस्य कियद्भागोऽपकृष्यते निकृष्टतरश्च भेदो जायते।
\vakya अपि च कोऽपि जीर्णासु कुतूषु नवीनं द्राक्षारसं न निधत्ते, अन्यथा नवीनो द्राक्षारसः कुतू र्विदारयति, तेन द्राक्षारसश्च विस्रवति कुत्वश्च नश्यन्ति। अपितु नूतनासु कुतूषु नवीनो द्राक्षारसो निहितव्यः।
\stitle{विश्रामवारमधि यीशोरुपदेशः।}
\vakya एकदा स विश्रामवारे शस्यक्षेत्रेणाव्रजत् तस्य शिष्याश्च गच्छन्तो मञ्जरी र्भङ्क्त्वादातुमारेभिरे।
\vakya तदा फरीशिनस्तमवदन्, पश्यतु, विश्रामवारे यन्न विधेयं भवतः शिष्यास्तदेव कथं कुर्वन्ति?
\vakya स तु तानब्रवीत्, अभावाद् दायूदस्य तत्सङ्गिनाञ्च क्षुधायां जातायां तेन यदकारि तत् किं युष्माभिः कदापि नापाठि?
\vakya महायाजकस्याबियाथरस्य काले स ईश्वरस्य गृहं प्रविष्टो ये च दर्शनीयपूपा याजकेभ्योऽन्यै र्न भोक्तव्यास्तानेव सोऽभक्षयत् स्वसङ्गिभ्यश्चाप्यदात्।
\vakya पुनः स तान् बभाषे, मानवस्य कृते विश्रामवारः सम्भूते न तु विश्रामवारस्य कृते मनुष्यः।
\vakya अतो मनुष्यपुत्रो विश्रामवारस्यापि प्रभुरस्ति\eoc