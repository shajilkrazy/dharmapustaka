\adhyAya
\stitle{अशौचमधि उपदेशः।}
\vakya एकदा फरीशिनः शास्त्राध्यापकानाञ्च केचिन्नरा यिरूशालेमत आगत्य
\vakya तस्यान्तिकं समाजग्मुस्तस्य कतिपयांश्च शिष्यान् अपवित्रैरर्थतोऽप्रक्षालितै र्हस्तैराहरतो दृष्ट्वा निनिन्दुः।
\vakya यतः फरीशिनो यिहूदीयाश्च सर्वे प्राचीनानां परम्परागतां शिक्षां रक्षन्तीतिहेतोः सयत्नं हस्तान् अप्रक्षाल्य न भक्ष्यन्ति।
\vakya हट्टाच्चागता न स्नात्वा न भक्ष्यन्ति। अन्या अपि बह्वो रीत्यस्तैः शिक्षित्वा रक्ष्यन्ते, यथा पानपात्राणां भाण्डानां तैजसभाजनानां शय्यानाञ्च स्नपनम्।
\vakya अतः फरीशिनः शास्त्राध्यापकाञ्च तमपृच्छन्, तव शिष्याः किमर्थं प्राचीनानां परम्परागतां शिक्षाम् अनाचरन्तोऽप्रक्षानितै र्हस्तैः खाद्यमाहरन्ति?
\vakya स तु तान् प्रतिजगाद, कपटिनो युष्मानधि यिशायाहः समीचीनां भावोक्तिं व्याहृतवान्, यथा लिखितमास्ते,
\begin{poem}
\startwithline “अधरैरेव कुर्वन्ति मत्सम्मानं जना इमे।
\pline किन्त्वन्तःकरणं तेषां मत्तो दूरमवस्थितं॥
\vakya अलीकार्थन्त्विमे सर्वे मम कुर्वन्ति सेवनं।
\pline धर्मशिक्षाछलेनैव शिक्षयन्तो नृणां विधीन्॥”
\end{poem}
\vakya यतो यूयमीश्वरस्याज्ञां विहाय मानवानां परम्परागतां शिक्षां रक्षन्तो भाण्डानां पानपात्राणाञ्च स्नपनं तत्सदृशा आन्याश्च बहूः क्रियाः कुरुथ।
\vakya स पुनस्तान् जगाद, युष्माकं परम्परागतां शिक्षां रक्षितुमाका यूयं बाढम् ईश्वरस्याज्ञां व्यर्थीकुरुथ।
\vakya यतो मोशिः कथयामास, त्वं निजपितर मातरञ्च सम्मन्यस्वेति, अपि च यो निजपितरं मातरं वा शपेत स वध्यो भविष्यतीति।
\vakya यूयन्तु वदथ, मत्तो यद्दानेन तवोपकारः समभविष्यत् तत् कर्बाणमर्थत ईश्वरायोपहृतमित्थं पित्रे मात्रे वा यो मनुष्यः कथयतीति। यूयञ्च तत्पश्चात् तं पुतु र्मातु र्वा कमप्युपकारं कर्तुं नानुजानीथ।
\vakya अनेन स्वपरम्परागतशिक्षयेश्वरस्य वाक्यं व्यर्थीकुरुथ।
\vakya ईदृशी र्बहूः क्रियाश्च कुरुथ।
\vakya अनन्तरं स जननिवहं पुनः स्वसमीपमाहूय बभाषे, सर्वै र्युष्माभिः श्रुत्वा बुद्धयतां।
\vakya बाह्यान्मनुष्यस्यान्तरं प्रविश्य यत् तम् अपवित्रीकर्तुं शक्नोति तादृशं किमपि नास्ति। प्रत्युत यद्यत् तस्मान्निःसरति तत्तदेव मनुष्यम् अपवित्रीकरोति।
\vakya शृणोतु यस्य श्रोतुं श्रोत्रे स्तः।
\vakya तस्मिंस्तु जननिवहं त्यक्त्वा गृहं प्रविष्टे शिष्यास्तं तस्या उपमाया अर्थं पप्रच्छुः।
\vakya ततः स तान् जगाद, किं यूयमप्येवं निर्बोधाः? यत् किञ्चन बाह्यान्मनुष्यं प्रविशति तत् तम् अपवित्रीकर्तुं न शक्नोत्येतत् किं न बुध्यध्वे?
\vakya तत् तावन्न तस्य हृदयं प्रविशत्यपि तूदरमेव प्रविश्य सर्वभुक्तस्य शूचीकारकं शौचकूपं निर्याति।
\vakya स पुन र्जगाद, मनुष्याद् यन्निःसरति तदेव मनुष्यम् अपवित्रीकरोति।
\vakya यतोऽन्तरान्मनुष्याणां हृदयादेव कुचिन्ता निःसरन्ति यथा व्यभिचारा वेश्यागमनानि नरहत्याश्चौर्याणि
\vakya लोभा अपकाराश्छलः स्वैरिता कुदृष्टि र्धर्मनिन्दाभिमानो मूढता।
\vakya सर्वाण्येतान्यशुभान्यन्तरान्निःसरन्ति मनुष्यम् अपवित्रीकर्वन्ति च।
\stitle{यीशुना भूतग्रस्तबालिकाया आरोग्यकरणम्।}
\vakya तस्मात् स्थानादुत्थाय स सोरसीदोनयोः सीमानं गत्वा किमपि गेहं प्रविवेश, तच्च केनापि न ज्ञातव्यमित्यवाञ्छत् न त्वशक्नोदलक्षितः स्थातुम्।
\vakya यतो यस्याः क्षुद्रा दुहिताशुच्यात्माविष्टासीत् तादृशी काचित् स्त्री तच्छ्रुत्वागत्य तस्य चरणयोः प्रणिपपात। सा योषित् सुरफैनीकीयजातीया यूनानीयासीत्।
\vakya स यथा तस्या दुहितुस्तं भूतं निःसारयेत् तदेव तयायाच्यत।
\vakya यीशुस्तु ताम् अब्रवीत्, प्रथमं बालकांस्त्रप्तुमनुमन्यस्व, यतो बालकानां खाद्यं हृत्वा सारमेयेभ्यो दानं न भद्रं।
\vakya अनेन सा तं प्रतिजगाद, सत्यं प्रभो, यतो भोज्यमञ्चस्याधो बालकानां फेल्यः सारमेयै र्भक्ष्यन्ते।
\vakya ततः स तां जगाद, अस्य वाक्यस्य हेतुतो याहि, तव दुहितुः स भूतो निर्गतः।
\vakya सा च स्वगृहं गत्वा तं भूतं निर्गतं स्वदुहितरञ्च पर्यङ्के शयानां ददर्श।
\vakya पुनः सोरसीदोनयोः सीमतः प्रस्थाय स दिकापलेः सीम्नां मध्येन गालीलीयसमुद्रस्यान्तिकमाजगाम।
\vakya तत्र मानवाः स्खलितवाचमेकं बधिरं तत्समीपमानीय स यत् तस्मिन् हस्तावर्पयेत् तदयाचन्त।
\vakya ततः स तं जननिवहान्निर्जनं स्थानमपनीय तस्य कर्णयोः स्वाङ्गुली र्निधाय निष्ठीवंस्तस्य जिह्वां पस्पर्श
\vakya स्वर्गं प्रत्युच्चदृष्टिं कृत्वा च निश्वस्य तं जगाद, इप्फाथाह, अस्यार्थो मुच्यस्वेति।
\vakya सपदि च तस्य कर्णावुद्घाटयाञ्चक्राते जिह्वाबन्धनञ्च मुमुचे स ऋजुवाग् बभूव च।
\vakya यीशुश्च तानादिदेश, यूयं कमप्येतन्मा वदत।
\vakya स तु यावदधिकं तानादिशत् तावदधिकं तेऽघोषयन्, अतिमात्रं चमत्कृताश्चावदन्, स सर्वमेव सम्यक् चकार, बधिरांश्च श्रवणे मूकांश्च कथने समर्थीकृतवान्\eoc