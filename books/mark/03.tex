\adhyAya
\stitle{शोषितहस्तलोकस्य तदन्येषाञ्चारोग्यकरणम्।}
\vakya ततः परं स पुनः समाजगृहं प्रविवेश।
\vakya तत्र च शोषितहस्तः कश्चिन्नर आसीत्, अतः स विश्रामवारे तं स्वस्थं करिष्यति न वेति ज्ञातुं ते तं निरैक्षन्त, यतस्ते तमभियोक्तुम् ऐच्छन्।
\vakya स तु तं शोषितहस्तं मानवं जगाद, मध्यस्थान उत्तिष्ठ, तांश्चाब्रवीत्,
\vakya विश्रामदिने किं विधेयं? हितक्रिया किंवाहितक्रिया? प्राणरक्षा किंवा प्राणनाशः? अनेन ते मौनमवललम्बिरे।
\vakya स तु सक्रोधं तान् समवलोक्य तेषां चित्तजडतायां शोतंस्तं नरं जगाद, स्वहस्तं प्रसारय। ततस्तेन प्रसारितः स हस्तः पुनरन्यतर इव स्वस्थः प्रतिपेदे।
\vakya तदा फरीशिनो निर्गत्य तूर्णं हेरोदीयैः सह तद्विनाशोपायाकाङ्क्षया तस्य विरुद्धं मन्त्रयाञ्चक्रिरे।
\vakya यीशुस्तु स्वशिष्यैः सार्धं समुद्रतटं जगाम। गालीलाच्च महान् जननिवहस्तमन्वगच्छत्।
\vakya यिहूदियातो यिरूशालेमाद् इदोमाद् यर्दनपारस्थदेशाच्चागता मानवाः सोरसीदोनयोश्चतुर्दिग्भ्यो महान् जननिवहश्च तेन यद्यत् क्रियते तच्छ्रुत्वा तदन्तिकमाजग्मुः।
\vakya तदा जननिवहस्य कारणाद् अर्थतः स यत्तै र्न पीड्येत तदर्थं स्वशिष्यान् जगाद, एका नौका मदर्थं सज्जीकृता तिष्ठत्विति।
\vakya यतः स बहून् निरामयान् अकार्षीत्, ततो व्याध्यापन्नाः सर्वे स्प्रष्टुं यतमानास्तम् आकुलयन्।
\vakya अशुचय आत्मानश्च यदा तम् अपश्यंस्तदा प्रणिपत्य क्रोशन्तोऽवदन्, भवान् ईश्वरस्य पुत्रः।
\vakya ते तु यत्तं न व्यक्तीकुर्युस्तदर्थं स तान् भूयोऽभर्त्सयत्।
\stitle{द्वादशशिष्याणां प्रेरितपदे नियोगः।}
\vakya ततः परं स गिरिमारुह्य स्वकीयेच्छातः कांश्चन नरान् स्वान्तिकम् आजुहाव ते च तत्समीपं जग्मुः।
\vakya ये च तेन सह वर्तेरन् घोषणार्थञ्च तेन प्रेष्येरन्
\vakya व्याधीनां प्रतीकाराय च भूतानां निःसारणाय च सामर्थ्यविशिष्टा भवेयुस्तादृशान् द्वादशनरान् स नियुयुजे।
\vakya (तेषां मध्ये गणिताय) शिमोनाय च पित्र इति नाम ददौ,
\vakya सिबदियस्य पुत्राय योकोबाय तस्य याकोबस्य भ्रात्रे योहनाय च स बनेरगश् इति नाम ददौ, एतस्यार्थो मेघनादसुताविति।
\vakya (शिष्याणां नामानि) अन्द्रियः, फिलिपः, बर्थलमयः, मथिः, थोमाः, आल्फेयसुतो याकोबः, थद्देयः, कानानिः शिमोनः,
\vakya ईष्करियोतीयो यिहूदाश्च। एष एव तस्य समर्पयिताभूत्।
\stitle{यीशुना कस्मेचित् भूतग्रस्ताय आरोग्यदानं उपदेशदानञ्च।}
\vakya अपरं ते गृहं प्रविवशुः। पुनश्च महान् जननिवहः समागमत्, ततस्त आहारमपि कर्तुं नाशक्नुवन्।
\vakya तच्छ्रुत्वा तु तस्य बन्धुजनास्तं धर्तुं निर्जग्मुः, यतस्तेऽवदन् स हतबुद्धिरभूत्।
\vakya यिरूशालेमाद् आगताः शास्त्राध्यापकाश्चाहुः, स बेल्‌सबूलेनाविष्टः, भूतानामधिपतेः साहाय्येन स भूतान् निःसारयतीति।
\vakya तदा स तान् समीपम् आहूयोपमाभि र्जगाद, शैतानः कथं शैतानं निःसारयितुं शक्नुयात्?
\vakya किञ्चन राज्यं यद्यात्मविरुद्धं विभज्यते तर्हि तद् राज्यं स्थातुं न शक्नोति,
\vakya किञ्चन कुलञ्च यद्यात्मविरुद्धं विभज्यते तत् कुलं स्थातुं न शक्नोति।
\vakya शैतानश्त यद्यात्मविरुद्धमुत्थाय विभक्तो जातस्तर्हि स स्थातुं न शक्नोति प्रत्युतान्तं प्राप्नोति।
\vakya तस्य बलवतो गृहं प्रविश्यैव तस्य सज्जाम् अपहर्तुं केनापि न शक्यं। प्रथमतः स बलवांस्तेन बध्यतां, तथानुष्ठिते तद्गृहस्य द्रव्याणि तेनापहारिष्यन्ते।
\vakya युष्मानहं सत्यं ब्रवीमि, मनुष्यसन्तानानां सर्वाणि पापानि तै र्व्याहृतानि सर्वाणि धर्मनिन्दावचांसि च मार्जयिष्यन्ते,
\vakya यस्तु पवित्रस्यात्मनो विरुद्धं निन्दां व्याहरति सोऽनन्ते कालेऽपि क्षमां नावाप्स्यति, अपि तु विचारेऽनन्तं दण्डमर्हति।
\vakya यतस्तेऽवदन्, असावशुचिनात्मनाविष्टः।
\vakya इतिमध्ये तस्य भ्रातरो माता चागत्य बहिस्तिष्ठन्तः प्रेष्यं प्रहित्य तम् आजुहुवुः।
\vakya तस्य परितस्तु जनौघ आसीन आसीत्। अतस्ते तं जगदुः, पश्यतु भवतो माता भ्रातरश्च बहिस्तिष्ठन्ति भवन्तमन्विष्यन्ति च।
\vakya स तु तान् प्रतिबभाषे, मम माता का? मम भ्रातरश्च के?
\vakya तदा स स्वपरित उपविष्टान् जनान् समवलोक्य बभाषे, पश्यत, मम माता मम भ्रातरश्चेमे।
\vakya यतो यः कश्चिद् ईश्वरस्येच्छाम् आचरति स एव मम भ्राता किंवा मम भगिनी किंवा मम माता\eoc