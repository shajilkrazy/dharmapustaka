\adhyAya
\stitle{अनेकभूतग्रस्तारोग्यकरणम्।}
\vakya ततः परं ते समुद्रपारस्थं गादारीयाणां देशं प्रापुः।
\vakya नौकातश्च निर्गते तस्मिन् तत्क्षणं शवागारेभ्य आगतः कश्चिद् अशुचिभूताविष्टो नरस्तस्य समक्षमुपतस्थे।
\vakya स शवागारेषु कृतनिवास आसीत् कश्चिच्च तं शृङ्खलैरपि बद्धुं नाशक्नोत्
\vakya यतः बहुकृत्वो निगडैः शृङ्खलैश्च बद्धेन तेन शृङ्खलानि विदीर्णानि निगडानि च चूर्णितानि तं शमयितुं शक्तिमान् कोऽपि नासीत्।
\vakya स दिने च रोत्रौ च निरन्तरं गिरिषु शवागारेषु वा तिष्ठन्नक्रोशत् प्रस्तरैश्चात्मानम् अकृन्तत्।
\vakya दूरात्तु यीशुं दृष्ट्वा सोऽभिधाव्य तस्य भजनं चकार महारवेण च क्रोशन् व्याजहार
\vakya भो परात्परस्येश्वरस्य पुत्र यीशो भवता सह मम कः सम्बन्धः? ईश्वरेण भवन्तं शापयामि मां मा यातयतु।
\vakya यतो यीशुस्तमवदन् रे अशुचे आत्मन् मनुष्यादस्मान्निःसार।
\vakya स च तं पप्रच्छ तव नाम किं? स प्रत्युवाच वाहिनीति मम नाम यतो वयं बहवः।
\vakya भवानस्मानेतद्देशाद् बहि र्मा प्रेषयत्वित्युक्त्वा च स तं भृशं प्रसादयामास।
\vakya तत्र गिरिनितम्बे तु महान् शूकरव्रजोऽचरत्।
\vakya अतस्ते सर्वे भूताः प्रार्थयमानास्तमवदन्, अमून् शूकरान् प्रत्यस्मान् विसृजतु, वयं तान् शूकरान् प्रविशाम।
\vakya ततो यीशुना तूर्णम् अनुज्ञातास्तेऽशुचय आत्मानो निःसृत्य तान् शूकरान् प्रविविशुस्तेन स कृत्स्नो व्रजः सवेगं धावन् शैलाग्रात् समुद्रे निपपात। ते शूकराः प्रायेण द्वे सहस्रे आसन्, सर्वे च समुद्रे पञ्चत्वं गताः।
\vakya शूकररक्षकास्तु पलाय्य नगरे पल्लीषु च संवादं व्याज्ञापयामासुः। ततो यद् वृत्तं तद् द्रष्टुं मनुष्या बहिराजग्मुः।
\vakya यीशोः समीपमुपस्थाय च वाहिन्याविष्टः स भूतग्रस्तो नर उपविष्टो वस्त्रान्वितः सुबुद्धिश्चेति पश्यन्तस्ते बिभ्युः।
\vakya ये च दृष्टवन्तस्ते तेभ्यस्तस्य भूताविष्टस्य वृत्तान्तं शूकराणाञ्च कथां निवेदयामासुः।
\vakya ततस्ते यीशुं स्वसीमातोऽपसरणं याचितुमारेभिरे।
\vakya तस्मिंस्तु नावं प्रविशति स भूताविष्टस्तं सविनयं ययाचे, अहं यद् भवता सार्धं तिष्ठेयं तदनुमन्यतामिति।
\vakya यीशुस्तु नानुमन्य तं जगाद, स्वगृहं स्वजनसमीपं याहि, प्रभुस्त्वामनुकम्प्य त्वदर्थं यत् कृतवांस्तत् सर्वं तेभ्यो निवेदय च।
\vakya ततः स गत्वा तं प्रति यीशुना यदकारि तत् सर्वं दिकापल्यां घोषयितुमारेभे, तेन मानवाः सर्व आश्चर्यममन्यन्त।
\stitle{यीशुना स्त्रिया आरोग्यकरणं मृतकन्यायै जीवनदानञ्च।}
\vakya ततः परं यीशौ नावा ह्रदं तरित्वा पुनः पारम् आगते महान् जनौघस्तत्समीपे समाययौ स तु समुद्रतट आसीत्।
\vakya पश्य च समाजाध्यक्षाणां मध्ये यायिरनामा नर एक आगच्छत् तञ्च दृष्ट्वा तस्य चरणयोः प्रणिपपात तञ्च भृशम् अनुनयन्नवदत्,
\vakya मम क्षुद्रा दुहिता मृतकल्पा। भवान् आगत्य तस्यां हस्तावर्पयितुम् अर्हति, तेन सा तरित्वा जीविष्यति।
\vakya ततः स तेन सह जगाम, महान् जननिवहश्च तम् अनुव्रजन्नपीडयत्।
\vakya तदा त्वाद्वादशवर्षेभ्यः प्रदररोगेण शीर्णा या काचिद् योषिद्
\vakya बहुभिश्चिकित्सकै र्भृशं क्लिष्टा सर्वस्वं त्यक्त्वापि कमप्युपशमं न लब्धवती प्रत्युताधिकमस्वस्था जाता
\vakya सा यीशोः कथां श्रुत्वा जनौघस्य मध्येन प्रच्छन्नं पश्चाद् उपस्थाय तस्य वसनं पस्पर्श
\vakya यतः सावदत् तस्य वासांसि स्पृष्ट्वैवाह तरिष्यामि।
\vakya सद्यश्च तस्याः शोणितोत्सः शुष्कीभूतः सा च यद् व्याधिमुक्ताभूत् तत् स्वदेहेऽनुबभूव।
\vakya यीशुस्तु सपदि स्वतः प्रभावस्य निर्गमनं ज्ञात्वा जननिवहस्य मध्ये परावृत्य पप्रच्छ, को मम वसनं स्पृष्टवान्?
\vakya तस्य शिष्यास्तमवादिषुः, जनौघेन भवान् पीड्यते तत् पश्यति, तथापि पृच्छति, को मां स्पृष्टवानिति।
\vakya तत्तु ययाकारि तां दिदृक्षुः स पर्यपश्यत्।
\vakya तदा सभमा सकम्पा च सा योषिद् आत्मनि यद् वृत्तं तज्ज्ञात्वोपस्थाय तत्समक्षं प्रणिपत्य सर्वं यथातथ्यं निवेदयामास।
\vakya स तु तां जगाद, वत्से तव विश्वासस्त्वां तारयामास, कुशलं याहि स्वव्याधितो मुक्ता तिष्ठ च।
\vakya यावत् स एतद् भाषते तावत् तस्य समाजाध्यक्षस्य गृहात् केचिन्नरा आगत्यावदन्, भवतो दुहिता ममार, किमर्थं गुरुम् अधिकं क्लिश्नीयात्?
\vakya यीशुस्तु ताम् उच्यमानां कथां सपद्युपश्रुत्य तं समाजाध्यक्षं जगाद, मा भैषीः केवलं विश्वसिहि।
\vakya स च पित्रं याकोबं याकोबस्य सहोदरञ्च योहनमपहायापरं कमपि स्वानुव्रजनाय नानुमेने।
\vakya ततः परं समाजाध्यक्षस्य गृहमागत्य भृशं रुदतां क्रोशताञ्च कोलाहलं दृष्ट्वा स प्रविश्य तान् जगाद,
\vakya किमर्थं कोलाहलं कुरुथ रुदिथ च? न मृता सा बाला सा निद्राणा।
\vakya ततस्ते तम् उपजहसुः। स तु तान् सर्वान् बहिः कृत्वा स्वसार्धं बालिकायाः पितरं मातरञ्च स्वसङ्गिनश्च नयन् बालिकायाः शयनागारं प्रविवेश।
\vakya तस्या बालिकाया हस्तञ्च गृहीत्वा तां जगाद, तालिथा-कूमीति। भाषान्तर एतस्यार्थोऽयं, बालिके, उत्तिष्ठेति ममाज्ञा।
\vakya सा च बालिका सपद्युत्थाय विहर्तुं प्रावर्तत, यतः सा द्वादशवर्षवयस्कासीत्। तदा मानवा महाचमत्कारं मेनिरे।
\vakya स तु तान् दृढमादिशत्, एतत् केनापि न ज्ञातव्यम्। अपि च स तस्यै भक्ष्यदानम् आज्ञापयामास\eoc