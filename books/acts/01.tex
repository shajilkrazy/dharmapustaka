\adhyAya
\stitle{आभाषः। प्रभोः यीशोः स्वर्गारोहणम्।}
\vakya भो महामहिमथियफिल, यीशु र्यद्यत् कर्तुमुपदेष्टुञ्च प्रववृते तत्सर्वस्य कथा मया पूर्वप्रबन्धे तद् दिनं यावद् रचिता
\vakya यस्मिन् स पवित्रेणात्मना स्ववरितेभ्यः शिष्येभ्य आदेशं दत्त्वोर्ध्वलोकमनायि।
\vakya स्वदुःखभोगात् परं स बहुभिः प्रत्यक्षैः प्रमाणैस्तेषां समक्षमात्मानं जीवन्तं दर्शितवान्, यतश्चत्वारिंशद्दिनेषु स पुनः पुनस्तेभ्यो दर्शनमददात् ईश्वरराज्यस्य कथाञ्चाकथयत्।
\vakya ततः स तान् एकत्रीकृत्यादिदेश, यूयं यिरूशालेमतो मापगच्छत, प्रत्युत मत्तः श्रुतां पितुः प्रतिज्ञां प्रतीक्षध्वं,
\vakya यतो योहनस्तोयेनास्नापयत्, यूयन्त्वल्पेषु दिनेष्वतीतेषु पवित्र आत्मनि स्नापयिष्यध्वे।
\vakya ते तदा समागत्य तमप्राक्षुः, प्रभो, कालेऽस्मिन् भवान् किमिस्रायेलाय राज्यं प्रतिपादयिष्यति?
\vakya स तु तानवादीत्, पिता यान् कालान् यांश्च समयान् स्वकर्तृत्वाधीनान् कृतवांस्तान् ज्ञातुं युष्माकमधिकारो नास्ति।
\vakya पवित्रेणात्मनाविष्टासु यूयं प्रभावं लप्स्यध्वे, यिरूशालेमे कृत्स्नायां यिहूदियायां शमरियायाञ्च मेदिन्याः प्रान्तञ्च यावन्मम साक्षिणो भविष्यथ।
\vakya तेनेदमुक्तास्ते यावन्निरैक्षन्त तावत् स ऊर्ध्वमनीयत मेघश्चैकस्तेषां लोचनगोचरात् तमपजहार।
\vakya तस्यापगमनकाले तेष्वेकाग्रदृष्ट्या गगनं निरीक्षमाणेषु, पश्य, शुल्कपरिच्छदौ द्वौ पुरुषौ तानुपस्थायावदतां, भो गालीलीयनराः, किमर्थं गगनमवलोकयन्तस्तिष्ठथ?
\vakya असौ यो यीशु र्युष्मत्समीपात् स्वर्गं नीतः स यादृशेन गमनेन स्वर्गं गच्छन् युष्माभिः सन्दृष्टस्तादृशेन प्रत्यागमिष्यति।
\vakya तदा ते जैतुनाख्यगिरितो यिरूशालेमं प्रत्याववृतिरे। स गिरि र्यिरूशालेमस्य निकटस्थः प्रायेण विश्रामवारीयमार्गैकव्यवधानः।
\vakya नगरं प्रविश्य ते गृहस्योपरिस्थप्रकोष्ठं जग्मुः। तद्गृहनिवासिनः पित्रो याकोबो योहन आन्द्रियश्च, फिलिपस्थोमाश्च बर्थलमयो मथिश्च,
\vakya आल्फेयसुतो याकोब उद्योगी शिमोनश्च याकोबस्य भ्राता यिहूदाश्चैते सर्वे योषिद्भि र्यीशो र्मात्रा मरियमा च तस्य भ्रातृभिश्च सार्धमेकचित्ताः सन्तः प्रार्थनायां याचनायाञ्चाध्यवसायिनोऽवर्तन्त।
\stitle{यिहूदाविनिमयेन मत्तथियनियोजनञ्च।}
\vakya तेषु दिनेष्वेकदा प्रायेण विंशत्यधिकशतजनेष्वेकत्रीभूतेषु पित्रो भ्रातॄणां मध्य उत्थाय बभाषे,
\vakya भ्रातरः, यीशु र्यैरधारि तेषां नायकीभूतं यिहूदामधि पवित्र आत्मा दायूदस्य वक्त्रेण यां शास्त्रीयोक्तिं व्याहृतवांस्तया सिद्धिः प्राप्तव्यासीत्।
\vakya स ह्यस्माकं मध्ये गणितोऽस्याः परिचर्याया अंशित्वं प्राप्तश्चासीत्।
\vakya सोऽधर्मस्य वेतनेन भूमिखण्डमेकमक्रैषीत् अवाक् पतित्वा तु द्विधा बभूव निःसस्रुश्च तस्य सर्वान्त्राणि।
\vakya तच्च यिरूशालेमनिवासिभिः सर्वैरज्ञायि, तस्मात् तेषां स्वभाषायां तत् क्षेत्रं हकलदामेत्यभिधीयते। अस्यार्थः शोणितक्षेत्रमिति। गीतसंहिताख्ये ग्रन्थे हि लिखितमास्ते।
\begin{poem}
\vakya “वाटी तस्य भवेच्छून्या तन्मध्ये कोऽपि मा वसेत्॥
\end{poem}
\eoc