\adhyAya
\stitle{पञ्चाशत्तमीत्यभिधोत्सवस्य दिने पवित्रस्यात्मन आवर्भावः।}
\vakya ततः परमुपस्थिते पञ्चाशत्तमीत्यभिधोत्सवस्य दिने ते सर्वे एकचित्तेनैकत्रासन्।
\vakya तदा कस्माद् गगनाद् बहमानस्य प्रचण्डवातस्येव शब्दः सञ्जातः, ते च यत्रोपविष्टा आसंस्तत् कृत्स्नं गृहं तेन व्याप्तम्,
\vakya अग्नेरिव विभज्यमाना तेषां प्रत्यक्षीबभूवुः स च वह्निस्तेषामेकैकस्मिन् निषसाद,
\vakya ते सर्वे च पवित्रेणात्मना पूरयाञ्चक्रिरे, आत्मना च यथोच्चारणमदीयत तथैवान्याभि र्भाषाभि र्गदितुमारेभिरे।
\vakya तदा गगनस्याधःस्थिताभ्यः सर्वजातिभ्य आगताः श्रद्धाशीला यिहूदीयनरा यिरूशालेमे न्यवसन्।
\vakya अतः सञ्जाते तस्मिन् शब्दे जनौघः समागत्य व्याकुलीबभूव, यतस्ते प्रत्येकं निजभाषया व्याहरतां तेषां कथामशृण्वन्।
\vakya ततः सर्व आश्चर्यं मत्वा चमत्कृताः परस्परमूचुः, पश्यतां, वक्तारोऽमी किं न सर्वे गालीलीयाः?
\vakya कथं तर्हि वयं प्रत्येकं स्वजन्मभाषया व्याहरतां तेषां कथां शृणुमः?
\vakya पार्थीया मादीया एलमीया जना मिसपतामिया-यिहूदिया-कापदकिया-पन्तदेशानाम् आशिया-फरुगिया-पम्फुलियादेशानां
\vakya मिसरस्य कुरीण्यन्तिकलिबुयादेशस्य च निवासिनः प्रवासिरोमीयाश्च यिहूदीया यिहूदिमतावलम्बिनश्च
\vakya क्रीतीया आरबीयाश्च वयं सर्वेऽस्मद्भाषाभिरीश्वरकृतानि महान्ति कर्माणि निवेदयताम् अमीषां कथाः शृणुमः।
\vakya इत्थं सर्वे विस्मिता बभूवुः सन्दिहानाश्च परस्परमूचुः, अस्य तात्पर्यं किं भवेत्?
\vakya अन्ये तूपहसन्तोऽवदन्, अमी सुस्वादुना द्राक्षारसेण मत्ताः।
\stitle{पित्रस्योपदेशवाक्यम्।}
\vakya तदा पित्र एकादशभिः सार्धमुत्थायोच्चस्वरमुदीरयंस्तानुद्दिश्येत्थं बभाषे, भो भ्रातरो यिरूशालेमनिवासिनश्च सर्वे, युष्माभिरेतज्ज्ञायताम् अतो मम वक्तव्यमाकर्ण्यतां।
\vakya युष्माभि र्यथानुमीयते तथामी मत्ताः सन्तितन्न सत्यं, यतोऽधुना दिवसस्य तृतीया घटिकास्ति।
\vakya प्रत्युत भाववादिना योयेलेन यदुक्तम् इदं तदेव, यथा।
\begin{poem}
\vakya इदं भाव्यन्तिमे काल इति व्याहरतीश्वरः।
\pline वर्षिष्यामि निजात्मानं सर्वमर्तस्य मूर्धनि।
\pline युष्माकं पुत्रपुत्रीभि र्भावोक्तिः कथयिष्यते॥
\pline भविष्यन्ति च युष्माकं युवानो लब्धदर्शनाः।
\pline स्वप्नान्यालोकयिष्यन्ते युष्माकं स्थविरा अपि॥
\vakya शीर्षेष्वपि स्वदासानां स्वदासीनाञ्च मूर्धसु।
\pline वर्षयिष्यामि निजात्मानमहं तेषु दिनेषु च॥
\vakya गगने लक्षणान्यूर्ध्वेऽभिज्ञानान्यप्यधो भुवि।
\pline रक्तं दास्यामि वह्निञ्च सधूमं बाष्पमेव च॥
\vakya विकृतोऽग्रे रविर्ध्वान्तं चन्द्रो रक्तं भविष्यति।
\pline तदायास्यति भीमं तत् प्रसिद्धञ्च प्रभोर्दिनं॥
\vakya भव्यमेतच्च यः कश्चिन्नरो नामाह्वयन् प्रभोः।
\pline प्रार्थनां कुरुते तस्य परित्राणं भविष्यति॥
\end{poem}
\vakya भो इस्रायेलीयनरा वचांसीमान्याकर्णयत। नासरतीयो यीशुः प्रभावसिद्धक्रियाभिरद्भुतलक्षणैरभिज्ञानार्थकर्मभिश्च युष्मत्समीपमीश्वरात् प्रहितो नर इव प्रतिपन्न आसीत्। ईश्वरो हि तेनैव युष्मन्मध्ये तानि कृतवान् एतद् यूयमपि जानीथ।
\vakya तमीश्वरस्य निर्णीतमन्त्रणया भाविज्ञानेन च समर्पितं यूयं धृत्वाधर्मिणां हस्तैः क्रूशविद्धं हतवन्तः।
\vakya ईश्वरस्तु मृत्यो र्यन्त्रणानि मोचयित्वा तमुत्थापयामास।
\vakya यतस्तं वशीकर्तुं मृत्युर्नाशक्नोत्। दायूदो हि तमुद्दिश्योवाच,
\begin{poem}
\startwithline सततं स्थापयामि स्म निजसाक्षादहं प्रभुं।
\pline स मद्दक्षिणपार्श्वस्थः स्खलनं मे निवारयन्॥
\vakya तस्माद्धृष्यति मच्चित्तं मम जिह्वा च नन्दति।
\pline शरीरं मामकञ्चाप्याश्वासितं विश्रमिष्यति॥
\vakya यतस्त्वं मामकप्राणान् पाताले न विहास्यसि।
\pline स्वीयपुण्यवतो नैव सोढासे क्षयदर्शनं।
\vakya जीवनस्य हि पन्थानं त्वं मां ज्ञापितवानसि।
\pline पूरयिष्यसि हर्षेण मां त्वदास्यस्य सङ्गिनं॥
\end{poem}
\vakya हे भ्रातरः, तं पितृकुलपतिं दायूदमधि युष्मभ्यं मुक्तकण्ठेन मयेदं कथयितुं शक्यते यत् स ममार च शवाधारे निदधे च तस्य शवागारञ्चाद्याप्यस्मदन्तिकं विद्यते।
\vakya स तु भाववाद्यासीदजानाच्च यच्छरीरसम्बन्धे तस्यौरसफलात् ख्रीष्टेनोत्पद्य तदीयसिंहासन उपवेष्टव्यमितीश्वरः शपथेन तस्मै प्रतिश्रुतवान्।
\vakya ततः स भवितव्यं दृष्ट्वा ख्रीष्टस्य पुनरुत्थानमधीदमब्रवीत्, यतस्तस्यैव प्राणाः पाताले न परित्यक्तास्तस्यैव देहश्च क्षयं न दृष्टवान्।
\vakya तमेव यीशुमीश्वरः पुनरुत्थापितवान्, अत्र सर्वे वयं साक्षिणः।
\vakya अत ईश्वरसय दक्षिणहस्तेनोच्चीकृतः स पितृतः प्रतिज्ञायाः फलं पवित्रमात्मानं लब्ध्वाधुना युष्माभि र्यद् दृश्यते श्रूयते च तदवतारयामास।
\vakya न हि स्वर्गमारूढवान् स एव तूवाच,
\begin{poem}
\startwithline मम प्रभुमिदं वाक्यं बभाषे परमेश्वरः।
\vakya त्वच्छत्रून् पादपीठं ते यावन्नहि करोम्यहं॥
\pline अवतिष्ठस्व तावत् त्वम् आसीनो मम दक्षिणो।
\end{poem}
\vakya अत इस्रायेलस्य कृत्स्नं कुलममोघं जानातु यद् युष्माभिः क्रुशारोपितं तमेव यीशपमीश्वरः प्रभुमभिषिक्तारञ्च चकार।
\stitle{त्रयाणां सहस्राणां लोकानां मण्डलिमुक्तव्यम्।}
\vakya इदं श्रुत्वा ते विदीर्णहृदाय भूत्वा पित्रमन्यान् प्रेरितांश्च जगदुः, भो भ्रातरः किं करवाम?
\vakya पित्रस्तदा तानुवाच, मनांसि परावर्तयत प्रत्येकञ्च पापानां माचनार्थं यीशोः ख्रीष्टस्य नाम्नि स्नाप्यध्वं, तथा कृते पवित्रमात्मानं दानवल्लप्स्यध्वे।
\vakya यतः सा प्रतिज्ञा युष्मासु युष्मत्सन्तानेषु चास्मदीश्वरेण प्रभुना समाह्वायिष्यमाणेषु दूरस्थेषु सर्वेष्वपि च वर्तते।
@* एतदन्यैरपि बहुभि र्वचोभिः स साक्ष्यं दत्त्वान्वनयत्, यूयमेतत्कालिकेभ्यः कुटिलेभ्यो मनुष्येभ्यः स्वनिस्तारं साधयितुमर्हथ।
@* तदा ते सानन्दं तस्य वाक्यं गृहीत्वा स्नापिता बभूवुः। इत्थं तस्मिन् दिने प्रायेण त्रयाणां सहस्राणां मनुष्याणामागमने मण्डली ववृधे।
@* ते च प्रेरितानां शिक्षायां सहभागितायां पूपभञ्जने प्रार्थनासु चाध्यवसायेन निविष्टा अतिष्ठन्।
@* यावतीयप्राणिनश्च भयं सञ्जातं प्रेरितैश्च बहून्यद्भुतानि लक्षणान्यभिज्ञानार्थकर्माणि चासाध्यन्त।
@* विश्वासिनश्च सर्व एकत्रातिष्ठन् सर्वञ्च साधारणं मत्वाधारयन्,
@* स्थावराणि जङ्गमानि वा वित्तानि च व्यक्रीणन् एकैकस्य प्रयोजनानुसारेण सर्वेषु व्यभजंश्च।
@* अध्यवसायेन च प्रत्यहमेकचित्ता धर्मधामन्यवर्तन्त गृहे गृहे च पूपं भञ्जन्त अह्लादेन हृदयस्य सारल्येन च भक्ष्यम् अभुञ्जतेश्वरमस्तुवंश्च सर्वजनस्यानुग्रहभाजनान्यभवंश्च। प्रभुस्तु प्रत्यहं परित्राणपात्रै र्मण्डलीमवर्धयत्\eoc