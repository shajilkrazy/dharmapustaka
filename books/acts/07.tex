\adhyAya
\stitle{स्तिफानस्योत्तरम्।}
\vakya अनन्तरं महायाजकः पप्रच्छ, कथा एताः किं सत्याः?
\vakya स्तिफानोऽवादीत्, भो भ्रातरः पितरश्च, शृणुत। अस्माकं पूर्वपुरुषोऽब्राहामो हारणनगरे निवासनात् प्राग् यदा मिसपतामियादेशेऽवर्तत, तदा प्रतापस्येश्वरस्तस्मै दर्शनं दत्त्वा तमवादीत्,
\vakya त्वं स्वदेशात् स्वज्ञातिजनमध्याच्च निर्गत्य मया प्रदर्शयितव्यं देशं याहि।
\vakya पुनस्तत्पितु र्मरणात् परं स तं ततोऽपि प्रस्थाप्येदानीं युष्माभिरध्युष्यमाणेऽस्मिन् देशे वासयामास।
\vakya तत्र तु तस्यै कमपि भूम्यधिकारं न ददौ पदैकमानामपि भूमिं न ददौ तथा च निजस्वविषयार्थं तं देशं तस्मै तदीयोत्तरवंशाय च दातुं प्रतिशुश्राव। तदा स निःसन्तान आसीत्।
\vakya ईश्वर इत्थमप्यभाषत, यत् तस्य वंशश्चत्वारिशतानि वर्षान् परदेशप्रवासी भविष्यति तज्जनैर्दासीकृतः पीडयिष्यते च।
\vakya ताः प्रजास्तु यस्या जाते र्दासा भविष्यन्ति, दामहं शासिष्यामीति प्रभुरुवाच, ततः परं ता निर्गत्य स्थानेऽस्मिन् मामाराधयिष्यन्तीति।
\vakya स तस्मै त्वक्छेदस्य नियममपि ददौ। तदनुरूपञ्च स इस्हाकं जनयित्वाष्टमे दिने तस्य त्वक्छेदं चकार।
\vakya पुनरिस्हाको याकोबं प्रति, याकोबश्चास्माकं द्वादश पितृकुलपतीन् प्रति तदेव चकार।
\vakya ते पितृकुलपतयो योषेफायेर्ष्यन्तो मिसरगामिनरेषु तं विचिक्रियिरे।
\vakya ईश्वरस्तु तेन सार्धमवर्तत तद्भुक्तात् सर्वस्मात् क्लेशात् तमुद्दधारे च, तस्मै मिसरराजस्य फरौणस्य समक्षमनुग्रहं प्राज्ञताञ्च विततार च, स च तं मिसरस्य स्वीयकृत्स्नकुलस्य चाधिपत्ये नियुयुजे।
\vakya तदा कृत्स्ने मिसरदेशे कनाने च दुर्भिक्षं महाक्लेशश्चोपतस्थाते, भक्ष्यद्रव्याणि चास्मत्पूर्वपुरुषैरलभ्यान्यभवन्।
\vakya तदा मिसरे गोधूमा विद्यन्त इति निशम्य याकोबः प्रथममस्मत्पूर्वपुरुषान् प्राहिणोत्।
\vakya अनन्तरं द्वितीये गमने योषेफः स्वभ्रातृभिः प्रत्यभिज्ञातः फरौणस्य समक्षञ्च योषेफस्य जातिः प्रकाशिता।
\vakya तदा योषेफः संवादं प्रहित्य स्वपितरं याकोबं निजान् पञ्चसप्ततिसङ्ख्यान् सर्वज्ञातींश्चानाययामास।
\vakya अनन्तरं योकोबो मिसरं जगाम, तत्रैव स चास्मत्पूर्वपुरुषाश्च प्राणांस्तत्याजुः।
\vakya तेषां शवाश्च शिखिमं नीताः शिखिमस्थहमोरस्य पुत्रेभ्यो रूप्यमूल्यं दत्त्वाब्राहामेण क्रीते शवागारे निदधिरे।
\vakya ईश्वरस्त्वाब्राहामाय शपथेन यत् प्रतिश्रुतवान् तस्याङ्गीकारवाक्यस्य सिद्धिसमयो यदा निकटोऽभवत् तदा मिसरे प्रजावर्धत बहुसङ्ख्या चाभवत्।
\vakya शेषे योषेफं यो नाजानात् तादृशोऽपरो मिसरराज उत्पन्नः।
\vakya सोऽस्मज्जाते र्विरुद्धं छलमाश्रित्यास्माकं पूर्वपुरुषानपीडयत् फलतस्तेषां शिशवो यथा न जीवेयुस्तदर्थं तेषां बहिःपरित्यागमादिदेश।
\vakya तस्मिन् काले मोशिरजनि, स ईश्वरस्य दृष्टौ मनोहर आसीत्, मासत्रयं पितृगृहे पर्यपाल्यत च।
\vakya ततः परं बहिः परित्यक्तं तं फरौणस्य दुहितोद्दधार स्वार्थं पुत्रं नियूुज्य परिपालयामास च।
\vakya ततो मोशि र्मिस्रीयाणां सर्वविज्ञानं शिक्षितो वाक्येषु क्रियासु च शक्तिमान् जातश्च।
\vakya इत्थं चत्वारिंशद्वर्षपरिमिते काले तेनातिवाहिते स्वभ्रातॄणामिस्रायेलसन्तानानां तत्त्वावधारणे प्रवृ्त्तिस्तस्य हृदाकाश उदियाय।
\vakya ततस्तेषामेकं प्रत्यन्यायः क्रियत इति दृष्ट्वा स उपकारं कुर्वंस्तं मिस्रीयं हत्वा तस्य पीड्यमानस्य कृतेऽन्यायस्य प्रतीकारं चकार।
\vakya अत्र सोऽन्वमन्यत, मम भ्रातरो भोत्स्यन्ते यन्मम हस्तेनेश्वरस्तेभ्यस्त्राणं ददातीति।
\vakya तत्तु तै र्नाबोधि। परदिने तेषु मिथो युध्यमानेषु स तेषां प्रत्यक्षो भूत्वा साग्रहं सन्धये प्रवर्तयंस्तानब्रवीत्,
\vakya भो नराः, भ्रातरो यूयं किमर्थं परस्परमकुरुथ? तदा निकटस्थस्यापकारी तं दुरीकुर्वन् जगाद, कस्त्वां शासकं प्राड्विवाकञ्च कृत्वास्मास्वधिकृतवान्।
\vakya त्वं यथा ह्यस्तं मिस्रीयं हतवांस्तथा मामपि हन्तुं किमिच्छसि?
\vakya तद् वचः श्रुत्वा मोशिः पलाय्य मिदियनदेशप्रवासी बभूव तत्र पुत्रद्वयं जनयामासे च।
\vakya इत्थं चत्वारिंशद्वर्षेष्वतीतेषु सीनयगिरे र्मरौ प्रभो र्दूतो गुल्मानलस्य शिखायां तस्मै दर्शनं ददौ।
\vakya तस्मिन् दर्शने मोशि र्विस्मितः। यदा तु तत समालोकयितुं समीपं याति, तदा तं प्रति प्रभोरियं वाणी बभूव,
\vakya तव पूर्वपुरुषाणामीश्वरोऽहम्, अब्राहामस्येश्वर इस्हाकस्येश्वरो याकोबस्य चेश्वरः। तदा मोशिर्वेपमानः समालोकयितुं प्रगल्भो न बभूव।
\vakya प्रभुस्तु तमब्रवीत्। तव चरणयोरुपानहौ मोचय यतस्त्वं यत्र तिष्ठसि सा पवित्रा भूमिः।
\vakya मिसरस्थानां मम प्रजानां पीडनं मया सम्यग् दृष्टं श्रुतश्च तसामार्तरावः। ता उद्धर्तुमेवाहमवतीर्णः। इदानीञ्च त्वमायाहि, अहं त्वां मिसरं प्रहेष्यामि।
\vakya तं यं मिशिं प्रत्याख्याय त उक्तवन्तः कस्त्वां शासकं प्राड्विवाकञ्च नियुक्तवांस्तमेवेश्वरः शासकं मोचयितारञ्च कृत्वा गुल्मे तस्मै दर्शनदातु र्दूतस्य हस्तेन प्रेषयामास।
\vakya स एव मिसरदेशे लोहितसमुद्रे च मरौ च चत्वारिंशद् वत्सरान् अद्भुतलक्षणान्यभिज्ञानार्थकर्माणि च साधयंस्तान् बहि र्निनाय।
\vakya स एव मोशिरिस्रायेलसन्तानान् जगाद युष्माकमीश्वरः प्रभु र्युष्मदर्थं युष्मद्भ्रातॄणां मध्यान्मादृशं भाववादिनमुत्पादयिष्यति, तस्य वाक्यं युष्माभि र्ग्रहीतव्यमिति।
\vakya स एव मरौ वर्तमानायां मण्डल्यां सीनयगिरौ तेन संलपतो दूतस्यास्मत्पूर्वपुरुषाणाञ्च मध्यस्थो जातोऽस्मभ्यं दातव्यानि जीवनावहानि वचांसि लब्धवांश्च।
\vakya अस्मत्पूर्वपुरुषास्तु तस्याज्ञावर्तिनो भवितुमनिच्छन्तस्तं दूरीकृत्य स्वमनःसु मिसरं प्रत्यावृत्ता हारोणमब्रुवन्
\vakya अस्मदर्थं देवान् निर्मिमीष्व येऽस्माकमग्रे यास्यन्ति, यतो येन मोशिना वयं मिसराद् बहिरानीतास्तस्य का दशा सम्भूता तन्न जानीम इति।
\vakya तस्मिन् काले ते गोवत्समूर्तिं निर्माय तां मूर्तिमुद्दिश्य यज्ञमनुतस्थुः स्वहस्तानां कर्मस्वाननन्दुश्च।
\vakya तत ईश्वरः पराङ्मुखीभूय गगनीयवाहिन्या आराधनार्थं तांस्तत्याज। भाववादिनां ग्रन्थे तदधीदं लिखितमास्ते,
\begin{poem}
\startwithline “चत्वारिंशत् समा यावत् किं मह्यं प्रान्तरे बलीन्।
\pline किं नैवेद्यानि वादत्त यूयं भो इस्रायेल्‌कुल॥
\vakya मोलकस्यापि दूष्यन्तु रिम्फन्देवस्य तारकां।
\pline युष्माभि र्निमिताः पूज्या अन्याश्चावहताकृतीः।
\pline ततोऽहं बाबिलः पारं वो नेष्यामि प्रवासयन्॥”
\end{poem}
\vakya यो मोशिं तेन दृष्टस्यादर्शस्यानुरूपं साक्ष्यदूष्यं निर्मातुमादिष्टवांस्तस्याज्ञातस्तद् दूष्यं मरावस्मत्पूर्वपुरुषाणां मध्येऽविद्यत,
\vakya यिहोशूयस्य सङ्गिनोऽस्मत्पूर्वपुरुषाश्च पैतृकविषयमध्ये तल्लब्ध्वा परजातीनामधिकारग्रणकाले तत्र नीत्वास्मत्पूर्वपुरुषाणां सम्मुखादीश्वरेण निरस्तानां तासां जातीनां देशे दायूदस्य काल, यावत् तदरक्षन्।
\vakya स ईश्वरस्यानुग्रहभाजनं भूत्वा योकोबेश्वरस्य कृत आवासस्याविष्कारं प्रर्थयाञ्चक्रे।
\vakya शलोमा तु तस्य कृते गृहं निर्ममे।
\vakya अपि तु परात्परो हस्तनिर्मितेषु निवासेषु न वसति, भाववादिना यथोक्तं,
\begin{poem}
\vakya “मम सिंहासनं स्वर्गे मम पादासनञ्च भूः।
\pline प्रभुराह गृहं कीदृग् विनिर्मास्यथ मत्कृते॥
\pline मद्विश्रामार्थकं स्थानं कुत्र वा लक्षयिष्यते।
\vakya मामकीनकरश्चेदं  साकल्यं किं सृष्टवान्॥”
\end{poem}
\vakya भो कठोरग्रीवा हृदयेषु कर्णेषु चाच्छिन्नत्वचश्च जनाः, यूयं सर्वदा पवित्रस्यात्मनः प्रतिकूलाचाराः, युष्माकं पूर्वपुरुषा यादृशा आसन्, यूयमपि तादृशाः।
\vakya भाववादिनां मध्ये को युष्मत्पूर्वपुरुषै र्नोपद्रुतः। तस्य धर्मवतो भाव्यागमनं यै र्ज्ञापितं तांस्ते जघ्नुः। इदानीञ्च यूयं तस्य समर्पयितारो घातकाश्च जाताः।
\vakya स्वर्गदूतानां नियममिव व्यवस्थां लब्ध्वा यूयं तां न रक्षितवन्तः।
\vakya एतच्छ्रुत्वा ते रुष्टहृदया जातास्तं प्रति दन्तानघर्षंश्च।
\vakya स तु पवित्रेणात्मना पूर्णः सन्ननन्यदृष्ट्या स्वर्गमालोकयन्नीश्वरस्य प्रतापमीश्वरस्य दक्षिणे तिष्ठन्तं यीशुञ्च ददर्श जगाद च,
\vakya पश्य, उद्घाटितं स्वर्गमीश्वरस्य दक्षिणे तिष्ठन्तं मनुष्यपुत्रञ्च वीक्षे,
\vakya तदा ते प्रोच्चरवेण क्रोशन्तः स्वकर्णान् रुद्ध्वा चैकचित्तेन तमभ्यद्रन्।
\vakya नगराद् बहिष्कृत्य च तं प्रस्तरैराघ्नन्। साक्षिणश्च शौलनाम्नः कस्यचिद् यूनश्चरणयोः स्ववासांसि निदधुः।
\vakya यदा ते स्तिफानं प्रस्तरैराघ्नंस्तदा स प्रार्थयमानोऽवदत्, प्रभो यीशो ममात्मानं गृहाण।
\vakya जानुनी पातयित्वा चाब्रवीत्, तेषामिदं पापं मा गणय। इत्युक्त्वा स निदद्रौ। शौलस्तु तस्य हत्यायां समप्रीयत\eoc