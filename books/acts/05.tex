\adhyAya
\stitle{अननियः साफीरा च।}
\vakya अननियाख्यस्त्वेकः पुरुषः स्वभार्यायाः साफीरायाः सम्मत्या कञ्चिद् विषयं विक्रीय
\vakya भार्यायामप्यवगतायां मूल्यस्यांशमात्मसात् कृत्वा कञ्चिदंशमानीय प्रेरितानां चरणेषु निदधौ।
\vakya पित्रस्तु तमवादीत्, अननिय, किमर्थं शैतानस्तव हृदयं पूरयित्वा त्वं यथा पवित्रायात्मनेऽनृतं कथयेस्तद्भूमिखण्डस्य मूल्यस्य चांशमात्मसात् कुर्यास्तथा त्वां प्रवर्तयामास?
\vakya स्थितिकाले स किं न तवातिष्ठत्? विक्रीते च तन्मूल्यं किं न तवाधीनमासीत्? किमर्थमेतत् कर्म स्वहृदयेऽग्राहीः? न मनुष्येभ्योऽपि त्वीश्वरायानृतं कथितं त्वया।
\vakya कथा एताः शृण्वन्ननियः पतित्वा प्राणांस्तत्याज, यावन्तश्चाश्रौषुस्ते सर्वे महाभयापन्नाः।
\vakya युवानस्तूत्थाय तं वस्त्रेण वेष्टयामासु र्बहिरूढ्वा शवागारे निदधुश्च।
\vakya तदनन्तरमतीते प्रायेणैकस्मिन् प्रहरे तस्य पत्नी तद् वृत्तमज्ञात्वा प्रविवेश।
\vakya पित्रस्तां सम्बोध्य जगाद, मां वद, सा भूमिः किं युवाभ्यामेतावन्मुद्राभि र्विक्रीता? साब्रवीत्, सत्यम् एतावतीभिरेव।
\vakya तदा पित्रस्तां जगाद, कथमेतद् यद् युवां प्रभोरात्मानं परीक्षितुमेकमन्त्रणौ जातौ? पश्य तव पति र्यैः शवागारे निहितस्तेषां पादा द्वारदेश उपस्थितास्ते त्वामपि बहि र्वक्षन्ति।
\vakya अनेन सा तत्क्षणं तस्य चरणयो र्निपत्य प्राणांस्तत्याज। ते युवानश्च प्रविश्य तां मृतां दृष्ट्वा बहिरूढ्वा पतिपार्श्वे तां शवागारे निदधुः।
\vakya ततः कृत्स्ना मण्डली यैश्चेदमश्रावि सर्वे ते च महाभयेनापन्नाः।
\vakya अपि च प्रेरितानां हस्तै र्जनेषु बहून्यभिज्ञानार्थकर्माण्यद्भुतलक्षणानि चासाध्यन्त।
\vakya सर्वे चैकचित्ताः शलोमनोऽलिन्देऽवर्तन्त। अपरेषान्तु कोऽपि मनुष्यस्तेष्वनुषंक्तुं प्रगल्भतां नागमत्। तथापि जनास्तान् समाद्रियन्त।
\vakya विश्वासिनां सङ्ख्या तु पुरुषाणां योषिताञ्च बाहुल्येनोत्तरोत्तरमवर्धत।
\vakya चत्वरेष्वपि मनुष्यै र्बहिरानीय व्याधिताः शय्यासु खट्वासु वा शायिता न्यधीयन्त, किंस्विदत्रागच्छतः पित्रस्य छायैव तेषां कमपि स्प्रक्ष्यतीत्याशया।
\vakya चतुर्दिक्स्थनगराणां नरनिकरश्च व्याधितान् अशुच्यात्मभिः क्लिष्टांश्च मनुष्यान् आनयन्तो यिरूशालेमे समागच्छन्, सर्वे ते चारोग्यमलभन्त।
\vakya तदा महायाजकः सर्वे तस्य सङ्गिनश्चार्थतः सद्दूकिमतावलम्बिनां सङ्घात ईर्ष्यया पूर्णः
\vakya प्रेरितेषु हस्तार्पणं कृत्वा तान् साधारणकारायां निदधुः।
\vakya प्रभो र्दूतस्तु रोत्रौ काराया द्वाराण्युद्घाटयामास, तांश्च बहि र्नीत्वा जगाद,
\vakya यूयं गत्वा धर्मधाम्नि तिष्ठन्तो जनेभ्यो एतज्जीवनावहानि सर्वाणि वाक्यानि कथयत।
\vakya श्रुत्वा त आसन्नप्रभाते धर्मधाम प्रविश्योपदेष्टुमारेभिरे। इतोमध्ये महायाजकस्तत्सङ्गिनश्चोपस्थाय सभात् इस्रायेलीयसन्तानानां कृत्स्नं प्राचीनवर्गञ्च समाहूय प्रेरितानामानयनार्थं पदातिनः कारागृहं प्रेषयामासुः।
\vakya ते तूपस्थाय तान् कारायां नासादयामासुः।
\vakya अतस्ते प्रत्यावृत्य जगदुः सम्पूर्णयत्नेन रुद्धं कारागृहं द्वारदेशेषु तिष्ठन्तो रक्षकाश्चास्माभि र्दृष्टाः, प्रविश्य त्वभ्यन्तरे कोऽपि नासादितः।
\vakya कथामिमां श्रुत्वा महायाजको धर्मधाम्नः सेनाध्यक्षो मुख्ययाजकाश्चेदं केन परिणंस्यतीति तेषु चिन्ताकुला अभवन्।
\vakya तदा कश्चिदागत्य तेभ्यो निवेदयामास, पश्यत, ये पुरुषा युष्माभिः कारायां समर्पितास्ते धर्मधाम्नि तिष्ठन्तो जनमुपदिशन्ति।
\vakya तदा सेनाध्यक्षः पदातयश्च गत्वा तानानिनाय, न तु बलेन, जनभयात् प्रस्तरै र्घानिष्यामह इति शङ्कमानाः।
\vakya तांश्चानीय सभायां स्थापयामासुः।
\vakya महायाजकस्तदा तान् पप्रच्छ, तेन नाम्ना मैवोपदिशतेति यूयं किं न दृढमस्माभिराज्ञापिताः? पश्यत तु, युष्मदुपदेशेन यूयं यिरूशालेमं पूरयितवन्तः, तस्य नरस्य शोणितमस्मासु वर्तयितुं यतध्वे च।
\vakya ततः पित्रोऽन्ये प्रेरिताश्च प्रतिजगदुः, मनुष्याणामाज्ञात ईश्वरस्याज्ञाधिकं ग्रहीतव्या।
\vakya यूयं काष्ठ उद्वध्य यं व्यापादितवन्तः स यीशुरस्मत्पूर्वपुरुषाणामीश्वरेण पुनरुत्थापितः।
\vakya ईश्वरः स्वदक्षिणहस्तेन तमेवोच्चीकृत्येस्रायेलाय मनःपरावर्तनस्य पापमोचनस्य च प्रदानार्थमधिपतिं त्रातारञ्च कृतवान्।
\vakya कथास्वेतासु तस्य साक्षिणो वयं स च पवित्र आत्मा यमीश्वरः स्वीयाज्ञाग्राहिभ्यः प्रदत्तवान्।
\vakya एतच्छ्रुत्वा ते रुष्टास्तेषां हत्यार्थममन्त्रयन्त।
\vakya सभायामुपस्थितस्त्वेकः फरीशी नरोऽर्थतो सर्वजनेन सम्मानितो गमलीयेल इत्यभिधो व्यवस्थाध्यापक उत्थाय प्रेरितानां कियत्कालार्थं बहिष्करणमादिश्य सभास्थान् जगाद,
\vakya भो इस्रायेलीयनराः, यूयं तान् मनुष्यान् प्रति यत् करिष्यथ तत्र सावधाना भवत।
\vakya एतेभ्यो दिनेभ्यः प्राक् थूदा आत्मानं विशिष्टं मत्वोत्तस्थौ न्यूनाधिकं चत्वारिशतानि नराश्च तस्मिन्ननुससज्जिरे, स तु जघ्ने तस्याज्ञाग्राहिणश्च सर्वे विकीर्णा अवास्तवाश्च बभूवुः।
\vakya तस्य पश्चात् नाम्नां लेख्यारोपणकाले गालीलीय यिहूदा उत्थाय बहुजनान् स्वानुगमनायोन्मार्गं निनाय। सोऽपि ननाश तद्वाक्यग्राहिणश्च सर्वे विकीर्णा बभूवुः।
\vakya अतोऽहमिदानीं युष्मान् ब्रवीमि, यूयं तान् नरांस्त्यक्त्वा मा बाधध्वं।
\vakya यतो मन्त्रणमिदं कार्यमिदं वा चेन्मनुष्यमूलकं तर्हि लोपं यास्यति, तत्तु चेदीश्वरमूलकं तर्हि यूयं तस्य लोपं साधयितुं न शक्ष्यथ। अपि त्वीश्वरेण सार्धं युद्धकारिणः प्रतिभास्यथ।
\vakya तदा ते तस्य वाक्यं जगृहुः प्रेरितांश्च स्वान्तिकमाहूय प्रहार्य च, यीशो र्नाम्ना पुन र्न भाषध्वमित्यादिश्य च तान् विससृजुः।
\vakya तदा तस्य नाम्नः कारणाद् वयमवमाननाया योग्या गणिता इत्यत्रानन्दन्तस्ते सभायाः सम्मुखात् प्रतस्थिरे।
\vakya कृत्स्नं दिनञ्चाविरतं धर्मधाम्नि गृहे चोपादिशन् यीशुरेव ख्रीष्ट इति संवादं व्यज्ञापयंश्च\eoc