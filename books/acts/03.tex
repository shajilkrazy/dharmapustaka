\adhyAya
\stitle{जनैकजन्मखञ्जस्य आरोग्यकरणम्।}
\vakya एकदा प्रार्थनायाः समयेऽर्थतस्तृतीये प्रहरे पित्रयोहनौ यदा युगपद् धर्मधाम
\vakya गच्छतस्तदामातृगर्भतः खञ्जो नर एक उह्यमान आगच्छत्। धर्मधामप्रवेशकान् भिक्षां याचिष्यमाणः स प्रत्यहं धर्मधाम्नः सुन्दराभिधो द्वारे न्यधीयत।
\vakya स धर्मधाम प्रवेष्टुमुद्यतौ पित्रयोहनौ निरीक्ष्य भिक्षां लिप्सुरयाचत।
\vakya तदा योहनेन सहितः पित्रोऽनन्यदृष्ट्या तं समालोक्याह, आवां प्रत्यवेक्षस्व।
\vakya स तदा किमपि लप्स्य इति प्रतीक्ष्य तयो र्निविष्टलोचनोऽतिष्ठत्।
\vakya ततः पित्रो जगाद, स्वर्णं रूप्यं वा मम नास्ति, यत्त्वस्ति तुभ्यं तद् ददामि। नासरतीयस्य यीशोः ख्रीष्टस्य नाम्ना समुत्थाय परिव्रज।
\vakya इत्युक्त्वा स तस्य दक्षिणहस्तं धृत्वा तमुत्थापयामास। तत्क्षणमेव तस्य पादतलौ गुल्फौ च दृढी बभूवुः,
\vakya स चोत्प्लवमान उत्तस्थौ व्रजितुं प्रववृते च व्रजन्नुत्प्लवमानश्चेश्वरं स्तुवंस्ताभ्यां सार्धं धर्मधाम प्रविवेश च।
\vakya व्रजन्नीश्वरं स्तुवंश्च स सर्वजनेनादर्शि
\vakya धर्मधाम्नः सुन्दरे द्वार उपविष्टो योऽभिक्षत स एवायमित्यभ्यज्ञायि च। तस्मिंश्च यत् सम्भूतं तेन ते चमत्कृता विस्मयापन्नाश्च बभूवुः।
\vakya स्वास्थ्यप्राप्ते तस्मिन् खञ्जे तु पित्रयोहनाववनम्ब्य स्थिते सर्वजनश्चमत्कृतो द्रुतगत्या शलेमनोऽलिन्दं तेषामन्तिकमाजगाम।
\vakya तद् दृष्ट्वा पित्रो जनानवादीत्, भो इस्रायेलीयनराः, यूयं किमर्थमत्राश्चर्यं मन्यध्वे निजशक्त्या निजभक्त्या वावामिमं व्रजने समर्थं कृतवन्ताविति मत्वा किमर्थमनन्यदृष्ट्यावां निरीक्ष्यध्वे च?
\vakya अब्राहामस्येसहाकस्य याकोबस्य च य ईश्वरोऽस्मत्पूर्वपुरुषाणामीश्वरः स एव स्वसेवकं तं यीशुं महिमाप्राप्तं कृतवान् यो युष्माभिः शत्रुहस्ते समर्पितो विचारे तद्विसर्जनकामस्य पीलातस्य समक्षमेव प्रत्याख्यातश्च।
\vakya यूयं हि पवित्रं धर्मवन्तञ्च तं नरं प्रत्याख्याय मनुष्यघातिनं नरं वरवद् याचितवन्तो जीवनस्यादिकर्तारन्तु हतवन्तः।
\vakya ईश्वरस्तु मृतानां मध्यात् तमुत्थापितवान् एतस्यावां साक्षिणौ।
\vakya तस्य नाम्नि विश्वसनहेतोश्च युष्माभि र्दृश्यमानमभिज्ञातञ्च नरमिमं तस्य नाम सबलीकृतवान् तज्जनतो विश्वासश्चास्मै सर्वेषां युष्माकं समक्षमिदं सर्वाङ्गव्यापि स्वास्थ्यं दत्तवान्।
\vakya अधुना, भो भ्रातरः, जानेऽहं तत् कर्म युष्माभि र्युष्मदीयाध्यक्षैश्चाज्ञानत्वादकारि।
\vakya ख्रीष्टेन तु दुःखं भोक्तव्यमिति यद्यद् भव्यमीश्वरः स्वीयभाववादिनां सर्वेषां मुखैर्विज्ञापितवान् तत्तदित्थं साधयामास।
\vakya अतो युष्मत्पापानां मार्जनार्थमनुतप्यध्वं मनांसि परावर्तयत च,
\vakya तथा कृते प्रभोः सकाशात् तापशान्तेः काल आयास्यति स च युष्मदर्थं पूर्वं निरूपितं ख्रीष्टं यीशुं प्रहेष्यति।
\vakya ईश्वरस्तु युगारम्भात् पवित्राणां स्वीयभाववादिनां सर्वेषां मुखैः सर्वविषयाणां सुदशायाः प्रतिपादनस्य यं कालमङ्गीकृतवान् तं प्रतीक्षमाणेन ख्रीष्टेन स्वर्गेऽवस्थातव्यं।
\vakya मोशि र्ह्यस्मत्पूर्वपुरुषान् उक्तवान् युष्माकमीश्वरः प्रभु र्युष्मदर्थं युष्मद्भ्रातॄणां भाववादिनमुत्पादयिष्यति, स युष्मान् यद्यद् वदिष्यति, तत् सर्वमधि तस्य वाक्यं युष्माभि र्ग्रहीतव्यं।
\vakya यः कश्चित् प्राणी तु तस्य भाववादिनो वाक्यं न ग्रहीष्यति स स्वजनमध्यात् उच्छेत्स्यत इति।
\vakya अपि च शमूयेलादयो यावन्तो भाववादिनः कालानुक्रमेण भाषितवन्तः सर्वे तेऽपीमं कालमधि भाविवाक्यं कथयामासुः।
\vakya यूयं हि भाववादिनां सन्ताना अस्मत्पूर्वपुरुषैः सार्धमीश्वरेण स्थापितस्य समयस्य चाधिकारिणः सन्तानाः। तं स्थापयन् सोऽब्राहाममवादीत्, तव वंशे मेदिन्याः सर्वाणि कुलान्याशिषं लप्स्यन्त इति।
\vakya प्रथमं युष्मदन्तिकमीश्वरः स्वसेवकं यीशुमुत्थाप्यैकैकं स्वखलतातः परावर्त्य युष्मान् आशिषो भाजनानि करिष्यन्तं तं प्राहिणोत्\eoc