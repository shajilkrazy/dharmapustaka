\adhyAya
\stitle{सप्रसहकारिणां नियोजनम्।}
\vakya तेषु दिनेषु वर्धमानायां शिष्याणां सङ्ख्यायाम् इब्रीयभाषिणां विरुद्धं यूनानीयभाषिणां मनोदुःखसूचकमिदं वाक्यमश्रावि यत् प्रात्यहिकोपकारेऽस्माकं विधवा योषित उपेक्ष्यन्त इति।
\vakya द्वादशप्रेरितास्तदा शिष्यनिकरं स्वसमीपमाहूय जगदुः, वयं यदीश्वरस्य वाक्यं विहाय भोजनमञ्चान् परिचरामस्तन्न प्रीतिजनकं।
\vakya अतो भो भ्रातरः, युष्मन्मध्ये पवित्रेणात्मना प्रज्ञया च पूर्णान् सप्त सुख्यातान् नरान् निरूपयत, तानेव वयं कार्येऽस्मिन्नधिकरिष्यामः।
\vakya वयन्तु प्रार्थनायां वाक्यस्य परिचर्यायाञ्चाध्यवसायिनः स्थास्यामः।
\vakya कथायामेतस्यां कृत्स्नस्य जननिवहस्य प्रीतिर्जाता, अतस्ते विश्वासेन पवित्रेणात्मना च पूर्णं स्तिफाननामानं नरं, फिलिपं, प्रखरं, नीकानरं, तीमोनं, प्रर्मिणां, नीकलायाभिधमेकञ्चान्तीयखीयं यिहूदिमतावलम्बिनं वरयामासुः
\vakya प्रेरितानां सम्मुखे स्थापयामासुश्च, ते च प्रार्थयित्वा तेषु हस्तानर्पयामासुः।
\vakya ईश्वरस्य वाक्यञ्चावर्धत, यिरूशालेमे शिष्याणां सङ्ख्याप्यतीवावर्धत। याजकानामपि महानिवहो विश्वासेनाज्ञाग्राही बभूव।
\stitle{स्तिफानस्य विवरणम्।}
\vakya अनन्तरं स्तिफानोऽनुग्रहेण प्रभावेण च पूर्णः सन् जनानां मध्येऽद्भुतलक्षणान्यभिज्ञानार्थकर्माणि चासाधयत्।
\vakya तदा लिबर्तीनानां समाज इत्यभिधस्य समाजस्य केचिदंशिनः कुरीणीयनां सिकन्दरीयाणाञ्च केचिन्नराः किलिकियाप्रदेशिनाम् आशियाप्रदेशिनाञ्च समाजस्य केचिदंशिनश्चोत्थाय स्तिफानेन सार्धं व्यवदन्त
\vakya स तु येन प्राज्ञत्वेन येनात्मना चाभाषत तयोः प्रतिरोधं कर्तुं नाशक्नुवन्।
\vakya तदा तैः प्रवर्तिताः केचिन्नरा अवदन् असौ मोशेरीश्वरस्य च विरुद्धं निन्दोक्ती र्व्याहत् ता वयमाकर्णितवन्तः।
\vakya इत्थं प्रजासु प्राचीनेषु शास्त्राध्यापकेषु च तैरुत्तेजितेषु ते तमुपस्थाय दध्रुः सभां निन्युश्च मृषासाक्षिण उपस्थापयामासुश्च।
\vakya तेऽब्रुवन् असौ नरोऽस्य पवित्रस्थानस्य व्यवस्थायाश्च विरुद्धं निन्दोक्ती र्व्याहर्तुं न विरमति।
\vakya वयं हि तस्येदं वाक्यमश्रौष्म, यथा, नासरतीयो यीशुरिदं स्थानं ध्वंसयिष्यति मोशिनास्माभ्यं दत्ता रीतीः परिवर्तयिष्यति चेति।
\vakya तदा सभायामासीनाः सर्वेऽनन्यदृष्ट्या तं समालोकयन्तस्तस्याननं स्वर्गदूतस्याननमिव प्रतिभातीत्यपश्यन्\eoc