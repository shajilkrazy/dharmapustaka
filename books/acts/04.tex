\adhyAya
\stitle{पित्रे योहने च राजविचारः।}
\vakya यदा तौ जनान् प्रतीत्थमभाषेतां, तदा याजका धर्मधाम्नः सेनापतिः सद्दूकीयाश्च तावुपतस्थुः,
\vakya यतस्तौ जनानुपादिशतां यीशौ मृतानां मध्यादुत्थानमज्ञापयताञ्चेत्यनेन ते पर्यतप्यन्त।
\vakya अतस्तयो र्हस्तार्पणं कृत्वा ते परदिनपर्यन्तं तौ कारायां निदधुः, यतस्तदा सन्ध्यासीत्।
\vakya ये तु वाक्यं श्रुतवन्तस्तेषां बहवो विश्वासिनो जाताः, पुरुषाश्च सङ्ख्यया न्यूनाधिकं पञ्च सहस्राण्यासन्।
\vakya परदिनेऽध्यक्षाः प्राचीनाः शास्त्राध्याका हानननामा महायाजकः कायाफा योहनः सिकन्दरश्च
\vakya महायाजकीयकुलजाता अन्ये सर्वे नराश्च यिरूशालेमे समाजग्मुः,
\vakya मध्यस्थाने तौ स्थापयित्वा पप्रच्छुश्च, केनाधिकारेण केन नाम्ना वा युवामिदं कृतवन्तौ?
\vakya तदा पवित्रेणात्मना पूर्णः पित्रस्तान् जगाद,
\vakya भो जनाध्यक्षा इस्रायेलस्य प्राचीनाश्च, अयं केन स्वास्थ्यं प्राप्तवानिति रोगिणोऽस्य मनुष्यस्योपकारमधि यद्यावां पृच्छ्यावहे,
\vakya तर्हि सर्वै र्युष्याभिः सर्वेणेस्रायेलीयजनेन चेदं ज्ञायतां, यो युष्माभिः क्रुशमारोपित ईश्वरेण तु मृतानां मध्यादुत्थापितस्यैव नासरतीयस्य यीशुख्रीष्टस्य नाम्नायं युष्मत्समक्षमरोगी तिष्ठति।
\vakya स एव स्थपतिभि र्युष्माभिरवज्ञातः स प्रस्तरो यः कोणस्य मुख्यप्रस्तरो जातः।
\vakya न च विद्यतेऽन्यस्मिन् कस्मिन्नपि परित्राणं, नापि वास्त्यन्यत् किमपि नाम स्वर्गस्याधो मनुष्येभ्यो दत्तं यद्द्वारास्माभिः परित्राणं प्राप्तव्यं।
\vakya तदा पित्रयोहनयो र्निर्भयतां पश्यन्तस्तावकृतविद्यौ सामान्यनराविति बुद्ध्वा च ते विस्मिता बभूवुस्तौ यीशोः सङ्गिनावास्तामित्यभिज्ञानंश्च।
\vakya तं स्वस्थीकृतं मनुष्यन्तु ताभ्यां सार्धं तिष्ठन्तं पश्यन्तः किमपि प्रतिवदितुं न शेकुः।
\vakya ततः परं सभातस्तेषां निर्गमनमादिश्य ते मिथ इत्थं मन्त्रयितुमारेभिरे, इमौ नरावुद्दिश्यास्माभिः किं कर्तव्यं? यतः प्रसिद्धमेकमभिज्ञानार्थकर्म ताभ्यां यदकारि तद् यिरूशालेमनिवासिनां सर्वेषां प्रत्यक्षमस्माभिश्च निह्नोतुमशक्यं।
\vakya परन्तु सा कथा प्रजासु यथाधिकं न व्याप्नुयात् तदर्थं वयं तौ तर्जयन्तोऽद्यप्रभृति तेन नाम्ना कमपि किमपि मा भाषेथामित्यादेक्ष्यामः।
\vakya ततः परं ते तावाहूयाज्ञापयामासुः, इतः परं यीशो र्नाम्ना कथञ्चित् किमपि मा व्याहरतं मा शिक्षयतं वेति।
\vakya पित्रयोहनौ तु प्रतिगदन्तौ तानवादिष्टाम् ईश्वरस्याज्ञातो युष्माकमाज्ञायामधिकमवधानमीश्वरस्य दृष्टौ न्याय्यं न वेति युष्माभि र्निर्णीयतां।
\vakya आवाभ्यां हि यद्यददर्श्यश्रावि च तन्न भाषितुमशक्यमिति।
\vakya तदा ते पुनस्तौ तर्जयित्वा विससृजुः, यतो यत् सम्भूतं तद्धेतोः सर्व ईश्वरमस्तुवन्, अतो जनभयात् ते तयोः शासनस्योपायं नालभन्त।
\vakya तत् स्वास्थ्यप्रदानरूपमभिज्ञानञ्च यस्मिन् नरे प्रदर्शितं तस्य वयश्चत्वारिंशद्वर्षेभ्योऽप्यधिकमासीत्।
\vakya इत्थं विसृष्टौ तौ स्वीयात्मीयानां समीपमगमतां मुख्ययाजकैः प्राचीनैश्च यद्यदुक्तौ तत् तेभ्यः कथयामासतुश्च।
\vakya श्रुत्वा ते चैकचित्तेनेश्वरमुद्दिश्य प्रोच्चैः प्रार्थयमाना अवदन्, भो नाथ, स्वर्गस्य भूमण्डलस्य समुद्रस्य च तत्र विद्यमानानां सर्वेषाञ्च त्वमेव स्रष्टेश्वरः,
\vakya त्वत्सेवकस्य दायूदस्य मुखेन च त्वमिदं भाषितवान्,
\begin{poem}
\startwithline “किमर्थं भिन्नजातीया लोकाः कुर्वन्ति गर्जनं।
\pline देशवासिप्रजाश्चापि चिन्तयन्ति व्यलीकतां॥
\vakya उत्तिष्ठन्ति महीपालाः समायान्ति च नायकाः।
\pline परमेशस्य तेनैव चाभिषिक्तस्य शात्रवात्॥
\end{poem}
\vakya सत्यं हि, पवित्रस्तव सेवको यो यीशुस्त्वयाभिषिक्तस्तस्य विरुद्धं नगरेऽस्मिन् हेरोदः पन्तीयः पीलातश्च परजातीयैरिस्रायेलीयवंशीयैश्च
\vakya जनैः सार्धं मिलित्वा साधितवन्तौ सर्वं तद् भव्यं यत् तव हस्तेन मन्त्रणया च पुरा निर्णीतमासीत्।
\vakya इदानीञ्च, भो प्रभो, तेषां विविधतर्जनमवलोकय
\vakya स्वदासेभ्यश्च सम्पूर्णनिर्भयत्वेन त्वद्वाक्यकथनस्य सामर्थ्यं देहि तदर्थञ्च प्रसारय स्वहस्तमारोग्यदानाय तव पवित्रसेवकस्य यीशो र्नाम्नाभिज्ञानार्थकर्मणामद्भुतलक्षणानाञ्च सिद्धये च।
\vakya तैरित्थं प्रार्थिते यत्र ते समवेता आसंस्तत् स्थानं चकम्पे, सर्वे ते च पवित्रेणात्मना पूर्णा ईश्वरस्य वाक्यं निर्भयमकथयन्।
\stitle{शिष्याणां प्रेम। प्रेरितानां क्षमतासाहसश्च।}
\vakya कृतविश्वासिनाञ्च निकरस्य चित्तमेकं मनंश्चैकमासीत्। नापि च कोऽपि सर्वस्वस्य किमपि नजस्वमवदत् अपि तु सर्वमेव तेषां साधारणमासीत्। प्रेरिताश्च महाप्रभावेण प्रभो र्यीशोः पुनरुत्थाने साक्ष्यं प्राददुः।
\vakya तेषु सर्वेषु च महानुग्रहोऽवर्तते।
\vakya फलतो दुर्गत एकोऽपि तेषु नाविद्यत, यतो यावन्तो भूमिखण्डानां गृहाणां वा स्वामिन आसंस्तैस्तानि विक्रीय यदा यद् व्यक्रीयत तदा तस्य मूल्यमानीय प्रेरितानां चरणेषु न्यधीयत ततः परम् एकैकस्मै प्रयोजनानुसाराद् व्यतीर्यत।
\vakya तदा यो प्रेरितै र्बार्णबा अर्थतः प्रबोधपुत्र इत्यभिहितः स कुप्रद्वीपे जातो योषिनामा लेवीयः क्षेत्रस्यैकस्य स्वाम्यासीत्, स तद् विक्रीय मूल्यमानीय प्रेरिताणां चरणेषु निदधौ\eoc