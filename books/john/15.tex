\adhyAya
\stitle{द्राक्षालता तस्य शाखाश्च।}
\vakya अहं यथार्था द्राक्षालता, मम पिता च कृषकः।
\vakya मयि स्थिता या काचिच्छाखा फलं न फलति सा तेन ह्रियते। या काचिच्छाखा च फलं फलति, सा तेन तथा संस्क्रियते यथाधिकं फलं फलिष्यति।
\vakya यूयं मया यद् वाक्यमुक्तास्तत्कारणात् परिष्कृता याताः।
\vakya मयि युष्माभिरवस्थीयतां, युष्मासु च मया। यथा द्राक्षालतायामनवस्थाय शाखया स्वतः फलं फलितुं न शक्यते, तथा मय्यनवस्थाय युष्माभिरपि न शक्यते।
\vakya अहं लता, यूयं तच्छाखाः। यो मय्यवतिष्ठतेऽहञ्च यस्मिन्नवतिष्ठे, स बहूनि फलानि फलति। यतो मत्तः पृथग् युष्माभिः किमपि कर्तुं न शक्यते।
\vakya मनुष्यश्चेन्मयि नावतिष्ठते, तर्हि स शाखेव बहि र्निक्षिप्तः शुष्कीभूतश्च, मनुष्यैश्च सञ्चित्याग्नौ निक्षिप्तास्ता ज्वलन्ति।
\vakya यूयं यदि मय्यवतिष्ठध्वे मम वचांसि च युष्मास्ववतिष्ठन्ते तर्हि यद्यद् वाच्छिष्यथ तत् प्रार्थयिष्यध्वे, युष्मदर्थं तत् सेत्स्यति च।
\vakya मम पितानेन महिमान्वितोऽभूत्, यद् यूयं बहु फलं फलथ मम शिष्या भविष्यथ च।
\vakya यथा मयि पित्रा प्रेमाकारि, तथा युष्मासु मया प्रेमाकारि। मम प्रेमन्यवतिष्ठध्वं।
\vakya यदि ममाज्ञा अनुपालयथ तर्हि मम प्रेमन्यवस्थास्यध्वे, यथाहं मत्पितुराज्ञा अनुपालितवांस्तत्प्रेमन्यवतिष्ठे च।
\vakya अहं युष्मभ्यमेतदर्थमिदं कथयामि यन्ममानन्दो युष्मास्ववस्थास्यते युष्माकमानन्दश्च सम्पूर्णा भविष्यति।
\vakya ममाज्ञेयं यद् युष्माभिः परस्परं तथा प्रेम कर्तव्यं यथा युष्मान् प्रति मया प्रेमाकारि।
\vakya स्वबन्धूनां निमित्तं प्राणत्यागान्महत्तरं प्रेम कस्यापि नास्ति।
\vakya यूयं मया यद्यदादिश्यध्वे तत् सर्वं चेदाचरथ तर्हि यूयं मम बन्धवः।
\vakya नाहं पुन र्युष्मान् दासानित्यभिदधामि, यतः प्रभुना किं क्रियते दासस्तन्न जानाति। अपि तु युष्मान् बन्धूनित्यभिहितवान्, यतः पितुः सकाशान्मया यद्यदश्रावि तत् सर्वमहं युष्मान् ज्ञापितवान्।
\vakya न यूयं मां वरितवन्तः, प्रत्युताहं युष्मान् वरितावांस्तदर्थं नियुक्तवांश्च यद् यूयं गत्वा फलं फलिष्यथ, युष्मत्फलञ्च स्थास्यति, यथा मम नाम्ना पितरं यत् किमपि प्रार्थयिष्यध्वे तत् तेन युष्मभ्यं दायिष्यते।
\vakya अहं युष्मानेतदर्थमिदमादिशामि, यद् यूयं परस्परं प्रेम करिष्यथ।
\stitle{जगद् युष्मान् द्वेष्टि।}
\vakya जगद् यदि युष्मान् द्वेष्टि तर्हीदं जानीत यद् युष्मादग्रतोऽहं तेन सन्द्विष्टः
\vakya यदि जगत्सम्बन्धीया अभविष्यत, तर्हि जग्निजस्वं प्रति प्रेमाकारिष्यत्। यूयन्तु न जगत्सम्बन्धीया अपि तु जगन्मध्यान्मया वरितास्तत्कारणाज्जगद् युष्मान् द्वेष्टि।
\vakya नास्ति दासः स्वप्रभुतो महत्तरः इति यद् वचो यूयं मयोक्तास्तत् स्मरत। तै र्यद्यहमुपद्रुतस्तर्हि यूयमप्युपद्राविष्यध्वे, तै र्यदि मम वाक्यमनुपालितं तर्हि युष्माकमपि वाक्यमनुपालयिष्यते।
\vakya एतत् सर्वन्तु युष्मान् प्रति मन्नाम्नः कारणात् तैः कारिष्यते, यतस्ते मत्प्रेषयितारं न जानन्ति।
\vakya यद्यहमनागत्य तेभ्यो नाकथयिष्यं, तर्हि ते निष्पापा अभविष्यन्। इदानीन्तु स्वपापप्रच्छादकोपायस्तेषां नास्ति।
\vakya यो मां द्वेष्टि स मत्पितरमपि द्वेष्टि।
\vakya यानि कर्माण्यन्येन केनापि न कृतानि, तानि चेदहं तेषां मध्ये नाकरिष्यं, तर्हि ते निष्पापा अभविष्यन्। इदानीन्तु ते दृष्ट्वापि माञ्च मत्पितरञ्च द्विष्टवन्तः।
\vakya किन्त्वित्थमेव तेषां शास्त्रे लिखितेनानेन वचनेन सिद्धेन भवितव्यं, द्विष्टोऽहं तैरकारणादिति।
\vakya यः शान्तिकर्ता तु पितुः सकाशाद् युष्मदन्तिकं मया प्रेषयिष्यते, पितुः सकाशान्निर्गच्छन् स सत्यस्वरूप आत्मा यदागमिष्यति, तदा स एव मामधि साक्ष्यं प्रदास्यति।
\vakya यूयमपि साक्षिणः, यत आदितो मत्सङ्गिनः स्थ\eoc