\adhyAya
\stitle{यीशोः समुद्रतीरे कतिपयशिष्येभ्यो दर्शनदानम्।}
\vakya ततः परं यीशुस्तिबिरीयस्य ह्रदस्य तीरे पुनरात्मानं शिष्याणां प्रत्यक्षं चकार। स इत्थमेवात्मानं प्रत्यक्षीचकार।
\vakya शिमोनः पित्रः, थोमा अर्थतो यमजः, गालीलस्थकान्नानिवासी नथनेलः, सिबदियस्य पुत्रौ, शिष्यौ च द्वावन्यावेकत्रासन्।
\vakya शिमोनः पित्रस्तान् ब्रवीति, अहं मीनान् धर्तुं यामि। ते तं वदन्ति वयमपि त्वया सार्धं यामः। ततस्ते तूर्णं बहि र्गत्वा नावमारुरुहुस्तस्यां रजन्यान्तु किमपि न दध्रुः।
\vakya जाते पुनः प्रत्यूषे यीशुस्तटे स्थितवान्। स तु यीशुरिति शिष्यै र्नाज्ञायि।
\vakya यीशुस्तदा तान् ब्रवीति, वत्साः, खाद्यं किमपि युष्माकं किमास्ते? ते तं प्रत्यवादिषुः, नास्ति।
\vakya यीशुस्तदा तान् वदति, नावो दक्षिणपार्श्वे जालं निक्षिपत, तर्हि लप्स्यध्वे। ततस्ते तन्निचिक्षिपुस्तदा च मीनानां महासङ्ख्यातः पुनस्तदाक्रष्टुं नाशक्नुवन्।
\vakya ततो यीशोः प्रियः स शिष्यः पित्रमवादीत्, असौ प्रभुः। तदा प्रभुरसाविति श्रुत्वा पित्र उत्तरीयवसनमादाय कटौ बबन्ध, यतः स विवस्त्र आसीत्। ततः परं समुद्रे निपपात।
\vakya अन्ये शिष्यास्तु पोतेनागमन्, यतस्ते स्थलाद् अदूराः केवलं प्रायेण द्विशतहस्तव्यवधाना आसन्। ते च मत्स्यपूर्णं जालमाकर्षन्।
\vakya स्थलमुत्तीर्य तु तैरङ्गारीयोऽग्निस्तस्योपरिस्थितो भृष्टो मीनः पूपश्च लक्ष्यन्ते।
\vakya यीशुस्तान् ब्रवीति, अधुना यान् मत्स्यान् अधार्ष्ट तेषां कतिपयानानयत।
\vakya ततः शिमोनः पित्र आरुह्य मत्स्यपूर्णं जालं कर्षन् स्थलेऽर्पयामास, तत्र त्रिपञ्चाशदधिकशतं महान्तो मत्स्या आसन् एतावत्सु सत्स्वपि जालं न व्यदीर्यत।
\vakya यीशुस्तदा तानवादीत्, आयात, प्रतराशं भुग्ध्वं। शिष्याणाञ्च कोऽपि कस्त्वमिति तं स्वतत्त्वं प्रष्टुं नोत्सेहे, यतः स प्रभुरिति तेऽजानन्।
\vakya यीशुस्तदोपस्थाय तं पूपं गृहीत्वा तेभ्यो वितरति, तं मत्स्यञ्च तथा।
\vakya इत्थं मृतानां मध्यादुत्थापितो यीशुस्तृतीयं वारं स्वशिष्याणां प्रत्यक्षो बभूव।
\stitle{पित्रेण सह यीशोः कथनम्।}
\vakya तेषां प्रातराशभोगात् परं यीशुः शिमोनं पित्रं ब्रवीति, योनासुत शिमोन, एतेभ्यस्त्वं किं मां प्रत्यधिकं प्रेम करोषि? स तं वदति, सत्यं प्रभो, भवान् जानाति यदहं भवदनुरक्तः। स तं ब्रवीति, मम मेषशावकान् चारय।
\vakya स पुन र्द्वितीयं वारं तं वदति, योनासुत शिमोन, मयि किं प्रेम करोषि? स तं वदति, सत्यं प्रभो, भवान् जानाति यदहं भवदनुरक्तः। स तं ब्रवीति, मम मेषान् पालय।
\vakya स तृतीयं वारं तं ब्रवीति, योनासुत शिमोन, त्वं किं मदनुरक्तः? पित्रस्तृतीयवारं त्वं किं मदनुरक्त इति तेन पृष्टस्तत्कारणाच्छोकार्तो बभूव, तञ्च जगाद, प्रभो भवान् सर्वविद्, भवान् जानाति यदहं भवदनुरक्तः। यीशुस्तं ब्रूते, मम मेषान् चारय।
\vakya सत्यं सत्यं, त्वामहं ब्रवीमि, यदा तरुणतर आसीस्तदा स्वयं स्वकटिमबन्धा यत्र चैच्छस्तत्र व्यहरः। यदा तु वृद्धो भविष्यसि, तदा स्वहस्तौ प्रसारयिष्यस्यन्यश्च कश्चित् तव कटिं भन्त्स्यति, त्वदनभीषटं स्थानं त्वां नेष्यति च।
\vakya कीदृशेन मृत्युना स ईश्वरं सम्मानयिष्यति तत् सूचयन् यीशुरिदं कथयामास। इदं कथयित्वा च स तमवादीत्, मामनुगच्छ।
\stitle{योहने तस्य कथा।}
\vakya पित्रस्तदा प्रत्यावृत्य यीशोः प्रियं तं शिष्यमनुगच्छन्तं व्यलोकयत्, यो भोज्ये तस्य वक्षस्यासीनः पृष्टवान्, प्रभो, भवतः समर्पयिता क इति।
\vakya तमेव दृष्ट्वा पित्रो यीशुं ब्रवीति, प्रभो, किं भवितव्यमनेन?
\vakya यीशुस्तं वदति यदि ममागमनं यावत् तस्यावस्थानं रोचये, तर्हि तत्र तव किं? त्वं मानुगच्छ।
\vakya ततः शिष्येषु कथेयं व्यानशे तत् तेन शिष्येण न मर्तव्यं। यीशुस्तु तं नैवेदं व्याजहार, तेन न मर्तव्यमिति, प्रत्युतेदं, यदि ममागमनं यावत् तस्यावस्थानं रोचये तर्हि तत्र तव किमिति।
\stitle{समाप्तेः कथा।}
\vakya स एव शिष्य एताः कथा अधि साक्ष्यं दत्तवान् एताः कथा लिखितवांश्च, तस्य साक्ष्यञ्च सत्यमिति वयं जानीमः।
\vakya एतेभ्योऽन्यान्यपि बहूनि कर्माणि यीशुना कृतानि, सर्वाणि तानि यत्येकैकशो लिख्यन्ते, मन्ये तर्हि तेषां लेखितव्यानां ग्रन्थानां धारणार्थं जगदप्यपर्याप्तं भविष्यति\eoc