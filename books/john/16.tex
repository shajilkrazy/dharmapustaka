\adhyAya
\stitle{यीशोः प्रस्थानं सहायस्यागमनं जगज्जनान् प्रति तत्कर्मप्रशंसनञ्च।}
\vakya यूयं यन्न स्खलिष्यथ तदर्थमहं युष्मभ्यं सर्वमेतत् कथितवान्।
\vakya जना युष्मान् समाजच्युतान् करिष्यन्ति। अपि तु समय आयाति यदा युष्मद्घातक एकैको मंस्यते, ईश्वरायाराधनमपहरामीति।
\vakya ते यद् युष्मान् प्रतीदृशमाचरिष्यन्ति, तत्कारणमिदं यत् तै र्न पिता नाहं वाभिज्ञातः।
\vakya यूयं मयेदमुक्ता इति यदुपस्थिते समये स्मरिष्यथ, तदर्थमहं युष्मभ्यमिदं कथितवान्। आदितस्तु युष्मभ्यमिदं नाकथयं, यतोऽहं युष्माभिः सार्धमासं।
\vakya अधुना तु मत्प्रेषयितुः समीपं गच्छामि, युष्माकञ्च कोऽपि मां न पृच्छति, कुत्र गच्छसीति।
\vakya अपि तु युष्मभ्यं मयैतत् कथितमिति कारणाद् युष्माकं हृदयं शोकपूर्णमभूत्।
\vakya तथापि युष्मानहं सत्यं ब्रवीमि, ममापगमनं युष्माकं हितावहं, यतोऽहं चेन्नापगच्छामि, तर्हि शान्तिकर्ता युष्मत्समीपं नागमिष्यति, यदि त्वपगच्छामि तर्हि तं युष्मत्समीपं प्रहेष्यामि।
\vakya ततः स आगत्य पापे धार्मिकतायां विचारे च दोषस्य प्रमाणं जगते दास्यति।
\vakya पापस्य प्रमाणमिदं यत् ते मयि न विश्वसन्ति।
\vakya धार्मिकतायाः प्रमाणमिदं यदहं पितुः समीपं गच्छामि युष्माभिश्च न पुन र्द्रक्ष्ये।
\vakya विचारस्य प्रमाणञ्चेदं यदस्य जगतोऽधिपते र्विचारो निरणायि।
\vakya युष्मभ्यं कथयितव्या अन्या बह्वः कथा मम सन्ति, यूयन्त्विदानीं ताः सोढुं न शक्नुथ।
\vakya स सत्यस्वरूप आत्मा तु यदायास्यति तदा पथप्रदर्शक इव युष्मान् कृत्स्नं सत्यं नेष्यति। स हि न स्वतः कथयिष्यति, प्रत्युत यद्यच्छ्रोष्यति तत् कथयिष्यति, यद्यद् भावि तद् युष्मान् ज्ञापयिष्यति च।
\vakya स मां महिमान्वितं करिष्यति, यतो यन्मम तल्लब्ध्वा युष्मान् ज्ञापयिष्यति।
\vakya पितु र्यद्यदास्ते तत् सर्वं मम ततो हेतो र्मया कथितं, यन्मम स तल्लब्ध्वा युष्मान् ज्ञापयिष्यति।
\vakya स्तोककालेऽतीते यूयं मां न द्रक्ष्यथ, पुनः स्तोककालेऽतीते मां द्रक्ष्यथ, यतोऽहं पितुः समीपं गच्छामि।
\vakya ततस्तच्छिष्याणां केचित् परस्परमवदन्, किमेतद् यत् तेन व्याह्रियते, स्तोककालेऽतीते यूयं मां द्रक्ष्यथ, पुनः स्तोककालेऽतीते मां द्रक्ष्यथेति, अपि च, यतोऽहं पितुः समीपं गच्छामीति।
\vakya तत्रैव तेऽवदन्, स्तोककाल इति तेन यदुदाह्रियते तत् किं? तेन किं कथ्यते तदस्माभि र्न बुध्यते।
\vakya ततस्ते तं प्रष्टुकामा इति ज्ञात्वा यीशुस्तानवादीत्, स्तोककालेऽतीते मां न द्रक्ष्यथ, पुनः स्तोककालेऽतीते मां द्रक्ष्यथेति मया यदुक्तं तदधि किं मिथो विचारयथ?
\vakya सत्यं सत्यं, युष्मानहं ब्रवीमि, यूयं क्रन्दिष्यथ विनपिष्यथ च, जगत् त्वानन्दिष्यति। यूयं शोचिष्यथ, किन्तु युष्माकं शोक आनन्देन परिणितो भविष्यति।
\vakya प्रसवन्ती नारी शोकार्ता भवति, यतस्तस्या दण्ड उपस्थितः। शिशुं प्रसूता तु सा तं क्लेशं पुन र्न स्मरति, प्रसवेन मनुष्यो जगत् प्रविष्ट इत्यानन्दात्।
\vakya यूयमप्यधुना शोकार्ताः, अहन्तु युष्मान् पुन र्द्रक्ष्यामि, ततो युष्माकं हृदयमानन्दिष्यति, युष्माकमानन्दश्च केनापि युष्मत्तो नापहर्तव्यः।
\vakya तस्मिन् देने च यूयं मां किमपि न प्रक्ष्यथ। सत्यं सत्यं, युष्मानहं ब्रवीमि, यूयं पितरं यद्यत् प्रार्थयिष्यध्वे मम नाम्ना स युष्मभ्यं तत् सर्वं दास्यति।
\vakya इतिपूर्वं यूयं मम नाम्ना किमपि न प्रार्थितवन्तः। प्रार्थयध्वं तेन लप्स्यध्वे, युष्माकमानन्दो यथा सम्पूर्णो भविष्यति।
\vakya सर्वमेतदहमुपमाभि र्युष्मभ्यमकथयं। आयाति तु समयो यदा न पुनरुपमाभि र्युष्माभिः सार्धं सम्भाषिष्येऽपि तु स्पष्टं युष्मान् पितुः कथां ज्ञापयिष्यामि।
\vakya तस्मिन् दिने यूयं मम नाम्ना प्रार्थयिष्यध्वे। युष्मानध्यहं पितरं याचिष्य इति युष्मान् न वदामि।
\vakya पिता हि स्वयं युष्मासु प्रेम कुरुते, यतो यूयं मयि प्रेम कृतन्तो विश्वसितवन्तश्च यदहं पितुः सकाशान्निर्गत्यायातः।
\vakya अहं पितुः सकाशान्निर्गत्य जगदायातः पुन र्जगत् त्यक्त्वा पितुः समीपं गच्छामि।
\vakya तस्य शिष्यास्तं वदन्ति, पश्यत्वधुना भवान् स्पष्टं भाषते, कामप्युपमां न कथयति।
\vakya भवान् सर्वं वेत्ति कस्यापि प्रश्नं न प्रतीक्षते, तदधुनास्माभिरनुभूतं। अनेन वयं विश्वसिमो यद् भवान् ईश्वरसकाशान्निर्गत्यायातः।
\vakya यीशुस्तान् प्रत्यवादीत्, अधुना किं विश्वसिथ?
\vakya पश्यतायाति स समय इदानीञ्चागतोऽस्ति यदा यूयं विकीर्णः प्रत्येकं निजस्थानं गमिष्यथ मामेकाकिनं विहास्यथ च। नास्मि त्वेकाकी, यतः पिता मम सार्धं वर्तते।
\vakya सर्वमेतदहं युष्मभ्यमकथयं, यथा मयि युष्माकं शान्ति र्भविष्यति। जगति युष्माकं क्लेशो जायते, किन्त्वाश्वसित, पराजितं मया जगत्\eoc