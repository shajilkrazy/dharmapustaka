\adhyAya
\stitle{एकस्या व्यभिचारिण्या मोचनं।}
\vakya ततस्ते प्रत्येकं स्वगृहं जग्मुः, यीशुश्च जैतूनगिरिं जगाम।
\vakya प्रत्यूषे तु स पुन र्धर्मधामन्युपतस्थे सर्वजनश्च तदन्तिकमागच्छत्, स चोपविश्य तानशिक्षयत्।
\vakya तदा शास्त्राध्यापकाः फरीशिनश्च व्यभिचारे धृतां स्त्रियं कञ्चित् तस्य समीपमानयन्ति, मध्यस्थाने स्थापयित्वा च तं वदन्ति,
\vakya गुरो, योषिदियं व्यभिचारं कुर्वन्त्येव धृता।
\vakya व्यवस्थायाञ्च मोशिरस्मान् प्रस्तराघातेनेदृशीनां प्राणदण्डमाज्ञापयामास, अतो भवान् किं वदति?
\vakya तस्य विरुद्धमभियोगसूत्रस्य लिप्सया ते परीक्षमाणास्तमेतदूचुः। यीशुस्तु प्रह्वीभूयाङ्गुल्या मृत्तिकायाम् अलिखत्।
\vakya तेषु तु तं पृष्ट्वावतिष्ठमानेषु, स ऊर्ध्वीभूय तानवादीत्, युष्माकं यो निष्पापः प्रथमं सोऽस्यां प्रस्तरं निक्षिपतु।
\vakya इत्युक्त्वा स पुनः प्रह्वीभूय मृत्तिकायामलिखत्।
\vakya ते तु तच्छ्रुत्वा संवेदन दोषबोधमनुभूय च प्राचीनेभ्य आरभ्यान्त्यान् यावदेकैकशः सर्वे निरक्रामन्। इत्थमेकाकी यीशु र्मध्यस्थाने तिष्ठन्ति सा स्त्री चाशिष्येतां।
\vakya ततो यीशुरूर्ध्वीभूय तां स्त्रियमृतेऽपरं कमपि न दृष्ट्वां च तां पप्रच्छ, नारि, त्वदभियोगिनस्ते नराः कुत्र? त्वं किं केनापि न दण्डार्हीकृता?
\vakya सावादीत्, न केनापि, प्रभो। ततो यीशुस्तामाह, अहमपि त्वां न दण्डार्हीकरोमि, याहि, पुनः पापं माकार्षीः।
\vakya ततो यीशुः पुन र्जनैः संलप्य बभाषे, अहं जगतो ज्योतिः, यो मामनुगच्छति स नैवान्धकारे व्रजिष्यत्यपि तु जीवनरूपमालोकं प्राप्स्यति।
\vakya फरीशिनस्तदा तमूचुः, त्वया स्वार्थं साक्ष्यं दीयते, तव साक्ष्यं न सत्यं।
\vakya यीशुः प्रतिभाषमाणस्तान् जगाद, यद्यप्यहं स्वार्थं साक्ष्यं ददामि, तथापि मम साक्ष्यं सत्यं, यतोऽहं कुत आगतः क्व वा गच्छामि, तज्जाने। यूयन्तु न जानीथ कुतोऽहमागतः क्व वा गच्छामि।
\vakya यूयं शरीरानुयायिनं विचारं कुरुथ, अहं कस्यापि विचारं न करोमि।
\vakya यद्यप्यहं विचारं करोमि, तथापि मम विचारः सत्यः, यतो नास्म्यहमेकाकी, किन्त्वहमस्मि मत्प्रेषयिता पिता चास्ति।
\vakya युष्मदीयव्यवस्थायामपि लिखितमस्ति, यद् द्वयोः साक्षिणोः साक्ष्यं सत्यं।
\vakya अहं स्वार्थे साक्ष्यदातास्मि, मत्प्रेषयिता पिता च मयि साक्ष्यं ददाति।
\vakya तदा ते तं पप्रच्छुः, कुत्रास्ति तव पिता? यीशुस्तान् प्रतिजगाद, यूयं माञ्च न जानीथ, मम पितरश्च न जानीथ। यदि मामज्ञास्यत, तर्हि मम पितरमप्यज्ञास्यत।
\vakya एतानि वचांसि धनागारे यीशुनाकथ्यन्त। स तदा धर्मधामन्युपादिशत्, तथापि कोऽपि तं न दधार, यतस्तदानीमपि तस्य समयोऽनुपस्थित आसीत्।
\vakya ततो यीशुः पुनस्तानवादीत्, अहं गच्छामि, यूयञ्च मां गवेषयिष्यथ, स्वपापे मरिष्यथ च। अहं यत् स्थानं गच्छामि तत्रोपस्थातुं युष्माभिरशक्यं।
\vakya ततो यिहूदीया अवदन्, स किमात्मघाती भवितुमुद्यतस्तद्धेतोश्च वदति, यत् स्थानमहं गच्छामि तत्रोपस्थातुं युष्माभिरशक्यमिति?
\vakya स तदा तानब्रवीत्, यूयमधःस्थानीयाः, अहमूर्ध्वस्थानीयः। यूयमेतज्जगत्सम्बन्धीयाः, नास्म्यहमेतज्जगत्सम्बन्धीयः।
\vakya ततो युष्मभ्यं कथितवान्, यूयं स्वपापेषु मरिष्यथ। यतोऽहं सोऽस्मीति चेन्न विश्वसिथ, तर्हि स्वपापेषु मरिष्यथ।
\vakya ते तदा तं पप्रच्छुः, कस्त्वं? यीशुस्तानुवाच, आदितस्तदेव युष्मभ्यं कथयामि।
\vakya युष्मानधि मया बहु वक्तव्यं बहु विचारणञ्च कर्तव्यं। मत्प्रेषयिता तु सत्यः, तस्य मुखाच्च मया यद्यदश्रावि तदेव जगते कथ्यते।
\vakya स यत् तेषां ज्ञानार्थं पितरमसूचयत् तत्तैर्नाबोधि।
\vakya ततो यीशुस्तानब्रवीत्, मनुष्यपुत्रे युष्माभिरुच्चीकृते ज्ञायिष्यते यदहं सोऽस्मि स्वतश्च किमपि न करोम्यपि तु पित्रा यथादिष्टस्तथैवैतानि भाषे।
\vakya मत्प्रेषयिता च मया सार्धमास्ते, स मामेकाकिनं न त्यक्तवान् यतस्तस्मै यद्यद् रोचते सर्वदा तदेव मया क्रियते।
\vakya स यदैताः कथा अभाषत, तदा बहवस्तस्मिन् व्यश्वसिषुः।
\vakya ततो ये यिहूदीयास्तस्मिन् व्यश्वसिषुस्तान् यीशु र्जगाद, यदि मम वाक्ये स्थिरा भवथ, तर्हि सत्यं यूयं मम शिष्याः स्थ,
\vakya सत्यञ्च ज्ञास्यथ सत्येन स्वाधीनीकारिष्यध्वे च।
\vakya ते तं प्रत्यूचुः, वयमब्राहामस्य वंशः, कदापि कस्यापि दास्यं नाचरितवन्तः। तत् कथं त्वयोच्यते, यूयं स्वाधीना भविष्यथेति?
\vakya यीशुस्तान् प्रत्यवादीत्, सत्यं सत्यं, युष्मानहं वदामि, यः कश्चित् पापमाचरति स पापस्य दासः,
\vakya दासस्तु न शाश्वतं गृहेऽवतिष्ठते, पुत्र एव शाश्वतमवतिष्ठते।
\vakya अतश्चेत् पुत्रेण स्वाधीनीक्रियध्वे, तर्हि सत्यरूपेण स्वाधीना भविष्यथ।
\vakya अहं जाने यद् यूयमब्राहामस्य वंशः, मान्तु जिघांसथ, यतो मदीयवाक्यं युष्मदन्तरे स्थानं नाप्नोति।
\vakya मत्पितुः समीपं मया यद्यददर्शि तदेव कथ्यते, तथा च युष्मत्पितुः समीपं युष्माभि र्यद्यददर्शि तत् क्रियते।
\vakya ते प्रतिभाषमाणास्तमूचुः, अब्राहामोऽस्माकं पिता। यीशुस्तान् वदति, यद्यब्राहामस्य सन्ताना अभविष्यत, तर्ह्यब्राहामस्य क्रिया अकरिष्यत।
\vakya इदानीन्तु मां जिघांसथ, नरं य ईश्वरस्य मुखात् सत्यं श्रुत्वा युष्मभ्यं कथितवान्। अब्राहामेणैतन्नाकारि।
\vakya युष्मत्पितुः क्रिया युष्माभिः क्रियन्ते। ततस्ते तमूचुः, न व्यभिचारजाता वयम्, अस्माकमेकः पितास्ति स ईश्वरः।
\vakya यीशुस्तदा तानुवाच, ईश्वरो यदि युष्माकं पिताभविष्यत्, तर्हि यूयं मयि प्रेमाकरिष्यत, यतोऽहमीश्वरान्निर्गत्यायातः, नैव हि स्वत आगतोऽहं प्रत्युत स एव मां प्रहितवान्।
\vakya मम भाषा किमर्थं युष्माभि र्नाभिज्ञायते? कोरणमेतद् यद् यूयं मम वाक्यं श्रोतुं न श्क्नुथ।
\vakya यूयं पितुर्दियावलस्य सम्बन्धीयाः, युष्मत्पितुरभिलाषानाचरितुकामाश्च। स आदितो नरघातक आसीत्, सत्ये च नावस्थितः, यतस्तस्मिन् सत्यं नास्ति। स यदानृतं भाषते तदा निजस्वं भाषते, यतः सोऽनृतवादी तस्य पिता चास्ति।
\vakya अहन्तु सत्यं भाषे, तत्कारणाद् यूयं मयि न विश्वसिथ।
\vakya मम पापमस्तीति प्रमाणं युष्माकं को ददाति? यदि तु सत्यं भाषे, किमर्थं तर्हि यूयं मयि न विश्वसिथ?
\vakya य ईश्वरसम्बन्धीयः स ईश्वरस्योक्तीः शृणोति। युष्माभिस्ता न श्रूयन्ते, तत्कारणमेतद् यद् यूयमीश्वरसम्बन्धीया न स्थ।
\vakya यिहूदीयास्तदा प्रतिभाषमाणास्तमवादिषुः, किं न युक्तमस्माभिः कथ्यते यत् त्वं शमरीयो भूताविष्टश्चेति।
\vakya यीशुः प्रतिजगाद, नास्म्यहं भाताविष्टः, प्रत्युत मम पितरं सम्मानायामि, यूयञ्च मामवमानयथ।
\vakya नाहं मत्सम्मानान्वेषी, अस्ति त्वेकस्तदन्वेषी तद्विचारयिता च।
\vakya सत्यं सत्यं, युष्मानहं ब्रवीमि, यः कश्चिन्मम वाक्यमनुपालयति, सोऽनन्तकालं यावन्मृत्युं न द्रक्ष्यति।
\vakya यिहूदीयास्तदा तमवदन्, अधुना वयं बुद्ध्वा जानीमो यत् त्वं भूताविष्टः। अब्राहामो ममार, भाववादिनश्च मम्रुः, त्वं पुन र्वदसि, यः कश्चिन्मम वाक्यमनुपालयति, सोऽनन्तकालं यावन्मृत्युं नास्वादयिष्यतीति।
\vakya अस्मत्पितुरब्राहामात् त्वं किं महत्तरः? स हि ममार, भाववदिनोऽपि मम्रुः, त्वमात्मानं कं कुरुषे?
\vakya यीशुः प्रत्युवाच, यद्यहं स्वं सम्मानयामि, तर्हि मम सम्मानः किमपि नास्ति, मम पितैव मत्सम्मानयिता, स युष्माकमीश्वर इति यूयं वदथ,
\vakya तथापि स युष्माकमविदितः, अहन्तु तं जानामि। स ममाविदित इति चेद् वदेयं, तर्हि युष्माभिः सदृशोऽनृतवादी भविष्यामि। अहं तं जानामि तस्य वाक्यमनुपालयामि च।
\vakya युष्मत्पिताब्राहामो मद्दिनदर्शनस्याशया प्रमुमुदे तद् दृष्ट्वा चाह्वष्यत्।
\vakya ततो यिहूदीयास्तमूचुः, इदानीमपि तव पञ्चाद्वर्षपरिमितं वयो नास्ति, त्वं किमब्राहामं दृष्टवान्?
\vakya यीशुस्तानुवाच, सत्यं सत्यं, युष्मानहं ब्रवीमि, अब्राहामस्य जन्मनः प्राक्कालमारभ्याहमस्मि।
\vakya तदा ते तमाहन्तुं प्रस्तरानाददिरे। यीशुस्तु प्रच्छन्नस्तेषां मध्येन व्रजन् धर्मधामतो निःससारेत्थं स्थानान्तरं जगाम च\eoc