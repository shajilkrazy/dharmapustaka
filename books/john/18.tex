\adhyAya
\stitle{यीशोः परकरेषु समर्पितत्वं।}
\vakya सर्वमेतत् कथयित्वा यीशुः स्वशिष्यान् सङ्गिनः कृत्वा किद्रोणाख्यस्रोतसः पारं जगाम शिष्यैः सार्धं तत्रस्थमेकमुद्यानं प्रविवेश च।
\vakya तस्य समर्पयिता यिहूदा अपि तत् स्थानमजानात्, यतो यीशुः पुनः पुनः स्वशिष्यैः सह तत्र समवेतः।
\vakya तदा यिहूदाः सैन्यगणं मुख्ययाजकेभ्यः फरीशिभ्यश्च पदातींश्च लब्ध्वोल्कादीपिकास्तैः सज्जितस्तत्रोपतिष्ठते।
\vakya ततो यीशुरात्मानं प्रति यद् भवितव्यं तद् दृष्ट्वा बहिरागत्य तानवादीत्, कं गवेषयथ?
\vakya ते प्रत्यूचुः, नासरतीयं यीशुं। यीशुस्तान् ब्रवीति, अहमेव सः। तैः सार्धं तस्य समर्पयिता यिहूदा अपि तिष्ठन्नासीत्।
\vakya अतोऽहं स इति तेन व्याहृते ते पश्चाद्दिशं व्रजित्वा भूमौ निपेतुः।
\vakya ततः स तान् पुनरप्राक्षीत्, कमन्विष्यथ? तेऽवादिषुः, नासरतीयं यीशुं।
\vakya यीशुस्तान् प्रत्यवादीत्, अहं सोऽस्मीति युष्मान् उक्तवान्। अतो मां चेदन्विष्यथ तर्ह्येतेषां गमनमनुजानीत।
\vakya एतस्याभिप्रायोऽयं यत् तेन कथितमिदं वचनं सिद्धिं भवेत्, त्वं मह्यं यान् दत्तवांस्तेषां कोऽपि मया न हारित इति।
\vakya तदा शिमोनः पित्रोऽसिना सज्जित आसीत्, अतः स तं कोशान्निष्कृष्य महायाजकस्य दासमाहत्य तस्य दक्षिणं कर्णं परिचिच्छेद। दसस्य तस्य नाम मल्क इति।
\vakya ततो यीशुः पित्रं जगाद, तवासिं कोषे निधेहि। पिता मह्यं यत् पानपात्रं दत्तवांस्तत्रस्थं पानीयं किं न पास्यामि?
\stitle{हाननस्यान्तिकं यीशोः गमनं।}
\vakya ततः परं सैन्यगणः सहस्रपति र्यिहूदियानां पदातयश्च यीशुं धृत्वा बद्ध्वा च प्रथमं हाननस्य समीपमनैषुः।
\vakya वत्सरस्य तस्य महायाजको यः कायाफास्तस्य श्वशुरो हाननः।
\vakya कायाफा एव यिहूदीयान् मन्त्रणारूपामिमां कथामुक्तवान्, प्रजावृन्दस्य निमित्तमेकस्य मनुष्यस्य मरणं श्रेय इति।
\stitle{पित्रस्यानङ्गीकारः।}
\vakya यीशुस्तु पित्रेण तदन्यैनैकेन शिष्येण चान्वगम्यत। स शिष्यो महायाजकस्य परिचित आसीत्, यीशुना सार्धं महायाजकस्य (निकेतनस्य) प्राङ्गणं प्रविवेश च।
\vakya पित्रस्तु बहि र्गोपुरसमीपं तिष्ठन्नासीत्। ततो महायाजकस्य परिचितः सोऽन्यतरः शिष्यो बहि र्गत्वा द्वाररक्षिकया दास्या संलप्य पित्रमभ्यन्तरमानैषीत्।
\vakya सा द्वारक्षिका दासी तदा पित्रं जगाद, त्वमपि नरस्य तस्य शिष्याणामेकः? सोऽवादीत्, नास्मि।
\vakya दासेषु पदातिषु च शीतकारणाद् अङ्गारैरग्निं प्रज्वाल्य तिष्ठत्सु तापं सेवमानेषु च पित्रोऽपि तैः सार्धं तिष्ठंस्तापमसेवत।
\stitle{महायाजकसमक्षं यीशोर्विचारः।}
\vakya महायाजकस्तदा यीशुं तस्य शिष्यानुपदेशञ्चाप्राक्षीत्। 
\vakya यीशुस्तं प्रतिजगाद, अहं जगते स्पष्टमकथयं। यिहूदीया यत्र नियतं समागच्छन्ति सर्वदा तत्र समाजगृहे धर्मधाम्नि चोपादिशं, गुप्तं किमपि न कथितं मया।
\vakya भवान् मां कथं पृच्छति? ये श्रुतवन्तस्ते मया किमुक्तास्तेनेव तत् पृच्छतु। पश्यतु मया यत् कथितं तत् तै र्ज्ञायते।
\vakya इदं भाषितवति तस्मिन् तत्र तिष्ठन्नेकः पदातिः करेण यीशुमाहत्य जगाद, महायाजकं किमित्थं प्रतिभाषसे?
\vakya यीशुस्तं प्रत्यवादीत्, यद्ययुक्तं व्याहृतं मया तर्हि तस्ययुक्तस्य प्रमाणं देहि, युक्तन्तु यदि व्याहृतं तर्हि कथं मां प्रहरसि?
\vakya ततो हाननस्तं बद्धं महायजकस्य कायाफाः समीपं प्राहिणोत्।
\stitle{पित्रेण यीशुः पुनरपि अस्वीकृतः।}
\vakya तदापि शिमोनः पित्रस्तिष्ठन् तापमसेवत। ततः केचित् तं वदन्ति, त्वमपि किं तस्य शिष्याणामेकः? स निह्नुत्य जगाद, नास्मि।
\vakya पित्रो यस्य कर्णं छिन्नवांस्तस्य ज्ञाति र्महायाजकस्यैको दासस्तदाब्रवीत्, उद्याने तेन सार्धं तिष्ठंस्त्वं किं न लक्षितो मया?
\vakya ततः पित्रः पुन र्निजुह्नुवे तत्क्षणञ्च कुक्कुटो रुराव।
\stitle{पीलातसमक्षं यीशोर्विचारः।}
\vakya ततो यीशुस्तैः कायाफाः सकाशात् राजगृहं नीतः। तदा प्रत्यूष आसीत्। ते तु राजगृहं न प्रविविशुः, यतस्तेऽशौचं परिहर्तुं निस्तारोत्सवीयं भाज्यं भाक्तुञ्चैच्छन्।
\vakya ततः पीलातस्तेषां समीपं बहिरागत्य बभाषे, नरस्यैतस्य विरुद्धं यूयं कमभियोगमुपस्थापयथ?
\vakya ते प्रतिभाषमाणास्तमूचुः, अयं यदि दुष्कर्मी नाभविष्यत्, तर्ह्यस्माभि र्भवतो हस्ते न समार्पयिष्यत।
\vakya ततः पीलातस्तानब्रवीत्, यूयमेव तं गृहीत्वा युष्मदीयव्यवस्थानुरूपं तस्य विचारं कुरुत। यिहूदीयास्तदा तमूचुः, कमपि हन्तुमस्माकं क्षमता नास्तीति।
\vakya एतस्याभिप्रायोऽयं यत् तेन कीदृशो मृत्यु भोक्तव्यस्तत्सूचकं तस्य वचनं सिद्धिं भवेत्।
\vakya ततः पीलातो राजगृहं पुनः प्रविश्य यीशुमाहूय जगाद, त्वं किं यिहूदीयानां राजा?
\vakya यीशुस्तं प्रत्यवादीत्, भवानेतत् स्वतो भाषते, किं वान्येन मम कथामुक्तः?
\vakya पीलातः प्रतिबभाषे, किं यिहूदीयोऽहं? तवैव जात्या मुख्ययाजकैश्च त्वं मयि समर्पितः। किमकारि त्वया?
\vakya यीशुः प्रत्युवाच, नास्ति मम राज्यम् एतज्जगत्सम्बन्धीयं। मम राज्यं यद्येतज्जगत्सम्बन्धीयमभविष्यत् तर्हि यिहूदीयानां हस्ते मत्समर्पणस्य निवारणार्थं मम भृत्याः (प्राणपणेन) व्यचेष्टिष्यन्त। वास्तवन्तु नास्ति मम राज्यमैहिकं।
\vakya ततः पीलातस्तमब्रवीत्, तर्हि नूनं राजासि त्वं? यीशुः प्रतिबभाषे भवांस्तथ्यं ब्रवीति, यतोऽस्म्यहं राजा। एतदर्थमजनिषि, एतदर्थं जगदागमञ्च, यत् सत्यार्थे मया साक्ष्यं दातव्यं। यः कश्चित् सत्यसम्बन्धीयः स मम वाणीं शृणोति।
\vakya पीलातस्तमाह, सत्यं किं? इदमुक्त्वा स पुन र्यिहूदीयानां समीपं बहि र्गत्वा तान् ब्रवीति, मया तस्मिन् कोऽपि दोषो न लक्ष्यते,
\vakya निस्तारोत्सवे तु काराबद्ध एको नरो युष्मदर्थं मया मोच्यते युष्माकमियं रीतिरस्ति। अतो यिहूदीयानां राजा युष्मदर्थं मया मोच्यत इदं किं वाञ्छथ?
\vakya ततः सर्वे पुनरुच्चैः स्वरेणावदन्, मैवायमपि तु बाराब्बाः। स बाराब्बा दस्युरासीत्\eoc