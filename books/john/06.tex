\adhyAya
\stitle{पञ्चसहस्रलोकानां भोजनं।}
\vakya ततः परं यीशु र्गालीलस्थस्य तिबिरियाख्यह्रदस्य पारं प्रतस्थे।
\vakya महाजननिवहश्च तमन्वगच्छत्, यतो रोगिषु तेन क्रियमाणान्यभिज्ञानार्थकर्माणि तैरदृश्यन्त।
\vakya एकदा यीशु र्गिरिमारुह्य स्वशिष्यैः सार्धं तत्रोपविष्टवान्।
\vakya तदा यीहूदीयानां निस्तारोत्सव आसन्न आसीत्।
\vakya ततो यीशुरूर्धदृष्टिं कृत्वा स्वसमीपमागच्छन्तं महान्तं जननिवहं निरीक्ष्य फिलिपं पप्रच्छ, एतेषां भोजनार्थं कुत्रास्माभिः पूपाः क्रेतेव्याः?
\vakya स तं परीक्षमाण इदमवदत्, यतः स किं कर्तुमुद्यत आसीत् तदजानात्।
\vakya फिलिपस्तं प्रत्यवादीत्, द्विशतमुद्रापादमूल्यः पूपा अप्येतेषां निमित्तं तथा पर्याप्ता न भविष्यन्ति यथैतेषामेकैकस्तोकं लप्स्यते।
\vakya तदीयशिष्याणामेकोऽर्थतः पित्रस्य भ्रातान्द्रियस्तं ब्रवीति,
\vakya अस्त्यत्र क्षुद्र एको बालकः, तत्समीपे पञ्च यावपूपा मीनौ च द्वौ क्षुद्रौ विद्यन्ते। एतावतः प्रति तु तत् किं?
\vakya यीशुस्तदाब्रवीत्, यूयं तान् मानवान् भोजनायोपवेशयत। तत्र प्रभूतं शष्पमासीत्। ये च पुरुषा उपविविशुस्ते संख्यया प्रायेण पञ्च सहस्राणि।
\vakya ततः परं यीशुस्तान् पूपान् जग्राह, धन्यवादं कृत्वा च शिष्येभ्यो ददौ, पुनश्च शिष्याः समासीनेभ्यो मनुष्येभ्यो ददुः। तथैव तयो र्मीनयोस्तेभ्यो यथेष्टमदीयत।
\vakya तेषु तु तृप्तेषु स स्वशिष्यान् जगाद, अवशिष्टा भग्नांशा युष्माभिः सञ्चीयन्तां, किमपि मा विनश्यतु।
\vakya अतस्ते तेषां पञ्चानां यावपूपानां भोक्तृभिः शेषितान् भग्नांशान् सञ्चित्य तै र्द्वादश डल्लकान् पूरयामासुः।
\vakya यीशुना कृतं तदभिज्ञानार्थकर्म दृष्ट्वा मनुष्या अवदन्, सत्यम्, अयं स भाववादी येन जगत्युपस्थातव्यमिति।
\vakya अतस्ते मां राजानं चिकीर्षन्तो मां धर्तुमागमिष्यन्तीति ज्ञात्वा यीशुरेकाकी पुन र्गिरिमाशिश्राय।
\stitle{तोयोपरि ख्रीष्टस्य पदव्रजनं।}
\vakya तस्य शिष्यास्तु सन्ध्यायामुपस्थितायां समुद्रतीरमवरूढा नावं प्रविश्य समुद्रपारस्थं कफरनाहूमं गन्तुं प्रवृत्ताश्च।
\vakya ततः परम् अन्धकारे जातेऽपि यीशुस्तेषां समीपं नागतवान्
\vakya समुद्रश्च महावायोराघातेनाक्षोभ्यत।
\vakya इत्थं वहमानास्ते यदा सार्धैकं क्रोशं क्रोशद्वयं वा गतास्तदा यीशुं समुद्रस्योपरि व्रजन्तं नावो निकटमुपस्थितञ्च लक्षयामासुरनेन बिभ्युश्च।
\vakya स तु तानाह, अयमहं, मा भैष्ट।
\vakya ततस्ते तं नावि ग्रहीतुमरोचयन्, नौश्च तत्क्षणं तेषां गन्तव्ये स्थल उपतस्थे।
\stitle{जीवनदायि खाद्यं।}
\vakya तत्परदिने समुद्रपारे जननिवहस्तिष्ठन्नासीत्। तदीयशिष्यैः प्रतिष्टात् पोतादन्यः पोतस्तत्र नासीत्, तदीयशिष्याणां प्रवेशकाले च सा नौ र्यीशुना न प्रविष्टा केवलं तदीयशिष्यैरेव प्रविष्टैतत् तै र्जनैरदर्शि।
\vakya तत्पश्चात्तु तिबिरियात आगता अन्या नावस्तत्स्थानसमीपमुपतस्थिरे, यत्र कृते प्रभुना धन्यवादे पूपास्तै र्भुक्ताः।
\vakya अतोऽत्र यीशुर्नास्ति तदीयशिष्याश्च न सन्तीति दृष्ट्वा ते जना अपि नौका आरुह्य यीशुमन्विष्यन्तः कफरनाहूममाजग्मुः
\vakya समुद्रपारे तं प्राप्त च पप्रच्छुः, रब्बिन्, भवानत्र कदोपस्थितवान्?
\vakya यीशुस्तान् प्रतिजगाद, सत्यं सत्यं, युष्मानहं ब्रवीमि, यूयमभिज्ञानानि दृष्टवन्तो नैतत्कारणान्मामन्विष्यथ, प्रत्युत तान् पूपान् भुक्त्वा यत् तृप्तवन्तस्तत्कारणादेव।
\vakya कार्यं मा क्रियतां युष्माभिः क्षीयमाणस्य भक्ष्यस्य लिप्सया, क्रियतान्त्वनन्तजीवनं यावत् स्थायिनो भक्ष्यस्य लिप्सया। मनुष्यपुत्रेण तद् युष्मभ्यं दातव्यं। स हि पित्रेश्वरेण मुद्राङ्कितः।
\vakya ततस्ते तमवादिषुः, ईश्वरादिष्टानि कार्याणि चिकीर्षुभिरस्माभिः किं कर्तव्यं?
\vakya यीशुः प्रतिभाषमाणस्तानुवाच, ईश्वरादिष्टं कार्यमिदं, यत् तेन यः प्रहितो युष्माभिस्तस्मिन् विश्वसितव्यं।
\vakya ततस्ते तमाहुः, त्वं नु किमभिज्ञानार्थकर्म करोषि यद् दृष्ट्वास्माभिस्त्वयि विश्वसितव्यं। त्वं किं समुपार्जयसि?
\vakya अस्मत्पूर्वपुरुषा मरौ मान्नामभुञ्जत, यतो लिखितमास्ते, स तेभ्यो भोजनार्थं स्वर्गोत्पन्नं खाद्यं ददाविति।
\vakya ततो यीशुस्तान् अवादीत्, सत्यं सत्यं, युष्मानहं ब्रवीमि, मोशि र्युष्मभ्यं स्वर्गोत्पन्नं खाद्यं न दत्तवान्, मम पिता तु युष्मभ्यं प्रकृतं स्वर्गोत्पन्नं खाद्यं ददाति।
\vakya यत ईश्वरीयं खाद्यं तदेव यत् स्वर्गादवतरति जगते जीवनं ददाति च।
\vakya ततस्ते तमूचुः, प्रभो, सर्वदास्मभ्यं तत् खाद्यं ददातु।
\vakya यीशुस्तु तानुवाच, अहमेव जीवनदायि खाद्यं। मत्समीपं य आगच्छति, स नैव क्षोत्स्यति, मयि यश्च विश्वसिति स कदापि न तर्षिष्यति।
\vakya युष्मांस्त्वहमुक्तवान्, यन्मां दृष्ट्वापि न विश्वसिथ।
\vakya पित्रा मह्यं यद्यद्दीयते सर्वं तन्मत्समीपमायास्यति। यश्च मत्समीपमायाति, स मया नैव बहि र्निक्षेप्स्यते।
\vakya यतो न मदीयाभीष्टमपि तु मत्प्रेषयितुरभीष्टं कर्तुमहं स्वर्गादवतीर्णः।
\vakya मत्प्रेषयितुः पितुरभीष्टञ्चेदं यन्मह्यं तेन यद्यद् दत्तं तस्य किमपि मया न हारयितव्यमपि त्वन्तिमदिने सर्वं मयोत्थापयितव्यं।
\vakya मत्प्रेषयितुरभीष्टञ्चेदं यत् पुत्रं निरीक्ष्य यः कश्चित् तस्मिन् विश्वसिति, तेनानन्तं जीवनं प्राप्तव्यमन्तिमदिने च स मयोत्थापयितव्यः।
\vakya अथ स्वर्गादवतीर्णं खाद्यमहमिति यीशुना युदुक्तं, तत्कारणाद् यिहूदीयास्तमधि विवदितुं प्रवृत्ताः।
\vakya तेऽवदन्, असौ किं न योषेफस्य पुत्रः स यीशु र्यस्य पितरं मातरञ्च जानीमः? कथं तर्हि स वदति, स्वर्गादवतीर्णोऽहमिति?
\vakya यीशु प्रतिभाषमाणस्तानब्रवीत्, मा विवदध्वं मिथः।
\vakya अपरः कोऽपि मत्समीपमायातुं न शक्नोति, केवलं स यो मत्प्रेषयित्रा पित्राकृष्यते, स चान्तिमदिने मयोत्थापयिष्यते।
\vakya भाववादिनां ग्रन्थे लिखितमास्ते,
\begin{poem}
\startwithline “भविष्यन्ति हि ते सर्व ईश्वरेणैव शिक्षिताः।”
\end{poem}
@V\eoc