\adhyAya
\stitle{लासारस्य मरणं।}
\vakya अथ बैथनियानिवासी लासाराभिधः कश्चिन्नरो व्याधित आसीत्। बैथनिया मरियमस्तस्या भगिन्या मार्थायाश्च ग्राम आसीत्।
\vakya मरियम् सा योषिद् या सुगन्धितैलेन प्रभुं समनीनिपत् स्वकचैश्च तच्चरणौ पर्यमार्क्षीत्। तस्या एव भ्राता लासारो व्याधित आसीत्।
\vakya अतस्तस्य भगिन्यौ दूतं प्रहित्य तस्मै निवेदयासतुः, प्रभो, पश्यतु, भवान् यस्मिन् प्रेम करोति स व्याधितः।
\vakya यीशुस्तु तच्छ्रुत्वाब्रवीत्, व्याधिरयं न मृत्यवे प्रत्युतेश्वरस्य महिमार्थः, ईश्वरस्य पुत्रो यथा तेन महीयेत।
\vakya यीशु र्मार्थायां तस्या भगिन्यां (मरियमि) लासारे च प्रेमाकरोत्।
\vakya स तु यत्राविद्यत तत्र तद्व्याधिसंवादश्रवणात् परमपि दिनद्वयमवतस्थे।
\vakya ततः परन्तु स्वशिष्यानवादीत्, पुनर्यिहूदीया गम्यतामस्माभिः।
\vakya शिष्यास्तं वदन्ति, रब्बिन्, दिनेऽमुष्मिन् यिहूदीयाः प्रस्तराघातेन भवन्तं मारयितुमयतन्त, भवांश्च किं पुनस्तत्र गच्छति?
\vakya यीशुः प्रतिबभाषे, दिवसे किं न द्वादश घटिका भवन्ति? दिवा यः परिव्रजति स न स्खलति, यतः स जगतोऽस्य ज्योतिः पश्यति,
\vakya निशायान्तु यः परिव्रजति स स्खलति, यतो ज्योति र्न तदन्तःस्थं।
\vakya एतत्कारणात् परं स तानब्रवीत्, अस्मद्बन्धु र्लासारः स्वपिति, तं जागरयितुन्तु गच्छामि।
\vakya शिष्यास्तदा तमूचुः, प्रभो, स यदि स्वपिति, तर्हि तरिष्यति।
\vakya यीशुस्तस्य मृत्युमुद्दिश्य कथितवान्, ते त्वमन्यन्त, स निद्रारूपस्वपनमुद्दिश्य कथयतीति।
\vakya अतएव यीशुस्तदा तान् स्पष्टमब्रवीत्, लासारो ममार, यूयञ्च विश्वसिष्यथेति बुद्ध्वा युष्मदर्थं तत्र मदनुपस्थितत्वादहं हष्यामि।
\vakya तद्यथा तथास्तु, तत्समीपं गम्यतां।
\vakya तदा थोमा अर्थतो दिदुमो (जमकः) सहशिष्यान् जगाद, अस्माभिरपि गम्यतां, तेन सार्धं तथा मिरिष्यामः।
\stitle{यीशुः पुनरुत्थानं जीवनञ्च।}
\vakya इत्थं यीशु र्यदागमत् तदाजानाद्, यदादिनचतुष्टयात् स (लासारः) शवागारे शेते।
\vakya बैथनिया च यिरूशालेमस्य समीपस्था प्रायेण क्रोशैकदूरासीत्,
\vakya बहवो यिहूदीयश्च मार्थामरियमो र्भ्रातृशोकसान्त्वनार्थं तयो र्गृहमागतवन्तः।
\vakya मार्था तदा यीशोरागमनसंवादं श्रुत्वैव तं प्रत्युज्जगाम, मरियम् तु गृह आसीनावातिष्ठत।
\vakya मार्था तदा यीशुमवादीत्, प्रभो, भवान् यद्यत्रास्थास्यत्, तर्हि मम भ्राता नामरिष्यत्;
\vakya जाने त्वधुनापि भवान् ईश्वरं यद्यत् प्रार्थयिष्यते, तदीश्वरेण भवते दायिष्यते।
\vakya यीशुस्तं ब्रूते, तव भ्राता पुनरुत्थास्यति।
\vakya मार्था तं ब्रूते, अन्तिमदिने पुनरुत्थाने स उत्थास्यतीति जाने।
\vakya यीशुस्तामुवाच, अहं पुनरुत्थानं जीवनञ्च। मयि यो विश्वसिति स मृत्वापि जीविष्यति,
\vakya यः कश्चिच्च जीवति मयि विश्वसिति च, सोऽनन्तकालेऽपि नैव मरिष्यति। अत्र त्वं किं विश्वसिषि?
\vakya सा तं ब्रवीति, तथैव, प्रभो। जगति येनागन्तव्यम् ईश्वरस्य पुत्रः स ख्रीष्टो भवानेवेति विश्वासो मयाकारि।
\vakya इत्युक्त्वा सापजगाम,गुप्तञ्च स्वभगिनीं मरियममाह्वयन्ती तामाह, गुररुपस्थितोऽस्ति त्वामाह्वयति च।
\vakya सा तच्छ्रुत्वैव सत्वरमुत्थाय तदन्तिकमायाति।
\vakya यीशुस्तदापि ग्रामं न प्रविष्टवान्, मार्था यत्र तं प्राप्तवती तत्रैवासीत्।
\vakya ततो मरियमा सार्धं गृहे विद्यमाना ये यिहूदीयास्तामसान्त्वयंस्ते तां सत्वरमुत्तिष्ठन्तीं निर्गच्छन्तीञ्च दृष्ट्वा, सा रोदितुं शवागारं यातीत्युक्त्वा च तामनुजग्मुः।
\vakya यीशुस्तु यत्रासीत् मरियम् तत्रोपस्थाय तं दृष्ट्वा च तच्चरणयोः प्रणिपत्योवाच, प्रभो, भवान् यद्यत्रास्थास्यत्, तर्हि मम भ्राता नामरिष्यत्।
\vakya तदा यीशुस्तं रुदतीं तया सार्धमागतान् यिहूदीयांश्चापि रुदतो विलोक्यात्मन्यधैर्यं गत्वा संक्षुभ्य च पप्रच्छ, स युष्माभिः कुत्र शायितः?
\vakya ते तं वदन्ति, प्रभो, एत्य निरीक्षतां।
\vakya यीशुरश्रूण्यमुञ्चत्।
\vakya ततो यिहूदीया अवदन्, पश्यत, स तस्मिन् कीदृक् प्रेमाकुरुत।
\vakya तेषां केचित्त्वाहुः,तस्यान्धस्य नेत्रे प्रसन्नीचकार योऽसौ किमस्य मृत्युं निवारयितुमपि समर्थो नासीत्?
\stitle{लासारस्योत्थापनं।}
\vakya यीशुस्तदान्तरे पुनरधैर्यं गत्वा शवागारसमीपमागच्छति। तद् गुहाकारं पाषाणेनापिहितञ्चासीत्।
\vakya यीशु र्ब्रवीति, तं पाषामपसारयत। तस्य मृतस्य भगिनी मार्था तं वदति, प्रभो, इदानीं स पूयते, यत आचतुर्भ्यो दिनेभ्यः स मृतः।
\vakya यीशुस्तां ब्रवीति, त्वं यदि विश्वसिषि, तर्हीश्वरस्य महिमानं द्रक्ष्यसीति किं नोक्ता मया?
\vakya ततः स मृतो यत्राशेत, तत्र स्थितः पाषाणस्तैरपसारितः। यीशुश्चोर्ध्वदृष्टिं कृत्वा व्याजहार, पितः, त्वामहं धन्यं वदामि, यतस्त्वं मे श्रुतवान्।
\vakya त्वं सर्वदा मे शृणोषि तदजानाम् अपि त्वयाहं यत् प्रहितस्तत्रैते विश्वसिष्यन्तीति बुद्ध्वा परितः स्थितस्य जननिवस्य कारणादिदं मयोदाहारि।
\vakya इदमुक्त्वा स प्रोच्चैः क्रोशन् व्याजहार, लासर, बहिरागच्छ।
\vakya तदा स मृतो बहिरागमत्, स च दीर्घवस्त्रखण्डै र्बद्धचरणहस्तः स्वेदहरवस्त्रेण वेष्टितमुखश्चासीत्। यीशुस्तान् ब्रवीति, बन्धनानि मुक्त्वेमं गन्तुमनुजानीत।
\stitle{महायाचकफरीशिनां विपक्षत्वकरणं।}
\vakya तदा मरियमः समीपमागता बहवो यिहूदीया यीशुना यद्यदकारि तद् दृष्ट्वा तस्मिन् व्यश्वसिषुः,
\vakya केचित्तु फरीशिनामन्तिकं गत्वा यीशुना यद्यदकारि तत् तेभ्यो निवेदयामासुः।
\vakya ततो मुख्ययाजकाः फरीशिनश्च मन्त्रिसभां समाहूयावादिषुः, किं कारवाम? स नरो बहून्यभिज्ञानार्थकर्माणि करोति।
\vakya यदि तस्य वृत्तिमित्थं सहामहे, तर्हि सर्वे तस्मिन् विश्वसिष्यन्ति, रोमीयाश्चागत्यास्माकं स्थानञ्च प्रावृन्दञ्च हरिष्यन्ते।
\vakya तेषां मध्ये गणितस्त्वेको नरोऽर्थतो यः कायाफास्तस्य संवत्सरस्य महायाजक आसीत् स तदा तानुवाच, यूयं किमपि न जानीथ नापीदं तर्कयथ,
\vakya यत् प्रजावृन्दस्य निमित्तमेकस्य मरणमस्मद्धितकरं, न च कृत्स्नस्य प्रजावृन्दस्य विनाशः।
\vakya एतत् तेन न स्वतो व्याहारि, प्रत्युत स तस्य संवत्सरस्य महायाजक आसीत्, तत्कारणात् तेन भावोक्तिरियं व्याहारि यत् प्रजावृन्दस्य निमित्तं यीशुना मर्तव्यं,
\vakya नापि च केवलं तस्य प्रजावृन्दस्य निमित्तं, प्रत्युत विकीर्णानाम् ईश्वरसन्तानानाम् एकत्र सङ्ग्रहणार्थमपि।
\vakya तद्दिनमारभ्य ते तस्य वधार्थम् अमन्त्रयन्त।
\vakya अतएव यीशु र्न पुनः सप्रकाशं यिहूदीयानां मध्ये पर्याटत्, प्रत्युत तस्मात् स्थानान्मरुसमीपस्थम् इफ्रयिमाख्यं ग्रामं प्रस्थाय स्वशिष्यैः सार्धं तत्रैवावर्तत।
\vakya तदा यिहूदीयानां निस्तारोत्सव आसन्न आसीत्, बहवो मनुष्याश्च शुचित्वसाधनार्थं निस्तारोत्सवात् प्राक् जनपदाद् यिरूशालेमं जग्मुः।
\vakya ते तदा यीशुमगवेषयन् धर्मधाम्नि स्थित्वा च मिथोऽवदन्, किं मन्यध्वे, स किमुत्सवे नैवोपस्थास्यत इति।
\vakya मुख्ययाजकाः फरीशिनश्च तं धर्तुमिच्छन्त आज्ञामपि दत्तवन्तः, यथा स कुत्र वर्तते तत् स्थानं यो वेत्ति स तत् सूचयोदिति\eoc