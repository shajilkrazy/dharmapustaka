\adhyAya
\stitle{यिरुशालेमे यीशोरुपदेशनम्।}
\vakya ततः परं यीशु र्गालीले प्रर्याटत्, यतो यिहूदीयास्तं जिघांसन्तीति ज्ञात्वा स पुन र्यिहूदियायां पर्यटितुं नैच्छत्।
\vakya यिहूदीयानामुटजनिर्माणाख्य उत्सवे त्वासन्ने सति तस्य भ्रातरस्तमवदन्,
\vakya त्वमतः स्थानादुत्थाय यिहूदियां याहि, त्वया यानि क्रियन्ते, तानि कर्माणि तव शिष्यैरप्यालोक्यन्तां,
\vakya यतः स्वयं सप्रकाशो भवितुमीहमानः कोऽपि गुप्तं नाचरति। त्वं यद्येतानि साधयसि, तर्ह्यात्मानं जगतः प्रत्यक्षं कुर्विति।
\vakya यतस्तदीयभ्रातरोऽपि तस्मिन्न व्यश्वसन्।
\vakya यीशुस्तदा तानवादीत्, मम समय इदानीमप्यनुपस्थितः, युष्माकं समयस्तु सर्वदा प्रस्तुतः।
\vakya जगद् युष्मान् द्वेष्टुं न शक्नोति, मान्तु द्वेष्टि, यतस्तस्मिन्नयेदं साक्ष्यं दीयते यत् तस्य कर्माणि दुष्टानि।
\vakya यूयमेवेनमुत्सवं गच्छत। अहमधुनाप्येनमुत्सवं न गच्छामि, यतो मम काल इदानीमप्यसम्पूर्णः।
\vakya तानिदमुक्त्वा स गालीलेऽवतस्थे।
\vakya प्रस्थितेषु तु तदीयभ्रातृषु सोऽपि तमुत्सवं जगाम, न सप्रकाशमपि तु गुप्तमिव।
\vakya ततो यिहूदीया उत्सवे तं गवेषयन्तोऽवदन्, स नरः कुत्र?
\vakya जननिवहेषु च तमधि बहुतरो विवादः समजायत। केचिदवदन्, स भद्र इति। केचित् पुनरवदन्, न तथा, प्रत्युत स जननिवहमुन्मार्गगामिनं करोतीति।
\vakya तथापि यिहूदीयानां भयात् कोऽपि तमधि सप्रकाशं नाकथयत्।
\vakya उत्सवस्यार्धकाले तु यूशु र्धर्मधाम गत्वोपदेष्टुं प्रववृते।
\vakya अनेन यिहूदीया विस्मयापन्ना अवदन्, असावनधीत्यापि कथं शास्त्रविज्जातः?
\vakya यीशुः प्रतिभाषमाणस्तानवादीत्, मदीयोपदेशो न मम, स मत्प्रेषयितुरेव।
\vakya कश्चिद् यदि तस्याभीष्टमाचरितुं रोचयति, स तर्ह्युपदेशमधि ज्ञास्यति स ईश्वरोत्पन्नोऽथवाहं स्वतो भाष इति।
\vakya यः स्वतो भाषते स स्वीयगौरवमीहते, यस्तु स्वप्रेषयितु र्गौरवमीहते, स सत्यस्तस्मिंश्चाधर्मो नास्ति।
\vakya व्यवस्था किं न मोशिना युष्मभ्यं दत्ता? तथापि युष्माकं कोऽपि व्यवस्थां न समाचरति। यूयं किमर्थं मां जिघांसथ?
\vakya जननिवहः प्रत्यवादीत्, त्वं भूताविष्टः, कस्त्वां जिघांसति?
\vakya यीशुः प्रतिभाषमाणस्तानब्रवीत्, कर्मैकमकारि मया, तत्र सर्वे यूयमाश्चर्यं मन्यध्वे।
\vakya तत्कारणाद् (वदामि), मोशि र्युष्मभ्यं त्वक्छेदस्य विधिं दत्तवान्, सोऽपि न मोशित उत्पन्नः प्रत्युत पितृभ्यः; यूयञ्च विश्रामवारे मनुष्यस्य त्वक्छेदं कुरुथ।
\vakya मोशे र्व्यवस्था यन्न लंघ्यते, तदर्थं यदि विश्रामवारे मनुष्येण त्वक्छेदः सोढव्यस्तर्हि विश्रामवारे सर्वाङ्गो मनुष्यो मया स्वस्थीकृतोऽत्र किं मह्यं क्रुध्यथ?
\vakya चाक्षुषं विचारं मा कुरुत, न्याय्यमेव विचारं कुरुत।
\vakya ततो यिरूशालेमनिवासिनां केचिदवदन्, (नायका) यं हन्तुं यतन्ते, मनुष्योऽयं किं न स एव?
\vakya पश्यत च स सप्रकाशं भाषते तैश्च किमपि नोच्यते। अयं ख्रीष्ट एवेति किं नायकैः सत्यमज्ञायि?
\vakya अयन्तु कुत उत्पन्नस्तदस्माभिर्विदितं। ख्रीष्टस्तु यदायास्यति तदा स कुत उत्पन्नस्तत् केनापि न ज्ञायिष्यते।
\vakya ततो यीशु र्धर्मधामन्युपदिशन् प्रोच्चैराह, यूयं माञ्च जानीथ कुतश्चाहमुत्पन्नस्तदपि जानीथ। अहन्तु न स्वत आगतः प्रत्युत सत्योऽस्ति मत्प्रेषयिता, स युष्माभिरविदितः।
\vakya अहन्तु तं जानामि, यतोऽहं तत्सकाशाद् (आगतो) ऽस्मि, स च मां प्रहितवान्।
\vakya ततस्ते तं धर्तुमयतन्त, तथापि कोऽपि तस्मिन् हस्तं नार्पयामास, यतस्तदानीमपि तस्य समयोऽनुपस्थित आसीत्।
\vakya जननिवहस्य बहवो नराश्च तस्मिन् व्यश्वसिषुरवादिषुश्च, ख्रीष्टो यदायास्यति, तदानेन यानि कृतानि तेभ्योऽधिकान्यभिज्ञानार्थकर्माणि किं स करिष्यति?
\vakya अथ जननिवहस्तमधीत्थं यद् व्यवदत तच्छ्रुत्वा फरीशिनो मुख्ययाजकाश्च तं धर्तुं पदातीन् प्रेषयामसुः।
\vakya अतो यीशुस्तान् जगाद, इतः परमहं स्तोकं कालं युष्माभिः सार्धं स्थास्यामि, ततो मत्प्रेषयितुः समीपं यास्यामि।
\vakya यूयं मां गवेषयिष्यथ न त्ववाप्स्यथ। अहञ्च यत्र विद्ये तत्रोपस्थातुं युष्माभिरशक्यं।
\vakya ततो यिहूदीया मिथोऽवदन्, अनेन किं स्थानं गन्तव्यं यत् सोऽस्माभिरप्राप्यो भविष्यति? अयं किं यूनानीयानां मध्ये विकीर्णजनानां समीपं गत्वा यूनानयान् शिक्षयिष्यति?
\vakya कीदृशमिदमनेन कथितं वचनं, यूयं मां गवेषयिष्यथ न त्ववाप्स्यथेत्यपरञ्च यत्राहं विद्ये तत्रोपस्थातुं युष्माभिरशक्यमिति?
\vakya अथोत्सवस्यान्तिमदिवसेऽर्थतो महादिवसे यीशुस्तिष्ठन्नुच्चरवेण बभाषे, यः पिपासुः स मदन्तिकमात्य पिबतु।
\vakya मयि यो विश्वसिति, शास्त्रीयोक्त्यनुसारेण तस्यान्तरादमृततोयस्य सरितः प्रस्रविष्यन्ति।
\vakya तस्मिन् विश्वासिभि र्य आत्मा लप्स्यते तमधि तेनैतदकथ्यत। पवित्रस्यात्मनो वितरणं तदाप्यसम्भूतमासीत्, यतस्तदापि यीशुरप्राप्तप्रताप आसीत्।
\vakya वचनमेतच्छ्रुत्वा जननिवहस्य बहवो जना अवदन्, सत्यम्, अयं स भाववादी।
\vakya अन्येऽवदन्, अयं ख्रीष्टः। अन्ये पुनरवदन्, कथमेतत्? ख्रीष्टेन किं गालीलादागन्तव्यं?
\vakya शास्त्रे किं नेदं लिखितमास्ते यद् दायूदस्य वंशाद् दायूदस्य वसतिग्रामाच्च बैतलेहमात् ख्रीष्टेनागन्तव्यं?
\vakya अतस्तमधि जननिवहस्य भेदः सञ्जातः।
\vakya तेषां केचिच्च तं धर्तुमैच्छन्, तथापि कोऽपि तस्मिन् हस्तं नार्पयामास।
\stitle{महायाजकानां फरीशिनाञ्च ख्रीष्टं प्रति विपक्षता।}
\vakya ततस्ते पदातयो मुख्ययाजकानां फरीशिनाञ्च समीपं प्रत्याजग्मुस्ते च तानप्राक्षुः, स किमर्थं युष्माभि र्नानीतः?
\vakya पदातयः प्रत्यूचुः, स यादृशं भाषते, तादृशं कोऽपि मनुष्य इतिपूर्वं कदापि न भाषितवान्।
\vakya फरीशिनस्तदा तान् प्रत्यूचुः, यूयमपि किमुन्मार्गं नीताः?
\vakya नायकानां फरीशिनां वा कोऽपि किं तस्मिन् विशश्वास?
\vakya प्रत्युत व्यवस्थानभिज्ञोऽसौ जननिवहः शप्तः।
\vakya तदा तेषां मध्ये गणितो यो नीकदीमो रजन्यां तस्यान्तिकमागतवान्, स तान् ब्रवीति,
\vakya प्रथमं मनुष्यस्य मुखाद् वाक्यमश्रुत्वा तेन किं क्रियते तदज्ञात्वा वास्माकं व्यवस्था किं मनुष्यं दोषिणं करोति?
\vakya ते प्रतिभाषमाणास्तमूचुः, त्वमपि किं गालीलीयः? अनुसन्धत्स्वालोचय च यद् गालीलतः कोऽपि भाववादी नोत्पद्यते\eoc