\adhyAya
\stitle{यीशुरेव पन्थाः।}
\vakya युष्माकं हृदयं मैवोद्विजताम्, ईश्वरे विश्वसित, मय्यपि विश्वसित।
\vakya मम पितु र्निकेतने बहवो वासाः सन्ति। नोचेद् युष्मभ्यमकथयिष्यं। युष्मदर्थं हि स्थानं सज्जीकर्तुं गच्छामि।
\vakya यदि च गत्वा युष्मदर्थं स्थानं सज्जीकरोमि, तर्हि पुनरागत्य मत्समीपं युष्मान् ग्रहीष्यामि, यत्राहं वर्ते तत्र यूयमपि यथा वर्तिष्यध्वे।
\vakya यद् स्थानमहं गच्छामि तद् यूयं जानीथ, पन्थानमपि जानीथ।
\vakya थोमास्तं ब्रूते, प्रभो, भवान् किं स्थानं गच्छति, तन्न जानीमः, कथं तर्हि पन्थानं ज्ञातुं शक्नुयाम?
\vakya यीशुस्तं वदति, अहमेव पन्थाः सत्यञ्च जीवनञ्च, नान्येनोपायेन मनुष्यः पितुः समीपमायाति, केवलं मया।
\vakya माञ्चेदज्ञास्यत, तर्हि मत्पितरमप्यज्ञास्यत। अधुनारभ्य च तं जानीथ, तं दृष्टवन्तश्च।
\vakya फिलिपस्तदा तं ब्रवीति, प्रभो, दर्शयत्वस्मभ्यं पितरम् अस्मदर्थं तत् पर्याप्तं भविष्यति।
\vakya यीशुस्तमवादीत्, एतावत्कालं यावदहं युष्माभिः सार्धं वर्ते, फिलिप। तथापि त्वं मां किं न जानासि? मां यो दृष्टवान् स पितरं दृष्टवान्। अतः कथं वदसि, दर्शयत्वस्मभ्यं पितरमिति?
\vakya त्वं किमिदं न विश्वसिषि यदहं पितरि, पिता च मयि स्थितः? युष्मभ्यमहं यानि वचांसि कथयामि तानि न स्वतः कथमाम्यपि तु मय्यवस्थितः पितैव कर्माणि करोति।
\vakya मयि विश्वसित यदहं पितरि पिता च मयि स्थितः। नोचेत् कर्मणां कारणाद् विश्वसित।
\vakya सत्यं सत्यं, युष्मानहं ब्रवीमि, मयि यो विश्वसिति, मया क्रियमाणानि कर्माणि तेनापि कारिष्यन्ते, तेभ्यो महत्तराणि च कारिष्यन्ते, यतो हेतोरहं मत्पितुः समीपं गच्छामि,
\vakya युष्माभिश्च मम नाम्ना यत् किमपि प्रार्थयिष्यते तत् करिष्यामि, पुत्रे पिता यथा महिमान्वितो भविष्यति।
\vakya यदि मम नाम्ना किञ्चन प्रार्थयध्वे तर्ह्यहं तत् करिष्यामि।
\stitle{पवित्रात्मानं प्रेरयितुं प्रतिज्ञा।}
\vakya यदि मां प्रति प्रेम कुरुथ, तर्हि ममाज्ञा अनुपालयत,
\vakya ततोऽहं पितरं याचिष्ये, स च युष्मभ्यमन्यमेकं शान्तिकर्तारं दास्यति युष्माभिः सार्धं येन शाश्वतं स्थातव्यं,
\vakya तं सत्यस्वरूपमात्मानं दास्यति यो जगता ग्रहीतुं न शक्यते, यतः स तेन न दृश्यते नापि वा ज्ञायते। युष्माभिस्तु स ज्ञायते, यतः स युष्मत्समीपमवतिष्ठते युष्मदन्तश्च वर्तिष्यते।
\vakya न परित्यक्ष्यामि युष्मान् अनाथान्, अहं युष्मदन्तिकमागच्छामि।
\vakya स्तोके काले व्यतीते जगन्मां पुन र्न द्रक्ष्यति, यूयन्तु द्रक्ष्यथ, यतोऽहं जीवामि (ततश्च) यूयमपि जीविष्यथ।
\vakya तस्मिन् दिने यूयं ज्ञास्यथ, यदहं मत्पितरि स्थितो यूयञ्च मयि स्थिता अहञ्च युष्मासु स्थितः।
\vakya यो ममाज्ञाः प्राप्तवांस्ता अनुपालयति च स एव मयि प्रेमकारी। यश्च मयि प्रेमकारी स मम पितुः प्रेमपात्रं भविष्यति, अहञ्च तं प्रति प्रेम करिष्यामि स्वं तत्प्रत्यक्षीकरिष्यामि च।
\vakya ईष्करियोतीयादन्यो यिहूदास्तदा तं ब्रूते, प्रभो, कथमेतद् यद् भवान् स्वं जगतः प्रत्यक्षं न कृत्वास्मत्प्रत्यक्षं करिष्यति?
\vakya यीशुः प्रतिभाषमाणस्तमाह, कश्चिद् यदि मां प्रति प्रेम करोति तर्हि मम वाक्यमनुपालयिष्यति, मम पिता च तं प्रति प्रेम करिष्यति, तत आवां तस्य समीपमागमिष्यावस्तेन सह वसतिं करिष्यावश्च।
\vakya यो मां प्रति प्रेम न करोति स मम वाक्यानि नानुपालयति। यच्च वाक्यं युष्माभिः श्रूयते, तन्न मम प्रत्युत मत्प्रेषयितुः पितु र्वाक्यम्।
\vakya युष्माभिः सार्धं वर्तमानोऽहं युष्मभ्यं सर्वमेतत् कथितवान्।
\vakya शान्तिकर्ता त्वर्थतो मन्नाम्नि पित्रा प्रेषयितव्यः पवित्र आत्मा युष्मान् सर्वं शिक्षयिष्यति, मया युष्मभ्यं यद्यत् कथितं तत् सर्वं स्मारयिष्यति च।
\vakya अहं युष्मदर्थं दायमिव शान्तिं त्यजामि, मदीयशान्तिं युष्मभ्यं ददामि। जगद् यथा ददाति नाहं युष्मभ्यं तथा ददामि। युष्माकं हृदयं मैवोद्विजतां मैव वा भीरु भवतु।
\vakya अहं गत्वा युष्मत्समीपमागमिष्यामीति युष्मभ्यं यदचकथं तद् यूयं श्रुतवन्तः। यदि मां प्रति प्रेमाकरिष्यत, तर्ह्यहं पितुः समीपं गच्छामीति यदचकथं तत्रानन्दिष्यत, यतः पिता मत्तो महत्तरः
\vakya तत्सिद्धिकाले च यद् विश्वसिष्यथ, तदर्थमहमिदानीं सिद्धेः प्राग् युष्मभ्यं कथितवान्।
\vakya इतः परमहं युष्माभिः सार्धं न पुन र्बहु सम्भाषिष्ये। यतो जगतोऽस्याधिपतिरागच्छति, मयि च तस्य किमपि नास्ति।
\vakya अहन्तु पितरं प्रति प्रेम करोमि पित्रा च यथादिष्टस्तथाचरामीति जगता ज्ञातव्यम्। उत्तिष्ठत, स्थानादस्मादस्माभिः प्रस्थीयताम्\eoc