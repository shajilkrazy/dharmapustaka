\adhyAya
\vspace{25pt}
\vakya तृतीये दिने तु गालीलस्थकान्नायां विवाहः समभूत्, यीशो र्माता च तत्रासीत्।
\vakya यीशुस्तच्छिष्याश्चापि तस्मै विवाहाय निमन्त्रिता आसन्।
\vakya जाते तदा द्राक्षारसस्य न्यूनत्वे यीशो र्माता तमवादीत्, अमीषां द्राक्षारसो नास्ति।
\vakya यीशुस्तं वदति, नारि, मया सह तव किं कार्यं? मम समय इदानीमप्यनुपस्थितः।
\vakya तस्य माता परिचारकान् ब्रवीति, असौ युष्मान् यद्यद् वक्ति तत् कुरुत।
\vakya तत्र तु यिहूदियानां शुचित्वसाधनार्थमुचितानि पाषाणमयानि षट् महान्ति तोयपात्राण्यासन्, तानि प्रत्येकमाढकद्वयमाढकत्रयं वाधारयन्।
\vakya यीशुस्तान् वक्ति, पात्राणीमानि तोयेन पूरयत। ततस्ते तान्याकर्णं पूरयामासुः।
\vakya तदा स तान् ब्रवीति, इदानीं किञ्चिदादाय भोज्याध्यक्षस्यान्तिकं वहत। ततस्ते किञ्चिदूहुः।
\vakya स भोज्याध्यक्षो यदा तोयं तद् द्राक्षारसेन परिणतमास्वादयत्, तदा तत् कुत आनीतं तन्नाजानात्, यैस्तु तदुत्तारितं ते परिचारका अजानन्।
\vakya ततः स भोज्याध्यक्षो वरमाहूयाह, सर्वमनुष्यः प्रथममुत्तमं द्राक्षारसं परिवेषयति, जने बाहुल्येन पीते च निकृष्टतरं परिवेषयति। त्वन्त्वेतावत्कालं यावदुत्तमं द्राक्षारसं रक्षितवान्।
\vakya गालीलस्थकान्नायां यीशुनाभिज्ञानानामयमारम्भोऽकारि स्वप्रतापश्च प्रत्यक्षीचक्रे। तच्छिष्याश्च तस्मिन् व्यश्वसन्।
\vakya ततः परं स तस्य माता भ्रातरश्च तस्य शिष्याश्च कफरनाहूममवतेरुः, तत्र त्वल्पदिनान्यवतस्थिरे।
\stitle{यीशो र्यिरुशालेमे यानं मन्दिरपरिष्करणञ्च।}
\vakya आसन्ने पुन र्यिहूदीयानां निस्तारपर्वाणि यीशु र्यिरूशालेममारुरोह।
\vakya धर्मधाम्नि च गोमेषकपोतविक्रेतॄन् मुद्राविनिमयकारिणश्चसीनान् दृष्ट्वा
\vakya स दामभिः कशां चरयित्वा सर्वांस्तान् मेषान् गाश्च धर्मधामतो बहिर्द्रावयामास, बणिजाञ्च मुद्राचयान् विकीर्यासनानि न्युब्जीचकार,
\vakya कपोतविक्रेतॄंश्चोवाच, सर्वमेतदतः स्थानादपसारयत, मम पितु र्गृहं मा कार्ष्ट वाणिज्यगृहमिति।
\vakya तच्छिष्याश्च सस्मरु र्यल्लिखितमास्ते,
\begin{poem}
\startwithline “त्वन्निकेतनपक्षीया व्यग्रतां मां ग्रसिष्यति।”
\end{poem}
\vakya ततो यिहूदीयाः प्रतिभाषमाणास्तमवादिषुः, एतत् त्वया यत् कर्तव्यं तस्य किमभिज्ञानमस्मान् दर्शयसि?
\vakya यीशुः प्रतिभाषमाणस्तानब्रवीत्, मन्दिरमिदं भङ्क्त, त्रिषु दिनेष्वहं तदुत्थापयिष्यामि।
\vakya यिहूदीयास्तदा जगदुः षट्चत्वारिंशद्‌वत्सरान् यावन्मन्दिरमिदं निर्मीयते स्म, त्वं कथं त्रिषु दिनेषु तदुत्थापयिष्यसि?
\vakya स तु स्वदेहरूपं मन्दिरमुद्दिश्याकथयत्।
\vakya अतो मृतानां मध्यतस्तस्योत्थानात् परं तस्य शिष्याः सस्मरुर्यत् ते तेनेदमुक्ताः। तदा ते शास्त्रे यीशुना कथिते वचने च व्यश्वसिषुः।
\vakya निस्तारपर्वाणि यदा स यिरूशालेमेऽवर्तत, तदा तेन क्रियमाणान्यभिज्ञानार्थकर्माणि विलोक्य बहवस्तस्य नाम्नि व्यश्वसन्।
\vakya यीशुस्तु तेष्वात्मसमर्पणं नाकार्षीत्, यतः स सर्वानजानात्।
\vakya मनुष्ये परस्य साक्ष्येण तस्य प्रयोजनं वा नाभवत्, यतो मनुष्यस्यान्तरे यदविद्यत, तत् स स्वयमजानात्\eoc