\adhyAya
\stitle{यीशुना जन्मान्धाय नेत्र दानम्।}
\vakya गमनकाले स नरमेकं जन्मान्धं लक्षयामास।
\vakya तस्य शिष्यास्तदा तमप्राक्षुः, रब्बिन्, असौ यदन्धोऽजनि, तादृशं पापं केनाकारि? अमुना किं वामुष्य पितृभ्यां?
\vakya यीशुः प्रत्युवाच, (तादृशं) पापं नामुनाकारि, नापि वामुष्य पितृभ्याम्, प्रत्युतामुष्मिन्नीश्वरस्य क्रियाभिः प्रत्यक्षीभवितव्यम्।
\vakya यावद् दिवसोऽस्ति तावन्मत्प्रेषयितुः क्रिया मया कर्तव्याः। आयाति सा रात्रि र्यदा कार्यं कर्तुं केनापि न शक्यते।
\vakya अहं यावज्जगति वर्ते तावज्जगतो ज्योतिरस्मि।
\vakya इत्थं भाषित्वा स भूमौ निष्ठीव्य तेन ष्ठीवनेन पङ्कं रचयामास, पङ्केन तस्यान्धस्य चक्षुषी लिप्त्वा च त जगाद,
\vakya याहि, शीलोहाख्यसरस्यास्यं प्रक्षालय च। भाषान्तरे नाम्नोऽस्यार्थः प्रहित इति। स तदा गत्वास्यं प्रक्षाल्य दृष्टिप्राप्तः प्रत्याजगाम।
\vakya तत्प्रतिवेशिवृन्दादयो ये मनुष्याः प्राक् तमन्धं ज्ञात्वालक्षयंस्ते तदाप्राक्षुः, अयं किं न स जनो यः (प्रत्यहम्) आसीनोऽभिक्षत?
\vakya अपरेऽवदन्, अयं स एव। अपरे चावदन्, अयं तेन सदृशः। स त्ववादीत्, अहं स एव।
\vakya ततस्ते तमप्राक्षुः, तव लोचने कथं प्रसन्नीभूते?
\vakya स प्रतिभाषमाणस्तानवादीत्, यीशुनामा नर एकः पङ्कं कृत्वा मम लोचने लिप्त्वा च मामादिदेश, शीलोहसरो गत्वास्यं प्रक्षालयेति। ततोऽहं गत्वा प्रक्षाल्य च दृष्टिं प्राप्तवान्।
\vakya ते तदा तमप्राक्षुः, स कुत्रास्ति? स आह, न जाने।
\vakya तदा ते तं पूर्वान्धं फरीशिनां समीपमनैषुः।
\vakya यीशुस्तु यस्मिन् दिने पङ्कं कृत्वा तस्य नेत्रे प्रसन्नीकृतवांस्तद् विश्रामदिनं।
\vakya फरीशिनस्तदा तं पुनरप्राक्षुः, कथं त्वया दृक्शक्ति र्लब्धा? स तान् जगाद, तेन मन्नेत्रयोः पङ्केऽर्पितेऽहमास्यं प्रक्षालितवान्, ततश्च पश्यामि।
\vakya फरीशिनः केचित् तदा तमूचुः, नास्ति स नर ईश्वरप्रहितः, यतः स विश्रामवारं नानुपालयति। अपरेऽवदन्, पापी मनुष्य ईदृशान्यभिज्ञानार्थकर्माणि कर्तुं कथं शक्नोति? इत्थं तेषु भेदो जातः।
\vakya ततस्ते तमन्धं पुनरप्राक्षुः, स तव नेत्रे प्रसन्नीकृतवानिति हेतोस्त्वं तमधि किं वदसि? स उवाच, स भाववादीति वदामि।
\vakya स यदन्ध आसीत् दृक्शक्तिञ्च प्राप्तवान्, यिहूदीयास्तन्न सत्यममन्यन्त, ततः शेषे तस्य दृष्टिप्राप्तस्य पितरावाहूय तावप्राक्षुः,
\vakya अयं युवयोः पुत्रो यो युवयोः प्रमाणादन्धोऽजनि, अधुनायं कथं पश्यति?
\vakya तत्पितरौ प्रतिभाषमाणौ तानूचतुः, अयमस्मत्पुत्रोऽस्त्यन्धोऽजनि च तज्जानीवः।
\vakya कथन्त्वधुना पश्यति, केन वास्य नेत्रे प्रसन्नीकृते तन्न जानीव आवां। वयःप्राप्तोऽयम्, इममेव पृच्छत, अयं स्वकथां स्वयं कथयिष्यति।
\vakya तत्पितरौ यिहूदीयानां भयादेवेत्थमभाषेतां, यतो यः कश्चित् तं ख्रीष्टं मत्वा स्वीकरिष्यति स समाजभ्रष्टो भविष्यतीति यिहूदीयास्तत्पूर्वं निश्चितवन्तः।
\vakya तत्कारणात् तस्य पितरावूचतुः, अयं वयःप्राप्तः, इममेव पृच्छतेति।
\vakya तदा ते द्वितीयवारं तं पूर्वान्धमाहूय जगदुः, ईश्वरस्य महिमानं स्वीकुरु, वयं जानीमो यत् स नरः पापी।
\vakya स प्रतिभाषमाणोऽब्रवीत्, स पापी न वेत्यहं न जाने, एकमेव जाने, पूर्वान्धोऽहमिदानीं पश्यामि।
\vakya ते तं पुनरप्राक्षुः, स त्वां प्रति किं कृतवान्? कथं तव नेत्रे प्रसन्नीकृतवान्?
\vakya स तान् प्रत्यवादीत्, पुरा मयोक्ता यूयं नाश्रौष्ट, किमर्थं पुनः श्रोतुमिच्छथ? यूयमपि किं तस्य शिष्या भवितुं वाञ्छथ?
\vakya तदा ते तं न्यक्कुर्वन्तोऽवादिषुः, त्वं तस्य शिष्यः, वयं मोशेः शिष्याः।
\vakya मोशिनेश्वरः संललाप तद् वयं जानीमः, असौ तु कुत उत्पन्नस्तन्न जानीमः।
\vakya स नरः प्रतिभाषमाणस्तान् जगाद, अत्रेतदेवाश्चर्यं, यत् स कुत उत्पन्नस्तद् युष्माभि र्न ज्ञायते।
\vakya स हि मम नेत्रे प्रसन्नीकृतवान्, वयञ्च जानीमो यदीश्वरः पापिनां न शृणोति, प्रत्युत यो मनुष्य ईश्वरभक्तस्तदभीष्टमाचरति च तस्यैव शृणोति।
\vakya जन्मान्धस्य नयने केनापि प्रसन्नीकृते इत्यायुगान्नाश्रावि।
\vakya असौ चेदीश्वरप्रहितो नाभविष्यत् तर्हि किमपि कर्तुं नाशक्ष्यत्।
\vakya ते तं प्रत्यवादिषुः, सर्वाङ्गः पापाविष्टोऽजनिष्ठा यस्त्वं त्वं किमस्मान् शिक्षयसि? इत्युक्त्वा ते तं बहिश्चक्रुः।
\vakya स तै र्बहिष्कृत इति श्रुत्वा यीशुस्तमनुसृत्य जगाद, त्वं किमीश्वरस्य पुत्रो विश्वसिषि?
\vakya स प्रतिबभाषे, स कः, प्रभो? अहं तस्मिन् विश्वसानि।
\vakya यीशुस्तमब्रवीत्, स त्वयादर्शि, त्वया यः संलपति स एव सः।
\vakya तेनावादि, विश्वसिमि, प्रभो। इत्युक्त्वा स प्रणिपत्य तमर्चयामास।
\vakya यीशुस्तदा बभाषे, ये न पश्यन्ति तै र्द्रष्टव्यं, ये च पश्यन्ति तैरन्धीभवितव्यमिति विचारार्थमहं जगदिदमागतः।
\vakya तच्छ्रुत्वा ये केचित् फरीशिनस्तेन सार्धमविद्यन्त ते तमवादिषुः, वयमपि किमन्धाः?
\vakya यीशुस्तानाह, यद्यन्धा अभविष्यत, तर्हि निष्पापा अभिविष्यत, युष्माभिस्त्विदानीमुच्यते पश्याम इति, ततो युष्माकं पापमवतिष्ठते\eoc