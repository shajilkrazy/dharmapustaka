\adhyAya
\stitle{ईश्वरस्य वाक्यं यीशोर्महत्वमवतारकथा च।}
\vakya आदौ वाद आसीत्, स च वाद ईश्वराभिमुख आसीत्, स च वाद ईश्वर आसीत्।
\vakya स आदावीश्वराभिमुख आसीत्।
\vakya तेन सर्वमुद्भूतं, यद्यदुद्भूतं तन्मध्ये च तं विना न किमप्युद्भूतं।
\vakya तस्मिन् जीवनमासीत्, तज्जीवनञ्च मनुष्याणां ज्योतिरासीत्।
\vakya तज्ज्योतिश्चान्धकारे राजतेऽन्धकारस्तु तन्न जग्राह।
\vakya अथेश्वरसकाशात् प्रहितो नर एकः समुद्बभूव, तस्य नाम योहन इति।
\vakya स साक्ष्यार्थमाजगाम, ज्योतिरधि तेन तथा साक्ष्यं दातव्यासीत्, यथा सर्वे तेन विश्वासिनो भवेयुः।
\vakya स ज्योति र्नासीत्, अपि तु ज्योतिपि साक्ष्यदाने नियुक्तः।
\vakya यत् सर्वमनुष्यं द्योतयति तद् यथार्थं ज्योतिरासीज्जगदायातुमुद्यतं।
\vakya तज्जगत्यासीत्, जगच्च तेन समुद्बभूव, जगत्तु तन्नाजानात्।
\vakya स ज्योतिःस्वरूपो निजविषयमाजगाम, निजस्वास्तु तं न जागृहुः।
\vakya यावन्तस्तु तं जागृहुस्तेभ्यः स तत् सामर्थ्यं ददौ येन त ईश्वरस्य सन्ताना भवन्ति यतस्तेऽमुष्य नाम्नि विश्वसन्ति।
\vakya न शोणितेभ्यो नापि वा मांसस्याभिलाषान्नापि वा पुरुषस्याभिलाषात् प्रत्युतेश्वरादेव ते जाताः।
\vakya स च वादो मांसरूपी बभूव प्रवासीवास्मन्मध्ये ववृते च, वयञ्च पितृसकाशादागतस्यैकजातस्य पुत्रस्य प्रतापमिव तस्य प्रतापं निरीक्षितवन्तः। परिपूर्णः स प्रसादेन सत्येन च।
\vakya योहनस्तस्य साक्षी, स प्रोच्चैरुक्तवान्, अयं स एव यमधि मयोक्तं, मम पश्चाद् य आयाति स मत्तोऽग्रगण्यो जातः, यतः स मत्तः प्रागासीदिति।
\vakya तस्य पूर्णतातश्चालम्भि सर्वैरस्माभिः प्रसादात् परं प्रसादः।
\vakya यतो मोशिद्वारा व्यवस्थादायि, प्रसादसत्ये तु यीशुख्रीष्टद्वारा समुद्भूते।
\vakya ईश्वरः कदापि केनापि नादर्शि, पितृक्रोडस्य एकजातः पुत्र एव तं व्याचख्यौ।
\stitle{यीशुविषये योहनस्य साक्ष्यदानम्।}
\vakya अथ योहनस्य साक्ष्यमिदं। यिहूदीयजना यिरूशालेमाद् याजकान् नेवीयांश्च प्रहित्य यदा तमप्रच्छयन्, को भवानिति, तदा सोऽङ्गीचकार नैवापजुह्नुवे।
\vakya सोऽङ्गीचकार, नाहं ख्रीष्टः।
\vakya ते तमप्राक्षुः, तर्हि किं? भवान् किमेलियः? सोऽब्रवीत्, नाहं सः। तर्हि भवान् किं स भाववादी? स प्रत्यवादीत्, नाहं सः।
\vakya ततस्ते तमाहुः, भवान् कस्तद् वदतु, यतो वयं यैः प्रेरितास्तेभ्योऽस्माभिः प्रत्युत्तरं दातव्यं। आत्मविषये भवान् किं वदति?
\vakya स उवाच, भाववादी यिशयाहो यथोक्तवान्, तथैव।
\begin{poem}
\startwithline “अस्म्यहं कस्यचिद् वाणी प्रोच्चै र्घोषयतो मरौ।
\pline भो युष्माभिः प्रभो र्मार्गः समानीक्रियतामिति॥”
\end{poem}
\vakya प्रहितास्ते जनाः फरीशिनां श्रेण्यां गणिताः।
\vakya ते तदा तमिमां कथामप्राक्षुः, भवान् यदि न ख्रीष्टो नाप्येलियो वा नापि वा स भाववादी, किमर्थं तर्हि स्नापयति?
\vakya योहनः प्रतिभाषमाणस्तानवादीत्, अहं तोयेन स्नापयामि, यूयन्तु यं न जानीथ तादृशो नर एको युष्माकं मध्ये स्थित्वास्ते।
\vakya मत्पश्चादागच्छन् स एव ममाग्रगण्यो जातः। नार्हाम्यहं तदीयोपानहो र्बन्धनं मोक्तुमिति।
\vakya योहनो यत्रास्नापयत् तत्र यर्दनपारस्थायां बैथनियायामिदं समभूत्।
\vakya परदिवसे योहनः स्वसमीपमागच्छन्तं यीशुं विलोक्य बभाषे, पश्यायमीश्वरस्य मेषशावको यो जगतः पापभारं हरति।
\vakya अयं स यमधि मयोक्तं, मत्पश्चान्नर एक आयाति यो ममाग्रगण्यो जातः, यतः स मत्तः प्रागासीदिति।
\vakya अहञ्च तं नाजानां, स तु यदिस्रायेलस्य प्रत्यक्षीक्रियेत तदर्थमहं तोये स्नापयन्नागत इति।
\vakya अपि च योहनः साक्ष्यं दत्त्वावादीत्, पवित्र आत्मा कपोत इव स्वर्गादवरोहन् मया नक्षितः, स चामुष्योपर्यवतस्थे।
\vakya अहमपि तं नाजानं, येन तु तोये स्नापयितुं प्रहितोऽहं तेनैव मह्यं कथितं, त्वं यस्योपरि पवित्रमात्मानमवरोहन्तमवतिष्ठमानञ्च द्रक्ष्यसि, स एव पवित्र आत्मनि स्नापयिता।
\vakya अहञ्च दृष्टवान् साक्ष्यं दत्तवांश्च यदयमीश्वरस्य पुत्र इति।
\stitle{यीशुना शिष्याणां प्रथमाह्वानम्।}
\vakya तत्परदिवसे पुन र्योहनस्तिष्ठन्नासीत्, तच्छिष्यौ च द्वौ तेन सार्धमास्तां।
\vakya स तदा यीशुं विहरन्तं निरीक्ष्य बभाषे, पश्यायमीश्वरस्य मेषशावकः।
\vakya तौ द्वौ शिष्यौ च तस्येदं वचनं श्रुत्वा यीशुमनुजग्मतुः।
\vakya यीशुः प्रत्यावृत्य तावनुगच्छन्तौ विलोक्य भाषते, किमन्विष्यथः? तौ तमवदतां, भवान् कुत्र वसति, रब्बिन्? भाषान्तरेऽस्यार्थो गुरो।
\vakya स तौ ब्रवीति, एत्य पश्यतं। ततस्तौ गत्वा स यत्रावसत् तत् स्थानं ददृशतुस्तद्दिनञ्च तेन सार्धमवतस्थाते। तदा प्रायेण दशमी घटिकासीत्।
\vakya यौ द्वौ योहनस्य वचनं श्रुत्वा यीशुमनुगतौ, आन्द्रियनामा शिमोनस्य पित्रस्य भ्राता तयोरेकतर आसीत्।
\vakya प्रथमः स निजभ्रातरं शिमोनमन्विष्य प्राप्नोति वदति च, आसादित आवाभ्यां मशीहः। भाषान्तरेऽस्यार्थः ख्रीष्टः (अभिषिक्तः)।
\vakya इदमुक्त्वा स तं यीशोरन्तिकं निनाय। यीशुस्तदा तं समालोक्य बभाषे, त्वं योनाः सुतः शिमोनः, त्वं कैफा इत्यभिधायिष्यसे। भाषान्तरेस्यार्थः पित्रः (पाषाणः)।
\stitle{नथनेलस्य कथा च।}
\vakya तत्परदिवसे यीशु र्गालीलं प्रस्थातुमैच्छत्। स च फिलिपमासाद्य वदति, मामनुगच्छ।
\vakya आन्द्रियपित्रौ यस्य पौरौ, स फिलिपोऽपि तस्य बैत्सैदाख्यपुरस्य पौर आसीत्।
\vakya पुनः फिलिपो नथनेलमासाद्य वदति, मोशि र्व्यवस्थायां भाववादिनश्च यस्य वर्णनां लिखितवन्तः सोऽस्माभिः प्राप्तः, स योषेफसुतो नासरतीयो यीशुः।
\vakya नथनेलस्तमब्रवीत्, नासरताद् भद्रं किमपि किं समुद्भवितुं शक्नोति? फिलिपस्तं ब्रवीति, एत्य पश्य।
\vakya यीशुः स्वान्तिकमागच्छन्तं नथनेलं निरीक्ष्य तमुद्दिश्य भाषते, पश्यायं सत्येनेस्रायेलीयो नास्त्यस्मिन् कपटः।
\vakya नथनेलस्तं ब्रवीति, कथं मां जानाति भवान्? यीशुः प्रतिभाषमाणस्तमाह, फिलिपेन तवाह्वानात् प्रागुडुम्बरवृक्षस्याधः स्थितस्त्वं मया लक्षितः।
\vakya नथनेलः प्रतिभाषमाणस्तं वदति, रब्बिन्, भवानीश्वरस्य पुत्रः, भवानिस्रायेलस्य राजा।
\vakya यीशुः प्रतिभाषमाणस्तमाह, उडुम्बरस्याधस्तात् त्वं मया लक्षित इत्यहं त्वामुक्तवांस्तत्कारणात् किं विश्वसिषि? अस्मादपि महत्तराणि (लक्षणानि) द्रक्ष्यसि।
\vakya अन्यच्च स तं वक्ति, सत्यं सत्यं युष्मानहं ब्रवीमि, उद्घाटितः स्वर्गो मनुष्यपुत्रस्योर्धेन चारोहन्तोऽवतरन्तश्च स्वर्गदूता युष्माभि र्द्रक्ष्यन्ते\eoc