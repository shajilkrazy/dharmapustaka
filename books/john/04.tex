\adhyAya
\stitle{शमरीयस्त्रियै यीशोरुपदेशकथा तत्‌फलञ्च।}
\vakya यीशुः स्वयं नास्नापयत्, तस्य शिष्या अस्नापयन्।
\vakya अतो योहनतो यीशुना बहवः शिष्याः क्रियन्ते स्नाप्यन्ते चेति कथा फरीशिभिरश्राव्येतज्ज्ञात्वा
\vakya प्रभु र्यिहूदियां परित्यज्य पुन र्गालीलं प्रतस्थे।
\vakya शमरियादेशेन गमनन्तु तस्यावश्यङ्कर्तव्यमासीत्।
\vakya ततः स शमरियादेशस्थं शुखराभिधं नगरमायाति। पुरा या भूमि र्याकोबेन स्वपुत्राय योषेफाय दत्ता तन्नगरं तत्समीपस्थं।
\vakya तत्र च याकोबस्य कूप आसीत्। अत यीशुः पथश्रान्तो भूत्वा तथैव तत्कूपान्तिकमुपविष्टवान्। तदा प्रायेण षष्ठी घटिवासीत्।
\vakya शमरीया काचिन्नारी तोयमुद्धर्तुमायाति। यीशुस्तं ब्रूते, मह्यं पानीयं देहि।
\vakya यतस्तस्यः शिष्याः खाद्यक्रयार्थं नगरमपगतवन्तः।
\vakya ततः सा शमरीया नारी तं ब्रवीति, अहं शमरीया नारी, त्वञ्च यिहूदीयः, तत् कथं मां पानीयं याचसे? यतः शमरीयैः सह यिहूदियानां व्यवहारो नास्ति।
\vakya यीशुः प्रतिभाषमाणस्तामवादीत्, यदि त्वमीश्वरस्य दानमज्ञास्यः, मह्यं पानीयं देहीति येनोच्यसे तस्य तत्त्वञ्च यद्यज्ञास्यस्तर्हि त्वमेव तमयाचिष्यथाः, स च तुभ्यं जीवि तोयमदास्यत्।
\vakya सा योषित् तं वदति, प्रभो, तोयोद्धारणोपायो भवतो नास्ति, प्रहिश्च गम्भीरः, कुतः पुनस्तज्जीवि तोयं लब्धं भवता?
\vakya कूपोऽयं येनास्मभ्यं दत्तस्तस्मादस्मत्पितामहाद् याकोबाद् भवान् किं महत्तरः? तेनापि तत्पुत्रैश्च तत्पोषितैः पशुभिश्चैतस्य तोयमपीयत।
\vakya यीशुः प्रतिभाषमाणस्तामवादीत्, तोयमेतद् येन केनचित् पीयते स पुनः पिपासिष्यति,
\vakya मद्दातव्यं तोयन्तु येन पीयते, सोऽनन्तकालं यावन्न पिपासिष्यति, अपि त्वहं तस्मै यत् तोयं दास्यामि तत् तस्यान्तरेऽनन्तं जीवनं यावदुत्प्लवमानस्य तोयस्योत्सो भविष्यति।
\vakya सा नारी तं ब्रूते, प्रभो, मह्यं तत् तोयं ददातु, अहं यथा न पिपासामि नापि वा तोयोद्धरणार्थमत्रागच्छामि।
\vakya यीशुस्तं वक्ति, याहि, तव पतिमाहूयात्रागच्छ।
\vakya सा योषित् प्रतिभाषमाणा तमवादीत्, पति र्मम नास्ति।
\vakya यीशुस्तं ब्रूते, पति र्मम नास्तीति युक्तं कथितं त्वया, यतः पतयः पञ्च तवासन्, अधुना तु तव योऽस्ति नास्ति स तव पतिः। सत्यमिदमुक्तं त्वया।
\vakya सा नारी तं वदति, प्रभो, निरूपयाम्यहं यद् भवान् भाववादी।
\vakya अस्मत्पितामहा अस्मिन् गिरावुपासनम् अकुर्वन्, यूयन्तु वदथ, उपासनं यत्र कर्तव्यं, तत् स्थलं यिरूशालेमेऽस्तीति।
\vakya यीशुस्तां ब्रूते, नारि, मयि विश्वसिहि तत् समयः स आयाति, यदा नास्मिन् गिरौ नापि वा यिरूशालेमे पितुरुपासनं युष्माभिः कारिष्यते।
\vakya यूयं यन्न जानीथ तस्योपासनं कुरुथ। वयं यज्जानीमस्तस्योपासनं कुर्मः। परित्राणं हि यिहूदीयजातिमूलकं।
\vakya प्रत्युत समयः स आयात्यधुना चास्ति यदा प्रकृता उपासका आत्मना सत्येन च पितुरुपासनं करिष्यन्ति, यतः पिता स्वोपासकांस्तादृशान् अनुसन्धत्ते।
\vakya ईश्वर आत्मा, यैश्च तस्योपासनं क्रियत आत्मना सत्येन चोपासनं तैः कर्तव्यं।
\vakya सा स्त्री तं गदति, जानेऽहं मशीहेनार्थतः ख्रीष्टाभिधेन पुरुषेणागन्तव्यं, स यदायास्यति तदास्मभ्यं सर्वं निवेदयिष्यति।
\vakya यीशुस्तां ब्रवीति, त्वया सम्भाषमाणो योऽहं सोऽहमेव सोऽस्मि।
\vakya तदैव तस्य शिष्या उपतस्थिरे, नार्या सह तस्य सम्भाषण आश्चर्यं मेनिरे च, तथापि भवान् किमन्विष्यतीति, अथवामुया सह भवान् किं सम्भाषत इति केनापि नावादि।
\vakya ततः सा नारी स्वकलसं विहाय नगरं गत्वा जनान् वदति,
\vakya आयात, मया यद्यदकारि तत् सर्वं मह्यं यः कथितवांस्तादृशं मनुष्यं पश्यत च।
\vakya स किं ख्रीष्टः स्यात्? ततस्ते नगरान्निर्गत्य तस्य समीपमगच्छन्।
\vakya तस्मिन्नवसरे शिष्याः सानुनयं तमवादिषुः, रब्बिन्, अश्नातु।
\vakya स तु तानब्रवीत्, भोजनार्थं युष्मदविदितं भक्ष्यं ममास्ते।
\vakya अनेन शिष्या मिथोऽपृच्छन्, केनापि किं तस्य समीपं भक्ष्यमानायि?
\vakya यीशुस्तान् ब्रूते, मम भक्ष्यमिदं यन्मत्प्रेषयितुरभीष्टमाचरामि तदीयकर्म साधयामि च।
\vakya अधुनापि मासचतुष्टयं शिष्यते, तदा शस्यकर्तनकाल उपस्थास्यत इति किं युष्माभि र्नोच्यते? पश्यताहं युष्मान् ब्रवीमि, ऊर्ध्वदृष्टिं कृत्वा क्षेत्राण्यवलोकयत, शस्यकर्तनार्थं तान्यधुनैव सितानि प्रतिभान्ति।
\vakya शस्यच्छेत्ता च वेतनं लभतेऽनन्तजीवनाय फलानि सञ्चिनोति च, यथा वप्तृच्छेत्तारावेकत्रानन्देतां।
\vakya अत्र हि प्रवादोऽयं सत्यः, वपत्येकः कृन्तत्यन्य इति। यूयं यत्र न श्रममनुष्ठितवन्तस्तत्र शस्यकर्तनार्थं प्रहिता मया।
\vakya अन्ये श्रममनुष्ठितवन्तः, यूयञ्च तेषां श्रमस्थानं प्रविष्टाः।
\vakya अथ मया यद्यदकारि तत् सर्वं स मह्यं कथितवानिति साक्ष्यदायिन्यास्तस्या नार्या वचनात् तन्नगरीया बहवः शमरीयमनुष्यास्तस्मिन् व्यश्वसिषुः।
\vakya ततस्ते शमरीया यदा तत्समीपमुपातिष्ठन्त, तदा तैः सार्धमवस्थानं तं प्रार्थयन्त, स च दिनद्वयं तत्रावतस्थे।
\vakya बहुगुणाधिकाश्च तस्य वचनाद् व्यश्वसिषुस्तां नारीमवादिषुश्च,
\vakya अधुनापि त्वदीयकथनकारणादेव वयं विश्वसिमस्तन्न, यतः स्वयमस्माभिः श्रुतमनुभूतञ्चेदं यत् सत्यमेवायं जगत्त्राता ख्रीष्टः।
\vakya दिनद्वये तस्मिन्नतीते स ततः प्रस्थाय गालीलं जगाम।
\vakya यीशुः स्वयं हि साक्ष्यमिदं दत्तवान् यत् स्वदेशे भाववादी सम्मानहीनः।
\vakya अतो गालील उपस्थितः स गालीलीयैरन्वग्राहि, तत्कारणमिदं यद् यिरूशालेमे पर्वाणि तेन यद्यत् कृतं तत् सर्वं ते दृष्टवन्तः, यतस्तेऽपि पर्वण्युपस्थितवन्तः।
\stitle{कफर्नाहूमे राजसभास्तारपुत्रस्य स्वास्थ्यञ्च।}
\vakya अत एव यीशु र्यत्र तोयं द्राक्षारसीकृतवांस्तां गालीलस्थकान्नां पुनराजगाम। तदा कफरनाहूमे यस्य पुत्रो व्याधितस्तादृशो राजपुरुषः कश्चिदासीत्।
\vakya स यिहूदियातो गालीले यीशोरागमनस्य संवादं श्रुत्वा तस्यान्तिकं जगाम, स च यत् (कफरनाहूमम्) अवरुह्य तं पुत्रं निरामयं कुर्यात् तत् प्रार्थयाञ्चक्रे, यतः स मृतकल्प आसीत्।
\vakya ततो यीशुस्तमुवाच, यूयम् अभिज्ञानान्यद्भुतलक्षणानि चादृष्ट्वा नैव विश्वसिष्यथ।
\vakya स राजपुरुषस्तं ब्रूते, प्रभो, मद्बालकस्य मरणात् प्रागवरोहतु।
\vakya यीशुस्तं ब्रवीति, याहि, तव पुत्रो जीवति। ततः स नरो यीशुना यदुक्तस्तस्मिन् वाक्ये विश्वस्य यात्रामकुरुत।
\vakya तस्य चावरोहणकाल एव तस्य दासास्तत्सम्मुखमुपस्थायेदं निवेदयामासुः, भवतः पुत्रो जीवतीति।
\vakya ततः स तान् पप्रच्छ, कस्मिन् दण्डे तस्योपशमो जातः? ते तमूचुः, ह्यः सप्तम्यां घटिकायां ज्वरस्तमत्याक्षीत्।
\vakya ततः पित्राबोधि तत् तस्यामेव घटिकायां सम्भूतं यदा स यीशुनोक्तस्तव पुत्रो जीवतीति। ततः स तदीयगृह्याश्च सर्वे व्यश्वसिषुः।
\vakya इदं पुन र्द्वितीयमभिज्ञानाथकर्माकारि यीशुना यिहूदियातो गालीलमागतेन\eoc