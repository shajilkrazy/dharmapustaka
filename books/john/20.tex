\adhyAya
\stitle{शून्यं शवागारं।}
\vakya सप्रताहस्य प्रथमे दिने तु मग्दलीनी मरियम् प्रत्यूषेऽन्धकारेऽनतीते तच्छवागारमायाति पाषाणं शवागारमुखादपसारितं लक्षयति च।
\vakya ततः सा धावन्ती शिमोनस्य पित्रस्य समीपं यीशोः प्रियस्यान्यतरशिष्यस्य च समीपमायाति तौ वदति च, शवागारात् प्रभुरपसारितोऽस्मदविदिते स्थाने निहितश्चेति।
\vakya ततः पित्रः सोऽन्यतरः शिष्यश्च निर्गत्य शवागारमगच्छतां।
\vakya तौ युगपदधावतां। सोऽन्यतरः शिष्यस्तु पित्रात् क्षिप्रतरं प्रधाव्य प्रथमं शवागार उपतस्थे,
\vakya प्रह्वीभूय पटानि लक्षयामास च, न तु प्रविवेश।
\vakya ततः परं तमनुगच्छन् शिमोनः पित्र उपस्थाय प्रविशति भूमौ निहितानि पटानि लक्षयति च,
\vakya तस्य शिरश्च येनाच्छादितं तत् स्वेदहरवस्त्रमपि लक्षयामास। तन्न पटैः सार्धं निहितं प्रत्युत पृथक् स्थानविशेषे सुरचितमासीत्।
\vakya तदैव शवागारे प्रथममागतः सोऽन्यतरः शिष्योऽपि प्रावेक्षीत् दृष्ट्वा च व्यश्वसीत्।
\vakya यतो मृतानां मध्यात् तेनोत्थातव्यमिति या शास्त्रीयोक्तिः सा तदानीमपि ताभ्यां नाबोधि।
\vakya तौ शिष्यौ तदा पुनर्निजस्थानं जग्मतुः।
\stitle{मग्दलीनीमरियमे यीशोः दर्शनदानम्।}
\vakya मरियम् तु रुदती शवागारसमीपे बहिरतिष्ठत्। रुदती शवागारं प्रति प्रह्वीभूय सा श्वेतवसनौ द्वौ स्वर्गदूतौ पश्यति,
\vakya यीशो र्देहो यत्राशेत तौ तत्रोपविष्टौ, एकः शिरसः स्थानेऽन्यतरश्च चरणयोः स्थाने।
\vakya तौ तां वदतः, नारि, किमर्थं रोदिषि? सा तौ ब्रवीति, मम प्रभुः स्थानान्तरीकृतः, न जाने कुत्र स निहितः।
\vakya एतदुक्त्वा सा पश्चाद्दिशि प्रत्यावृत्य यीशुं तिष्ठन्तं पश्यति, न तु जानाति यत् स यीशुः।
\vakya यीशुस्तां ब्रूते, नारि, किमर्थं रोदिषि? कमन्विष्यसि? सा तम् उद्यानरक्षकं मत्वा ब्रवीति, महेच्छ, स यदि भवतापवाहितस्तर्हि कुत्र निहितस्तन्मां वक्तुमर्हति, अहञ्च तं स्थानान्तरीकरिष्यामि।
\vakya यीशुस्तां वदति, भो मरियम्। सा प्रत्यावृत्येब्रीयभाषया तं वदति, रब्बूणे। अस्यार्थः, भो गुरो।
\vakya यीशुस्तां ब्रूते, मां मा धर, यतोऽहमिदानीमपि मत्पितुः समीपं नारूढवान्, प्रत्युत मम भ्रातॄणां समीपं याहि तान् ब्रूहि च, यो मम पिता युष्माकञ्च पिता, ममेश्वरो युष्माकञ्चेश्वरस्तस्य समीपमहमारोहामीति।
\vakya तदा मग्दलीनी मरियम् गत्वा, प्रभुस्तस्यै दर्शनं दत्तवान् सर्वमेतत् कथितवांश्चेदं शिष्यान् ज्ञापयामास।
\stitle{शिष्येभ्यो दर्शनदानम्।}
\vakya तस्मिन्नेव सप्ताहस्य प्रथमदिने सन्ध्यायां जातायां समवेताः शिष्या यत्रासन् यिहूदीयानां भयात् तस्य स्थानस्य द्वारेषु रुद्धेषु यीशुरागत्य मध्यस्थाने स्थित्वा तान् ब्रूते, युष्माकं शान्ति र्भूयात्।
\vakya इदमुक्त्वा स तान् स्वहस्तौ स्वपार्श्वञ्च दर्शयामास। ततः प्रभुं निरीक्ष्य शिष्या आननन्दुः।
\vakya यीशुस्तदा तान् पुनर्ब्रवीति, युष्माकं शान्ति र्भूयात्। यथा पिता मां प्रहितवांस्तथाहं युष्मान् प्रहिणोमि।
\vakya इदमुक्त्वा स तानुद्दिश्य दध्मौ तान् जगाद च, पवित्रमात्मानं गृह्णीत,
\vakya येषां पापानि मोचयिष्यथ तेषां मोचयिष्यन्ते, येषान्तु न मोचयिष्यथ तेषाममुक्तानि स्थास्यन्ति।
\stitle{थोमै र्दर्शनदानम्।}
\vakya इत्थ यदा यीशुरागमत्, तदा द्वादशानां मध्ये गणितः थोमा अर्थतो यमजस्तैः सार्धं नासीत्।
\vakya ततोऽन्ये शिष्यास्तमूचुः, वयं प्रभुं दृष्टवन्तः। स तु तानवादीत्, तस्य करयो र्लौहकीलचिह्नमदृष्ट्वा तत्र लौहकीलचिह्ने ममाङ्गुलीमनिधाय च तस्य पार्श्वे मम करमनिधाय चाहं नैव विश्वसिष्यामीति।
\vakya ततोऽष्टदिनेभ्यः परं शिष्याः पुनरभ्यन्तर आसन्, थोमा अपि तैः सार्धमासीत्। तदा रुद्धेषु द्वारेषु यीशुरुपस्थाय मध्यस्थाने तिष्ठन्नवादीत्, युष्माकं शान्ति र्भूयात्।
\vakya ततः परं स थोमां ब्रवीति, तवाङ्गुलीमत्र नय मम हस्तौ निरीक्षस्व च, तव करञ्च नय मम पार्श्वे निधेहि च। अविश्वासी च मा भव, प्रत्युत विश्वासी भव।
\vakya थोमास्तदा प्रतिभाषमाणस्तमवादीत् भो मम प्रभो, भो ममेश्वर।
\vakya यीशुस्तं ब्रूते, त्वं मां दृष्टवांस्थोमाः, तस्माद् व्यश्वसीः, धन्यास्ते ये न दृष्ट्वापि व्यश्वसिषुः।
\stitle{सुसंवादलेखकस्याभिप्रायः।}
\vakya वास्तवमेतेभ्योऽन्यानि बहून्यभिज्ञानार्थकर्माणि स्वशिष्याणां समक्षं यीशुनाक्रियन्त, तानि त्वस्मिन् ग्रन्थे न लिखितानि।
\vakya एतेषां वृत्तान्तस्त्वलेखि, यतो यीशु र्यदीश्वरस्य पुत्रः ख्रीष्टोऽत्र युष्माभि र्विश्वसितव्यं, विश्वसद्भिश्च तस्य नाम्नि जीवनं प्राप्तव्यम्\eoc