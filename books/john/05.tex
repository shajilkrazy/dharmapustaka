\adhyAya
\stitle{अष्टात्रिंशद्वत्सरान् यावद् रोगग्रस्तमनुष्यस्य विश्रामवारे स्वास्थ्यकरणं।}
\vakya ततः परमुपस्थिते यिहूदीयानामुत्सवे यीशु र्यिरूशालेमं जगाम।
\vakya यिरूशालेमे मेषद्वाराख्यस्य गोपुरस्य निकटे सर एकमस्ति, इब्रीयभाषया तस्य नाम बैथेषदेति, तच्च पञ्चभिरलिन्दै र्युक्तं।
\vakya तेषु शयाना व्याधिता अन्धाः खञ्जाः शुष्काङ्गा वा बहवो मनुष्यास्तोयस्य कम्पनं प्रत्यैक्षन्त।
\vakya यतः समये स्वर्गदूत एकस्तत् सरोऽवतरंस्तोयमकम्पयत्। तोयकम्पनात् परञ्च प्रथमो यो मनुष्यस्तोयं प्राविशत् स येन केनचिद् व्याधिना व्याधितस्तस्मान्मुक्तो निरामयश्चाभवत्।
\vakya तत्र चाष्टत्रिंशद्वर्षाणि यावद् व्याधिग्रस्तो नरः कश्चिदविद्यत।
\vakya यीशुस्तमेव शयानं दृष्ट्वा स आदीर्घकालाद् व्याधितस्तज्ज्ञात्वा च पप्रच्छ, त्वं किं निरामयो भवितुं वाञ्छसि?
\vakya स रोगी तं प्रत्यवादीत्, प्रभो, नास्ति मम जनो यस्तोयकम्पनकाले मां सरसि निक्षिपेत्, यावच्चाहमुपागच्छामि, तावदपरः कश्चिन्मत्तः प्रागवतरति।
\vakya यीशुस्तं ब्रवीति, उत्तिष्ठ, तव खट्वामादाय विहर च।
\vakya ततः स मनुष्यस्तत्क्षणं निरामयो जातः स्वखट्वामादाय च विहर्तुं प्रवृत्तः। तद् दिनन्तु विश्रामदिनं।
\vakya ततो यिहूदीयास्तमारोग्यप्राप्तं नरं जगदुः, अद्य विश्रामदिनं, तव खट्वावहनमविधेयं।
\vakya स तान् प्रत्यवादीत्, अहं येन स्वस्थीकृतस्तेनैवोक्तः, तव खट्वामादाय विहरेति।
\vakya ततस्ते तमपृच्छन्, तव खट्वामादाय विहरेति त्वं येनोक्तः स मनुष्यः कः?
\vakya स तु कस्तत् तेनारोग्यप्राप्तेन नाज्ञायि, यतस्तत्र जननिवहस्योपस्थितितो यीशुरपसृतः।
\vakya ततः परं यीशु र्धर्मधाम्नि तमासाद्यावादीत्, पश्य त्वयारोग्यमलम्भि, पुनः पापं मा कुरु, नोचेत् तव धोरतरा दुर्गति र्भविष्यति।
\vakya स मनुष्यस्तदा गत्वा यिहूदीयान् ज्ञापयामास, यद् यीशुस्तस्यारोग्यकारी।
\vakya ततो यिहूदीया यीशुमबाधन्त हन्तुमयतन्त च, तत्कारणमेतदेव यत् स विश्रामवार एतान्यकरोत्।
\vakya यीशुस्तु तान् प्रत्यवादीत्, मम पिताद्य यावत् कर्म करोत्यहमपि करोमि।
\vakya एतत्कारणादेव यिहूदीयास्तं हन्तुमधिकमप्ययतन्त, यतः स न केवलं विश्रामवारमलङ्घयदपि त्वीश्वरं स्वीयपितरमभ्यदधात्, इत्थञ्चात्मानमीश्वरेण तुल्यमकुरुत।
\stitle{पुत्राधिकारमधि शिक्षा।}
\vakya ततो यीशुः प्रतिभाषमाणस्तानुवाच, सत्यं सत्यं, युष्मानहं ब्रवीमि, पितरं कर्म कुर्वन्तं न दृष्ट्वा पुत्रः स्वतः किमपि कर्तुं न शक्नोति। स हि यद्यत् करोति, पुत्रोऽपि तथैव तत् करोति।
\vakya यतः पिता पुत्रो प्रेम कुरुते, स्वयञ्च यद्यत् करोति पुत्रं सर्वं दर्शयति। स च तमेतेभ्योऽपि महत्तराणि कर्माणि दर्शयिष्यति, यूयं यथाश्चर्यं मंस्यध्वे।
\vakya यतः पिता यादृशं मृतानुत्थापयति सञ्जीवयति च, तादृशं पुत्रोऽपि यान् रोचयति तान् सञ्जीवयति।
\vakya पिता हि कस्यापि विचारमपि न करोत्यपि तु समस्तं विचारं पुत्रो समर्पयामास,
\vakya सर्वे यथा पितरं मानयन्ति तथैव पुत्रं मानयन्त्विति तस्याभिप्रायः।
\vakya पुत्रो येन न मान्यते, तत्प्रेषयिता पितापि तेन न मान्यते। सत्यं सत्यं, युष्मानहं ब्रवीमि, यो मम वाक्यं शृणोति मत्प्रेषयितरि विश्वसिति च, सोऽनन्तजीवनप्राप्तः, स च विचारं न नेतव्योऽपि तु मृत्युतो जीवनं प्रत्युत्तीर्णः।
\vakya सत्यं सत्यं, युष्मानहं ब्रवीमि, समयः स आयात्यधुना चास्ति, यदा मृता ईश्वरपुत्रस्य गिरं श्रोष्यन्ति, ये च श्रोष्यन्ति ते जीविष्यन्ति।
\vakya यतः पिता यादृक् स्वयञ्जीवी तादृक् पुत्रायापि स्वयञ्जीवित्वं दत्तवान् विचारसाधनस्य सामर्थ्यञ्च तस्मै दत्तवान्, यतः स मनुष्यपुत्रः।
\vakya अत्र माश्चर्यं मन्यध्वं।
\vakya यत आयाति स समयो यदा शवागारस्थाः सर्वे तस्य गिरं श्रोष्यन्ति
\vakya बहिरायास्यन्ति च कृतसदाचारा जीवनार्थपुनरुत्थानाय, कृतदुराचाराश्च विचारार्थुपनरुत्थानाय।
\stitle{ख्रीष्टस्य साक्ष्यं।}
\vakya स्वतः किमपि कर्तुं मया न शक्यते, यच्छृणोमि तदनुरूपं विचारं करोमि, मम विचारश्च न्याय्यः, यतोऽहं न मदीयाभीष्टमपि तु मत्प्रेषयितुः पितुरभीष्टमीहे।
\vakya यद्यहं स्वार्थे साक्ष्यं ददामि, तर्हि मम साक्ष्यं न।
\vakya अपरो मयि साक्ष्यं ददाति, अहञ्च जाने यन्मयि दीयमानं तस्य साक्ष्यं सत्यं।
\vakya यूयं योहनं प्रति दूतान् प्रहितवन्तः स च सत्यस्य स्वपक्षं साक्ष्यं दत्तवान्।
\vakya नाहन्तु मनुष्यात् साक्ष्यं गृह्णामि। यूयं यत् त्राणमाप्नुयात तदर्थमेतत् कथयामि।
\vakya स ज्वलन्ती विराजमाना च दीपिकासीत्, यूयञ्च क्षणं यावत् तस्य दीप्त्यां मोदितुमरोचयत।
\vakya योहनस्य साक्ष्यात्तु महत्तरं साक्ष्यं ममास्ते। येषां कार्याणां साधनार्थं तद्भारः पित्रा मयि समर्पितस्तानि मया क्रियमाणानि कार्याण्यव मयीदं साक्ष्यं ददति यदहं पित्रा प्रहितः।
\vakya मत्प्रेषयिता पितापि स्वयं मामधि साक्ष्यं दत्तवान्। युष्माभिः कदापि तस्य वाणी न श्रुत्वा नापि वा तस्य रूपं दृष्टं, नापि वा लब्धं तस्य वाक्यं युष्मदन्तरवतिष्ठमानं।
\vakya यतस्तेन यः प्रहितस्तस्मिन् युष्माभि र्न विश्वस्यते।
\vakya शास्त्राण्यनुसन्धद्ध्वं यूयं हि तेषु स्वान् अनन्तजीवनप्राप्तान् मन्यध्वे, तान्येव च मयि साक्ष्यं ददति।
\vakya तथापि यूयं जीवनलाभार्थं मदन्तिकमायातुमसम्मताः।
\vakya मनुष्येभ्योऽहं सम्मानं न गृह्णामि,
\vakya युष्मांस्तु जानामि, यन्न लब्धं युष्माभिरीश्वरस्य प्रेम युष्मदन्तरवतिष्ठमानं।
\vakya मत्पितु र्नाम्नागतोऽहं युष्माभि र्न गृह्ये। अपरः कश्चिच्चेन्निजनाम्नागच्छेत्, स एव तर्हि युष्माभि र्ग्राहीष्यते।
\vakya परस्परं सम्मानं गृह्णन्त एकस्मादीश्वराल्लभ्यं सम्मानन्त्वनीहमाना ये यूयं, कथं विश्वसितुं शक्यते युष्माभिः?
\vakya मा मन्यध्वं यत् पितुरन्तिकमहं युष्माकमभियोक्ता भविष्यामि। अस्ति युष्माकमभियोक्ता स मोशि र्यो युष्मदीयाशाभूमिः।
\vakya यतो यदि मोशौ व्यश्वसिष्यत, तर्हि मयि व्यश्वसिष्यत, स हि मामधि लिखितवान्।
\vakya तस्य लेखेषु तु यदि न विश्वसिथ, कथं तर्हि मम वाक्येषु विश्वसिष्यथ?\eoc