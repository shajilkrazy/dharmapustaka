\adhyAya
\stitle{बैथनियायां यीशोस्तैलाभिषेकं।}
\vakya ततो यीशु र्निस्तारोत्सवात् षट् दिनानि प्राग् बैथनियायामुपतस्थे। यो मृतो लासारस्तेन मृतानां मध्यादुत्थापितः स तत्रासीत्।
\vakya तत्र च तस्य निमित्तं रात्रिभोज्यमकारि, मार्था तत्र पर्यचरत्, लासारश्च तत्सहभोजिनामेक आसीत्।
\vakya तदा मरियम् अर्धसेटकं बहुमूल्यं प्रकृतं जटामांसीरसयुक्तं सुगन्धितैलमादाय यीशोश्चरणौ समलेपयत् स्वकचैश्च तौ चरणौ पर्यमार्क्षीत्, कृत्स्नं गृहञ्च तस्य तैलस्य सौरभेण पर्यपूर्यत।
\vakya तदा तच्छिष्याणामेकोऽर्थतः शिमोनसुतो य ईष्करियोतीयो यिहूदास्तत्समर्पयिताभूत्, स ब्रवीति,
\vakya सुगन्धतैलमिदं किमर्थं न त्रिभिः शतै र्मुद्रापदै र्विक्रीतं दरिद्रैभ्यो दत्तञ्च?
\vakya न दरिद्राणां हितचिन्तनकारणात् तेनेदमकथ्यत, प्रत्युत स चौरः सम्पुटवाही चासीत् सम्पुटे च यद्यन्न्यक्षिप्यत तदहरत् तत्कारणादेव।
\vakya यीशुस्तदाब्रवीत्, अमूमनुजानीहि, मत्समाधिदिनार्थं तयेदं रक्षितं।
\vakya तदिद्रा हि सततं युष्मत्सङ्गिनः, अहन्तु न सततं युष्मत्सङ्गी।
\stitle{लासारं द्रष्टुं बहुलोकानामागमनं।}
\vakya अथ यीशुस्तत्रास्तीति ज्ञात्वा यिहूदीयानां महान् जननिवहस्तत्रोपतस्थे, न केवलं यीशोः कारणात्, प्रत्युत मृतानां मध्यात् तेनोत्थापितस्य लासारस्यापि दिदृक्षया।
\vakya मुख्ययाजकास्तु लासारस्यपि हत्यार्थं मन्त्रयाञ्चक्रिरे,
\vakya यतस्तस्य कारणाद् बहवो यिहूदीया गत्वा यीशौ व्यश्वसन्।
\stitle{ख्रीष्टस्य राजकीयप्रवेशनं।}
\vakya अथ परदिने यीशु र्यिरूशालेममायातीति श्रुत्वा पर्वार्थमागतो महान् जननिवहः
\vakya खर्जूरपल्लवानादाय यीशोः प्रत्युद्गमनार्थं निर्ययावुच्चैरवदच्च, जय, प्रभो र्नाम्ना य आयाति धन्यः स इस्रायेलस्य राजा।
\vakya यीशुस्तु यवगर्दभ प्राप्य तत्पृष्ठ उपविवेश, यथा निखतमास्ते,
\begin{poem}
\startwithvakya “मैवानुभूयतां त्रासो भो सीयोनसुतो त्वया।
\pline पश्य राजा तवायात्यासीनो गर्दभशावके॥”
\end{poem}
\vakya सर्वमेतत् तस्य शिष्यैः प्रथमं नाबोधि, यीशोस्तु महिमप्राप्तेः परं ते सस्मरु र्यत् तमधि कथेयं लिखितमासीत् तं प्रतीत्थमकारि चेति।
\vakya यीशु र्यल्लासारं शवागारान्निर्गमनार्थमाहूतवान् मृतानां मध्यादुत्थापितवांश्चेति साक्ष्यं तस्य तदानीन्तनसङ्गिनिवहेनादीयत।
\vakya एतत्कारणादेवार्थतः स तद् अभिज्ञानार्थकर्म कृतवानितिकथाश्रवणकारणादेव जननिवस्तं प्रत्युज्जगाम।
\vakya फरीशिनस्तदा परस्परमाहुः, युष्माकं सर्वयत्नो व्यर्थस्तद् विलोकयथ, जगदेव तस्य पश्चाद् अपगतम्।
\stitle{यूनानीयलोकानां यीशुमन्वेषणं।}
\vakya पर्वण्युपासनार्थं ये आगमंस्तेषां मध्ये कतिपया यूनानीया नरा आसन्।
\vakya ते तदा गालीलस्थबैत्सैदानिवासिनं फिलिपमभिगम्य प्रार्थयाञ्चक्रिरे, प्रभो, वयं यीशुं दिदृक्षामहे।
\vakya फिलिप आगत्यान्द्रियं तद् वदति। पुनरान्द्रियफिलिपौ यीशवे निवेदयतः।
\vakya यीशुस्तु प्रतिभाषमाणस्ताववादीत्, मनुष्यपुत्रस्य महिमप्राप्तेः समय उपस्थितः।
\vakya सत्यं सत्यं, युवामहं ब्रवीमि, गोधूमबीजं चेन्न मृत्तिकायां पतित्वा म्रियते, तर्ह्येकं तिष्ठति, यदि तु म्रियते, तर्हि बहु फलं फलति।
\vakya यः स्वप्राणेषु प्रेम कुरुते, स तान् हारयिष्यति, यस्त्विहलोके तान् प्रति विरज्यते, सोऽनन्तजीवनाय तान् रक्षिष्यति।
\vakya कश्चिनमनुष्यो यदि मां परिचरितुं स्वीकरोति, तर्हि मामनुगच्छतु, तथा सत्यहं यत्रास्मि मम परिचारकोऽपि तत्रैव स्थस्यति। यश्च कश्चिन्मां परिचरति, मम पिता तं सम्मानयिष्यति।
\stitle{यीशो र्निजनिधनस्य भविष्यद्वाक्यं।}
\vakya अधुना तु मम प्राणा उद्विग्ना जाताः, अत्र किं वदानि? पितः, दण्डादस्मान्मां तारय। प्रत्युतैतदर्थमहं दण्डमिमं यावदागतवान्।
\vakya पितः, तव नाम महिमान्वितं कुरु। तद गगनादियं वाणी सञ्जाता, तन्मया महिमान्वितमकारि पुनश्च महिमान्वितं कारिष्यत इति।
\vakya तत्र स्थितो जननिवहस्तदा श्रुत्वाब्रवीत्, मेघनादो जात इति। अन्येऽवदन्, कश्चित् स्वर्गदूतोऽनेन समभाषतेति।
\vakya यीशुस्तु प्रतिभाषमाणोऽब्रवीत्, न मदर्थं, किन्तु युष्मदर्थं वाणीयं समभूत्।
\vakya अधुना जगतोऽस्य विचारः साध्यते, अधुना जगतोऽस्याधिपति र्बहि र्निक्षेप्स्यते।
\vakya अहञ्च चेत् भूतलादुच्चीकृतो भवामि, तर्हि सर्वान् मदन्तिकमाक्रक्ष्यामि।
\vakya स कीदृशं मृत्युं सहिष्यते वचनेनैतेन तदसूचयत्।
\vakya ततो जननिवह उवाच, व्यवस्थापाठेन वयं श्रुतवन्तो यत् ख्रीष्टेन शाश्वतं स्थातव्यं, तद् भवान् कथं ब्रूते, मनुष्यपुत्र उच्चीकर्तव्य इति? स मनुष्यपुत्रः कः?
\vakya यीशुस्तदा तानवादीत्, इतः परं ज्योतिः स्तोकं कालं युष्माभिः सार्धं विद्यते। ततो यावज्ज्योतिस्तिष्ठति तावत् परिव्रजत, नोचेदन्धकारेणावेक्ष्यध्वे। अन्धकारे यः परिव्रजति स कुत्र गच्छति तन्न जानाति।
\vakya यावद् युष्माकं ज्योतिरास्ते तावज्ज्योतिषि तथा विश्वसित यथा ज्योतिषः पुत्रा भविष्यथ। एतानि वचनानि यीशुस्तेभ्यः कथयामास, ततः परमपगत्य तेभ्यः प्रच्छन्नो बभूव।
\stitle{यिहूदीयानामविश्वासं।}
\vakya यद्यपि स तेषां समक्षमेतावन्त्यभिज्ञानार्थकर्माणि कृतवांस्तथापि ते तस्मिन् न व्यश्वसन्।
\vakya इत्थं भववादिनो यिशायाहस्यैतद् वचनं सिद्धिं गतं,
\begin{poem}
\startwithline “अस्मच्छावितसंवादे विश्वासं कोऽकरोत् प्रभो।
\pline परमेशस्य बाहुर्वाभूत् कस्मै सम्प्रकाशितः॥”
\end{poem}
\vakya एतत्कारणात् ते विश्वसितुं नाशक्नुवन्, यतो यिशायाहः पुनरिदमपि व्याहृतवान्,
\begin{poem}
\startwithvakya “तेषां नेत्राणि सोऽन्धानि चित्तं स्थूलं चकार च।
\pline नोचेत् ते लोचनै र्दृष्ट्वा चित्तै र्बुद्धिमवाप्य च।
\pline परावृत्ता भविष्यन्ति मोक्तव्याश्चामयान्मया॥”
\end{poem}
\vakya यिशायाहस्तस्य प्रतापमपश्यत् तत इदं व्याहार्षीत्, तमेवोद्दिश्य स बभाषे।
\vakya तथापि नायकानामपि बहवस्तस्मिन् व्यश्सिषुः, फरीशिनां कारणात्तु तं न स्वीकृतवन्तः, समाजभ्रष्टा भविष्याम इति भयात्।
\vakya यत ईश्वरदत्तगौरवात् ते मानवदत्तं गौरवं प्रेयोऽमन्यन्त।
\vakya अपि तु यीशुरुच्चैःस्वरेण व्याहृतवान्, यो मयि विश्वसिति, स न मयि विश्वसित्यपि तु मत्प्रेषयितरि।
\vakya यश्च मां निरीक्षते स मत्प्रेषयितारं निरीक्षते।
\vakya अहं ज्योतिःस्वरूपो जगदागतवान्, यः कश्चिन्मयि विश्वसिति स यथा तिमिरे न स्थास्यति।
\vakya यश्च कश्चिन्मम वचांसि श्रुत्वा न विश्वसिति तस्य विचारो मया न क्रियते, यस्मान्न जगतो विचारं कर्तुमागतोऽहं, प्रत्युत जगत् तारयितुमेव।
\vakya यो मां निराकरोति मम वचांसि न गृह्णाति च तस्य विचारयितास्ते। मया कथितं वाक्यमेवान्तिमदिने तस्य विचारं करिष्यति।
\vakya यतोऽहं न स्वतः कथितवान्, प्रत्युत मया किं कथयितव्यं किञ्च भाषितव्यं, तन्निरूपिकाज्ञा मत्प्रेषयित्रा पित्रा मह्यमदायि।
\vakya तस्याज्ञा च यद् अनन्तजीवनावहा तदहं जानामि। अतोऽहं यद् भाषे तत् पित्रा यथोक्तस्तथैव भाषे\eoc