\adhyAya
\stitle{मेषपालकस्य दृष्टान्तं।}
\vakya सत्यं सत्यं, युष्मानहं ब्रवीमि, यो न द्वारेण मेषवाटं प्रविश्येतरोपायेनारोहति स चोरो दस्युश्च,
\vakya यस्तु द्वारेण प्रविशति स मेषाणां रक्षकः।
\vakya तस्य निमित्तं दौवारिको द्वारं मुञ्चति, मेषाश्च तस्य गिरं शृण्वन्ति, स चैकैस्य नाम्ना निजमेषानाह्वयति बहि र्नयति च।
\vakya स च यदा निजस्वान् बहिः करोति, तदा तेषामग्रतो व्रजति, मेषाश्च तमनुगच्छन्ति, यतस्ते तस्य गिरं जानन्ति।
\vakya परकीयन्तु नैवानुगमिष्यन्ति प्रत्युत तत्समीपात् पलायिष्यन्ते, यतस्ते परकीयाणां गिरं न जानन्ति।
\vakya उपमेयं यीशुना तेभ्योऽकथ्यत, ते तु तेन यदुच्यते तन्नाबुध्यन्त।
\stitle{यीशुरेव भद्रो मेषरक्षकः।}
\vakya यीशुस्तदा पुनस्तानुवाच, सत्यं सत्यं, युष्मानहं ब्रवीमि, अहमेव मेषवाटद्वारं।
\vakya यावन्तो मदग्रत आगतास्ते सर्वे चोरा दस्यवश्च, मेषान्तु तेषां नाशृण्वन्।
\vakya अहं द्वारस्वरूपः। मया यः प्रविशति, स त्राणं लप्स्यते, स च प्रवेक्ष्यति निर्यास्यति च प्रचारमाप्स्यति च।
\vakya चोरो नान्यकार्यार्थमायाति, केवलं चौर्यहत्याविनाशार्थमेव। मेषास्तु यथा जीवनम् अतिपूर्णताञ्च प्राप्नुयुस्तदर्थमहमागतः।
\vakya अहं स भद्रो मेषरक्षकः। भद्रो मेषरक्षको मेषाणां निमित्तं स्वप्राणांस्त्यजति।
\vakya यो वैतनिकस्तु न मेषरक्षकः, मेषाश्च यस्य निजस्वा न सन्ति, स वृकमागच्छन्तं निरीक्ष्य मेषान् विजहाति पलायते च। ततो वृकस्तान् हरति मेषान् विकिरति च।
\vakya वैतनिकः पलायते, यतो हेतोः स वैतनिको मेषेषु निश्चिन्तश्च।
\vakya अहं स भद्रो मेषरक्षको मदीयांश्च जानामि मदीयाश्च मां जानन्ति,
\vakya यथा पिता मां जानात्यहञ्च पितरं जानामि। मेषाणां निमित्तं स्वप्राणांस्त्यजामि च।
\vakya सन्ति ममैतन्मेषवाटस्थेभ्योऽप्यन्ये मेषाः, तेऽपि मया नेतव्याः, ते च मम गिरं श्रोष्यन्ति, इत्थं मेषव्रज एको भविष्यति, मेषरक्षकश्चैकः।
\vakya एतत्कारणान्मत्पिता मयि प्रेम करोति यतोऽहं मम प्राणांस्तथा त्यजामि यथा तान् पुन र्ग्रहीष्यामि।
\vakya नान्यः कोऽपि मत्तस्तानपहरति, स्वयमहंस्तांस्त्यजामि। तास्त्यक्तुं मम सामर्थ्यमास्ते, पुनस्तान् ग्रहीतुमपि मम सामर्थ्यमास्ते। आदेशोऽयं मत्पितृतो मयालम्भि।
\vakya एतेभ्यो वाक्येभ्यो यिहूदियेषु पुन र्भेदः सञ्जातः।
\vakya तेषां मध्ये बहवोऽवदन्, स भूताविष्ट उन्मत्तश्च, किमर्थं तस्य शृणुथ?
\vakya अन्येऽवदन्, न सन्ति भूताविष्टस्योक्तय इमाः। भूतः किमन्धानां नेत्राणि प्रसन्नीकर्तुं शक्नोति?
\stitle{यीशुरीश्वरस्य पुत्रः।}
\vakya अथ यिरूशालेमे मन्दिरोत्सर्गपर्वाभूत्। तदा हेमन्तकाल आसीत्।
\vakya एकदा धर्मधाम्नि शलोमनोऽनिन्दे विरहन् यीशु र्यिहूदीयै र्वेष्टितः पृष्टश्च,
\vakya कति कालमस्मत्प्राणान् संशयारूढान् धारयसि? त्वं चेत् ख्रीष्टस्तर्ह्यस्मान् स्पष्टं वद।
\vakya यीशुस्तान् प्रत्यवादीत्, यूयमुक्ता मया, न तु विश्वसिथ। मत्पितु र्नाम्नाहं याः क्रियाः करोमि ता एव मामधि साक्ष्यं ददति यूयन्तु विश्वसिथ।
\vakya यूयं हि न मदीयमेषाणां श्रेण्यां गण्याः। (युष्मानहं तदनुरूपमुक्तवान्।)
\vakya मम मेषा मम गिरं शृण्वन्ति, अहञ्च तान् जानामि, ते च मामनुगच्छन्ति तेभ्योऽहञ्चानन्तं जीवनं ददामि।
\vakya अनन्तकालेऽपि ते नैव विनंक्ष्यन्ति, मम कराच्च कोऽपि तान् नापहरिष्यति।
\vakya ते मह्यं येन दत्ताः स मदीयपिता सर्वेभ्यो महत्तरः। मत्पितु र्हस्ताच्च तानपहर्तुं कोऽपि न शक्नोति।
\vakya अहं पिता चैकं स्वः।
\vakya यिहूदीयास्तदा पुनस्तमाहन्तुं प्रस्तरानाददिरे।
\vakya यीशुस्तान् प्रत्यवादीत्, मत्पितुः सकाशादहं युष्मान् बहूनि हितकर्माणि दर्शितवान्, तेषां कस्य कर्मणः कारणान्मां प्रस्तरैराहथ?
\vakya यिहूदीयाः प्रतिभाषमणास्तमवादिषुः, नैव हितकर्मणः कारणात् त्वं प्रस्तरैराहन्मः, प्रत्युतेश्वरनिन्दाकारणात्, यतस्त्वं मनुष्यः सन्नात्मानमीश्वरं कुरुषे।
\vakya यीशुस्तान् प्रतिजगाद, युष्माकं शास्त्रे किं नेदं लिखितमास्ते, मयोक्तमीश्वरा यूयमिति?
\vakya स यदि तान् ईश्वरान् अभिहितवान् यान् प्रतीश्वरस्य वाक्यं प्रादुरभूत्, यदि च शास्त्रस्य लोपो न सम्भवति,
\vakya तर्हि पिता यं पवित्रीकृत्य जगति प्रहितवान् सोऽहं कथितवान् ईश्वरस्य पुत्रोऽस्मीति कारणात् किं मां वदथ त्वमीश्वरं निन्दसीति?
\vakya यदि मत्पितुः क्रिया न करोमि तर्हि मयि मा विश्वसित,
\vakya यदि तु करोमि, मयि च न विश्वसिथ, तर्हि क्रियासु तथा विश्वसित, यथा ज्ञास्यथ विश्वसिष्यथ च यन्मयि पिता तस्मिंश्चाहं स्थितः।
\vakya तदा ते पुनस्तं धर्तुमयतन्त, स तु तेषां हस्ततो निःससार।
\vakya ततः परं यीशुः पुन र्यर्दनपारस्थं तत् स्थानं प्रतस्थे यत्र प्रथमं योहनोऽस्नापयत्, तत्रैवावतस्थे च।
\vakya बहवश्च तदन्तिकमागच्छन्नवदंश्च, योहनः किमप्यभिज्ञानार्थकर्म न कृतवान्, इममधि तु योहनो यद्यदुक्तवांस्तत् सर्वं सत्यमासीत्।
\vakya बहवश्च तत्र तस्मिन् व्यश्वसिषुः\eoc