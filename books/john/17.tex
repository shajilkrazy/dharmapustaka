\adhyAya
\stitle{शिष्याणामर्थं यीशोर्प्रार्थनम्।}
\vakya सर्वमेतत् कथयित्वा यीशुः स्वर्गं प्रत्यूर्ध्वदृष्टिं कृत्वा बभाषे, पितः, समय उपस्थितः, तव पुत्रं महिमान्वितं कुरु, पुत्रोऽपि यथा त्वां महिमान्वितं करिष्यति।
\vakya त्वं ह्येतदर्थं तस्मै यावतीयमर्त्यानां कर्तृत्वं दत्तवान् यत् तस्मै त्वया यद्यददायि तत्सर्वस्मै सोऽनन्तं जीवनं दास्यति।
\vakya अनन्तं जीवनन्त्विदं यत् त एकं सत्यमीश्वरं त्वं त्वत्प्रेरितं यीशुं ख्रीष्टञ्च ज्ञास्यन्ति।
\vakya पृथिव्यामहं त्वां महिमान्वितं कृतवान् साधितवांश्च तत् कर्म यत् त्वं मया करणार्थं मयि समर्पितवान्।
\vakya अतोऽधुना, पितः, त्वत्समीपं मां महिमान्वितं कुरु तेन महिम्ना यो जगत उद्भवात् प्राक् त्वत्समीपं ममासीत्।
\vakya जगतो मध्याद् ये मनुष्यास्त्वया मह्यं दत्तास्तेषां समक्षमहं तव नाम प्रकाशितवान्। ते तवासन्, त्वञ्च मह्यं तान् दत्तवान्, ते च तव वाक्यं रचितवन्तः।
\vakya इदानीं तैरज्ञायि यत् त्वं मह्यं यद्यद् दत्तवान्, तत् सर्वं त्वत्सकाशादुत्पन्नं।
\vakya यतस्त्वं मह्यं यानि वाक्यानि दत्तवांस्तानि मया तेभ्यो दत्तानि, तैश्च तानि गृहीत्वा सत्यमज्ञायि यदहं त्वत्सकाशान्निर्गत्यायातः। ते चात्र व्यश्वसिषु र्यत् त्वं मां प्रहितवान्।
\vakya तेषां निमित्तमहं याचे। नाहं जगतो निमित्तं याचे, किन्तु त्वया ये मह्यं दत्तास्तेषां निमित्तं, यतस्ते तव।
\vakya यद्यन्मम तत् सर्वं तव, तथा यद्यत् तव तन्मम। अहञ्च तेषु महिमान्वितो जातः।
\vakya इतः प्रभृति चाहं न जगति वर्ते। इमे तु जगति वर्तन्तेऽहञ्च त्वत्समीपं गच्छामि। पवित्र पितः, मह्यं दत्ते तव नाम्नि तान् रक्ष, आवामिव ते यथैकं भविष्यन्ति।
\vakya यदा तैः सार्धं जगत्यासं तदाहं तव नाम्नि तारक्षं। त्वं मह्यं यान् दत्तवांस्तानहं रक्षितवान्। तेषां मध्ये कोऽपि न हारितः, केवलं तद् विनाशभाजनं हारितं, यतः शास्त्रेण सिद्धेन भवितव्यं।
\vakya अधुना त्वहं तव समीपं गच्छामि, भाषे चेदं जगति, ते यथा ममानन्दं सम्पूर्णमन्तरे प्राप्य भोक्ष्यन्ते।
\vakya अहं तेभ्यस्तव वाक्यं दत्तवान् जगच्च तान् अद्विषत्, यतोऽहं यथा न जगत्सम्बन्धीयस्तथा तेऽपि न जगत्सम्बन्धीयाः।
\vakya ते त्वया जगतो बहि र्नीयन्तां नाहमेतद् याचे, प्रत्युत ते त्वया दुरात्मतो रक्ष्यन्तामिति याचे।
\vakya अहं यथा जगत्सम्बन्धीयो नास्मि तेऽपि तथा जगत्सम्बन्धीया न सन्ति।
\vakya तव सत्येन तान् पवित्रीकुरु, तव वाक्यमेव सत्यस्वरूपं।
\vakya त्वं यथा मां जगति प्रहितवांस्तथाहमपि तान् जगति प्रहितवान्।
\vakya तेषां निमित्तं स्वं पवित्रीकरोमि च, यथा तेऽपि सत्येन पवित्रीकृता भविष्यन्ति।
\vakya अहं न केवलमेतेषां निमित्तं याचे, प्रत्युत तेषामपि निमित्तं य एतेषां वाक्येन मयि विश्वसिष्यन्ति,
\vakya सर्वे ते यद् एकीभविष्यन्ति। यथा मयि त्वं, पितः, त्वयि चाहं, तथावयोस्तेऽपि यद् एकीभविष्यन्ति, त्वं मां प्रहितवानत्र जगद् यद् विश्वसिष्यति।
\vakya त्वया यो महिमा मह्यं दत्तस्तमहं तेभ्यो दत्तवान्, यथावामेकं स्वस्तथा तेऽपि यदेकीभविष्यन्ति,
\vakya तेष्वहं मयि च त्वं, ते यत् सिद्धिं प्राप्यैकीभविष्यन्ति, त्वञ्च मां प्रहितवान् यथा च मयि प्रेम कृतवांस्तथा तेष्वपि प्रेम कृतवानिति जगद् यज्ज्ञास्यति।
\vakya पितः, ममाभीष्टमिदं, त्वं मह्यं यान् दत्तवान्, यत्राहं वर्ते तत्र मया सार्धं तेऽपि वर्तन्तां, जगत्स्थापनात् प्राक् च त्वं मयि प्रेम कृतवानिति हेतो र्मह्यं यं महिमानं दत्तवान्, मम तं महिमानं ते निरीक्षन्तां।
\vakya धर्ममय पितः, जगत् त्वां नाजानात्, अहन्तु त्वामजानां, त्वयाहं प्रहित इतीमेऽप्यजानन्।
\vakya अहञ्च तांस्तव नाम ज्ञापितवान् पुन र्ज्ञापयिष्यामि च, त्वं मयि यत् प्रेम कृतवांस्तद् यथा तेषु स्थास्यत्यहञ्च यथा तेषु स्थास्यामि\eoc