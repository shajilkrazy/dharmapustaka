\adhyAya
\vspace{25pt}
\vakya अत एव तदा पीलातो यीशुं गृहीत्वा कशाभिराजघान।
\vakya सैनिकजनाश्च कण्टकैः स्रजं रचयित्वा तस्य शिरस्यर्पयामासुस्तञ्च कृष्णलोहितं वसनं परिधापयामासुस्ततो
\vakya नमो यिहूदीयानां राज्ञ इति वदन्तस्तमताडयन्।
\vakya पीलातस्तदा पुन र्बहिरागत्य जनानब्रवीत्, पश्यताहं तं पुन र्युष्मत्समीपमानयामि, यतस्तस्मिन् कमप्यपराधं न लक्षयामीति युष्माभि र्ज्ञातव्यं।
\vakya ततस्तया कण्टकस्रजा तेन कृष्णलोहितवर्णेन वसनेन च राजवेशी यीशु र्बहिराजगाम, पीलातश्च बभाषे, निरीक्ष्यतामयं मनुष्य इति।
\vakya मुख्ययाजकाः पदातयश्च तदा तं दृष्ट्वा रुरुवुः, क्रुशारोपणं कुरु, क्रुशारोपणं कुरु। पीलातस्तान् ब्रूते, यूयं तं गृहीत्वा क्रुशमारोपयत, यतो मया तस्मिन् कोऽप्यपराधो न लक्ष्यते।
\vakya यिहूदीयास्तं प्रतिजगदुः, अस्माकं व्यवस्थास्ति, तयास्मदीयव्यवस्थया स मृत्युमर्हति, यतः स स्वमीश्वरस्य पुत्रं कृतवान्।
\vakya ततः पीलातो वचनमिदं श्रुत्वाधिकमभैषीत् पुनश्च राजगृहं प्रविश्य यीशुं ब्रवीति, कुत्रत्यस्त्वं?
\vakya यीशुस्तु तस्मै किमपि प्रत्युत्तरं नादात्।
\vakya ततः पीलातस्तमवादीत्, मया किं न सम्भाषसे? किं न जानासि यत् त्वां क्रुशमारोपयितुं ममाधिकारोऽस्ति, त्वां मोचयितुमधिकारोऽप्यस्ति?
\vakya यीशुः प्रतिबभाषे, ऊर्ध्वतो यदि भवते न प्रादायिष्यत, तर्हि मद्विरुद्धं भवतः कोऽप्यधिकारो नाभविष्यत्।
\vakya एतत्कारणात् स महत्तरेण पापेन लिप्तो येनाहं भवतः करे समर्पितः। तदारभ्य पीलातस्तं मोचयितुमयतत। यिहूदीयास्तूच्चैरवदन् भवान् यद्येनं मोचयति, तर्हि कैमरस्य प्रणयी न भवति। यः कश्चित् स्वं राजानं कुरुते, स कैसरं प्रति विसंवादीति।
\vakya ततः पीलातः कथामिमां श्रुत्वा यीशुं बहिरानिनाय, स्वयञ्च शिलास्तर इत्यभिहिते स्थाने विचारासन उपविवेश। इब्रीयभाषया तस्य स्थानस्य नाम गब्बथेति।
\vakya तद् दिनं निस्तारोत्सवस्य सज्जनदिनं, घटिका प्रायेण षष्ठी। ततः स यिहूदीयान् ब्रूते, निरीक्ष्यतां युष्माकं राजा।
\vakya ते तु रुरुवुः, संहरैनं संहरैनं क्रुशमारोपय। पीलातस्तान् ब्रूते, युष्माकं राजानं किं क्रुशमारोपयिष्यामि? मुख्ययाजकाः प्रतिबभाषिरे, नास्त्यस्माकं राजा कैसरादन्यः।
\vakya तदैव स तं क्रुशारोपणार्थं तेषु समर्पयामास। ते च यीशुं धृत्वापनिन्युः।
\stitle{यीशोः क्रुशारोपणं।}
\vakya ततः स स्वक्रुशं वहन् कपालस्थलमित्यभिधमर्थत इब्रीयभाषया गल्गथेतिनामकं स्थानं प्रतस्थे।
\vakya तत्रैव ते तञ्च तेन सार्धमन्यौ द्वौ च क्रुशान्यारोपयामासुः, तौ द्वौ तस्य पार्श्वयोर्यीशुश्च मध्यस्थाने।
\vakya पीलातश्चापराधविज्ञापकं पत्रं लेखयित्वा क्रुशेऽर्पयामास। तत्र लिखितमासीत्, यिहूदीयानां राजा नासरतीयो यीशुरिति।
\vakya अत एव बहवो यिहूदीयास्तत् पत्रमपठन्, यतो यीशु र्यत्र क्रुशमारोपितस्तत् स्थानं नगरस्य निकटस्थं, तत् पत्रञ्चेब्रीयै र्यूनानीयै रोमीयैश्चाक्षरै र्लिखितमासीत्।
\vakya अत एव यिहूदीयानां मुख्ययाजकाः पीलातमवादिषुः यिहूदीयानां राजेति मा लिख्यतां, प्रत्युतेदं लिख्यताम् अयं कथितवान्, यिहूदीयानां राजाहमिति।
\vakya पीलातः प्रतिबभाषे, अहं यल्लिखितवांस्तदेव लिखितवान्।
\vakya सैनिकजनास्तु यीशुं क्रुशमारोप्य तस्य वासांसि गृहीत्वैकैकस्य सैनिकस्य निमित्तमेकैकमंशमिति चतुरोऽंशान् कृत्वा बिभेजिरे, तस्याङ्गाच्छादकमप्याददुश्च। तदङ्गाच्छादकन्तु सीवनरहितम् ऊर्ध्वप्रान्तमारभ्याखिलमूतमासीत्।
\vakya ततस्ते परस्परमवदन्, मैव छिद्यतामिदं प्रत्युतेदं कस्य भविष्यति गुटिकापातं कृत्वा तन्निश्चीयताम्। अनेन शास्त्रीयमिदं वचः सिद्धं बभूव,
\begin{poem}
\startwithline “मामकीनानि वस्त्राणि स्वमध्ये विभजन्ति ते।
\pline मम परिच्छदार्थञ्च गुटिकां पातयन्ति हि॥”
\end{poem}
@V\eoc