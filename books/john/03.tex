\adhyAya
\stitle{नीकदीमं प्रति ख्रीष्टस्योपदेशकथा।}
\vakya अथ फरीशिनां मध्ये नीकदीमनामा नर आसीत्, स यिहूदीयानामेको नायकः।
\vakya स रात्रौ यीशोरन्तिकमागत्य तमवादीत्, रब्बिन्, वयं जानीमो यद् भवानीश्वरादागतो गुरुः, यतो भवान् यान्यभिज्ञानार्थकर्माणि करोति, तानि कर्तुं केनापि न शक्यानीश्वरस्य सङ्गित्वं विना।
\vakya यीशुः प्रतिभाषमाणस्तमब्रवीत्, सत्यं सत्यं, त्वामहं ब्रवीमि, पुनरादितो न जनित्वा मनुष्य ईश्वरस्य राज्यं द्रष्टुं न शक्नोति।
\vakya नीकदीमस्तं ब्रूते, वार्द्धक्यप्राप्तो मनुष्यः कथं जनितुं शक्नुयात्? स किं द्वितीयवारं मातु र्जठरं प्रविश्य जनितुं शक्नोति?
\vakya यीशुः प्रतिबभाषे, सत्यं सत्यं, त्वामहं ब्रवीमि, तोयात्मभ्यां न जनित्वा मनुष्य ईश्वरस्य राज्यं प्रवेष्टुं न शक्नोति।
\vakya यन्मांसतो जातं तन्मांसं, यदात्मतो जातं तदात्मा।
\vakya युष्माभिः पुनरादितो जनितव्यमिति मया त्वं यदुक्तस्तत्राश्चर्यं मा मन्यस्व।
\vakya आत्मरूपो वायु र्यत्रेच्छति तत्र वाति, तत्स्वनश्च त्वया श्रूयते, न ज्ञायते तु स कुत आयाति कुत्र याति वा। आत्मतो जातः सर्वमनुष्यस्तादृग्भूतः।
\vakya नीकदीमः प्रतिभाषमाणस्तमवादीत्, सर्वमेतत् कथं भवितुं शक्नोति?
\vakya यीशुः प्रतिभाषमाणस्तमाह, त्वमिस्रायेलस्य गुरुस्तथाप्येतन्न जानासि?
\vakya सत्यं सत्यं, त्वामहं ब्रवीमि, वयं यज्जानीमस्तत् कथयामः, यच्च दृष्टवन्तस्तत्र साक्ष्यं दद्मः, तथाप्यस्मत्साक्ष्यं युष्माभि र्न गृह्यते।
\vakya पार्थिवविषयान् कथितवति मयि यदि यूयं न विश्वसिथ, कथं तर्हि स्वर्गीयविषयान् कथितवति मयि विश्वसिष्यथ?
\vakya नापरः कोऽपि स्वर्गमारुरोह, केवलं स्वर्गादवरूढः स्वर्गवासी मनुष्यपुत्रः।
\vakya अपि च यथा मरौ मोशिना सर्प उच्चीकृतस्तथा मनुष्यपुत्रो यद् उच्चीकारिष्यते तत्तस्यावश्यम्भावि,
\vakya तस्मिन् विश्वासी सर्वमनुष्यो यथा न विनश्यानन्तं जीवं लप्स्यते।
\vakya यत ईश्वरो जगतीत्थं प्रेम चकार, यन्निजमेकजातं पुत्रं ददौ, तस्मिन् विश्वासी सर्वमनुष्यो यथा न विनश्यानन्तं जीवनं लप्स्यते।
\vakya ईश्वरो हि स्वपुत्रं जगति प्रहितवान् न जगतो विचारसाथनार्थं, प्रत्युत जगद् यत् तेन त्राणं लभेत तदर्थं।
\vakya यस्तस्मिन् विश्वसिति तस्य विचारो न क्रियते। यो न विश्वसिति तस्य विचारः सम्भूतः, यतः स ईश्वरस्यैकजातपुत्रस्य नाम्नि न विश्वसितवान्।
\vakya विचारश्चाय, यदागतं जगति ज्योतिः, मनुष्यास्तु ज्योतिषस्तिमिरेऽधिकमप्रीयन्त, यतस्तेषां कर्माणि दुष्टानि।
\vakya कदाचारवान् सर्वमनुष्यो हि ज्योति र्द्वेष्टि, तदीयकर्मणां दोषस्तेन व्यक्तीभविष्यतीति भयाच्च स ज्योतिषो निकटं नायाति।
\vakya यस्तु सत्यमाचरति तस्य कर्माणीश्वरे साधितानीव प्रत्यक्षीभविष्यन्तीत्याकाङ्क्षया स ज्योतिषो निकटमायाति।
\stitle{यीशुविषये योहनस्य साक्ष्यदानम्}
\vakya ततः परं यीशुस्तदीयशिष्याश्च यिहूदियाया जनपदं जग्मुः स च तत्रैव तैः सार्धं वर्तमानोऽस्नापयत्।
\vakya योहनोऽपि शालीमान्तिकस्थ ऐनोने वर्तमानोऽस्नापयत्, यतस्तत्र बह्वाप आसन् मनुष्याश्चोपस्थाय स्नापिताः।
\vakya योहनस्तदापि कारायामनिक्षिप्त आसीत्।
\vakya अथ शुचित्वसाधनमधि योहनस्य केषाञ्चिच्छिष्याणां कस्यचिद् यिहूदीयनरस्य च मिथो वादानुवादो बभूव।
\vakya तो तदा योहनस्य समीपमागत्य तमवादिषुः, रब्बिन्, यर्दनपारे यो भवता सार्धमासीत्, भवता च यस्य स्वपक्षं साक्ष्यमदायि, पश्यतु स स्नापयति, सर्वे च तत्समीपमुपतिष्ठन्ते।
\vakya योहनः प्रतिभाषमाणो जगाद, यस्मै यत् स्वर्गाददायि, तद्भिन्नं किमपि तेन मनुष्येणादातुं न शक्यते।
\vakya नाहं ख्रीष्टोऽपि तु तस्याग्रे प्रहितोऽस्मीति मया यदूचे तत्र यूयमेव मम साक्षिणः।
\vakya यः कन्यां प्राप्तवान् स एव वरः, वरस्य यो बन्धुस्तु सन्निधौ तेष्ठंस्तस्य वाक्यान्याकर्णयति, स वरस्य वाणीकारणादानन्देन मोदिते। नूनं मम स आनन्दः पूर्णो बभूव।
\vakya अमुना वर्धितव्यं, मया तु ह्रसितव्यं।
\vakya य ऊर्ध्वत आयाति स सर्वेषामुपरिस्थः। यो भूम्युत्पन्नो भौमः सः, भौमविषयान् कथयति च।
\vakya यः स्वर्गादायाति स सर्वेषामुपरिस्थः, यच्च दृष्टवान् श्रुतवांश्च तदधि साक्ष्यं ददाति।
\vakya तस्य साक्ष्यान्तु केनापि न गृह्यते। तस्य साक्ष्यं येन गृहीतं, तेनेश्वरः सत्य इति कथा मुद्रयाङ्किता।
\vakya यतो य ईश्वरेण प्रहितः स ईश्वरस्योक्ती र्भाषते। ईश्वरो ह्यामित्वात्मानं ददाति।
\vakya पिता पुत्रे प्रेम करोति, तस्य हस्ते सर्वं समर्पयामास च।
\vakya यः पुत्रे विश्वसिति तस्यानन्तं जीवनमास्ते, यस्तु पुत्रं न श्रद्दधाति स जीवनं न द्रक्ष्यति प्रत्युतेश्वरस्य क्रोधस्तस्योपर्यवतिष्ठते\eoc