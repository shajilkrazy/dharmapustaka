\adhyAya
\stitle{यीशुना स्वशिष्याणां चरणक्षालनम्।}
\vakya अथ निस्तारोत्सवात् प्राग् यीशु र्यदा जगतोऽस्मात् पितुरन्तिकमात्मगमनस्य समयमुपस्थितमजानात् तदा जगति स्थितान् यान् निजस्वान् प्रति प्रेम कृतवांस्तान् प्रत्यन्तं यावत् प्रेमाकुरुत।
\vakya अपि च शत्रुषु तत्समर्पणस्य बीजे दियावलेन शिमोनसुतस्येष्करियोतीयस्य यिहूदा हृदि निक्षिप्ते सति यदा रात्रिभोज्यमभवत्,
\vakya तदा सर्वं पित्रा मत्करयोः समर्पितमहञ्चेश्वरसकाशादागत ईश्वरसमीपं गन्तुमुद्यतश्चेति ज्ञात्वा
\vakya यीशु र्भोज्यादुत्तस्थौ स्ववसनानि मोचयित्वा च गात्रमार्जनवस्त्रमादाय तेन स्वकटिं बबन्ध।
\vakya ततः परं प्रक्षालनपात्रे तोयं निधाय शिष्याणां चरणान् प्रक्षालयितुं स्वकटौ निबद्धेन मार्जनवस्त्रेण मार्जयितुञ्चारेभे।
\vakya इत्थं स यदा शिमोनपित्रस्य समीपमागच्छत् तदा स तमवादीत्, प्रभो, भवान् किं मम चरणौ प्रक्षालयेत्?
\vakya यीशुः प्रतिभाषमाणस्तमुवाच, मया यत् क्रियते, त्वया तदिदानीं न ज्ञायते, इतः परन्तु ज्ञायिष्यते।
\vakya पित्रस्तं ब्रूते, भवान् अनन्तकालेऽपि नैव मच्चरणौ प्रक्षालयिष्यति। यीशुस्तं प्रत्यब्रवीत्, यद्यहं त्वां न प्रक्षालयामि, तर्हि मया सार्धं तवांशित्वं न भवति।
\vakya शिमोनः पित्रस्तं वदति, तर्हि न केवलं मम चरणौ, प्रत्युत करौ शिरश्चापि (प्रक्षालयतु)।
\vakya यीशुस्तं ब्रूते, यः स्नातः केवलं चरणयोः प्रक्षालनं तस्यावश्यकं स हि सर्वाङ्गः शुचिः। यूयमपि शुचयः स्थ, न तु सर्वे।
\vakya यतो यीशुः स्वसमर्पयितारमजानात्, तत्कारणाज्जगाद, न सर्वे यूयं शुचयः।
\vakya इत्थं तेषां चरणान् प्रक्षाल्य स स्ववसनान्यादहे, पुन र्भोजनायोपविश्य च तानवादीत्, युष्मान् प्रति मया यदकारि तत् किं जानीथ?
\vakya युष्माभिरहं गुरः प्रभुश्चेत्यभिधीये, तच्च भद्रं कथ्यते, यतोऽहं स एवास्मि।
\vakya अतः प्रभु र्गुरुश्च योऽहं, सोऽहं यदि युष्मच्चरणान् प्रक्षालितवांस्तर्हि परस्परं चरणप्रक्षालनं युष्माभिरपि कर्तव्यं।
\vakya यतोऽहं युष्मान् प्रति यादृशं कृतवान् तादृशं कर्म यूयमपि यत् करिष्यथ, तदर्थं युष्मान् दृष्टान्तं दर्शितवान्।
\vakya सत्यं सत्यं, युष्मानहं ब्रवीमि, नास्ति दासः स्वप्रभुतो महत्तरो नापि वा प्रेरितः प्रेषयितु र्महत्तरः।
\vakya यद्येतत् सर्वं जानीथ तर्हि तदाचरन्तो धन्या भविष्यथ।
\vakya न सर्वान् युष्मानधीदं मया कथ्यते। अहं यान् वरितवांस्तान् जानामि, अपि त्वनेन शास्त्रीयवचनेन सिद्धेन भवितव्यं,
\begin{poem}
\startwithline “\textunderscore\textunderscore\textunderscore यः पूपाशी मया सह।
\pline सोऽपि मत्प्रातिकूल्येन पार्ष्णिक्षेपं चकार ह॥”
\end{poem}
\vakya इदानीमेव सिद्धेः प्राग् यूयमुच्यध्वे मया, सिद्धिकाले यथा विश्वसिष्यथ यदहं सोऽस्मि।
\vakya सत्यं सत्यं युष्मानहं ब्रवीमि, यो मया प्रहितं नरं गृह्णाति स मां गृह्णाति, माञ्च यो गृह्णाति स मत्प्रेषयितारं गृह्णाति।
\stitle{यिहूदा विश्वासघाती भविष्यतीति ज्ञापनम्।}
\vakya इदमुक्त्वा यीशुरात्मन्युदविजत, साक्ष्यञ्च ददत् कथयामास, सत्यं सत्यं, युष्मानहं ब्रवीमि, युष्माकमेको मां शत्रुषु समर्पयिष्यति।
\vakya ततः स कमधि भाषते तत्र सन्दिहानाः शिष्याः परस्परमालोकयन्।
\vakya तदीयशिष्याणामेकस्त्वर्थतो यो यीशोः प्रियः स तस्य क्रोडमालम्ब्यासीन आसीत्।
\vakya शिमोनः पित्रः सङ्केतेन तमेवेदमादिशत्, यत् कमुद्दिश्य यीशु र्भाषते, तत् तेन पृष्ट्वा वक्तव्यं।
\vakya ततः स यीशो र्वक्षसि शिरो निपात्य तं वदति, प्रभो, स कः?
\vakya यीशुः प्रत्यवादीत्, अहं (सूपे) पूपखण्डं मज्जयित्वा यस्मै दास्यामि स एव सः। ततः स पूपखण्डं मज्जयित्वा शिमोनसुतायेष्करियोतियाय यिहूदै ददौ।
\vakya तस्मात् पूपखण्डात् परं शैतानस्तं प्रावेक्षीत्। यीशुस्तदा तमाह, त्वं यत् करोषि तत् सत्वरमेव कुरु।
\vakya किमुद्दिश्येदं तेनाकथ्यत, तद् भोजनायासीनानां केनापि नाज्ञायत।
\vakya स हि यिहूदाश्चेलसम्पुटवाहक आसीत्, तस्मात् कैश्चिदन्वमीयत, यीशुस्तमुत्सवार्थमस्मत्प्रयोज्यानि द्रव्याणि क्रेतुं किंवा दरिद्रेभ्यः किञ्चिद् दातुमाज्ञापयतीति।
\vakya स तु तं पूपखण्डं गृहीत्वा सपदि निश्चक्राम। तदा रात्रिरासीत्।
\stitle{यीशो र्नवीनाज्ञा।}
\vakya तस्मिन् निष्क्रान्ते यीशुरुवाच, इदानीं मनुष्यपुत्रो महिमान्वितोऽभूत्, ईश्वरश्च तस्मिन् महिमान्वितोऽभूत्।
\vakya तस्मिंश्चेदीश्वरो महिमान्वितोऽभूत्, तर्हीश्वरोऽपि स्वस्मिंस्तं महिमान्वितं करिष्यति, सत्वरमेव करिष्यति।
\vakya वत्साः, युष्माभिः सार्धमहमितः परं स्तोकं कालं स्थास्यामि। यूयं मां गवेषयिष्यथ, यथा तु यिहूदीयानहमुक्तवांस्तथाधुना युष्मानपि वच्मि, अहं यत् स्थानं गच्छामि तत्रोपस्थातुं युष्माभि र्न शक्यं।
\vakya नवीनामेकामाज्ञां युष्मभ्यं ददामि, यद् युष्माभिः परस्परं प्रेम कर्तव्यं। यथा युष्मान् प्रति मयैतदर्थं प्रेमाकारि यद् यूयमपि परस्परं प्रेम करिष्यत।
\vakya यदि परस्परं प्रेमाचरथ, तर्ह्यनेन सर्वे ज्ञास्यन्ति यद् यूयं मम शिष्या इति।
\vakya शिमोनः पित्रस्तं ब्रवीति, प्रभो, कुत्र याति भवान्? यीशुस्तं प्रत्युवाच, यत्र गच्छामि, तत्रेदानीं मामनुगन्तुं न शक्नोषि, पश्चात् त्वनुगमिष्यसि।
\vakya पित्रस्तं ब्रूते, किमर्थमधुनैव भवन्तमनुयातुं न शक्नुयां? भवदर्थमहं प्राणांस्त्यक्ष्यामि।
\vakya यीशुः प्रतिबभाषे, मदर्थं त्वं किं प्राणांस्त्यक्ष्यसि? सत्यं सत्यं त्वामहं ब्रवीमि, त्वं यावन्न त्रिकृत्वो मां निन्होष्यसे तावत् कुक्कुटो न रविष्यति\eoc